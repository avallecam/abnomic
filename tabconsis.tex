\section{ANEXOS}\label{anexos}

\subsection{Matriz de consistencia}\label{matriz-de-consistencia}

Ver \autoref{tab:consis} al final del documento.


\afterpage{
    \clearpage
    \newgeometry{left=2.5cm,right=0.5cm,top=0.5cm,bottom=0.5cm}
    \thispagestyle{empty}
    \begin{landscape}
        \centering
\begin{center}

\begin{tabular}{|m{3.2cm}m{3.2cm}m{3.2cm}m{3.2cm}m{3.2cm}m{3.2cm}m{3.2cm}|}
  \hline
  \textbf{Título:}
  &
  \multicolumn{6}{l|}{ %>{\centering}m{19cm}
  \begin{minipage}{19.2cm}
  Comparación de la respuesta de anticuerpos ante la %infección con 
  malaria vivax
  en pacientes de la Amazonía peruana %ciudad de Iquitos (Loreto - Perú)
  según su severidad y episodios previos %exposición previa
  mediante microarreglos de proteínas  
  \end{minipage}  
  }\\
  \cline{1-7}
  \textbf{Problema} & \textbf{Objetivos} & \textbf{Hipótesis} & \textbf{Variables} & 
  \textbf{Diseño} & \textbf{Muestra} & %\textbf{Instrumentos} & 
  \textbf{Análisis}\\
  \hline
  \begin{minipage}{3.2cm} 
  \textbf{Principal}\\
  1. ¿Cuáles son los antígenos de \textit{P. vivax} con reactividad serológica diferenciada
  ante la infección con malaria vivax entre pacientes severos y no-severos?\\
  \newline
  \textbf{Secundario}\\
  2. ¿Cuáles son los antígenos de \textit{P. vivax}  con reactividad serológica diferenciada
  ante la infección con malaria vivax entre pacientes con y sin episodios previos?\\
  \newline
  3. ¿Cuáles son las características protéicas de los antígenos de \textit{P. vivax} 
  con reactividad diferenciada o predominante
  en pacientes con malaria vivax?
  \end{minipage} 
  & 
  \begin{minipage}{3.2cm} 
  \textbf{Principal}\\
  1. Identificar antígenos de \textit{P. vivax} con reactividad serológica 
  diferenciada ante la infección con malaria vivax entre pacientes 
  severos y no-severos.\\
  \newline
  \textbf{Secundario}\\
  2. Identificar antígenos de \textit{P. vivax} con reactividad serológica 
  diferenciada ante la infección con malaria vivax entre pacientes 
  con y sin episodios previos.\\
  \newline
  3. Describir las características protéicas de los antígenos con reactividad 
  diferenciada o predominante
  \end{minipage} 
  & 
  \begin{minipage}{3.2cm} 
  .\\
  \textbf{De diferencia}\\ \textbf{entre grupos:}\\
  1. Los pacientes con malaria vivax no-severa poseen 
  mayor reactividad serológica contra antígenos de \textit{P. vivax}
  asociados a exposición, invasión o adhesión celular
  con respecto a los pacientes severos.\\
  \newline
  2. Los pacientes con episodios previos de malaria poseen
  mayor reactividad serológica contra antígenos de \textit{P. vivax}
  asociados a exposición
  con respecto a los pacientes sin episodios previos.\\
  \newline
  \textbf{Descriptiva:}\\
  3. Los antígenos con reactividad diferenciada o predominante
  se caracterizan por poseer una localización extracelular 
  y estar bajo presión selectiva por el sistema inmune.\\
  \end{minipage} 
  &
  \begin{minipage}{3.2cm} 
  \textbf{Dependiente}\\ Reactividad\\ serológica\\
  \newline 
  \textbf{Independiente}\\ Severidad\\
  \newline
  \textbf{Independiente}\\ Episodios previos\\
  \newline
  \underline{Instrumentos}:\\
  %\textbf{Reactividad serológica:}\\
  -Microarreglo de proteínas PfPv500.\\%Pf498/Pv516\\
  %\newline
  %\textbf{Severidad:}\\
  -Diagnóstico clínico y exámenes de laboratorio (criterio OMS).\\%Criterios de la OMS para malaria severa
  %\newline
  %\textbf{Episodios}\\ \textbf{previos:}\\
  -Encuesta.\\
  \newline
  \underline{Operacionalización}:\\
  -Ir a \autoref{tab:opera}
  \end{minipage} 
  &
  \begin{minipage}{3.2cm} 
  \textbf{Tipo:}\\
  Caso-Control.\\
  \newline
  \textbf{Clasificación:}\\
  -Por la finalidad: Analítico.\\
  %\newline
  -Por el Control de la asignación:\\ Observacional.\\
  %\newline
  -Por el seguimiento: Transversal.\\
  %\newline
  -Por la relación cronológica:\\ Prospectivo.\\
  %\newline
  -Por la unidad de análisis:\\ Basado en individuo.
  \end{minipage}   
  &
  \begin{minipage}{3.2cm} 
  \textbf{Universo teórico:}\\ 
  Pacientes con malaria vivax 
  de la cuenca amazónica del Perú.\\
  \newline
  \textbf{Marco}\\ \textbf{Muestral:}\\
  Pacientes diagnosticados con malaria vivax de la ciudad de Iquitos, Loreto-Perú, 
  entre enero del 2012 y junio del 2013.\\
  \newline
  \textbf{Muestra:}\\
  Selección aleatoria simple
  de 30 pacientes con más de 2 criterios de malaria severa (casos) y 
  30 no-severos (control).\\
  \newline
  \textbf{Tipo:}\\ Probabilística.
  \end{minipage}   
  &
%  \begin{minipage}{3.2cm} 
%  \textbf{Reactividad serológica:}\\
%  Microarreglos de\\proteínas Pf498/Pv516\\
%  \newline
%  \textbf{Severidad:}\\
%  Diagnóstico \\clínico \\y exámenes de \\laboratorio\\ según criterios de la OMS.\\
%  \newline
%  \textbf{Episodios}\\ \textbf{previos:}\\
%  Encuesta
%  \end{minipage}   
%  &
  \begin{minipage}{3.2cm} 
  \underline{Control de Calidad}:\\
  \newline
  \textbf{1. Validez y}\\ \textbf{Reproducibilidad:}\\
  correlación de Pearson o Spearman\\
  \newline
  \underline{Prueba de Hipótesis}:\\
  \newline
  \textbf{2. Amplitud}\\ \textbf{e intensidad de}\\ \textbf{respuesta:}\\
  prueba t-Student o Mann-Whitney\\
  \newline
  \textbf{3.}\\ \textbf{Reactividad}\\ \textbf{diferenciada:}\\%\textbf{3. Inferencia}\\
  test-t moderado con\\
  corrección del FDR\\por el método\\Benjamini-Hochberg\\
  \newline
  \textbf{4. Clasificación:}\\
  agrupamiento jerárquico o\\ \textit{hierarchical}\\ \textit{clustering}\\
  en base a la\\ distancia euclidiana.\\
  \newline
  \textbf{5. Descripción:}\\
  características disponibles en\\ PlasmoDB.
  \end{minipage}   
  \\
  %\cline{1-5}
  %y & z & m & n & y & z & m & n\\
  \hline
  % etc. ...
\end{tabular}

\end{center}
        \captionof{table}{Matriz de consistencia}
        \label{tab:consis}
    \end{landscape}
    \restoregeometry
    \clearpage
}

