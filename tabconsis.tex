\afterpage{
    \clearpage
    \newgeometry{left=6cm,right=3cm,top=1cm,bottom=1cm}
    \thispagestyle{empty}
    \begin{landscape}
        \centering
\begin{center}
\begin{tabular}{ll >{\centering}m{2cm} m{2cm}m{2cm}m{2cm}m{1.5cm}m{1.6cm} @{}m{0pt}@{} }
  \hline
  
  \multirow{2}{*}{Problema} &
  \multirow{2}{*}{Objetivo} & 
  \multirow{2}{*}{Variable} & 
  \multicolumn{5}{c}{Operacionalización de variable} &\\[1ex]
  \cline{4-8}
  
   & 
   & 
   & 
   Dimensión & 
   Indicador & 
   Instrumento & 
   Escala & 
   Fuente &\\[1ex]
  \hline
  
  \multirow{3}{*}{
  \begin{minipage}{5cm}
  \textit{Plasmodium vivax} es responsable del 90\% de los casos 
  de malaria en el Perú, por los que el desarrollo de una vacuna es 
  urgente.\\
  \newline
  Si bien se conocen marcadores de exposición e inmunidad, 
  las respuestas de anticuerpos a larga escala aún no están 
  completamente identificadas.\\
  \newline
  El presente estudio identificará antígenos con reactividad serológica 
  y relevancia clínica en \textit{P. vivax}
  a partir del perfil de anticuerpos en respuesta a la infección.
  \end{minipage}
  } & 
  
  \begin{minipage}{4cm}
  \underline{General}\\
  Identificar un subconjunto de antígenos con reactividad serológica
  y relevancia clínica en \textit{P. vivax}.
  \end{minipage} & 
  
  
  Reactividad serológica & 
  Reactividad antígeno-anticuerpo & 
  \begin{minipage}{2cm} 
  Intensidad de fluorescencia:\\
  \textbf{0-6000} MFI
  \end{minipage} & 
  Lector de microarreglos & 
  Razón &
  Plasma sanguíneo &\\[14ex]
  \cline{2-8}
  
   & 
  \begin{minipage}{4cm}
  \underline{Específico}\\
  Identificar antígenos de \textit{P. vivax} con reactividad diferenciada
  entre pacientes con y sin eventos previos reportados.
  \end{minipage} & 
  
  Exposición & 
  Eventos previos reportados & 
  \begin{minipage}{2cm} 
  \textbf{0:} sin eventos previos.\\
  \textbf{1 o más:} con eventos previos.
  \end{minipage} & 
  Entrevista & 
  Nominal &
  Paciente &\\[16ex]
  \cline{2-8}

   & 
  \begin{minipage}{4cm}
  \underline{Específico}\\
  Identificar antígenos de \textit{P. vivax} con reactividad diferenciada
  entre pacientes con malaria severa y no-severa.
  \end{minipage} & 
  
  Severidad & 
  Criterios OMS de malaria severa & 
  \begin{minipage}{2cm} 
  \textbf{0:} malaria no-severa.\\
  \textbf{1 o más:} malaria severa.
  \end{minipage} & 
  Diagnóstico clínico y exámenes de laboratorio & 
  Nominal &
  Médico y muestra de sangre &\\[16ex]
  \hline

  % etc. ...
\end{tabular}
\end{center}
        \captionof{table}{Matriz de consistencia}
        \label{tab:consis}
    \end{landscape}
    \restoregeometry
    \clearpage
}

