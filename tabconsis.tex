\afterpage{
    \clearpage
    \newgeometry{left=3cm,right=3cm,top=0.5cm,bottom=0.5cm}
    \thispagestyle{empty}
    \begin{landscape}
        \centering
\begin{center}

\begin{tabular}{ll >{\centering}m{2.4cm} m{2.2cm}m{2.2cm}m{2cm}m{2.2cm}m{1.7cm}m{1.5cm}m{1.6cm} @{}m{0pt}@{} }
  
  \hline

  
  \multirow{3}{*}{Problema} &
  \multirow{3}{*}{Objetivo} & 
  \multicolumn{8}{c}{Operacionalización de variable} &\\[1ex]
  \cline{3-10}


  &
  &
  \multirow{2}{*}{Variable}
  & 
  \multicolumn{2}{c}{Definición} 
  %&
  %\begin{minipage}{2.2cm}
  %Definición\\conceptual
  %\end{minipage}
  %&
  %\begin{minipage}{2.2cm}
  %Definición\\operacional
  %\end{minipage}
  & 
  \multirow{2}{*}{
  \begin{minipage}{2.2cm}
  Instrumento\\de medición
  \end{minipage}
  }
  &
  \multirow{2}{*}{
  \begin{minipage}{2.2cm}
  Criterios\\de medición
  \end{minipage}
  }
  &
  \multirow{2}{*}{
  \begin{minipage}{1.7cm}
  Tipo de\\variable
  \end{minipage}
  }
  &
  \multirow{2}{*}{
  \begin{minipage}{1.5cm}
  Escala de \\medición
  \end{minipage}
  }
  &
  \multirow{2}{*}{
  Fuente
  } &\\[0ex]
  %\hline
  \cline{4-5}
  
  &
  &
  &
  Conceptual
  &
  Operacional
  & 
  &
  &
  & &\\[1ex]
  \hline
 
  \multirow{3}{*}{
  \begin{minipage}{4cm}
  \textit{Plasmodium vivax} es responsable del 90\% de los casos 
  de malaria en el Perú, por los que el desarrollo de una vacuna es 
  urgente.\\
  \newline
  Si bien se conocen marcadores de exposición e inmunidad, 
  las respuestas de anticuerpos a larga escala aún no están 
  completamente identificadas.\\
  \newline
  El presente estudio identificará antígenos con reactividad serológica 
  y relevancia clínica en \textit{P. vivax}
  a partir del perfil de anticuerpos en respuesta a la infección.
  \end{minipage}
  } & 

  \begin{minipage}{3cm}
  \underline{General}\\
  Identificar un subconjunto de antígenos de \textit{P. vivax} 
  con reactividad serológica 
  predominante o discriminante de condiciones clínicas
  ante infecciones con malaria vivax.
  \end{minipage} & 

  \textbf{Dependiente} Reactividad serológica
  & 
  % esCONCEPTUAL: 
  \begin{minipage}{2.2cm} 
  Especificidad \\de anticuerpos \\de respuesta contra un antígeno
  \end{minipage} 
  &
  % aOPERACIONAL: 
  \begin{minipage}{2.2cm} 
  Medida \\indirecta de \\la reacción antígeno-anticuerpo
  \end{minipage} 
  % aDETALLES: medida indirecta de la reacción antígeno-anticuerpo 
  % mediante la lectura de la reacción fluorescente entre 
  % anticuerpo secundario y fluoroforo por spot
  & 
  \begin{minipage}{2.2cm} 
  Lector de\\
  microarreglos
  \end{minipage}
  & 
  \begin{minipage}{2.2cm} 
  \textbf{0-6000} MFI o intensidad\\
  fluorescente \\promedio.
  \end{minipage} 
  &
  Numérica contínua
  & 
  Razón
  &
  Plasma sanguíneo &\\[28ex]
  \cline{2-10}

   & 
  \begin{minipage}{3cm}
  \underline{Específico}\\
  Identificar antígenos con reactividad diferenciada según 
  la severidad de los pacientes con 
  malaria vivax.
  \end{minipage} & 


  \textbf{Independiente} Severidad
  & 
  % aCONCEPTUAL: 
  Presencia de manifestaciones clínicas severas o complicaciones sistémicas
  &
  % aOPERACIONAL:
  Número de criterios OMS para malaria severa
  & 
  \begin{minipage}{2.2cm} 
  Diagnóstico \\clínico \\y exámenes de \\laboratorio 
  \end{minipage}
  & 
  \begin{minipage}{2.2cm} 
  \textbf{No-severa:} 0 criterios.\\
  \textbf{Severa:} 1 o más criterios.
  \end{minipage}
  &
  Categórica dicotómica
  & 
  Nominal
  &
  Historia clínica y muestra de sangre &\\[20ex]
  \cline{2-10}

   & 
  \begin{minipage}{3cm}
  \underline{Específico}\\
  Identificar antígenos con reactividad diferenciada según 
  la exposición previa de los pacientes con 
  malaria vivax.
  \end{minipage} & 

  
  \textbf{Independiente} Exposición previa
  & 
  % aCONCEPTUAL: 
  Presencia de infecciones de malaria en el pasado
  &
  % aOPERACIONAL:
  Número de eventos previos reportados 
  & 
  Encuesta
  & 
  \begin{minipage}{2.2cm} 
  \textbf{Sin:} 0 eventos.\\
  \textbf{Con:} 1 o más eventos.
  \end{minipage}
  &
  Categórica dicotómica
  & 
  Nominal
  &
  Historia clínica &\\[20ex]
  \hline
  %\cline{2-10}

%  \textbf{Interviniente} Edad %Confusora
%  & 
%  % aCONCEPTUAL: 
%  Edad del paciente
%  &
%  % aOPERACIONAL:
%  Años de vida reportados
%  & 
%  Encuesta
%  & 
%  \begin{minipage}{2.2cm} 
%  \textbf{0-90} años.
%  \end{minipage}
%  &
%  Numérica discreta
%  & 
%  Razón
%  &
%  Historia clínica &\\[10ex]
%  \hline


  % etc. ...
\end{tabular}

\end{center}
        \captionof{table}{Matriz de consistencia}
        \label{tab:consis}
    \end{landscape}
    \restoregeometry
    \clearpage
}

