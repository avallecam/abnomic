\documentclass[]{article}
\usepackage{lmodern}
\usepackage{amssymb,amsmath}
\usepackage{ifxetex,ifluatex}
\usepackage{fixltx2e} % provides \textsubscript
\ifnum 0\ifxetex 1\fi\ifluatex 1\fi=0 % if pdftex
  \usepackage[T1]{fontenc}
  \usepackage[utf8]{inputenc}
\else % if luatex or xelatex
  \ifxetex
    \usepackage{mathspec}
  \else
    \usepackage{fontspec}
  \fi
  \defaultfontfeatures{Ligatures=TeX,Scale=MatchLowercase}
\fi
% use upquote if available, for straight quotes in verbatim environments
\IfFileExists{upquote.sty}{\usepackage{upquote}}{}
% use microtype if available
\IfFileExists{microtype.sty}{%
\usepackage{microtype}
\UseMicrotypeSet[protrusion]{basicmath} % disable protrusion for tt fonts
}{}
\usepackage[margin=1in]{geometry}
\usepackage{hyperref}
\hypersetup{unicode=true,
            pdfborder={0 0 0},
            breaklinks=true}
\urlstyle{same}  % don't use monospace font for urls
\usepackage{longtable,booktabs}
\usepackage{graphicx,grffile}
\makeatletter
\def\maxwidth{\ifdim\Gin@nat@width>\linewidth\linewidth\else\Gin@nat@width\fi}
\def\maxheight{\ifdim\Gin@nat@height>\textheight\textheight\else\Gin@nat@height\fi}
\makeatother
% Scale images if necessary, so that they will not overflow the page
% margins by default, and it is still possible to overwrite the defaults
% using explicit options in \includegraphics[width, height, ...]{}
\setkeys{Gin}{width=\maxwidth,height=\maxheight,keepaspectratio}
\usepackage[normalem]{ulem}
% avoid problems with \sout in headers with hyperref:
\pdfstringdefDisableCommands{\renewcommand{\sout}{}}
\IfFileExists{parskip.sty}{%
\usepackage{parskip}
}{% else
\setlength{\parindent}{0pt}
\setlength{\parskip}{6pt plus 2pt minus 1pt}
}
\setlength{\emergencystretch}{3em}  % prevent overfull lines
\providecommand{\tightlist}{%
  \setlength{\itemsep}{0pt}\setlength{\parskip}{0pt}}
\setcounter{secnumdepth}{5}
% Redefines (sub)paragraphs to behave more like sections
\ifx\paragraph\undefined\else
\let\oldparagraph\paragraph
\renewcommand{\paragraph}[1]{\oldparagraph{#1}\mbox{}}
\fi
\ifx\subparagraph\undefined\else
\let\oldsubparagraph\subparagraph
\renewcommand{\subparagraph}[1]{\oldsubparagraph{#1}\mbox{}}
\fi

%%% Use protect on footnotes to avoid problems with footnotes in titles
\let\rmarkdownfootnote\footnote%
\def\footnote{\protect\rmarkdownfootnote}

%%% Change title format to be more compact
\usepackage{titling}

% Create subtitle command for use in maketitle
\newcommand{\subtitle}[1]{
  \posttitle{
    \begin{center}\large#1\end{center}
    }
}

\setlength{\droptitle}{-2em}
  \title{}
  \pretitle{\vspace{\droptitle}}
  \posttitle{}
  \author{}
  \preauthor{}\postauthor{}
  \date{}
  \predate{}\postdate{}

\usepackage{multirow}
\usepackage{pdflscape}
\usepackage{afterpage}
\usepackage{capt-of}
\usepackage{array}
\usepackage{color}

\begin{document}

\renewcommand{\contentsname}{Índice General} 
\renewcommand{\tablename}{Tabla}
\renewcommand{\tableautorefname}{Tabla}

\pagenumbering{gobble}

\clearpage
\newgeometry{top=0.2cm,bottom=1cm} \vspace*{\fill}

\begin{centering}

\begin{figure}[!ht]
  \begin{center}
    \includegraphics[width=.8in]{figure/UNMSM_escudo-2000px.png}
  \end{center}
\end{figure}

\Large
UNIVERSIDAD NACIONAL MAYOR DE SAN MARCOS

\large
(Universidad del Perú, DECANA DE AMÉRICA)

\vspace{.5 cm}

\Large
FACULTAD DE CIENCIAS BIOLÓGICAS

\vspace{.5 cm}

\normalsize
ESCUELA ACADÉMICO PROFESIONAL DE

GENÉTICA Y BIOTECNOLOGÍA

\vspace{1.5 cm}

\Large
Comparación de la respuesta de anticuerpos ante la %infección con 
malaria vivax 
en pacientes de la Amazonía peruana %ciudad de Iquitos (Loreto - Perú)
según su severidad y episodios previos %exposición previa
mediante microarreglos de proteínas
% aPerfil de anticuerpos en respuesta a la infección con malaria vivax
%mediante un enfoque inmunómico .
%% con sintomatología severa y no complicada/ pre-inmunes/ semi-inmunes

\vspace{1.5 cm}

\Large
Proyecto de Tesis para obtar al Título Profesional de 

Biólogo Genetista y Biotecnólogo

\vspace{1 cm}

\Large
Bach. Andree Adolfo Valle Campos

\vspace{1.2 cm}

\Large
Asesores 

Interno: <por confirmar>%Prof. Walter Cabrera-Febolá

Externo: PhD. G. Christian Baldeviano


\vspace{1.2 cm}

Lima - Perú

\vspace{.5 cm}

2017

\end{centering}

\vfill
\restoregeometry
\clearpage

\newpage

\tableofcontents

\newpage

\pagenumbering{arabic}

\section*{RESUMEN}\label{resumen}
\addcontentsline{toc}{section}{RESUMEN}

\sout{RESUMEN A ESCRIBIR Y CORREGIR AL FINAL!!!! El presente proyecto
tiene como objetivo identificar antígenos con relevancia clínica contra
la malaria vivax. Para ello, se realizará un perfil a larga escala de
anticuerpos en pacientes sintomáticos a través de microarreglos de 500
proteínas de \emph{Plasmodium vivax} y 500 de \emph{P. falciparum}. Se
contrastarán parámetros generales de la respuesta humoral y evaluará la
reactividad diferenciada de antígenos en pacientes estratificados por
severidad y episodios previos. Finalmente, se propondrán nuevos
antígenos de \emph{P. vivax} con posible aplicación en la vigilancia
sero-epidemiológica. Finalmente, se propondrán nuevos antígenos de
\emph{P. vivax} con posible aplicación en la vigilancia
sero-epidemiológica. Finalmente, se propondrán nuevos antígenos de
\emph{P. vivax} con posible aplicación en la vigilancia
sero-epidemiológica. Finalmente, se propondrán nuevos antígenos de
\emph{P. vivax} con posible aplicación en la vigilancia
sero-epidemiológica. Finalmente, se propondrán nuevos antígenos de
\emph{P. vivax} con posible aplicación en la vigilancia
sero-epidemiológica.}

\section{PLANTEAMIENTO DEL PROBLEMA}\label{planteamiento-del-problema}

\subsection{Formulación del problema}\label{intro}

Malaria es una enfermedad infecciosa de importancia mundial, causada por
protozoarios parásitos del género \emph{Plasmodium} y transmitida por
mosquitos del género \emph{Anopheles}. En el 2015, se estimaron 212
millones de casos y 429,000 muertes atribuidas a esta
infección.\textsuperscript{\protect\hyperlink{ref-WHO2016world}{1}}
Aunque \emph{P. falciparum} representó el 96 y 99\% de estas cifras,
fuera de África se estimó que \emph{P. vivax} fue responsable del 41 y
86\%, respectivamente. Más aún, la región de las Américas tuvo la mayor
proporción de estos casos (69\%) donde Perú fue el tercer país con más
reportes (n=63,153), detrás de Brasil y Venezuela, atribuidos en un 80\%
a dicha
especie.\textsuperscript{\protect\hyperlink{ref-rosas2016peru}{2}}

Reportes recientes de malaria severa y fatal causados por \emph{P.
vivax} han desafiado su tradicional condición de enfermedad
benigna.\textsuperscript{\protect\hyperlink{ref-baird2009}{3},\protect\hyperlink{ref-quispe2014}{4}}
El incremento de estos casos puede ser consecuencia de un retrazo en la
adquisición de inmunidad en la población por una reducción en la
intensidad de
transmisión.\textsuperscript{\protect\hyperlink{ref-reyburn2015}{5}} Por
ello, tanto su actual
subestimación\textsuperscript{\protect\hyperlink{ref-llanoschea2015}{6}}
como la auscencia de marcadores serológicos de protección contra la
severidad en las actuales estrategias de
eliminación\textsuperscript{\protect\hyperlink{ref-accelerate2016}{7}}
podría incrementar su prevalencia a largo plazo. En contraste con dicha
hipótesis, pacientes de la amazonía peruana con malaria vivax severa y
no complicada mostraron un grado similar de exposición previa, basado en
la respuesta humoral contra un marcador
tradicional.\textsuperscript{\protect\hyperlink{ref-baldevi2013}{8}} Sin
embargo, recientes estudios a larga escala han desafiado su validez como
indicador de inmunidad o
exposición.\textsuperscript{\protect\hyperlink{ref-crompton2010}{9},\protect\hyperlink{ref-Helb2015exposure}{10}}

Por esta razón, el presente estudio tiene como objetivo comparar la
respuesta de anticuerpos ante la infección con malaria vivax contra más
de 500 antígenos de \emph{P. vivax} en pacientes severos y no-severos de
la ciudad de Iquitos, empleando microarreglos de proteínas.
Hipotetizamos que los no-severos poseen mayor reactividad serológica
contra antígenos de exposición o invasión y adhesión celular, con
respecto a los severos. Primero, se identificarán los antígenos con
reactividad diferenciada entre ambos grupos. Luego, se determinarán los
antígenos de respuesta secundaria en pacientes con episodios previos de
esta infección. Finalmente, se describirán sus características protéicas
junto a los antígenos con mayor reactividad en toda la muestra con el
propósito de proponerlos como candidatos a vigilancia
sero-epidemiológica.

\subsection{Preguntas de
investigación}\label{preguntas-de-investigacion}

\begin{enumerate}
\def\labelenumi{\arabic{enumi}.}
\item
  ¿Cuáles son los antígenos de \emph{P. vivax} con reactividad
  serológica diferenciada ante la infección con malaria vivax entre
  pacientes severos y no-severos?
\item
  ¿Cuáles son los antígenos de \emph{P. vivax} con reactividad
  serológica diferenciada ante la infección con malaria vivax entre
  pacientes con y sin episodios previos?
\item
  ¿Cuáles son las características protéicas de los antígenos de \emph{P.
  vivax} con reactividad diferenciada o predominante en pacientes con
  malaria vivax?
\end{enumerate}

\subsection{Objetivos}\label{objetivos}

\subsubsection{General}\label{general}

\begin{itemize}
\tightlist
\item
  Identificar un subconjunto de antígenos de \emph{P. vivax} con
  reactividad serológica discriminante de condiciones clínicas
  relevantes ante la infección con malaria vivax.
\end{itemize}

\subsubsection{Específicos}\label{especificos}

\begin{itemize}
\item
  Identificar antígenos de \emph{P. vivax} con reactividad serológica
  diferenciada ante la infección con malaria vivax entre pacientes
  severos y no-severos.
\item
  Identificar antígenos de \emph{P. vivax} con reactividad serológica
  diferenciada ante la infección con malaria vivax entre pacientes con y
  sin episodios previos.
\item
  Describir las características protéicas de los antígenos con
  reactividad diferenciada o predominante.
\end{itemize}

\subsubsection{Exploratorio}\label{exploratorio}

\begin{itemize}
\tightlist
\item
  Comparar la amplitud e intensidad de respuesta de anticuerpos según la
  edad de los pacientes con malaria vivax.
\end{itemize}

\subsection{Justificación}\label{justif}

Ante la reemergencia repetitiva de la malaria en la amazonía
peruana,\textsuperscript{\protect\hyperlink{ref-rosas2016peru}{2},\protect\hyperlink{ref-griffing2013history}{11},\protect\hyperlink{ref-soto2017spatio}{12}}
la implementación de vigilancias serológicas programáticas en nuestro
país es una
prioridad.\textsuperscript{\protect\hyperlink{ref-hotspots2015}{13}} En
zonas con baja transmisión, estos ensayos poseen una mayor sensibilidad
y representan un menor gasto económico en comparación a las estrategias
convensionales de monitoreo y
control.\textsuperscript{\protect\hyperlink{ref-elliott2014}{14}} Por
esta razón, el presente estudio se justifica en el descubrimiento de
antígenos potencialmente discriminantes de condiciones clínicas como
severidad o exposición que permitirán optimizar y acelerar su ejecución
en programas de salud pública contra la malaria en el Perú en miras a su
control y posterior
eliminación.\textsuperscript{\protect\hyperlink{ref-accelerate2016}{7}}

\subsection{Limitaciones}\label{limit}

Resumimos tres limitaciones del estudio.

Primero, las muestras a evaluar provienen de pacientes con infecciones
no sincronizadas. El muestreo se ejecutó en la fase sintomática, momento
en el que acuden al Hospital, con un desconocimiento del inicio de la
infección por paciente. Sin embargo, infecciones experimentales han
estimado que dicha fase ocurre normalmente entre los 11 y 13 días de
infección.\textsuperscript{\protect\hyperlink{ref-arevalo2014}{15}}

Segundo, el estudio posee debilidades propias del diseño experimental a
emplear. El diseño de tipo Caso Control es suceptible a errores
sistemáticos de selección y clasificación, posee dificultad para
establecer relaciones causa-efecto y a través de él no se pueden
calcular prevalencias o incidencias. Además, la auscencia de un
seguimiento activo impidió el registro de covariables relevantes para la
caracterización de la enfermedad.

Tercero, los microarreglos de proteínas poseen limitantes propias a su
fabricación.\textsuperscript{\protect\hyperlink{ref-vigil2010}{16}} Cada
paso (amplificación, clonamiento y expresión de genes a larga escala)
posee una eficiencia límite que afectará la calidad final de las
proteínas. Además, el plegamiento protéico no será posible de verificar
a dicha escala. Por último, la identificación de antígenos con
modificaciones post-transcriptacionales, particularmente relevantes en
\emph{Plasmodium}\textsuperscript{\protect\hyperlink{ref-leroch2009postmod}{17}},
no serán posibles de reproducir en su integridad en el sistema de
expresión procarionte. A pesar de ello, se evaluará la validez y
reproducibilidad del ensayo con controles internos a detallar en la
\protect\hyperlink{validez}{sección 4.3.2}.

\subsection{Viabilidad}\label{viabilidad}

El proyecto es viable dada la factibilidad de su ejecución y el nivel de
interés para el área de conocimiento.

En primer lugar, su factibilidad se sustenta en que el proyecto se
ejecutará sobre los resultados de un estudio prospectivo ya realizado,
facilitando el proceso de muestreo y limitándolo al planteamiento del
problema y análisis de los datos. Además, dado que los microarreglos de
proteínas requieren de pequeño volúmen de plasma por paciente
(\textasciitilde{}2\(\mu\)L), se contará con un adecuado tamaño muestral
(n=60) el cual permitirá poner a prueba las hipótesis formuladas con
suficiente poder estadístico.

Finalmente, un aspecto novedoso del proyecto será el uso del lenguaje de
programación y software libre
R\textsuperscript{\protect\hyperlink{ref-R}{18}} para el análisis de
datos. Esta plataforma facilitará la reproducibilidad del estudio,
promoviendo su reutilización y contribución por otros
grupos.\textsuperscript{\protect\hyperlink{ref-CienciaReproducible2016}{19}}
Este enfoque de análisis permitirá acelerar el progreso en esta área de
investigación, automatizar procesos de análisis y reducir las brechas
técnicas que que continúan limitando el descubrimiento de biomarcadores
relevantes para la aplicación de vigilancias sero-epidemiológicas en
nuestro país.

\section{FORMULACIÓN DE HIPÓTESIS Y
VARIABLES}\label{formulacion-de-hipotesis-y-variables}

\subsection{Hipótesis}\label{hipotesis}

\subsubsection{De diferencia entre
grupos}\label{de-diferencia-entre-grupos}

\begin{enumerate}
\def\labelenumi{\arabic{enumi}.}
\item
  Los pacientes con malaria vivax no-severa poseen mayor reactividad
  serológica contra antígenos de \emph{P. vivax} asociados a exposición
  o invasión y adhesión celular con respecto a los pacientes severos.
\item
  Los pacientes con episodios previos de malaria poseen mayor
  reactividad serológica contra antígenos de \emph{P. vivax} asociados a
  exposición con respecto a los pacientes sin episodios previos.
\end{enumerate}

\subsubsection{Descriptiva}\label{descriptiva}

\begin{enumerate}
\def\labelenumi{\arabic{enumi}.}
\setcounter{enumi}{2}
\tightlist
\item
  Los antígenos con reactividad diferenciada o predominante se
  caracterizan por poseer una localización extracelular y estar bajo
  presión selectiva por el sistema inmune.
\end{enumerate}

\subsection{Variables}\label{variables}

El presente estudio busca contrastar la respuesta de anticuerpos ante la
infección con malaria vivax entre pacientes clasificados según su
severidad y episodios previos. Para ello se medirá la variable de
reactividad serológica de antígenos de \emph{P. vivax} luego de sondear
plasma de pacientes en un microarreglo de proteínas. Además se
clasificará a los pacientes como severos o no-severos de acuerdo a
diagnóstico clínico y bioquímico, y como con o sin episodios previos de
acuerdo a lo reportado en la encuesta con el médico tratante.

Primero, la reactividad serológica cuantificará la especificidad de los
anticuerpos de respuesta contra los antígenos del patógeno \emph{P.
vivax}. Esta variable medirá indirectamente la intensidad fluorescente
producto de la interacción entre antígeno y anticuerpo, a través de un
lector de microarreglos.

Segundo, la clasificación de pacientes como severos representará la
presencia de manifestaciones clínicas severas o complicaciones
sistémicas. Esta variable será asignada con el cumplimiento de al menos
dos criterios clínicos o de laboratorio para malaria severa, tal como lo
recomienda la OMS.

Tercero, la clasificación por episodios previos reportados representará
la presencia de infecciones previas con malaria y, consecuentemente, la
presencia de una respuesta inmune humoral secundaria. Esta variable será
reportada por el paciente como el número de eventos previos. Dada la
incertidumbre de dicho dato, la variable será dicotomizada con el fin de
facilitar su inclusión.

\subsubsection{Operacionalización de
variables}\label{operacionalizacion-de-variables}

Ver \autoref{tab:opera}.

\begin{table}[ht]
\begin{center}
\hspace*{-1cm}
\begin{tabular}{>{\centering}m{2.4cm} m{2.2cm}m{2.2cm}m{2cm}m{2.2cm}m{1.7cm}m{1.5cm}m{1.6cm} @{}m{0pt}@{} }
  
  \hline
  \multirow{2}{*}{Variable}
  & 
  \multicolumn{2}{c}{Definición} 
  %&
  %\begin{minipage}{2.2cm}
  %Definición\\conceptual
  %\end{minipage}
  %&
  %\begin{minipage}{2.2cm}
  %Definición\\operacional
  %\end{minipage}
  & 
  \multirow{2}{*}{
  \begin{minipage}{2.2cm}
  Instrumento\\de medición
  \end{minipage}
  }
  &
  \multirow{2}{*}{
  \begin{minipage}{2.2cm}
  Criterios\\de medición
  \end{minipage}
  }
  &
  \multirow{2}{*}{
  \begin{minipage}{1.7cm}
  Tipo de\\variable
  \end{minipage}
  }
  &
  \multirow{2}{*}{
  \begin{minipage}{1.5cm}
  Escala de \\medición
  \end{minipage}
  }
  &
  \multirow{2}{*}{
  Fuente
  } &\\[0ex]
  %\hline
  \cline{2-3}
  
  &
  Conceptual
  &
  Operacional
  & 
  &
  &
  & &\\[1ex]
  \hline
  
  \textbf{Dependiente} Reactividad serológica
  & 
  % esCONCEPTUAL: 
  \begin{minipage}{2.2cm} 
  Especificidad \\de anticuerpos \\de respuesta contra un antígeno
  \end{minipage} 
  &
  % aOPERACIONAL: 
  \begin{minipage}{2.2cm} 
  Medida \\indirecta de \\la reacción antígeno-anticuerpo
  \end{minipage} 
  % aDETALLES: medida indirecta de la reacción antígeno-anticuerpo 
  % mediante la lectura de la reacción fluorescente entre 
  % anticuerpo secundario y fluoroforo por spot
  & 
  \begin{minipage}{2.2cm} 
  Lector de\\
  microarreglos
  \end{minipage}
  & 
  \begin{minipage}{2.2cm} 
  \textbf{0-6000} MFI o intensidad\\
  fluorescente \\promedio.
  \end{minipage} 
  &
  Numérica contínua
  & 
  Razón
  &
  Plasma sanguíneo &\\[13ex]
  \hline

  \textbf{Independiente} Severidad
  & 
  % aCONCEPTUAL: 
  Presencia de manifestaciones clínicas severas o complicaciones sistémicas
  &
  % aOPERACIONAL:
  Número de criterios OMS para malaria severa
  & 
  \begin{minipage}{2.2cm} 
  Diagnóstico \\clínico \\y exámenes de \\laboratorio 
  \end{minipage}
  & 
  \begin{minipage}{2.2cm} 
  \textbf{No-severa:} 0 criterios.\\
  \textbf{Severa:} 2 o más criterios.
  \end{minipage}
  &
  Categórica dicotómica
  & 
  Nominal
  &
  Historia clínica y muestra de sangre &\\[15ex]
  \hline
  
  \textbf{Independiente} Episodios previos
  & 
  % aCONCEPTUAL: 
  Exposición a la infección de malaria en el pasado
  &
  % aOPERACIONAL:
  Número de episodios previos reportados 
  & 
  Encuesta
  & 
  \begin{minipage}{2.2cm} 
  \textbf{Sin:} 0 episodios.\\
  \textbf{Con:} 1 o más episodios.
  \end{minipage}
  &
  Categórica dicotómica
  & 
  Nominal
  &
  Historia clínica &\\[10ex]
  \hline

%  \textbf{Interviniente} Edad %Confusora
%  & 
%  % aCONCEPTUAL: 
%  Edad del paciente
%  &
%  % aOPERACIONAL:
%  Años de vida reportados
%  & 
%  Encuesta
%  & 
%  \begin{minipage}{2.2cm} 
%  \textbf{0-90} años.
%  \end{minipage}
%  &
%  Numérica discreta
%  & 
%  Razón
%  &
%  Historia clínica &\\[10ex]
%  \hline


  % etc. ...
\end{tabular}
\hspace*{-1cm}
\end{center}
        \captionof{table}{Operacionalización de variables}
        \label{tab:opera}
\end{table}

\section{MARCO TEÓRICO}\label{marco-teorico}

\subsection{Antecedentes de la
investigación}\label{antecedentes-de-la-investigacion}

\begin{enumerate}
\def\labelenumi{\alph{enumi}.}
\item
  En Africa

  Un hito en la aplicación de microarreglos de proteínas para el estudio
  de la respuesta humoral a escala epidemiológica fue la publicación de
  Crompton et al.
  2010.\textsuperscript{\protect\hyperlink{ref-crompton2010}{9}} Este
  estudio comparó la respuesta de anticuerpos contra el 23\% del
  proteoma de \emph{P. falciparum} antes y después de la temporada de
  malaria en 220 individuos de Mali, en dos grupos poblacionales: 2-10 y
  18-25 años. Dentro del grupo de niños entre 8 y 10 años se
  identificaron 49 proteínas con mayor reactividad serológica en el
  grupo de infectados asintomáticos, en comparación a los sintomáticos.
  Cinco de los principales candidatos a vacuna (CSP, LSA-3, MSP1, MSP2,
  AMA1) no lograron discriminar ambos grupos. Sin embargo, cuatro
  candidatos secundarios (STARP, LSA-1, RESA, antígeno 332), con
  expresión en diferentes estadíos del ciclo biológico, sí lograron tal
  discriminación.

  Un segundo hito de interés representa el trabajo de Helb et al.
  2015.\textsuperscript{\protect\hyperlink{ref-Helb2015exposure}{10}}
  Ellos reportaron una estrategia para identificar combinanciones de
  respuestas contra más de un anticuerpo que maximicen la información de
  la exposición reciente al nivel de individuos. Para ello emplearon
  modelos basados en el aprendizaje automático o \emph{machine learning}
  para el análisis de las respuestas contra 865 antígenos de \emph{P.
  falciparum} en 186 niños (3-6 años) de Uganda en base al registro
  activo y pasivo de sus historias clínicas a los largo de un año. En
  contraste a los marcadores tradicionalmente emplados para evaluar
  exposición a nivel poblacional (CSP, MSP1, MSP2,
  AMA1),\textsuperscript{\protect\hyperlink{ref-elliott2014}{14}} se
  identificaron marcadores más informativos como hyp2, GEXP18, EMP1,
  ETRAMP4, HSP40-II y PF70. La validación de este método permitirá
  seleccionar mejores marcadores para la vigilancia sero-epidemiológica,
  relevante para el guiado y evaluación de los programas de control y
  eliminación de malaria a nivel mundial.
\item
  En Perú

  El primer estudio en publicarse fue de Torres et al.
  2014\textsuperscript{\protect\hyperlink{ref-Torres2014asymptomatic}{20}}
  donde reportaron marcadores de inmunidad clínica -no esterilizante- de
  adquisición natural a la malaria falciparum. Además de ello, el
  resultado con mayor implicancias fue que estos marcadores presentaron
  un enrriquecimiento de polimorfismos no-sinónimos, indicativo de una
  presión de selección positiva por parte del sistema inmune, a pesar de
  provenir de una zona con baja transmisión. Las 51 proteínas con mayor
  reactividad en 14 pacientes infectados y asintomáticos se obtuvieron
  al comparar las respuestas con 24 paciente sintomáticos provenientes
  de la Amazonía peruana (Departamento de Loreto, Provincia de Maynas y
  Requena) contra 824 fragmentos (699 proteínas) de \emph{P.
  falciparum}.

  El segundo y último estudio en ser publicado ha sido el de Chuquiyauri
  et al.
  2015.\textsuperscript{\protect\hyperlink{ref-chuquiyauri2015vivax}{21}}
  Ellos compararon la respuesta contra \emph{P. vivax} de pacientes con
  relapsos y reinfección, sin encontrar diferencia alguna entre ambos
  grupos. Sin embargo, al igual que el anterior estudio, identificaron
  un enriquecimiento de proteínas con polimorfismos no-sinónimos en el
  grupo de antígenos reactivos. Además, dentro del grupo con mayor
  reactividad en toda la muestra, resaltaron a PvMSP-10 como potencial
  candidato a vacuna al presentar la expresión más consistente y validar
  lo reportado por dos estudios previos con enfoque tradicional donde
  emplean ambos sistemas de expresión: eucariota y procariota. El
  estudio empleó un arreglo con 2233 fragmentos (1936 proteínas) en 106
  individuos de la ciudad de Maynas, Loreto - Perú.
\end{enumerate}

\subsection{Bases teóricas}\label{bases-teoricas}

\subsubsection{Malaria Vivax}\label{malaria-vivax}

\begin{enumerate}
\def\labelenumi{\alph{enumi}.}
\item
  Epidemiología

  \begin{enumerate}
  \def\labelenumii{\roman{enumii}.}
  \item
    \textbf{A nivel mundial}

    \emph{P. vivax} y \emph{P. falciparum} son los principales
    responsables de los casos de malaria en humanos. Ambas especies
    exponen aproximadamente a 2.5 mil millones de personas en riesgo de
    infección.\textsuperscript{\protect\hyperlink{ref-howes2016global}{22}}
    Sin embargo, \emph{P. vivax} es el parásito dominante en las
    regiones fuera del África Sub-Sahariana, en su mayoría densamente
    pobladas y empobrecidas. Entre ellas, Etiopía, India, Indonesia y
    Pakistán acumularon el 78\% de casos de \emph{P.vivax} a nivel
    mundial. A su vez, la región de las Américas tuvo la mayor
    proporción de estos, con un
    69\%.\textsuperscript{\protect\hyperlink{ref-WHO2016world}{1}} A
    pesar de ello, hasta el momento la mayoría de la investigación y
    financiamiento está destinado a la prevención, tratamiento y control
    de \emph{P.
    falciparum}.\textsuperscript{\protect\hyperlink{ref-path2011}{23}}
  \item
    \textbf{En el Perú}

    En el 2015, Perú fue el tercer país con más casos reportados en
    Latinoamérica (19\%), detrás de Brasil (24\%) y Venezuela
    (30\%).\textsuperscript{\protect\hyperlink{ref-WHO2016world}{1}} El
    80\% fueron casuados por \emph{P.vivax} (63,153 en total), en
    regiones con endemismo y transmisión
    heterogénea\textsuperscript{\protect\hyperlink{ref-rosas2016peru}{2}}.
    El 95\% pertenecieron al noreste amazónico (con una razón Pv/Pf de
    4/1) y el resto al norte costero y la región minera del suroeste.
    Notablemente, una comunidad en Madre de Dios no ha tenido un solo
    caso de malaria falciparum en una década. Con respecto a factores
    ambientales, en el año 1998 durante el fenómeno El Niño-Oscilación
    Sur (ENSO) se produjo el mayor pico de casos anuales en la historia
    (200,000 casos), con mayor efecto en el norte
    costero.\textsuperscript{\protect\hyperlink{ref-gagnon2002enso}{24}}
    Los casos anuales provenientes de Tumbes, Piura y Lambayeque a nivel
    nacional pasaron de representar menos del 7\% en 1996, a un 48\%
    entre 1998 y 1999.
  \end{enumerate}
\item
  Biología

  \begin{enumerate}
  \def\labelenumii{\roman{enumii}.}
  \item
    \textbf{Ciclo Biológico}

    El ciclo de vida del parasito de la malaria involucra a dos
    hospederos: el humano y el mosquito. En el humano el parásito se
    desarrolla asexualmente al invadir a las células hepáticas y luego a
    los glóbulos rojos (RBC). A través de la picadura del mosquito
    infectado, los esporozoitos ingresan a las células hepáticas
    iniciandose el estadio hepático. Ahí se multiplican hasta formar
    esquizontes, los cuales egresan al torrente sanguíneo en la forma de
    merozoitos. El estadio eritrocítico se inicia al invadir RBC,
    desarrollando en forma consecutiva trofozoitos inmaduros (en forma
    de anillo), maduros y esquizontes, los cuales a su ruptura los
    nuevos merozoitos reinfectan más glóbulos rojos. La rapidez de este
    ciclo determina la razón de multiplicación parasitaria. Por motivos
    aún desconocidos, a partir de los trofozoitos inmaduros se inicia el
    desarrollo de gametocitos diferenciandose sexualmente dentro del
    torrente sanguíneo del humano. La fase sexual en el mosquito
    mosquito se inicia mediante la ingestión accidental de gametocitos
    en la alimentación sanguínea de mosquitos hembra, con la intención
    inicial de proveer nutrientes a sus huevos. En su tracto digestivo,
    se desarrollan los esporozoitos, los cuales migran a las glándulas
    salivares para continuar la transmisión en la siguiente alimentación
    del moquitos hembra.
  \item
    \textbf{Particularidades}

    \emph{P.vivax} posee importantes variantes biológicas que
    caracterizan su epidemiología y dinámica de
    infección\textsuperscript{\protect\hyperlink{ref-howes2016global}{22}}.
    Dos de las más importantes son (i) la presencia de relapsos por la
    activación de hipnozoitos, estado de latencia o dormacia en el
    estadio hepático, condición que genera reservorios de infección que
    prolongan su transmisión; y (ii) el tropismo hacia reticulocitos o
    RBC inmaduros (1-2\% de RBC circulantes), condición que genera bajas
    parasitemias en comparación a \emph{P. falciparum}. Por otro lado,
    el antígeno sanguíneo \emph{Duffy}, predominante en problación con
    ancestría africana, es considerado un factor de resistencia a
    \emph{P.vivax}.
  \end{enumerate}
\end{enumerate}

\subsubsection{Malaria Severa}\label{malaria-severa}

\begin{enumerate}
\def\labelenumi{\alph{enumi}.}
\item
  General

  \begin{enumerate}
  \def\labelenumii{\roman{enumii}.}
  \item
    \textbf{Definición}

    Ante la auscencia de una descripción especie-específica, la Malaria
    Vivax Severa está definida por el criterio para \emph{P. falciparum}
    otorgado por la OMS en el
    2014,\textsuperscript{\protect\hyperlink{ref-WHO2014severe}{25}} el
    cual incluye uno o más de las siguientes características (todas
    registradas en mono-infecciones por \emph{P. vivax}): 1. condición
    neurológica: coma, mareo, pérdida conciencia; 2. condición
    hematológica: anemia/trombocitopenia severa; 3. síntomas sistémicos:
    shock circulatorio; y 4. fallo de órganos vitales: disfución
    respiratoria, estrés respiratorio agudo, daño renal agudo, ruptura
    esplénica, disfunción hepática e ictericia (hiperbilirrubina).
  \item
    \textbf{Riesgo}

    La vulnerabilidad a la Malaria Severa ha sido asociada a la
    intensidad de transmisión y el desarrollo de la inmunidad
    independiente a la
    edad.\textsuperscript{\protect\hyperlink{ref-reyburn2015}{5}} Por
    ejemplo, en zonas de alta transmisión como en el África
    Sub-Sahariana, las poblaciones más vulnerables son: niños menores de
    5 años con un desarrollo incompleto de inmunidad parcial contra la
    malaria,\textsuperscript{\protect\hyperlink{ref-Stanisic2015}{26}}
    mujeres embarazadas en parte a la adhesión placentaria de
    iRBC,\textsuperscript{\protect\hyperlink{ref-rogerson2007preg}{27}}
    y viajeros o migrantes sin inmunidad provenientes de áreas con baja
    o ninguna transmisión de malaria. Por otro lado, en zonas de baja
    transmisión como en Asia y América Latina, al haber una menor
    exposición a la infección, la mayoría de la población llega a la
    adultez sin haber desarrollado una inmunidad protectiva. Como
    consecuencia, tanto en la Amazonía peruana como en el Norte costero
    la población adolescente y adulta joven es el más suceptible a
    desarrollar esta
    patología,\textsuperscript{\protect\hyperlink{ref-quispe2014}{4},\protect\hyperlink{ref-llanoschea2015}{6}}
    comúnmente al iniciar trabajos a campo abierto, e.g.~actividades
    madereras o mineras, en zonas de alto riesgo de contacto con
    mosquitos
    infectados.\textsuperscript{\protect\hyperlink{ref-factores2001}{28}}

    Cabe agregar que la presencia de comorbilidades, coinfecciones y
    malnutrición son reconocidos como factores contribuyentes a la
    Malaria Severa, los cuales deben ser adecuadamente registrados. Sin
    embargo, se ha sugerido que la coinfección de \emph{Plasmodium} con
    Dengue no está asociada con una peor enfermedad, asemejándose a una
    mono-infección con Dengue tanto en frecuencia de síntomas como en el
    nivel de marcadores
    inmunológicos.\textsuperscript{\protect\hyperlink{ref-baldevi2016}{29}}
  \end{enumerate}
\item
  Epidemiología

  \begin{enumerate}
  \def\labelenumii{\roman{enumii}.}
  \item
    \textbf{A nivel mundial}

    Entre el 2005 y 2015, entre el 1-3\% de casos no-complicados fueron
    asumidos como malaria severa, causante de la muerte de 429,000
    personas\textsuperscript{\protect\hyperlink{ref-WHO2016world}{1}} y
    en un 90\% niños menores de 5 años en
    África.\textsuperscript{\protect\hyperlink{ref-wassmer2015}{30}}
    Entre las estrategias preventivas desarrolladas hasta el momento
    (control vectorial, quimioprevención, vacunas e indicadores), la
    única vacuna en completar un ensayo de fase 3, RTS,S/AS01, solo
    redujo en un 39\% la incidencia de enfermedad y en 31.5\% la de
    malaria severa en niños entre 5-17
    meses.\textsuperscript{\protect\hyperlink{ref-rts2015}{31}}

    Con respecto a la malaria vivax severa, un reciente meta-análisis
    con 46,411 casos registrados desde el año 1900 identificó patrones
    geográficos en la prevalencia de manifestaciones
    clínicas.\textsuperscript{\protect\hyperlink{ref-rahimi2014meta}{32}}
    Concluyeron que en las zonas de alta transmisión (e.g.~Papua Nueva
    Guinea) la anemia severa fue el síntoma con mayor prevalencia en
    niños jóvenes. Por otro lado, en zonas con baja transmisión (e.g.~la
    región de las Américas) los adultos eran los más susceptibles a la
    severidad con una mayor variedad de disfunciones orgánicas. Por
    ejemplo, en Brasil Alexandre et al.
    2010\textsuperscript{\protect\hyperlink{ref-alexandre2010}{33}} en
    un estudio retrospectivo identifico 17/11,251 casos de malaria vivax
    severa, entre 28-80 años de edad, con hiperbilirrubina (59\%),
    anemia severa (29\%), daño renal agudo (12\%), daño pulmonar (12\%)
    y shock circulatorio (6\%).
  \item
    \textbf{En el Perú}

    En el norte costero, Quispe et al.
    2014\textsuperscript{\protect\hyperlink{ref-quispe2014}{4}} mediante
    un estudio retrospectivo (2008-2009) identificó 81/6502 casos de
    malaria vivax severa con anemia severa (57\%), shock circulatorio
    (25\%), hiperbilirrubina (25\%), daño pulmonar (21\%), daño renal
    agudo (14\%) y malaria cerebral (11\%). Comparados con los pacientes
    no complicados, los severos fueron mayores (38 vs 26 años,
    P\textless{}0.001).

    En el noreste amazónico, un reciente estudio prospectivo dirigido
    por El Centro de Enfermedades Tropicales de la Marina de los Estados
    Unidos NAMRU-6
    (2012-2013)\textsuperscript{\protect\hyperlink{ref-smith2013}{34}}
    identificó 67/164 casos de malaria vivax severa con síndrome
    respiratorio agudo (94\%), hiperbilirubina (80\%), shock (46\%),
    sangrado (12\%) e ictericia conjuntiva (12\%). No se encontraron
    diferencias con respecto a la edad entre no-severos y severos (33.5
    vs 30, P=0.956). Sin embargo, sí se identificó una mayor proporción
    de pacientes con episodios previos en el grupo de no-severos (67 vs
    49\%, P=0.007).
  \end{enumerate}
\item
  Biología

  \begin{enumerate}
  \def\labelenumii{\roman{enumii}.}
  \item
    \textbf{Comparación}

    Históricamente, la malaria vivax ha sido considerada como
    ``benigna'' en comparación a \emph{P.falciparum} debido a su: (i)
    baja invasión parasitaria, sezgada a reticulocitos y rutas alternas
    de menor efectividad; y (ii) pobre citoadhesión de sus glóbulos
    rojos infectados (iRBC), dada por la auscencia de protruciones
    abastonadas o \emph{knob protrusions}, auscencia de genes homólogos
    a \emph{var} o genes \emph{vir}, y a la baja proporción de
    esquizontes tardíos que sugieren bajo sequestramiento tisular,
    corroborado en
    autopsias.\textsuperscript{\protect\hyperlink{ref-wassmer2015}{30}}

    Sin embargo, recientemente se ha demostrado que en los casos severos
    por \emph{P. vivax} la parasitemia periférica subestima la biomasa
    parasitaria
    total\textsuperscript{\protect\hyperlink{ref-barber2015}{35}}.
    Además, la parasitemia oculta fue la mayor contribuyente de
    citoquinas pro-inflamatorias, semejante a lo observado con \emph{P.
    falciparum}. Los autores también sugirieron una posible acumulación
    parasitaria en partes de órganos sin endotelio.
  \item
    \textbf{Patogénesis}

    Aún no está completamente entendida. Sin embargo, se han propuesto
    mecanismos en base a lo observado en los casos de malaria cerebral
    por \emph{P. falciparum}, donde los mecanismos de invasión y
    adhesión parasitaria son relevantes. La respuesta celular contra la
    parasitemia, dependiente de su tasa de multiplicación, sumada a la
    citoadherencia de iRBC en estadío esquizonte a endotelio o RBC no
    infectados (rosetamiento), desencadena una obstrucción microvascular
    que activa a sus células endoteliales. Estas liberan una mayor
    cantidad de citoquinas pro-inflamatorias, provocando la pérdida de
    perfusión y una consecuente disfunción microvasular que progresa por
    retroalimentación positiva.
  \end{enumerate}
\end{enumerate}

\subsubsection{Anticuerpos}\label{anticuerpos}

\begin{enumerate}
\def\labelenumi{\alph{enumi}.}
\item
  Relevancia

  Los anticuerpos han demostrado ser los marcadores ideales para el
  monitoreo rápido y preciso de la efectividad de intervenciones contra
  la malaria a nivel mundial. A partir de una sola muestra de sangre,
  con métodos análiticos aplicables a una escala epidemiológica, es
  posible evaluar tanto la exposición acumulada como los niveles de
  inmunidad contra patógenos específicos. Incluso con el potencial de
  identificar \emph{hotspots} y cambios en la transmisión, evaluar el
  impacto de intervenciones, confirmar los estados de control o
  eliminación, y monitorear
  reemergencias.\textsuperscript{\protect\hyperlink{ref-elliott2014}{14}}

  Además, la identificación de antígenos homólogos a \emph{P.
  falciparum} e inductores de inmunidad por adquisión natural contra
  \emph{P. vivax} en poblaciones endémicas ha sido la fuente de mayor
  importancia para la propuesta de candidatos a vacuna. A diferencia de
  \emph{P. falciparum}, \emph{P. vivax} posee dificulatdes técnicas para
  el desarrollo de cultivos \emph{in vitro} y modelos animales de
  infección. Esto se ve reflejado en los dos candidatos a vacuna de
  \emph{P. vivax} (PvDBP en ensayos clínicos fase I y PvCSP en estudios
  pre-clínicos) contra los 23 de \emph{P.
  falciparum}.\textsuperscript{\protect\hyperlink{ref-rainbow2016}{36}}
\item
  Exposición

  Estudios longitudinales han sugerido que la intensidad de la respuesta
  de anticuerpos en poblaciones con poca inmunidad, en un inicio pueden
  actuar como marcadores de exposición. Luego de una exposición
  constante, esta respuesta incrementa hasta superar un umbral de
  protección, donde los anticuerpos pueden actuar como marcadores de
  inmunidad.\textsuperscript{\protect\hyperlink{ref-Stanisic2015}{26}}

  En el caso de \emph{P. vivax}, estudios seroepidemiológicos han
  logrado asociar marcadores con exposición (PvCSP,
  PvMSP-1\textsubscript{19}, PvMSP-9\textsubscript{RIRII} y PvAMA-1) e
  inmunidad (PvMSP-1\textsubscript{19}, PvMSP-1\textsubscript{NT},
  PvMSP-3\(\alpha\) y
  PvMSP-9\textsubscript{NT}).\textsuperscript{\protect\hyperlink{ref-cutts2014meta}{37}}
  Sin embargo, su principal limitante está en la carecia de evaluaciones
  en su cinética de adquisición, i.e.~su desarrollo y mantenimiento, en
  cohortes longitudinales. 
\item
  Inmunidad

  En áreas de alta transmisión, los niños adquieren resistencia a la
  manifestación severa de la malaria a la edad de cinco años,
  aproximadamente. Sin embargo, continuan siendo susceptibles a
  episodios no-complicados hasta la adolescencia, donde adquieren un
  estado resistente a la malaria sintomática. A pesar de ello, no se ha
  demostrado que esta adquisición natural por la exposición acumulada a
  malaria otorgue una protección esterilizante o de resistencia a la
  infección.\textsuperscript{\protect\hyperlink{ref-crompton2014rev}{38}}
  Los detalles de los procesos involucrados en cada una de estas fases
  no están completamente entendidos, pero se tiene detalles de algunos
  componentes:

  \begin{enumerate}
  \def\labelenumii{\roman{enumii}.}
  \item
    \textbf{contra la severidad}

    Se conoce que la respuesta contra esta manifestación está guiada por
    una excesiva respuesta inflamatoria que incluye la producción de las
    citoquinas proinflamatorias TNF-\(\alpha\), INF-\(\alpha\),
    IL-1\(\beta\) e
    IL-6\textsuperscript{\protect\hyperlink{ref-baird2013}{39}} junto a
    una reducción en la producción de la citoquina antiinflamatoria
    IL-10, un regulador central en la respuesta contra la
    malaria.\textsuperscript{\protect\hyperlink{ref-jagannathan2014}{40}}

    Es por ello que esta primera fase implicaría la adquisición de
    inmunidad a componentes parasitarios que gatillan una respuesta
    inflamatoria innata. Una hipótesis está en anticuerpos que
    interfieran la unión de glucoporteínas pro-inflamatorias con
    receptores
    Toll.\textsuperscript{\protect\hyperlink{ref-schofield2006toll}{41},\protect\hyperlink{ref-coban2005toll}{42}}

    Por otro lado, la adquisición de anticuerpos que bloqueen la
    invasión parasitaria o adhesión de iRBC también involucraría una
    protección a la
    severidad.\textsuperscript{\protect\hyperlink{ref-wassmer2015}{30}}
    Antígenos candidatos son genes \emph{vir}, homologos a los genes
    \emph{var} de \emph{P. falciparum} e involucrados en la secreción de
    proteínas adhesivas a RBC y
    endotelio,\textsuperscript{\protect\hyperlink{ref-portillo2001vir}{43}}
    y genes PvRBP, alternativo a PvDBP y tradicionalmente identificado
    como exclusivo para la invasión a
    reticulocitos.\textsuperscript{\protect\hyperlink{ref-galinski1992rbp}{44}}
  \item
    \textbf{contra la enfermedad}

    En 1961, la transferencia pasiva de anticuerpos donados por adultos
    con inmunidad adquirida de forma natural tuvo una alta efectividad
    en la disminución de la parasitemia periférica y resolución de
    síntomas clínicos en niños con
    malaria.\textsuperscript{\protect\hyperlink{ref-cohen1961}{45},\protect\hyperlink{ref-sabchareon1991}{46}}
    Este experimento sugirió que los anticuerpos contra antígenos del
    estadio eritrocítico son marcadores importantes de inmunidad.

    En el caso de \emph{P. vivax}, la adquisición de esta inmunidad
    ocurre con una mayor rapidez. Se ha propuesto que este proceso es
    facilitado por las particularidades biológicas de la
    especie.\textsuperscript{\protect\hyperlink{ref-mueller2013}{47}}

    Los antígenos propuestos hasta el momento para \emph{P. vivax} son
    proteínas del micronema de los merozoitos como DBP y AMA1, y
    proteínas de superficie como MSP1, MSP1-P (región C-terminal), MSP3
    (PvMSP-3\(\alpha\), PvMSP-3\(\beta\)) y
    MSP9.\textsuperscript{\protect\hyperlink{ref-lopez2017}{48}}
  \item
    \textbf{contra la parasitemia}

    La adquisición de una inmunidad esterilizante solo ha sido
    comprobada mediante la inmunización con esporozoitos atenuados por
    radiación. Tanto en \emph{P. falciparum} como en \emph{P. vivax} se
    ha confirmado esta respuesta. Un reciente ensayo clínico de fase II
    en Colombia con \emph{P. vivax} encontró que luego de siete
    inmunizaciones a lo largo de 56 semanas y una reexposición con
    esporozoitos infecciosos, 5/12 voluntarios obtuvieron dicha
    protección y las respuestas de anticuerpos Ig1 anti-PvCSP estuvieron
    asociadas a
    ella.\textsuperscript{\protect\hyperlink{ref-arevalo2016spz}{49}} 
  \end{enumerate}
\item
  Enfoque a larga escala

  El parásito de la malaria, al tener un ciclo de vida complejo con
  múltiples estadios de desarrollo en el humano, posee también un
  transcriptoma, proteoma y, por lo tanto, un inmunoma
  estadio-específico que deriva de un total de \textasciitilde{}5300
  proteínas putativas. Por este motivo, tanto los esfuferzos de vacunas
  por subunidades (disponibles para hepatitis B) como por organismos
  completos (atenuados o inactivados) han fracazado. Ensayos clínicos
  han demostrado que los actuales candidatos poseen un pobre
  inmunogenicidad y protección contra la malaria (e.g.~RST,S).

  \begin{enumerate}
  \def\labelenumii{\roman{enumii}.}
  \item
    \textbf{Inmunómica}

    La era genómica permitió un cambio de perspectiva en la
    investigación inmunológica: se pasó de una procupación en
    plataformas a una en antígenos de vacunación. Es por ello que,
    actuamente, el objetivo de mayor importancia está en identificar el
    subgrupo de antígenos capaz de inducir una respuesta inmune
    protectiva.

    El primer enfoque en respuesta a esta problemática fue llamado
    \emph{Vacunología Reversa}, el cual hace uso de la información
    genómica para la predicción bioinformática de antígenos candidatos a
    vacuna a partir de características protéticas asociadas a
    inmunogenicidad y efectividad en vacunación. La principal limitación
    de este enfoque radica en el que el genoma es estático en el tiempo,
    no aplicable en organismos con ciclos de vida complejos.

    Por ello, un segundo enfoque que se construyó sobre aquél fue
    llamado \emph{Inmunómica}, el cual hace uso de la información
    transcriptómica y proteómica de patógenos dinámicos junto a métodos
    computacionales predictivos e información del hospedero para la
    selección inmunógenos en forma empírica. Su principal ventaja está
    en la capacidad de priorizar la selección de antígenos en base a
    criterios clínicamente relevantes, como el de individuos con
    inmunidad por adquicisión natural.

    Este enfoque ha permitido obtener datos empíricos sobre diferencias
    en la amplitud, intensidad, cinética y longevidad de respuestas
    inmunes inducidas por patógenos. Además de respuestas a preguntas
    clave de la respuesta inmune como el porcentaje del proteóma
    reconocido por el sistema inmune, si los anticuerpos pueden ser
    predictores de estados de la enfermedad, o si la inmunogenicidad
    puede ser predicha solamente por aminoácidos. Entre las plataformas
    empleadas en este enfoque están los microarreglos de proteínas y
    péptidos, los cuales complementados con el mapeo de epítopes, han
    alimentado a los actuales algoritmos de predicción computacional.
  \end{enumerate}
\end{enumerate}

\subsubsection{Microarreglos}\label{microarreglos}

\begin{enumerate}
\def\labelenumi{\alph{enumi}.}
\item
  De proteínas

  Dado el tamaño y complejidad de los genomas de \emph{Plasmodium}, la
  ejecución de un \emph{screening} de proteoma completo es desafiante.
  Por ejemplo, \emph{P. falciparum} posee más de 3kb de longitud,
  \textasciitilde{}5300 genes, genes multi-exónicos complejos y
  segmentos largos de secuencias repetitivas por el bajo contenido de
  GC. Por esta razón, el diseño de microarreglos de proteínas requiere
  de procesos de selección en base a características protéicas y
  conocimiento empírico. De esta manera se ha logrado desarrollar
  plataformas costo-efectivas y rápidas de emplear, capaces de facilitar
  el descubrimiento de antígenos para vacunas o serovigilancia con el
  menor trabajo en laboratorio e independiente de algoritmos
  predictivos.

  \begin{enumerate}
  \def\labelenumii{\roman{enumii}.}
  \item
    \textbf{Primera generación}

    Finney et al.
    2014\textsuperscript{\protect\hyperlink{ref-Finney2014}{50}}
    diseñaron los microarreglos de \emph{P. vivax} y \emph{P.
    falciparum} en base a las proteínas predichas para el estadío
    eritrocítico con capacidad de ser expuestas a superficie, presencia
    de evidencia en microarreglos, proteómica, EST, secretoma predicho,
    presencia de péptidos señal, dominios transmembrana, proteínas
    putativas citplasmáticas carentes de péptidos señal y dominios
    transmembrana en base a filtros por punto isoeléctro, exclusión de
    proteínas con variabilidad antigénica (e.g.~PfEMP1, RIFIN, surfin,
    stevors; Pv \emph{var} y pseudogenes), inclusión de genes
    multi-exonicos y genes largos didividos en segmentos sobrelapantes,
    limitando amplicones a 300-3000nt.
  \item
    \textbf{Segunda generación}

    En el 2015, el Centro Internacional de Excelencia para la
    Investigación de la Malaria ICEMR realizó una subselección empírica
    del microarreglo anteriormente
    detallado.\textsuperscript{\protect\hyperlink{ref-King2015FOC}{51}}
    Para ello sondearon los microarreglos de \emph{P. falciparum} con 20
    muestras de Papua Nueva Guinea, 20 de Kenya, 20 de Mali y 10
    controles norteamericanos, y los de \emph{P. vivax} con 15 de Papua
    Nueva Guinea, 15 de China, 22 de Perú, 10 de Tailandia y 10
    controles. Luego se seleccionó a los antígenos seroreactivos por
    país, cumpliendo la condición de ser mayores a dos veces la
    desviasión estandar de la media de la reactividad serologica en los
    controles. Finalmente, se seleccionó el top 500 para ambas especies
    empleando un filtrado jerárquico, dándole prioridad a los antígenos
    con seroreactividad en todos los paises y en las posiciones
    restantes los antígenos en orden descendiente a la reactividad en
    todos los paises. Este diseño ha sido depositado en la base de datos
    GEO con el código GPL 18316.
  \end{enumerate}
\item
  Análisis de datos

  \begin{enumerate}
  \def\labelenumii{\roman{enumii}.}
  \item
    Métodos

    \begin{enumerate}
    \def\labelenumiii{\arabic{enumiii}.}
    \item
      Preprocesamiento

      Transformación Normalización Filtrado
    \item
      Test estadístico

      Bayes t-test y Comparación múltiple
    \item
      Parámetros

      y se emplearán a tres parámetros: reactividad serológica, amplitud
      e intensidad de respuesta

      los perfiles de anticuerpos a larga escala han demostrado la
      amplitud y heterogeneidad de las respuestas
      humorales\textsuperscript{\protect\hyperlink{ref-crompton2010}{9}}.

      Los dos principales parámetros de comparación entre grupos, con
      respecto a la respuesta general de anticuerpos, son la proporción
      de antígenos reactivos y el promedio de sus intensidades por
      individuo o amplitud e intensidad de respuesta,
      respectivamente.\textsuperscript{\protect\hyperlink{ref-crompton2010}{9},\protect\hyperlink{ref-Helb2015exposure}{10},\protect\hyperlink{ref-King2015FOC}{51}}
    \end{enumerate}
  \item
    Softwares

    VDVFSVFSVFS
  \end{enumerate}
\end{enumerate}

\subsection{Definiciones conceptuales}\label{definiciones-conceptuales}

\subsubsection{Acrónimos}\label{acronimos}

\textbf{IVTT:} (Proteína) transcrita/traducida \emph{in vitro}.

\textbf{MFI:} Unidad de lectura cruda expresada como intensidad
fluorescente promedio de todos los pixeles de cada \emph{spot},
normalizada localmente mediante la sustracción de la intensidad de fondo
presente a su alrededor.

\textbf{RTS:} Sistema rápido de traducción de proteínas recombinantes
libre de células.

\subsubsection{Términos}\label{terminos}

\begin{enumerate}
\def\labelenumi{\alph{enumi}.}
\item
  Generales

  \textbf{Antígeno.-} Molécula que se une con productos de la respuesta
  inmune, como anticuerpos o receptores de lifocitos T o
  B.\textsuperscript{\protect\hyperlink{ref-abbas2012}{52}}

  \textbf{Inmunógeno.-} Un antígeno que induce una respuesta
  inmunitaria.\textsuperscript{\protect\hyperlink{ref-abbas2012}{52}}

  \textbf{Inmunoma.-} Conjunto de todos los inmunógenos que interactúan
  con el sistema inmune de determinado
  hospedero\textsuperscript{\protect\hyperlink{ref-immunomics2016}{53},\protect\hyperlink{ref-sette2005}{54}}

  \textbf{Inmunómica.-} Estudio del
  inmunoma.\textsuperscript{\protect\hyperlink{ref-immunomics2016}{53}}

  \textbf{Anticuerpo.-} Tipo de molécula glucoprotéica, también llamada
  inmunoglobulina (Ig), producida por los linfocitos B, que se une a
  antígenos con un grado alto de especificidad y
  afinidad.\textsuperscript{\protect\hyperlink{ref-abbas2012}{52}}

  \textbf{Anticuerpo secundario.-} Anticuerpo de unión específica a la
  fracción constante de los anticuerpos de un hospedero, comúnmente
  conjugado con biotina.

  \textbf{Fluoróforo.-} Componente de una molécula que brinda la
  cualidad de fluorescencia, comúnmente conjugada con estreptavidina
  para la detección de moléculas biotiniladas.
\item
  Específicos

  \textbf{Antígeno-IVTT:} Antígeno objetivo o \emph{spot} con proteína
  IVTT a partir de un plásmido con DNA insertado del polipéptido,
  segmento o exón de interés.

  \textbf{Control-IVTT:} Control negativo o \emph{spot} con mix de
  expresión RTS y plásmido sin DNA insertado, representante de la
  intensidad de fondo específica del paciente.

  \textbf{Proteína purificada:} Control de comparación o \emph{spot} con
  proteína de antigenicidad conocida expresada dentro de célula.

  \textbf{Transformación:} Logaritmo en base dos de los MFI de
  antígenos-IVTT y control-IVTT.

  \textbf{Normalización:} Sustracción de la mediana de los control-IVTT
  a cada antígeno-IVTT por individuo.

  \textbf{Intensidad de antígeno:} Lectura normalizada de cada
  antígeno-IVTT entre individuos. También llamada ``Reactividad de
  anticuerpos''.

  \textbf{Antígeno reactivo:} Antígenos-IVTT con una lectura
  transformada mayor o igual a dos veces la mediana de los control-IVTT
  por individuo.

  \textbf{Frecuencia del antígeno:} Porcentaje de individuos con
  antígeno reactivo por antígeno-IVTT.

  \textbf{Filtrado:} Remoción de todo antígeno que posea una frecuencia
  de reactividad menor al 10\%.

  \textbf{Intensidad de respuesta:} Promedio de intensidades de
  antígenos reactivos por individuo

  \textbf{Amplitud de respuesta:} Porcentaje de antígenos reactivos por
  individuo.
\end{enumerate}

\section{METODOLOGÍA}\label{metodologia}

\subsection{Diseño}\label{diseno}

El diseño del estudio es de tipo Caso-Control. Su aplicación consistirá
en la selección de la población objetivo por presencia (casos) o
auscencia (controles) del evento en estudio. Además se fijará el número
de eventos a estudiar, así como el número de sujetos sin evento que se
incluirán en la población de comparación.

\subsubsection{Tipo de investigación}\label{tipo-de-investigacion}

\textbf{Por la finalidad: Analítico.} A diferencia de un descriptivo,
realizaremos compraciones entre grupos y evaluaremos una posible
relación causal entre el factor y el efecto.

\textbf{Por el Control de la asignación: Observacional.} A diferencia de
un experimental, no controlaremos la asignación de los factores
(severidad y episodios previos). Es decir, se procederemos a observar el
fenómeno, ejecutar la medición y analizar los resultados.

\textbf{Por el seguimiento: Transversal.} A diferencia de uno
longitudinal, no ejecutaremos un seguimiento. Las variables se medirán
una sola vez, en un mismo instante.

\textbf{Por la relación cronológica: Prospectivo.} A diferencia de un
retrospectivo, recolectaremos los datos después de planificado el
estudio.

\textbf{Por la unidad de análisis: Basado en individuo.} A diferencia de
uno ecológico, la unidad de observación son individuos, llamados aquí
pacientes.

\subsection{Población y muestra}\label{poblacion-y-muestra}

\subsubsection{Población}\label{poblacion}

El universo teórico está representado por los pacientes con malaria
vivax de la cuenca amazónica del Perú.

El marco muestral está delimitado por pacientes diagnosticados con
malaria vivax de la ciudad de Iquitos, Loreto - Perú, entre enero del
2012 y junio del 2013 a lo largo de un estudio prospectivo ejecutado en
dos hospitales de referencia: Hospital de Apoyo y Hospital Regional.

\subsubsection{Criterios de inclusión}\label{criterios-de-inclusion}

En el estudio prospectivo mencionado, 164 pacientes con malaria vivax
fueron enrolados mediante vigilancia pasiva. Bajo los criterios clínicos
y de laboratorio de la Organización Mundial de la Salud (OMS o WHO por
sus siglas en inglés) para la Malaria
Severa\textsuperscript{\protect\hyperlink{ref-WHO2014severe}{25}}, 67
fueron clasificados como severos y 97 como no-severos.

Los criterios clínicos empleados fueron: Shock circulatorio (presión
sanguínea sistólica \textless{} 80 mmHg), Deterioro del nivel de
conciencia (puntaje Glasgow \(\le\) 9/14), Daño del sistema nerviosos
central (convulsión, postración, coma, confusión), Daño pulmonar
(disnea, taquipnea, infiltracion, edema), Síndrome de dificultad
respiratoria aguda o SDRA.

Los criterios de laboratorio empleados fueron: Hipoglicemia (glucosa
\textless{} 40 mg/dL), Anemia severa (hemoglobina \textless{} 7mg/dL),
Daño renal (creatinina \textgreater{} 3mg/dl), Hiperbilirrubina
(bilirrubina sérica \textgreater{} 2.5 mg/dL),

\subsubsection{Criterios de exclusión}\label{criterios-de-exclusion}

Dos criterios de exclusión fueron empleados: Presencia de
mono-infecciones de \emph{P. vivax} determinadas por PCR y auscencia de
co-infecciones como leptospirosis, dengue u otra arbovirosis por las
técnicas de aislamiento viral e inmunofluorescencia.

\subsubsection{Selección de
participantes}\label{seleccion-de-participantes}

A partir de esta clasificación, se ejecutará una selección aleatoria
simple de 30 pacientes con más de 2 criterios de malaria severa para el
grupo de casos y 30 no-severos para el grupo control.

\textbf{Tipo de muestra: Probabilística.} Cada elemento de la muestra
será seleccionado al azar, con probabilidad de selección conocida.

\subsection{Recolección de los datos e
Instrumento}\label{recoleccion-de-los-datos-e-instrumento}

\subsubsection{Técnica para la para recolección de
datos}\label{tecnica-para-la-para-recoleccion-de-datos}

Previo al tratamiento antimalárico del paciente, se extrajeron las
muestras de sangre tanto para el diagnóstico de malaria vivax por la
técnica de frotís como para las pruebas bioquímicas. Al momento del
diagnóstico positivo, bajo consentimiento informado y de forma
voluntaria, el plasma sanguíneo fue colectado y conservado a -80°C hasta
su uso.

\subsubsection{Instrumento de medición}\label{instrumento-de-medicion}

El microarreglo de proteínas es una herramienta que permite medir a
larga escala los anticuerpos reactivos a determinados antígenos de un
patógeno presentes en plasma. El presente estudio hará uso de
microarreglos diseñados con 1014 fragmentos protéicos recombinantes (498
de \emph{P. falciparum} y 516 de \emph{P. vivax}, expresado como
Pf498/Pv516) representando 873 proteínas predichas (427 de \emph{P.
falciparum} y 446 de \emph{P. vivax}, \textasciitilde{}8\% del total
predicho para el genoma de \emph{P. vivax} Sal1) seleccionadas luego de
una extensiva evaluación serológica con uno de mayor escala
(Pf2208/Pv2233).\textsuperscript{\protect\hyperlink{ref-Finney2014}{50}}
Las proteínas son transcritas/traducidas \emph{in vitro} empleando un
sistema de expresión de \emph{E. coli} libre de células e impresas en
bloques de nitrocelulosa sobre una lámina portaobjetos con 8 bloques
paralelos, cada uno con 4 arreglos paralelos compuestos por cuadrículas
de 17x17 \emph{spots}. Este diseño ha sido validado mediante el sondeo
con plasma colectado de pacientes con malaria y controles sanos
alrededor del
mundo.\textsuperscript{\protect\hyperlink{ref-King2015FOC}{51}}

\paragraph{Aplicación}\label{aplicacion}

La aplicación del instrumento consiste en tres pasos: sondeo, escaneo y
análisis, tal como ha sido publicado
previamente.\textsuperscript{\protect\hyperlink{ref-Driguez2015}{55}}
Primero, previo al sondeo los microarreglos se hidratan con buffer de
bloqueo dentro de cámaras de incubación. Paralelamente, el plasma de los
pacientes se diluye con buffer de bloqueo en 1:100 y se pre-absorbe con
lisado de \emph{E. coli} en 10\%(w/v), con el objetivo de reducir ruido
de fondo en las mediciones tanto por uniones inespecíficas con el
sustrato como por antígenos bacterianos del sistema de expresión RTS.
Segundo, aspirado el buffer de bloqueo del microarreglo, se procede con
el sondeo al agregar el plasma pre-absorbido e incubar \emph{overnight}
en cámara húmeda a 4°C con leve agitación. Tercero, luego de lavados y
aspirado repetitivos antes y después de cada paso, se agrega la solución
con anticuerpos secundarios biotinilados y posteriormente la solución
con fluoróforos conjugados con estreptavidina. Cuarto, luego de
centrifugar las láminas, se procede al escaneo de las fluorescencias con
lectores de microarreglos de laser confocal (e.g., Genepix 4300A). Las
medidas crudas se obtienen luego de una normalización local mediante la
sustracción de la intensidad de fondo presente alrededor de cada
\emph{spot}, ejecutada en el software del lector (Genepix Pro 7).
Finalmente, se procede al análisis de los datos el cual será detallado
en \protect\hyperlink{procanal}{la sección 4.4}.

\hypertarget{validez}{\paragraph{Validez y
confiabilidad}\label{validez}}

Ambos parámetros será evaluados seguiendo la metodología empleada en
estudios
previos\textsuperscript{\protect\hyperlink{ref-crompton2010}{9}} y
estará incluido en la \protect\hyperlink{procanal}{sección 4.4}.

Primero, la validez o exactitud del experimento será evaluada mediante
la correlación lineal entre las lecturas de los antígenos-IVTT y sus
correspondientes proteínas purificadas por muestra, compuestas por
proteínas de inmunogenicidad conocida y expresadas por un sistema dentro
de célula.

Segundo, la confiabilidad o reproducibilidad será evaluada mediante la
correlación lineal entre controles positivos agregados a las lecturas
del primer y segundo día de sondeo de las muestras.

\subsubsection{Codificación y creación del archivo de
datos}\label{codificacion-y-creacion-del-archivo-de-datos}

Cada paciente del estudio estará identificado con un código de
estructura \texttt{LIM\#\#\#\#}, e.g.: \texttt{LIM1063}. Los antígenos
protéicos estarán identificados con el código asignado a sus genes en la
base de datos PlasmoDB, e.g.: \texttt{PF3D7\_0202500} o
\texttt{PVX\_091315}, ya sea un gen de \emph{P. falciparum} o \emph{P.
vivax}, respectivamente. En caso los genes poseean multiples exones, se
amplificarán por separado y se extenderá el código de cada uno con el
número del exón y el total de exones, e.g.: \texttt{\_1o2} exón 1 de un
gen con 2 exones. En caso los genes poseean una longitud mayor a 3000
nucleótidos, se dividirán en segmentos sobrelapantes entre 300 y 3000 nt
y se extenderá el código de cada uno con su respectivo número, e.g.:
\texttt{\_S1} para el primer segmento de un gen.

Los datos serán organizados en dos matrices: (i) archivo
\texttt{samples.csv} con los códigos de los pacientes y sus covariables
epidemiológicas y (ii) archivo \texttt{RawData.csv} con los códigos de
las proteínas y sus lecturas crudas en MFI por código de paciente.

\hypertarget{procanal}{\subsection{Procesamiento y Análisis de
datos}\label{procanal}}

Todo el análisis estadístico se realizará en el software de computación
estadística R\textsuperscript{\protect\hyperlink{ref-R}{18}}, el cual se
complementará con funciones provenientes de distintos paquetes.

\subsubsection{Procesamiento}\label{procesamiento}

Preliminarmente, se describirá la distribución y proporción de las
covariables epidemiológicas, clínicas y de laboratorio de los pacientes
de la muestra. Luego, con las lecturas crudas en MFI se evaluará la
validez y reproducibilidad del ensayo mediante un test de asociación
entre variables contínuas, usando correlación de Pearson (\(r\)) o
Spearman (\(\rho\)), dependiendo de la distribución. Posteriormente, se
procederá con su transformación a escala logarítmica, normalización
entre muestras y filtrado en base al punto de corte establecido. Por
último, se asociarán ambas matrices de datos en un
\texttt{ExpressionSet}, a través de los códigos de pacientes, empleando
el paquete
\texttt{Biobase}.\textsuperscript{\protect\hyperlink{ref-Biobase}{56}}

\subsubsection{Confirmación de
hipótesis}\label{confirmacion-de-hipotesis}

Se pondrán a prueba las hipótesis con el siguiente protocolo. Primero,
se contrastará la amplitud e intensidad de respuesta con un test de
diferencias entre variables contínuas y no pareadas de dos grupos,
usando t-Student o Mann-Whitney, dependiendo de la distribución.
Segundo, se realizará un test de reactividad diferenciada de anticuerpos
entre dos grupos, usando el test t-moderado empírico de Bayes o
eBayes\textsuperscript{\protect\hyperlink{ref-smyth2004ebayes}{57}} con
corrección por comparación múltiple de la razón de falsos
descubrimientos por el método de Benjamini-Hochberg, disponible en el
paquete
\texttt{limma}.\textsuperscript{\protect\hyperlink{ref-limma}{58}}
Tercero, se realizará un agrupamiento jerárquico o \emph{hierarchical
clustering} de los antígenos identificados en base a su distancia
euclidiana, disponible en el paquete
\texttt{NMF}.\textsuperscript{\protect\hyperlink{ref-Gaujoux2010NMF}{59}}
Finalmente, se mostrarán las siguiente características protéicas de los
antígenos identificados: presencia de dominios transmembrana, péptido
señal, número de ortólogos en \emph{Plasmodium}, ontología génica y
razón de mutaciones no-sinónimas sobre sinónimas, disponibles en la base
de datos PlasmoDB.\textsuperscript{\protect\hyperlink{ref-plasmodb}{60}}

\subsubsection{Visualización de
resultados}\label{visualizacion-de-resultados}

Los resultados se visualizarán mediante cinco tipos de gráficos.
Diagrama de dispersión para mostrar la correlación entre dos variables
contínuas, Diagramas de cajas para variables contínuas, Diagrama de
barras para frecuencias, Diagramas tipo volcán o \emph{Volcano plots}
para mostrar la reactividad diferenciada de antígenos entre grupos
contra el valor P no ajustado resaltando los que posean un P ajustado
significativo, y Mapas de calor o \emph{Heatmaps} para el perfil
serológico en escala de dos colores segmentado por racimos o
\emph{clusters}.

\section{ASPECTOS ADMINISTRATIVOS}\label{aspectos-administrativos}

\subsection{Cronograma de actividades}\label{cronograma-de-actividades}

\begin{longtable}[]{@{}lllllll@{}}
\toprule
\textbf{ACTIVIDAD PROGRAMADA} & & & & & &\tabularnewline
\midrule
\endhead
& \textbf{Jul} & \textbf{Ago} & \textbf{Sep} & \textbf{Oct} &
\textbf{Nov} & \textbf{Dic}\tabularnewline
Aprovación del proyecto de tesis & X & & & & &\tabularnewline
Procesamiento de datos & & X & & & &\tabularnewline
Interpretación de resultados & & & X & & &\tabularnewline
Redacción final & & & & X & &\tabularnewline
Correcciones & & & & & X &\tabularnewline
\textbf{Sustentación} & & & & & & X\tabularnewline
\bottomrule
\end{longtable}

\subsection{Presupuesto y
financiamiento}\label{presupuesto-y-financiamiento}

El proyecto será realizado en El Centro de Enfermedades Tropicales de la
Marina de los Estados Unidos NAMRU-6 y
\textcolor{red}{financiado por ****}.

\begin{longtable}[]{@{}lcc@{}}
\toprule
\begin{minipage}[b]{0.46\columnwidth}\raggedright\strut
\textbf{DESCRIPCIÓN}\strut
\end{minipage} & \begin{minipage}[b]{0.22\columnwidth}\centering\strut
\textbf{MONTO (S/)}\strut
\end{minipage} & \begin{minipage}[b]{0.22\columnwidth}\centering\strut
\textbf{PORCENTAJE (\%)}\strut
\end{minipage}\tabularnewline
\midrule
\endhead
\begin{minipage}[t]{0.46\columnwidth}\raggedright\strut
\textbf{Bienes}\strut
\end{minipage} & \begin{minipage}[t]{0.22\columnwidth}\centering\strut
\strut
\end{minipage} & \begin{minipage}[t]{0.22\columnwidth}\centering\strut
\strut
\end{minipage}\tabularnewline
\begin{minipage}[t]{0.46\columnwidth}\raggedright\strut
Papelería, útiles y material de oficina\strut
\end{minipage} & \begin{minipage}[t]{0.22\columnwidth}\centering\strut
50\strut
\end{minipage} & \begin{minipage}[t]{0.22\columnwidth}\centering\strut
0.16\strut
\end{minipage}\tabularnewline
\begin{minipage}[t]{0.46\columnwidth}\raggedright\strut
Insumos, instrumental y accesorios de laboratorio\strut
\end{minipage} & \begin{minipage}[t]{0.22\columnwidth}\centering\strut
200\strut
\end{minipage} & \begin{minipage}[t]{0.22\columnwidth}\centering\strut
0.63\strut
\end{minipage}\tabularnewline
\begin{minipage}[t]{0.46\columnwidth}\raggedright\strut
Productos químicos\strut
\end{minipage} & \begin{minipage}[t]{0.22\columnwidth}\centering\strut
25\strut
\end{minipage} & \begin{minipage}[t]{0.22\columnwidth}\centering\strut
0.08\strut
\end{minipage}\tabularnewline
\begin{minipage}[t]{0.46\columnwidth}\raggedright\strut
\textbf{Servicios}\strut
\end{minipage} & \begin{minipage}[t]{0.22\columnwidth}\centering\strut
\strut
\end{minipage} & \begin{minipage}[t]{0.22\columnwidth}\centering\strut
\strut
\end{minipage}\tabularnewline
\begin{minipage}[t]{0.46\columnwidth}\raggedright\strut
Compra, sondeo y lectura de microarreglos\strut
\end{minipage} & \begin{minipage}[t]{0.22\columnwidth}\centering\strut
29610\strut
\end{minipage} & \begin{minipage}[t]{0.22\columnwidth}\centering\strut
92.79\strut
\end{minipage}\tabularnewline
\begin{minipage}[t]{0.46\columnwidth}\raggedright\strut
Gastos en el transporte de muestras\strut
\end{minipage} & \begin{minipage}[t]{0.22\columnwidth}\centering\strut
2000\strut
\end{minipage} & \begin{minipage}[t]{0.22\columnwidth}\centering\strut
6.27\strut
\end{minipage}\tabularnewline
\begin{minipage}[t]{0.46\columnwidth}\raggedright\strut
Impresiones, encuadernación y empastado\strut
\end{minipage} & \begin{minipage}[t]{0.22\columnwidth}\centering\strut
25\strut
\end{minipage} & \begin{minipage}[t]{0.22\columnwidth}\centering\strut
0.08\strut
\end{minipage}\tabularnewline
\begin{minipage}[t]{0.46\columnwidth}\raggedright\strut
\textbf{TOTAL}\strut
\end{minipage} & \begin{minipage}[t]{0.22\columnwidth}\centering\strut
31910\strut
\end{minipage} & \begin{minipage}[t]{0.22\columnwidth}\centering\strut
100\strut
\end{minipage}\tabularnewline
\bottomrule
\end{longtable}

\section{BIBLIOGRAFÍA}\label{bibliografia}

\hypertarget{refs}{}
\hypertarget{ref-WHO2016world}{}
1. WHO. \emph{World Malaria Report 2016}. Vol 13. Geneva: World Health
Organization; 2016.
\url{http://www.who.int/malaria/publications/world-malaria-report-2016/report/en/}.

\hypertarget{ref-rosas2016peru}{}
2. Rosas-Aguirre A, Gamboa D, Manrique P, et al. Epidemiology of
plasmodium vivax malaria in peru. \emph{The American Journal of Tropical
Medicine and Hygiene}. 2016;95(6 Suppl):133-144.
doi:\href{https://doi.org/10.4269/ajtmh.16-0268}{10.4269/ajtmh.16-0268}.

\hypertarget{ref-baird2009}{}
3. Baird JK. Severe and fatal vivax malaria challenges' benign tertian
malaria'dogma. \emph{Annals of tropical paediatrics}.
2009;29(4):251-252.

\hypertarget{ref-quispe2014}{}
4. Quispe AM, Pozo E, Guerrero E, et al. Plasmodium vivax
hospitalizations in a monoendemic malaria region: Severe vivax malaria?
\emph{The American journal of tropical medicine and hygiene}.
2014;91(1):11-17.
doi:\href{https://doi.org/10.4269/ajtmh.12-0610}{10.4269/ajtmh.12-0610}.

\hypertarget{ref-reyburn2015}{}
5. Reyburn H, Mbatia R, Drakeley C, et al. Association of transmission
intensity and age with clinical manifestations and case fatality of
severe plasmodium falciparum malaria. \emph{JAMA}.
2005;293(12):1461-1470.
doi:\href{https://doi.org/10.1001/jama.293.12.1461}{10.1001/jama.293.12.1461}.

\hypertarget{ref-llanoschea2015}{}
6. Llanos-Chea F, Martínez D, Rosas A, Samalvides F, Vinetz JM,
Llanos-Cuentas A. Characteristics of travel-related severe plasmodium
vivax and plasmodium falciparum malaria in individuals hospitalized at a
tertiary referral center in lima, peru. \emph{The American Journal of
Tropical Medicine and Hygiene}. 2015;93(6):1249-1253.
doi:\href{https://doi.org/10.4269/ajtmh.14-0652}{10.4269/ajtmh.14-0652}.

\hypertarget{ref-accelerate2016}{}
7. Quispe AM, Llanos-Cuentas A, Rodriguez H, et al. Accelerating to
zero: Strategies to eliminate malaria in the peruvian amazon. \emph{The
American Journal of Tropical Medicine and Hygiene}.
2016;94(6):1200-1207.
doi:\href{https://doi.org/10.4269/ajtmh.15-0369}{10.4269/ajtmh.15-0369}.

\hypertarget{ref-baldevi2013}{}
8. Baldeviano GC, Leiva KP, Quispe AM, et al. Serum markers of severe
clinical complications during plasmodium vivax malaria monoinfections in
the peruvian amazon basin. In: \emph{Abstract Book of the Astmh 62nd
Annual Meeting, Nov. 13--17, Washington d.C., United States}.; 2013:340.
\url{http://www.astmh.org/ASTMH/media/Documents/AbstractBook2013Final.pdf}.

\hypertarget{ref-crompton2010}{}
9. Crompton PD, Kayala MA, Traore B, et al. A prospective analysis of
the ab response to plasmodium falciparum before and after a malaria
season by protein microarray. \emph{Proceedings of the National Academy
of Sciences}. 2010;107(15):6958-6963.

\hypertarget{ref-Helb2015exposure}{}
10. Helb DA, Tetteh KKA, Felgner PL, et al. Novel serologic biomarkers
provide accurate estimates of recent plasmodium falciparum exposure for
individuals and communities. \emph{Proceedings of the National Academy
of Sciences}. 2015;112(32):E4438-E4447.
doi:\href{https://doi.org/10.1073/pnas.1501705112}{10.1073/pnas.1501705112}.

\hypertarget{ref-griffing2013history}{}
11. Griffing SM, Gamboa D, Udhayakumar V. The history of 20 th century
malaria control in peru. \emph{Malaria journal}. 2013;12(1):303.
doi:\href{https://doi.org/10.1186/1475-2875-12-303}{10.1186/1475-2875-12-303}.

\hypertarget{ref-soto2017spatio}{}
12. Soto-Calle V, Rosas-Aguirre A, Llanos-Cuentas A, et al.
Spatio-temporal analysis of malaria incidence in the peruvian amazon
region between 2002 and 2013. \emph{Scientific reports}. 2017;7:40350.
doi:\href{https://doi.org/10.1038/srep40350}{10.1038/srep40350}.

\hypertarget{ref-hotspots2015}{}
13. Rosas-Aguirre A, Speybroeck N, Llanos-Cuentas A, et al. Hotspots of
malaria transmission in the peruvian amazon: Rapid assessment through a
parasitological and serological survey. \emph{PLOS ONE}.
2015;10(9):1-21.
doi:\href{https://doi.org/10.1371/journal.pone.0137458}{10.1371/journal.pone.0137458}.

\hypertarget{ref-elliott2014}{}
14. Elliott SR, Fowkes F, Richards JS, Reiling L, Drew DR, Beeson JG.
Research priorities for the development and implementation of
serological tools for malaria surveillance. \emph{F1000Prime Rep}.
2014;6:100.

\hypertarget{ref-arevalo2014}{}
15. Arévalo-Herrera M, Forero-Peña DA, Rubiano K, et al. Plasmodium
vivax sporozoite challenge in malaria-naive and semi-immune colombian
volunteers. \emph{PLoS One}. 2014;9(6):e99754.
doi:\href{https://doi.org/10.1371/journal.pone.0099754}{10.1371/journal.pone.0099754}.

\hypertarget{ref-vigil2010}{}
16. Vigil A, Davies DH, Felgner PL. Defining the humoral immune response
to infectious agents using high-density protein microarrays.
\emph{Future microbiology}. 2010;5(2):241-251.

\hypertarget{ref-leroch2009postmod}{}
17. Chung D-WD, Ponts N, Cervantes S, Le Roch KG. Post-translational
modifications in plasmodium: More than you think! \emph{Molecular and
biochemical parasitology}. 2009;168(2):123-134.

\hypertarget{ref-R}{}
18. R Core Team. \emph{R: A Language and Environment for Statistical
Computing}. Vienna, Austria: R Foundation for Statistical Computing;
2016. \url{https://www.R-project.org/}.

\hypertarget{ref-CienciaReproducible2016}{}
19. Rodríguez-Sanchez F, Pérez-Luque AJ, Bartomeus I, Varela S. Ciencia
reproducible: qué, por qué, cómo? \emph{ECOS}. 2016;25(2):83-92.
doi:\href{https://doi.org/10.7818/ecos.2016.25-2.11}{10.7818/ecos.2016.25-2.11}.

\hypertarget{ref-Torres2014asymptomatic}{}
20. Torres KJ, Castrillon CE, Moss EL, et al. Genome-level determination
of plasmodium falciparum blood-stage targets of malarial clinical
immunity in the peruvian amazon. \emph{Journal of Infectious Diseases}.
November 2014.
doi:\href{https://doi.org/10.1093/infdis/jiu614}{10.1093/infdis/jiu614}.

\hypertarget{ref-chuquiyauri2015vivax}{}
21. Chuquiyauri R, Molina DM, Moss EL, et al. Genome-scale protein
microarray comparison of human antibody responses in plasmodium vivax
relapse and reinfection. \emph{The American journal of tropical medicine
and hygiene}. 2015;93(4):801-809.

\hypertarget{ref-howes2016global}{}
22. Howes RE, Battle KE, Mendis KN, et al. Global epidemiology of
plasmodium vivax. \emph{The American Journal of Tropical Medicine and
Hygiene}. 2016;95(6 Suppl):15-34.
doi:\href{https://doi.org/10.4269/ajtmh.16-0141}{10.4269/ajtmh.16-0141}.

\hypertarget{ref-path2011}{}
23. PATH. \emph{Staying the Course? Malaria Research and Development in
a Time of Economic Uncertainty}. Seattle, WA: PATH; 2011.
\url{www.malariavaccine.org/files/RD-report-June2011.pdf}.

\hypertarget{ref-gagnon2002enso}{}
24. Gagnon AS, Smoyer-Tomic KE, Bush AB. The el nino southern
oscillation and malaria epidemics in south america. \emph{International
Journal of Biometeorology}. 2002;46(2):81-89.

\hypertarget{ref-WHO2014severe}{}
25. WHO. Severe malaria. \emph{Trop Med Int Health}. 2014;19:7-131.
doi:\href{https://doi.org/10.1111/tmi.12313_2}{10.1111/tmi.12313\_2}.

\hypertarget{ref-Stanisic2015}{}
26. Stanisic DI, Fowkes FJI, Koinari M, et al. Acquisition of antibodies
against plasmodium falciparum merozoites and malaria immunity in young
children and the influence of age, force of infection, and magnitude of
response. \emph{Infection and Immunity}. 2015;83(2):646-660.
doi:\href{https://doi.org/10.1128/IAI.02398-14}{10.1128/IAI.02398-14}.

\hypertarget{ref-rogerson2007preg}{}
27. Rogerson SJ, Hviid L, Duffy PE, Leke RF, Taylor DW. Malaria in
pregnancy: Pathogenesis and immunity. \emph{The Lancet infectious
diseases}. 2007;7(2):105-117.

\hypertarget{ref-factores2001}{}
28. MINSA. \emph{Factores de Riesgo de La Malaria Grave En El Perú}.
Proyecto Vigía (MINSA-USAID); 2001.
\url{http://bvs.minsa.gob.pe/local/minsa/1772.pdf}.

\hypertarget{ref-baldevi2016}{}
29. Halsey ES, Baldeviano GC, Edgel KA, Vilcarromero S, Sihuincha M,
Lescano AG. Symptoms and immune markers in plasmodium/dengue virus
co-infection compared with mono-infection with either in peru.
\emph{PLOS Neglected Tropical Diseases}. 2016;10(4):1-16.
doi:\href{https://doi.org/10.1371/journal.pntd.0004646}{10.1371/journal.pntd.0004646}.

\hypertarget{ref-wassmer2015}{}
30. Wassmer SC, Taylor TE, Rathod PK, et al. Investigating the
pathogenesis of severe malaria: A multidisciplinary and
cross-geographical approach. \emph{The American journal of tropical
medicine and hygiene}. 2015;93(3 Suppl):42-56.
doi:\href{https://doi.org/10.4269/ajtmh.14-0841}{10.4269/ajtmh.14-0841}.

\hypertarget{ref-rts2015}{}
31. RTSS CTP. Efficacy and safety of rts,s/as01 malaria vaccine with or
without a booster dose in infants and children in africa: Final results
of a phase 3, individually randomised, controlled trial. \emph{The
Lancet}. 2015;386(9988):31-45.
doi:\href{https://doi.org/10.1016/S0140-6736(15)60721-8}{10.1016/S0140-6736(15)60721-8}.

\hypertarget{ref-rahimi2014meta}{}
32. Rahimi BA, Thakkinstian A, White NJ, Sirivichayakul C, Dondorp AM,
Chokejindachai W. Severe vivax malaria: A systematic review and
meta-analysis of clinical studies since 1900. \emph{Malaria journal}.
2014;13(1):481.
doi:\href{https://doi.org/10.1186/1475-2875-13-481}{10.1186/1475-2875-13-481}.

\hypertarget{ref-alexandre2010}{}
33. Alexandre MA, Ferreira CO, Siqueira AM, et al. Severe plasmodium
vivax malaria, brazilian amazon. \emph{Emerging infectious diseases}.
2010;16(10):1611.
doi:\href{https://doi.org/10.3201/eid1610.100685}{10.3201/eid1610.100685}.

\hypertarget{ref-smith2013}{}
34. Smith-Nuñez ES, Durand S, Baldeviano GC, et al. WHO criteria for
severe malaria in identifying severe vivax malaria: Preliminary data
from a study in iquitos, peru. In: \emph{Abstract Book of the Astmh 62nd
Annual Meeting, Nov. 13--17, Washington d.C., United States}.; 2013:398.
\url{http://www.astmh.org/ASTMH/media/Documents/AbstractBook2013Final.pdf}.

\hypertarget{ref-barber2015}{}
35. Barber BE, William T, Grigg MJ, et al. Parasite biomass-related
inflammation, endothelial activation, microvascular dysfunction and
disease severity in vivax malaria. \emph{PLoS Pathog}. 2015;11(1):1-13.
doi:\href{https://doi.org/10.1371/journal.ppat.1004558}{10.1371/journal.ppat.1004558}.

\hypertarget{ref-rainbow2016}{}
36. WHO. Malaria vaccine rainbow tables. In: World Health Organization.
Accessed: 15-june-2017.
\url{http://www.who.int/vaccine_research/links/Rainbow/en/index.html}.

\hypertarget{ref-cutts2014meta}{}
37. Cutts JC, Powell R, Agius PA, Beeson JG, Simpson JA, Fowkes FJ.
Immunological markers of plasmodium vivax exposure and immunity: A
systematic review and meta-analysis. \emph{BMC medicine}.
2014;12(1):150.
doi:\href{https://doi.org/10.1186/s12916-014-0150-1}{10.1186/s12916-014-0150-1}.

\hypertarget{ref-crompton2014rev}{}
38. Crompton PD, Moebius J, Portugal S, et al. Malaria immunity in man
and mosquito: Insights into unsolved mysteries of a deadly infectious
disease. \emph{Annual review of immunology}. 2014;32:157-187.
doi:\href{https://doi.org/10.1146/annurev-iy-32-060414-200001}{10.1146/annurev-iy-32-060414-200001}.

\hypertarget{ref-baird2013}{}
39. Baird JK. Evidence and implications of mortality associated with
acute plasmodium vivax malaria. \emph{Clinical Microbiology Reviews}.
2013;26(1):36-57.
doi:\href{https://doi.org/10.1128/CMR.00074-12}{10.1128/CMR.00074-12}.

\hypertarget{ref-jagannathan2014}{}
40. Jagannathan P, Eccles-James I, Bowen K, et al. IFN\(\gamma\)/il-10
co-producing cells dominate the cd4 response to malaria in highly
exposed children. \emph{PLoS Pathog}. 2014;10(1):e1003864.
doi:\href{https://doi.org/10.1371/journal.ppat.1003864}{10.1371/journal.ppat.1003864}.

\hypertarget{ref-schofield2006toll}{}
41. Schofield L, Mueller I. Clinical immunity to malaria. \emph{Current
molecular medicine}. 2006;6(2):205-221.
doi:\href{https://doi.org/10.2174/156652406776055221}{10.2174/156652406776055221}.

\hypertarget{ref-coban2005toll}{}
42. Coban C, Ishii KJ, Kawai T, et al. Toll-like receptor 9 mediates
innate immune activation by the malaria pigment hemozoin. \emph{Journal
of Experimental Medicine}. 2005;201(1):19-25.
doi:\href{https://doi.org/10.1084/jem.20041836}{10.1084/jem.20041836}.

\hypertarget{ref-portillo2001vir}{}
43. Portillo HA del, Fernandez-Becerra C, Bowman S, et al. A superfamily
of variant genes encoded in the subtelomeric region of plasmodium vivax.
\emph{Nature}. 2001;410(6830):839-842.
doi:\href{https://doi.org/10.1038/35071118}{10.1038/35071118}.

\hypertarget{ref-galinski1992rbp}{}
44. Galinski MR, Medina CC, Ingravallo P, Barnwell JW. A
reticulocyte-binding protein complex of plasmodium vivax merozoites.
\emph{Cell}. 1992;69(7):1213-1226.
doi:\href{https://doi.org/10.1016/0092-8674(92)90642-P}{10.1016/0092-8674(92)90642-P}.

\hypertarget{ref-cohen1961}{}
45. Cohen S, McGregor I, Carrington S. Gamma-globulin and acquired
immunity to human malaria. \emph{Nature}. 1961;192(4804):733-737.

\hypertarget{ref-sabchareon1991}{}
46. Sabchareon A, Burnouf T, Ouattara D, et al. Parasitologic and
clinical human response to immunoglobulin administration in falciparum
malaria. \emph{The American journal of tropical medicine and hygiene}.
1991;45(3):297-308.
doi:\href{https://doi.org/10.4269/ajtmh.1991.45.297}{10.4269/ajtmh.1991.45.297}.

\hypertarget{ref-mueller2013}{}
47. Mueller I, Galinski MR, Tsuboi T, Arévalo-Herrera M, Collins WE,
King CL. Natural acquisition of immunity to plasmodium vivax:
Epidemiological observations and potential targets. \emph{Adv
Parasitol}. 2013;81:77-131.
doi:\href{https://doi.org/10.1016/B978-0-12-407826-0.00003-5}{10.1016/B978-0-12-407826-0.00003-5}.

\hypertarget{ref-lopez2017}{}
48. López C, Yepes-Pérez Y, Hincapié-Escobar N, Díaz-Arévalo D,
Patarroyo MA. What is known about the immune response induced by
plasmodium vivax malaria vaccine candidates? \emph{Frontiers in
immunology}. 2017;8.
doi:\href{https://doi.org/10.3389/fimmu.2017.00126}{10.3389/fimmu.2017.00126}.

\hypertarget{ref-arevalo2016spz}{}
49. Arévalo-Herrera M, Vásquez-Jiménez JM, Lopez-Perez M, et al.
Protective efficacy of plasmodium vivax radiation-attenuated sporozoites
in colombian volunteers: A randomized controlled trial. \emph{PLoS Negl
Trop Dis}. 2016;10(10).
doi:\href{https://doi.org/10.1371/journal.pntd.0005070}{10.1371/journal.pntd.0005070}.

\hypertarget{ref-Finney2014}{}
50. Finney OC, Danziger SA, Molina DM, et al. Predicting antidisease
immunity using proteome arrays and sera from children naturally exposed
to malaria. \emph{Molecular \& Cellular Proteomics}.
2014;13(10):2646-2660.
doi:\href{https://doi.org/10.1074/mcp.M113.036632}{10.1074/mcp.M113.036632}.

\hypertarget{ref-King2015FOC}{}
51. King CL, Davies DH, Felgner P, et al. Biosignatures of
exposure/transmission and immunity. \emph{American Journal of Tropical
Medicine and Hygiene}. 2015;93(3 Suppl):16-27.
doi:\href{https://doi.org/10.4269/ajtmh.15-0037}{10.4269/ajtmh.15-0037}.

\hypertarget{ref-abbas2012}{}
52. Abbas A, Lichtman A, Pillai S. \emph{Inmunología Celular Y Molecular
+ Student Consult}. Elsevier Health Sciences Spain; 2012.
\url{https://books.google.es/books?id=iQCupVehIXQC}.

\hypertarget{ref-immunomics2016}{}
53. De Sousa KP, Doolan DL. Immunomics: A 21st century approach to
vaccine development for complex pathogens. \emph{Parasitology}.
2016;143(02):236-244.

\hypertarget{ref-sette2005}{}
54. Sette A, Fleri W, Peters B, Sathiamurthy M, Bui H-H, Wilson S. A
roadmap for the immunomics of category a--C pathogens. \emph{Immunity}.
2005;22(2):155-161.

\hypertarget{ref-Driguez2015}{}
55. Driguez P, Doolan DL, Molina DM, et al. Protein microarrays for
parasite antigen discovery. In: Peacock C, ed. \emph{Parasite Genomics
Protocols}. New York, NY: Springer New York; 2015:221-233.
doi:\href{https://doi.org/10.1007/978-1-4939-1438-8_13}{10.1007/978-1-4939-1438-8\_13}.

\hypertarget{ref-Biobase}{}
56. Huber, W., Carey, et al. Orchestrating high-throughput genomic
analysis with Bioconductor. \emph{Nature Methods}. 2015;12(2):115-121.
\url{http://www.nature.com/nmeth/journal/v12/n2/full/nmeth.3252.html}.

\hypertarget{ref-smyth2004ebayes}{}
57. Smyth GK, others. Linear models and empirical bayes methods for
assessing differential expression in microarray experiments. \emph{Stat
Appl Genet Mol Biol}. 2004;3(1):3.

\hypertarget{ref-limma}{}
58. Ritchie ME, Phipson B, Wu D, et al. limma powers differential
expression analyses for RNA-sequencing and microarray studies.
\emph{Nucleic Acids Research}. 2015;43(7):e47.

\hypertarget{ref-Gaujoux2010NMF}{}
59. Gaujoux R, Seoighe C. A flexible r package for nonnegative matrix
factorization. \emph{BMC Bioinformatics}. 2010;11(1):367.
doi:\href{https://doi.org/10.1186/1471-2105-11-367}{10.1186/1471-2105-11-367}.

\hypertarget{ref-plasmodb}{}
60. Aurrecoechea C, Brestelli J, Brunk BP, et al. PlasmoDB: A functional
genomic database for malaria parasites. \emph{Nucleic acids research}.
2009;37(suppl 1):D539-D543.


\end{document}
