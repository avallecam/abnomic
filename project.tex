% Options for packages loaded elsewhere
\PassOptionsToPackage{unicode}{hyperref}
\PassOptionsToPackage{hyphens}{url}
\PassOptionsToPackage{dvipsnames,svgnames*,x11names*}{xcolor}
%
\documentclass[
  a4paper]{article}
\usepackage{lmodern}
\usepackage{amsmath}
\usepackage{ifxetex,ifluatex}
\ifnum 0\ifxetex 1\fi\ifluatex 1\fi=0 % if pdftex
  \usepackage[T1]{fontenc}
  \usepackage[utf8]{inputenc}
  \usepackage{textcomp} % provide euro and other symbols
  \usepackage{amssymb}
\else % if luatex or xetex
  \usepackage{unicode-math}
  \defaultfontfeatures{Scale=MatchLowercase}
  \defaultfontfeatures[\rmfamily]{Ligatures=TeX,Scale=1}
\fi
% Use upquote if available, for straight quotes in verbatim environments
\IfFileExists{upquote.sty}{\usepackage{upquote}}{}
\IfFileExists{microtype.sty}{% use microtype if available
  \usepackage[]{microtype}
  \UseMicrotypeSet[protrusion]{basicmath} % disable protrusion for tt fonts
}{}
\makeatletter
\@ifundefined{KOMAClassName}{% if non-KOMA class
  \IfFileExists{parskip.sty}{%
    \usepackage{parskip}
  }{% else
    \setlength{\parindent}{0pt}
    \setlength{\parskip}{6pt plus 2pt minus 1pt}}
}{% if KOMA class
  \KOMAoptions{parskip=half}}
\makeatother
\usepackage{xcolor}
\IfFileExists{xurl.sty}{\usepackage{xurl}}{} % add URL line breaks if available
\IfFileExists{bookmark.sty}{\usepackage{bookmark}}{\usepackage{hyperref}}
\hypersetup{
  colorlinks=true,
  linkcolor=Blue,
  filecolor=Maroon,
  citecolor=Blue,
  urlcolor=Blue,
  pdfcreator={LaTeX via pandoc}}
\urlstyle{same} % disable monospaced font for URLs
\usepackage[margin=1in]{geometry}
\usepackage{longtable,booktabs}
\usepackage{calc} % for calculating minipage widths
% Correct order of tables after \paragraph or \subparagraph
\usepackage{etoolbox}
\makeatletter
\patchcmd\longtable{\par}{\if@noskipsec\mbox{}\fi\par}{}{}
\makeatother
% Allow footnotes in longtable head/foot
\IfFileExists{footnotehyper.sty}{\usepackage{footnotehyper}}{\usepackage{footnote}}
\makesavenoteenv{longtable}
\usepackage{graphicx}
\makeatletter
\def\maxwidth{\ifdim\Gin@nat@width>\linewidth\linewidth\else\Gin@nat@width\fi}
\def\maxheight{\ifdim\Gin@nat@height>\textheight\textheight\else\Gin@nat@height\fi}
\makeatother
% Scale images if necessary, so that they will not overflow the page
% margins by default, and it is still possible to overwrite the defaults
% using explicit options in \includegraphics[width, height, ...]{}
\setkeys{Gin}{width=\maxwidth,height=\maxheight,keepaspectratio}
% Set default figure placement to htbp
\makeatletter
\def\fps@figure{htbp}
\makeatother
\setlength{\emergencystretch}{3em} % prevent overfull lines
\providecommand{\tightlist}{%
  \setlength{\itemsep}{0pt}\setlength{\parskip}{0pt}}
\setcounter{secnumdepth}{5}
\usepackage{multirow}
\usepackage{pdflscape}
\usepackage{afterpage}
\usepackage{capt-of}
\usepackage{array}
\usepackage{color,soul}
\usepackage{geometry}
\ifluatex
  \usepackage{selnolig}  % disable illegal ligatures
\fi
\newlength{\cslhangindent}
\setlength{\cslhangindent}{1.5em}
\newlength{\csllabelwidth}
\setlength{\csllabelwidth}{3em}
\newenvironment{CSLReferences}[2] % #1 hanging-ident, #2 entry spacing
 {% don't indent paragraphs
  \setlength{\parindent}{0pt}
  % turn on hanging indent if param 1 is 1
  \ifodd #1 \everypar{\setlength{\hangindent}{\cslhangindent}}\ignorespaces\fi
  % set entry spacing
  \ifnum #2 > 0
  \setlength{\parskip}{#2\baselineskip}
  \fi
 }%
 {}
\usepackage{calc}
\newcommand{\CSLBlock}[1]{#1\hfill\break}
\newcommand{\CSLLeftMargin}[1]{\parbox[t]{\csllabelwidth}{#1}}
\newcommand{\CSLRightInline}[1]{\parbox[t]{\linewidth - \csllabelwidth}{#1}\break}
\newcommand{\CSLIndent}[1]{\hspace{\cslhangindent}#1}

\author{}
\date{\vspace{-2.5em}}

\begin{document}

\renewcommand{\contentsname}{Índice General} 
\renewcommand{\tablename}{Tabla}
\renewcommand{\tableautorefname}{Tabla}

\pagenumbering{gobble}

\clearpage
\newgeometry{left=0.5cm,right=0.5cm,top=2cm,bottom=1cm}

\begin{centering}

\begin{figure}[!ht]
  \begin{center}
    \includegraphics[width=.8in]{figure/UNMSM_escudo-2000px.png}% en informe: 8
  \end{center}
\end{figure}

\LARGE %https://tex.stackexchange.com/questions/24599/what-point-pt-font-size-are-large-etc
UNIVERSIDAD NACIONAL MAYOR DE SAN MARCOS

\large
(Universidad del Perú, DECANA DE AMÉRICA)

%\vspace{.3 cm}

\Large
FACULTAD DE CIENCIAS BIOLÓGICAS

%\vspace{.3 cm}

%\normalsize
\large
ESCUELA PROFESIONAL DE GENÉTICA Y BIOTECNOLOGÍA

\vspace{.5 cm}

\Large
PROYECTO DE TESIS PARA OPTAR AL TÍTULO PROFESIONAL DE 

BIÓLOGO GENETISTA BIOTECNÓLOGO

\vspace{2.5 cm}

\Large
Comparación de la respuesta de anticuerpos ante %la %infección con 
\textit{Plasmodium vivax}
en \\pacientes de la Amazonía Peruana %ciudad de Iquitos (Loreto - Perú)
según su severidad y episodios \\previos %exposición previa
mediante microarreglos de proteínas
% aPerfil de anticuerpos en respuesta a la infección con malaria vivax
%mediante un enfoque inmunómico .
%% con sintomatología severa y no complicada/ pre-inmunes/ semi-inmunes

\vspace{2.5 cm}

\large
TESISTA:

Bach. Andree Adolfo Valle Campos

\vspace{.3 cm}

ASESORES:

INTERNO: Dr. Juan Jiménez Chunga %Prof. Walter Cabrera-Febolá

EXTERNO: PhD. G. Christian Baldeviano

\vspace{.3 cm}

Lugar de Ejecución:

Centro de Enfermedades Tropicales de la Marina de los Estados Unidos 
NAMRU-6

\vspace{.3 cm}

Duración: 1 año

\vspace{1.5 cm}

\Large
Lima - Perú

%\vspace{.5 cm}

2020

\end{centering}

\vfill
\restoregeometry
\clearpage

\newpage
\tableofcontents

\newpage
\pagenumbering{arabic}

\hypertarget{resumen}{%
\section*{RESUMEN}\label{resumen}}
\addcontentsline{toc}{section}{RESUMEN}

\begin{quote}
\emph{Plasmodium vivax} es responsable del 80\% de la malaria en el
Perú. Casos de enfermedad severa por mono-infecciones de \emph{P. vivax}
han sido reportados tanto en el Noreste amazónico como en Norte costero.
Sin embargo, dicha condición aún está subestimada en la Norma Técnica a
nivel nacional, incrementando el riesgo de posibles diagnósticos tardíos
o inapropiados por el personal de salud. Con el fin de identificar
biomarcadores de relevancia clínica contra la malaria severa por
\emph{P. vivax}, el presente proyecto de tesis propone comparar la
respuesta de anticuerpos ante la infección en pacientes severos y
no-severos provenientes de un estudio prospectivo caso-control ejecutado
en la ciudad de Iquitos, Loreto - Perú, empleando microarreglos de
proteínas. Los resultados permitirán proponer marcadores serológicos
discriminantes de severidad con el potencial de implementarse en
programas de serovigilancia en miras al control y eliminación de la
malaria por \emph{P. vivax} en la región.
\end{quote}

\hypertarget{planteamiento-del-problema}{%
\section{PLANTEAMIENTO DEL PROBLEMA}\label{planteamiento-del-problema}}

\hypertarget{intro}{%
\subsection{Formulación del problema}\label{intro}}

Malaria es una enfermedad parasitaria causada por protozoarios del
género \emph{Plasmodium} y transmitida por mosquitos del género
\emph{Anopheles}. En el 2019, se estimaron 229 millones de casos en en
87 paises endémicos y 409,000 muertes atribuidas a esta infección
(\protect\hyperlink{ref-world2020world}{WHO 2020}). Aunque \emph{P.
falciparum} representó el 96 y 99\% de estas cifras, fuera de África se
estimó que \emph{P. vivax} fue responsable del 41 y 86\%,
respectivamente. Más aún, la región de las Américas tuvo la mayor
proporción de estos casos (69\%) donde Perú fue el tercer país con más
reportes (n=63,153), detrás de Brasil y Venezuela, atribuidos en un 80\%
a dicha especie (\protect\hyperlink{ref-rosas2016peru}{Rosas-Aguirre et
al. 2016}).

Reportes recientes de malaria severa y fatal causados por \emph{P.
vivax} han desafiado su tradicional condición de enfermedad benigna
(\protect\hyperlink{ref-baird2009}{Baird 2009};
\protect\hyperlink{ref-quispe2014}{Antonio M. Quispe et al. 2014}). El
incremento de estos casos puede ser consecuencia de un retraso en la
adquisición de inmunidad en la población por una reducción en la
intensidad de transmisión (\protect\hyperlink{ref-reyburn2015}{Reyburn
et al. 2005}). Por ello, tanto su actual subestimación
(\protect\hyperlink{ref-norma2001}{MINSA 2007}) como la ausencia de
marcadores serológicos de protección contra la severidad en las actuales
estrategias de eliminación
(\protect\hyperlink{ref-accelerate2016}{Antonio M. Quispe et al. 2016})
podría incrementar su prevalencia a largo plazo. En contraste con dicha
hipótesis, pacientes de la Amazonía Peruana con malaria por \emph{P.
vivax} severa y no complicada mostraron un grado similar de exposición
previa, basado en la respuesta humoral contra un marcador tradicional
(\protect\hyperlink{ref-baldevi2013}{Baldeviano et al. 2013}). Sin
embargo, recientes estudios a larga escala han desafiado su validez como
indicador de inmunidad o exposición
(\protect\hyperlink{ref-crompton2010}{Crompton et al. 2010};
\protect\hyperlink{ref-Helb2015exposure}{Helb et al. 2015}).

Por esta razón, el presente estudio tiene como objetivo comparar la
respuesta de anticuerpos ante la infección con \emph{P. vivax} contra
más de 500 antígenos del parásito en pacientes severos y no-severos de
la ciudad de Iquitos, empleando microarreglos de proteínas.
Hipotetizamos que los no-severos poseen mayor reactividad serológica
contra antígenos de exposición, invasión o adhesión celular, con
respecto a los severos. Primero, se identificarán los antígenos con
reactividad diferenciada entre ambos grupos. Luego, se determinarán los
antígenos de respuesta secundaria en pacientes con episodios previos de
esta infección. Finalmente, se describirán sus características proteicas
junto a los antígenos con mayor reactividad en toda la muestra con el
propósito de proponerlos como candidatos a vigilancia
seroepidemiológica.

\hypertarget{preguntas-de-investigaciuxf3n}{%
\subsection{Preguntas de
investigación}\label{preguntas-de-investigaciuxf3n}}

\begin{enumerate}
\def\labelenumi{\arabic{enumi}.}
\item
  ¿Cuáles son los antígenos con reactividad serológica diferenciada ante
  la infección por \emph{P. vivax} entre pacientes severos y no-severos?
\item
  ¿Cuáles son los antígenos con reactividad serológica diferenciada ante
  la infección por \emph{P. vivax} entre pacientes con y sin episodios
  previos?
\item
  ¿Cuáles son las características proteicas de los antígenos con
  reactividad serológica diferenciada o predominante en pacientes con
  malaria por \emph{P. vivax}?
\end{enumerate}

\hypertarget{justificaciuxf3n-de-la-investigaciuxf3n}{%
\section{JUSTIFICACIÓN DE LA
INVESTIGACIÓN}\label{justificaciuxf3n-de-la-investigaciuxf3n}}

\hypertarget{justif}{%
\subsection{Justificación}\label{justif}}

Ante la reemergencia repetitiva de la malaria en la Amazonía Peruana
(\protect\hyperlink{ref-griffing2013history}{Griffing, Gamboa, and
Udhayakumar 2013}; \protect\hyperlink{ref-rosas2016peru}{Rosas-Aguirre
et al. 2016}), la implementación de vigilancias serológicas
programáticas en nuestro país es una prioridad
(\protect\hyperlink{ref-hotspots2015}{Rosas-Aguirre et al. 2015}). En
zonas con baja transmisión, estos ensayos poseen una mayor sensibilidad
y representan un menor gasto económico en comparación a las estrategias
convencionales de monitoreo y control
(\protect\hyperlink{ref-elliott2014}{Elliott et al. 2014}). Por esta
razón, el presente estudio se justifica en el descubrimiento de
antígenos potencialmente discriminantes de condiciones clínicas como
severidad o exposición. Estos permitirán optimizar y acelerar la
ejecución de vigilancias serológicas en los programas de salud pública
contra la malaria en el Perú, en miras a su control y posterior
eliminación (\protect\hyperlink{ref-accelerate2016}{Antonio M. Quispe et
al. 2016}).

\hypertarget{limit}{%
\subsection{Limitaciones}\label{limit}}

Resumimos tres limitaciones del estudio.

Primero, las muestras a evaluar provienen de pacientes con infecciones
no sincronizadas. El muestreo se ejecutó en la fase sintomática, momento
en el que acuden al Hospital, con un desconocimiento del inicio de la
infección por paciente. Sin embargo, infecciones experimentales han
estimado que dicha fase ocurre normalmente entre los 11 y 13 días de
infección (\protect\hyperlink{ref-arevalo2014}{Arévalo-Herrera et al.
2014}).

Segundo, el estudio posee debilidades propias del diseño experimental a
emplear. El diseño de tipo Caso Control es susceptible a errores
sistemáticos de selección y clasificación, posee dificultad para
establecer relaciones causa-efecto y a través de él no se pueden
calcular prevalencias o incidencias. Además, la ausencia de un
seguimiento activo impide el registro de covariables relevantes para la
caracterización de la enfermedad.

Tercero, los microarreglos de proteínas poseen limitantes propias a su
fabricación (\protect\hyperlink{ref-vigil2010}{Vigil, Davies, and
Felgner 2010}). Cada paso (amplificación, clonamiento, expresión de
genes a larga escala e impresión del arreglo) posee una eficiencia
límite que afectará la calidad final de las proteínas. Además, el
plegamiento proteico no será posible de verificar a dicha escala. Por
último, la identificación de antígenos con modificaciones
postranscriptacionales, particularmente relevantes en \emph{Plasmodium}
(\protect\hyperlink{ref-leroch2009postmod}{Chung et al. 2009}), no serán
posibles de reproducir en su integridad en el sistema de expresión
procarionte. A pesar de ello, se evaluará la validez del ensayo con
controles internos a detallar en la \protect\hyperlink{validez}{sección
4.3.2.2}.

\hypertarget{viabilidad}{%
\subsection{Viabilidad}\label{viabilidad}}

El proyecto es viable por la factibilidad de su ejecución, el interés de
sus resultados para el área, la novedad de su enfoque y la relevancia de
su método para la investigación biomédica.

Primero, su ejecución es factible por la disponibilidad inmediata de
muestras provenientes de un estudio prospectivo ya realizado, su
adecuado tamaño muestral (n=60), la objetividad de la pregunta planteada
y la experiencia técnica del investigador en la herramienta de análisis
R (\protect\hyperlink{ref-R2016}{R Core Team 2016})/Bioconductor
(\protect\hyperlink{ref-bioconductor2004}{Gentleman et al. 2004}).
Segundo, el problema es de interés para la comunidad por la posibilidad
de que sus resultados generen un cambio en la práctica médica. Tercero,
la novedad de su enfoque a larga escala permitirá corroborar, refutar o
extender el estado del conocimiento y brindar mayor evidencia con
respecto a la controversia mencionada. Cuarto, la relevancia de su
método de análisis reproducible con software libre permitirá
transparentar la generación de resultados y promover su puesta en
práctica en futuras investigaciones de esta u otras áreas.

\hypertarget{formulaciuxf3n-de-objetivos}{%
\section{FORMULACIÓN DE OBJETIVOS}\label{formulaciuxf3n-de-objetivos}}

\hypertarget{objetivos}{%
\subsection{Objetivos}\label{objetivos}}

\hypertarget{general}{%
\subsubsection{General}\label{general}}

\begin{itemize}
\tightlist
\item
  Identificar un subconjunto de antígenos con reactividad serológica
  discriminante de condiciones clínicas relevantes ante la infección por
  \emph{P. vivax}.
\end{itemize}

\hypertarget{especuxedficos}{%
\subsubsection{Específicos}\label{especuxedficos}}

\begin{itemize}
\item
  Identificar antígenos de \emph{P. vivax} con reactividad serológica
  diferenciada ante la infección entre pacientes severos y no-severos.
\item
  Identificar antígenos de \emph{P. vivax} con reactividad serológica
  diferenciada ante la infección entre pacientes con y sin episodios
  previos.
\item
  Describir las características proteicas de los antígenos con
  reactividad diferenciada o predominante.
\end{itemize}

\hypertarget{exploratorio}{%
\subsubsection{Exploratorio}\label{exploratorio}}

\begin{itemize}
\tightlist
\item
  Comparar la amplitud e intensidad de respuesta de anticuerpos según la
  edad de los pacientes con malaria por \emph{P. vivax}.
\end{itemize}

\hypertarget{hipuxf3tesis-y-variables-de-investigaciuxf3n}{%
\section{HIPÓTESIS Y VARIABLES DE
INVESTIGACIÓN}\label{hipuxf3tesis-y-variables-de-investigaciuxf3n}}

\hypertarget{hipuxf3tesis}{%
\subsection{Hipótesis}\label{hipuxf3tesis}}

\hypertarget{de-diferencia-entre-grupos}{%
\subsubsection{De diferencia entre
grupos}\label{de-diferencia-entre-grupos}}

\begin{enumerate}
\def\labelenumi{\arabic{enumi}.}
\item
  Los pacientes con malaria no-severa poseen mayor reactividad
  serológica contra antígenos de \emph{P. vivax} asociados a exposición,
  invasión o adhesión celular con respecto a los pacientes severos.
\item
  Los pacientes con episodios previos de malaria poseen mayor
  reactividad serológica contra antígenos de \emph{P. vivax} asociados a
  exposición con respecto a los pacientes sin episodios previos.
\end{enumerate}

\hypertarget{descriptiva}{%
\subsubsection{Descriptiva}\label{descriptiva}}

\begin{enumerate}
\def\labelenumi{\arabic{enumi}.}
\setcounter{enumi}{2}
\tightlist
\item
  Los antígenos con reactividad diferenciada o predominante se
  caracterizan por poseer una localización extracelular y estar bajo
  presión selectiva por el sistema inmune.
\end{enumerate}

\hypertarget{variables}{%
\subsection{Variables}\label{variables}}

El presente estudio busca contrastar la respuesta de anticuerpos ante la
infección con \emph{P. vivax} entre pacientes clasificados según su
severidad y episodios previos. Para ello se medirá la variable de
reactividad serológica de antígenos de \emph{P. vivax} luego de sondear
plasma de pacientes en un microarreglo de proteínas. Además se
clasificará a los pacientes como severos o no-severos de acuerdo a
diagnóstico clínico y bioquímico, y como con-episodios o sin-episodios
previos de acuerdo a lo reportado en la encuesta con el médico tratante.

Primero, la reactividad serológica cuantificará la especificidad de los
anticuerpos de respuesta contra los antígenos del patógeno \emph{P.
vivax}. Esta variable medirá indirectamente la interacción entre
antígenos y anticuerpos, a través de la lectura de la intensidad
fluorescente producto de esta reacción.

Segundo, los pacientes severos se definirán por la presencia de
manifestaciones clínicas severas o complicaciones sistémicas. Esta
variable será asignada por la presencia de uno o más criterios clínicos
y de laboratorio para la malaria severa, según la clasificación
recomendada por la Organización Mundial de la Salud (OMS)
(\protect\hyperlink{ref-WHO2014severe}{WHO 2014}).

Tercero, la clasificación por episodios previos reportados representará
la presencia de infecciones previas con malaria y, consecuentemente, la
presencia de una respuesta inmune humoral secundaria. Esta variable será
reportada por el paciente como el número de eventos previos a la actual
infección. Dada la incertidumbre de dicho dato, la variable será
dicotomizada con el fin de facilitar su inclusión.

\hypertarget{operacionalizaciuxf3n-de-variables}{%
\subsubsection{Operacionalización de
variables}\label{operacionalizaciuxf3n-de-variables}}

Ver \autoref{tab:opera}.

\begin{table}[ht]
        \captionof{table}{Operacionalización de variables}
        \label{tab:opera}
        \vspace{-1mm}
\begin{center}
\hspace*{-1cm}
\begin{tabular}{>{\centering}m{2.4cm} m{2.2cm}m{2.2cm}m{1.8cm}m{2cm}m{1.7cm}m{1.5cm}m{1.6cm} @{}m{0pt}@{} }
  
  \hline
  \multirow{2}{*}{Variable}
  & 
  \multicolumn{2}{c}{Definición} 
  %&
  %\begin{minipage}{2.2cm}
  %Definición\\conceptual
  %\end{minipage}
  %&
  %\begin{minipage}{2.2cm}
  %Definición\\operacional
  %\end{minipage}
  & 
  \multirow{2}{*}{
  \begin{minipage}{1.8cm}
  Instrumento\\de medición
  \end{minipage}
  }
  &
  \multirow{2}{*}{
  \begin{minipage}{2cm}
  Criterios\\de medición
  \end{minipage}
  }
  &
  \multirow{2}{*}{
  \begin{minipage}{1.7cm}
  Tipo de\\variable
  \end{minipage}
  }
  &
  \multirow{2}{*}{
  \begin{minipage}{1.5cm}
  Escala de \\medición
  \end{minipage}
  }
  &
  \multirow{2}{*}{
  Fuente
  } &\\[0ex]
  %\hline
  \cline{2-3}
  
  &
  Conceptual
  &
  Operacional
  & 
  &
  &
  & &\\[1ex]
  \hline
  
  \textbf{Dependiente} Reactividad serológica
  & 
  % esCONCEPTUAL: 
  \begin{minipage}{2.2cm} 
  Especificidad \\de anticuerpos \\de respuesta contra un antígeno
  \end{minipage} 
  &
  % aOPERACIONAL: 
  \begin{minipage}{2.2cm} 
  Medida \\indirecta de \\la reacción antígeno-anticuerpo
  \end{minipage} 
  % aDETALLES: medida indirecta de la reacción antígeno-anticuerpo 
  % mediante la lectura de la reacción fluorescente entre 
  % anticuerpo secundario y fluoroforo por spot
  & 
  \begin{minipage}{2.2cm} 
  %Lector\\ óptico de\\
  Microarreglo\\de proteínas
  \end{minipage}
  & 
  \begin{minipage}{2cm} 
  \textbf{0-6000} MFI o intensidad\\
  fluorescente \\promedio.
  \end{minipage} 
  &
  Numérica contínua
  & 
  Razón
  &
  Plasma sanguíneo &\\[13ex]
  \hline

  \textbf{Independiente} Severidad
  & 
  % aCONCEPTUAL: 
  Presencia de manifestaciones clínicas severas o complicaciones sistémicas
  &
  % aOPERACIONAL:
  Número de criterios de la clasificación OMS para malaria severa
  & 
  \begin{minipage}{2.2cm} 
  Diagnóstico \\clínico \\y pruebas \\bioquímicas 
  \end{minipage}
  & 
  \begin{minipage}{2cm} 
  \textbf{No-severa:} 0 criterios\\
  \textbf{Severa:} 1 o más criterios.
  \end{minipage}
  &
  Categórica dicotómica
  & 
  Nominal
  &
  Historia clínica y muestra de sangre &\\[15ex]
  \hline
  
  \textbf{Independiente} Episodios previos
  & 
  % aCONCEPTUAL: 
  Exposición a la infección de malaria en el pasado
  &
  % aOPERACIONAL:
  Número de episodios previos reportados 
  & 
  Encuesta
  & 
  \begin{minipage}{2.1cm} 
  \textbf{Sin:} 0 episodios\\
  \textbf{Con:} 1 o más episodios
  \end{minipage}
  &
  Categórica dicotómica
  & 
  Nominal
  &
  Historia clínica &\\[10ex]
  \hline

%  \textbf{Interviniente} Edad %Confusora
%  & 
%  % aCONCEPTUAL: 
%  Edad del paciente
%  &
%  % aOPERACIONAL:
%  Años de vida reportados
%  & 
%  Encuesta
%  & 
%  \begin{minipage}{2.2cm} 
%  \textbf{0-90} años.
%  \end{minipage}
%  &
%  Numérica discreta
%  & 
%  Razón
%  &
%  Historia clínica &\\[10ex]
%  \hline


  % etc. ...
\end{tabular}
\hspace*{-1cm}
\end{center}
\end{table}

\hypertarget{antecedentes}{%
\section{ANTECEDENTES}\label{antecedentes}}

\hypertarget{antecedentes-de-la-investigaciuxf3n}{%
\subsection{Antecedentes de la
investigación}\label{antecedentes-de-la-investigaciuxf3n}}

\begin{enumerate}
\def\labelenumi{\alph{enumi}.}
\item
  En África

  Un hito en la aplicación de microarreglos de proteínas para el estudio
  de la respuesta humoral a escala epidemiológica fue la publicación de
  Crompton et al.~2010 (\protect\hyperlink{ref-crompton2010}{Crompton et
  al. 2010}). Este estudio comparó la respuesta de anticuerpos contra el
  23\% del proteoma de \emph{P. falciparum} antes y después de la
  temporada de malaria en 220 individuos de Mali, en dos grupos
  poblacionales: 2-10 y 18-25 años. Dentro del grupo de niños entre 8 y
  10 años se identificaron 49 proteínas con mayor reactividad serológica
  en el grupo de infectados asintomáticos, en comparación a los
  sintomáticos. Cinco de los principales candidatos a vacuna (CSP,
  LSA-3, MSP1, MSP2, AMA1) no lograron discriminar ambos grupos. Sin
  embargo, cuatro candidatos secundarios (STARP, LSA-1, RESA, antígeno
  332), con expresión en diferentes estadios del ciclo biológico, sí
  lograron tal discriminación.

  Un segundo hito de interés representa el trabajo de Helb et al.~2015
  (\protect\hyperlink{ref-Helb2015exposure}{Helb et al. 2015}). Ellos
  reportaron una estrategia para identificar combinaciones de respuestas
  de anticuerpos contra más de un antígeno que maximicen la información
  de la exposición reciente al nivel de individuos. Para ello emplearon
  modelos basados en el aprendizaje automático o \emph{machine learning}
  para el análisis de las respuestas contra 865 antígenos de \emph{P.
  falciparum} en 186 niños (3-6 años) de Uganda en base al registro
  activo y pasivo de sus historias clínicas a los largo de un año. En
  contraste a los marcadores tradicionalmente empleados para evaluar
  exposición a nivel poblacional (CSP, MSP1, MSP2, AMA1)
  (\protect\hyperlink{ref-elliott2014}{Elliott et al. 2014}), se
  identificaron marcadores más informativos como hyp2, GEXP18, EMP1,
  ETRAMP4, HSP40-II y PF70. La validación de este método permitirá
  optimizar la selección tradicional de marcadores.
\item
  En Perú

  El primer estudio en publicarse fue de Torres et al.~2014
  (\protect\hyperlink{ref-Torres2014asymptomatic}{Torres et al. 2014})
  donde reportaron marcadores de inmunidad clínica -no esterilizante- de
  adquisición natural a la malaria por \emph{P. falciparum}. Además de
  ello, el resultado con mayor implicancias fue que estos marcadores
  presentaron un enriquecimiento de polimorfismos no-sinónimos,
  indicativo de una presión de selección positiva por parte del sistema
  inmune, a pesar de provenir de una zona con baja transmisión. Las 51
  proteínas con mayor reactividad en 14 pacientes infectados y
  asintomáticos se obtuvieron al comparar las respuestas con 24 paciente
  sintomáticos provenientes de la Amazonía Peruana (Departamento de
  Loreto, Provincia de Maynas y Requena) contra 824 fragmentos (699
  proteínas) de \emph{P. falciparum}.

  El segundo y último estudio en ser publicado ha sido el de Chuquiyauri
  et al.~2015 (\protect\hyperlink{ref-chuquiyauri2015vivax}{Chuquiyauri
  et al. 2015}). Ellos compararon la respuesta contra \emph{P. vivax} de
  pacientes con relapsos y reinfección, sin encontrar diferencia alguna
  entre ambos grupos. Sin embargo, al igual que el anterior estudio,
  identificaron un enriquecimiento de proteínas con polimorfismos
  no-sinónimos en el grupo de antígenos reactivos. Además, dentro del
  grupo con mayor reactividad en toda la muestra, resaltaron a PvMSP-10
  como potencial candidato a vacuna al presentar la expresión más
  consistente y validar lo reportado por dos estudios previos con
  enfoque tradicional donde emplean ambos sistemas de expresión:
  eucariota y procariota. El estudio empleó un arreglo con 2233
  fragmentos (1936 proteínas) en 106 individuos de la ciudad de Maynas,
  Loreto - Perú.
\end{enumerate}

\hypertarget{bases-teuxf3ricas}{%
\subsection{Bases teóricas}\label{bases-teuxf3ricas}}

\hypertarget{malaria-por-plasmodium-vivax}{%
\subsubsection{\texorpdfstring{Malaria por \emph{Plasmodium
vivax}}{Malaria por Plasmodium vivax}}\label{malaria-por-plasmodium-vivax}}

\begin{enumerate}
\def\labelenumi{\alph{enumi}.}
\item
  Epidemiología

  \begin{enumerate}
  \def\labelenumii{\roman{enumii}.}
  \item
    \textbf{A nivel mundial}

    \emph{P. vivax} y \emph{P. falciparum} son los principales
    responsables de los casos de malaria en humanos. Ambas especies
    exponen aproximadamente a 2.5 mil millones de personas en riesgo de
    infección (\protect\hyperlink{ref-howes2016global}{Howes et al.
    2016}). Sin embargo, \emph{P. vivax} es el parásito dominante en las
    regiones fuera del África subsahariana, en su mayoría densamente
    pobladas y empobrecidas. Entre ellas, Etiopía, India, Indonesia y
    Pakistán acumularon el 78\% de casos de \emph{P. vivax} a nivel
    mundial. A su vez, la región de las Américas tuvo la mayor
    proporción de estos, con un 69\%
    (\protect\hyperlink{ref-WHO2016world}{WHO 2016}). A pesar de ello,
    hasta el momento la mayoría de la investigación y financiamiento
    está destinado a la prevención, tratamiento y control de \emph{P.
    falciparum} (\protect\hyperlink{ref-path2011}{PATH 2011}).
  \item
    \textbf{En el Perú}

    En el 2015, Perú fue el tercer país con más casos reportados en
    Latinoamérica (19\%), detrás de Brasil (24\%) y Venezuela (30\%)
    (\protect\hyperlink{ref-WHO2016world}{WHO 2016}). El 80\% fueron
    causados por \emph{P. vivax} (63,153 en total), en regiones con
    endemismo y transmisión heterogénea
    (\protect\hyperlink{ref-rosas2016peru}{Rosas-Aguirre et al. 2016}).
    El 95\% pertenecieron al Noreste amazónico (con una razón Pv/Pf de
    4/1) y el resto al Norte costero y la región minera del Suroeste.
    Notablemente, comunidades en Madre de Dios no han registrado caso
    por \emph{P. falciparum} en una década (2001-2012)
    (\protect\hyperlink{ref-rosas2016peru}{Rosas-Aguirre et al. 2016}).
    Con respecto a factores ambientales, en el año 1998 durante el
    fenómeno El Niño-Oscilación Sur (ENSO) se produjo el mayor pico de
    casos anuales en la historia (200,000 casos), donde la región del
    Norte costero representó el 48\% de los casos a nivel nacional
    (\protect\hyperlink{ref-gagnon2002enso}{Gagnon, Smoyer-Tomic, and
    Bush 2002}).
  \end{enumerate}
\item
  Biología

  \begin{enumerate}
  \def\labelenumii{\roman{enumii}.}
  \item
    \textbf{Ciclo de vida}

    La ecología natural del parásito de la malaria involucra a dos
    hospederos, el humano y el mosquito, y tres ciclos de vida:

    \begin{enumerate}
    \def\labelenumiii{\arabic{enumiii}.}
    \tightlist
    \item
      El \emph{ciclo hepático o exo-eritrocítico} inicia con la picadura
      del mosquito infectado, la liberación de los esporozoitos al
      fluido sanguíneo del humano y su ingreso a las células hepáticas.
      Aquí se da su desarrollo asexual donde se multiplican hasta formar
      esquizontes, los cuales egresan al torrente sanguíneo en la forma
      de merozoitos.
    \item
      El \emph{ciclo eritrocítico} se inicia con la invasión de glóbulos
      rojos (RBC) y el desarrollo consecutivo de trofozoitos inmaduros ,
      maduros y esquizontes, los cuales a su ruptura liberan nuevos
      merozoitos que reinfectan a más RBC. Por motivos aún no
      determinados, a partir de los trofozoitos inmaduros se inicia el
      desarrollo de gametocitos diferenciándose sexualmente dentro del
      torrente sanguíneo del humano.
    \item
      El \emph{ciclo esporogónico} o fase sexual se inicia en el
      mosquito hembra mediante la ingestión accidental de gametocitos en
      la alimentación sanguínea, con la intención inicial de proveer de
      nutrientes a sus huevos. En su tracto digestivo se generan cigotos
      mótiles que invaden las paredes para formar un oocisto donde los
      esporozoitos se desarrollan y replican hasta su ruptura y
      liberación. Finalmente migran a las glándulas salivares para
      continuar la transmisión en la siguiente alimentación del
      mosquito.
    \end{enumerate}
  \item
    \textbf{Particularidades}

    \emph{P. vivax} posee importantes variantes biológicas que
    caracterizan su epidemiología y dinámica de infección
    (\protect\hyperlink{ref-howes2016global}{Howes et al. 2016}). Tres
    de las más importantes son:

    \begin{enumerate}
    \def\labelenumiii{\arabic{enumiii}.}
    \tightlist
    \item
      Presencia de relapsos por la activación de hipnozoitos, el cual es
      un estado de latencia o dormacia en el \emph{ciclo hepático},
      condición que puede generar tanto una liberación prolongada y
      lenta de merozoitos como reservorios de infección que prolongan su
      transmisión;
    \item
      Tropismo hacia reticulocitos o RBC inmaduros (1-2\% de RBC
      circulantes) en el \emph{ciclo eritrocítico}, condición que genera
      bajas parasitemias en comparación a \emph{P. falciparum}; y
    \item
      Dependencia del antígeno \emph{Duffy} para la invasión de RBC,
      cuya ausencia en individuos con ancestría africana ha sido
      considerada como un factor de resistencia a \emph{P. vivax}.
    \end{enumerate}
  \end{enumerate}
\end{enumerate}

\hypertarget{malaria-severa}{%
\subsubsection{Malaria Severa}\label{malaria-severa}}

\begin{enumerate}
\def\labelenumi{\alph{enumi}.}
\item
  Definición

  Ante la ausencia de una descripción especie-específica, la malaria
  severa por \emph{P. vivax} está definida por los criterios para
  \emph{P. falciparum} otorgados por la OMS en el 2014
  (\protect\hyperlink{ref-WHO2014severe}{WHO 2014}), el cual incluye una
  o más de las siguientes (todas registradas en mono-infecciones por
  \emph{P. vivax}):

  \begin{enumerate}
  \def\labelenumii{\arabic{enumii}.}
  \tightlist
  \item
    Condición neurológica: coma, mareo, pérdida conciencia;
  \item
    Condición hematológica: anemia/trombocitopenia severa;
  \item
    Síntomas sistémicos: shock circulatorio; y
  \item
    Fallo de órganos vitales: disfución respiratoria, estrés
    respiratorio agudo, daño renal agudo, ruptura esplénica, disfunción
    hepática e ictericia (hiperbilirrubina).
  \end{enumerate}
\item
  Epidemiología

  \begin{enumerate}
  \def\labelenumii{\roman{enumii}.}
  \item
    \textbf{A nivel mundial}

    Entre el 2005 y 2015, 1-3\% de casos no-complicados fueron asumidos
    como malaria severa. Esta causó la muerte de 429,000 personas
    (\protect\hyperlink{ref-WHO2016world}{WHO 2016}), en su mayoría
    (90\%) niños menores de 5 años en África
    (\protect\hyperlink{ref-wassmer2015}{Wassmer et al. 2015}). La
    vulnerabilidad a esta manifestación ha sido asociada con la
    intensidad de transmisión y el desarrollo de la inmunidad
    dependiente a la edad (\protect\hyperlink{ref-reyburn2015}{Reyburn
    et al. 2005}).

    \begin{enumerate}
    \def\labelenumiii{\arabic{enumiii}.}
    \tightlist
    \item
      En zonas de \emph{alta transmisión} como en el África
      subsahariana, las poblaciones más vulnerables son: niños menores
      de 5 años con un desarrollo incompleto de inmunidad parcial contra
      la malaria (\protect\hyperlink{ref-Stanisic2015}{Stanisic et al.
      2015}), mujeres embarazadas en parte a la adhesión placentaria de
      glóbulos rojos infectados (iRBC)
      (\protect\hyperlink{ref-rogerson2007preg}{Rogerson et al. 2007}),
      y viajeros o migrantes sin inmunidad provenientes de áreas con
      baja o ninguna transmisión de malaria.
    \item
      En zonas de \emph{baja transmisión} como en Asia y América Latina,
      al haber una menor exposición a la infección, la mayoría de la
      población llega a la adultez sin haber desarrollado una inmunidad
      protectiva. Como consecuencia, la población adolescente y adulta
      joven es la más susceptible a desarrollar esta patología
      (\protect\hyperlink{ref-llanoschea2015}{Llanos-Chea et al. 2015}),
      comúnmente al iniciar trabajos a campo abierto, e.g.~actividades
      madereras o mineras, en zonas de alto riesgo de contacto con
      mosquitos infectados (\protect\hyperlink{ref-factores2001}{MINSA
      2001}).
    \end{enumerate}
  \item
    \textbf{En el Perú}

    En el Norte costero, Quispe et al.~2014
    (\protect\hyperlink{ref-quispe2014}{Antonio M. Quispe et al. 2014})
    mediante un estudio retrospectivo (2008-2009) identificaron 81/6502
    casos de malaria severa por \emph{P. vivax} con anemia severa
    (57\%), shock circulatorio (25\%), hiperbilirrubina (25\%), daño
    pulmonar (21\%), daño renal agudo (14\%) y Glasgow \(\le\) 9/14
    (11\%). Comparados con los pacientes no complicados, los severos
    fueron mayores (38 vs 26 años, P\textless0.001).

    En el Noreste amazónico, un reciente estudio prospectivo dirigido
    por El Centro de Enfermedades Tropicales de la Marina de los Estados
    Unidos NAMRU-6 (2012-2013)
    (\protect\hyperlink{ref-smith2013}{Smith-Nuñez et al. 2013})
    identificó 55/164 casos de malaria severa por \emph{P. vivax} con
    shock circulatorio (27.6\%), daño pulmonar (24.5\%) hiperbilirubina
    (18.1\%), daño renal agudo (2\%), Glasgow \(\le\) 9/14 (2\%) y
    anemia severa (1.7\%). No se encontraron diferencias con respecto a
    la edad entre no-severos y severos (33 vs 29 años, P=0.497). Sin
    embargo, sí se identificó una mayor proporción de malaria severa en
    mujeres (52 vs 35\%, P=0.039).
  \end{enumerate}
\end{enumerate}

\begin{enumerate}
\def\labelenumi{\alph{enumi}.}
\setcounter{enumi}{2}
\item
  Biología

  \begin{enumerate}
  \def\labelenumii{\roman{enumii}.}
  \item
    \textbf{Patogénesis}

    Se han propuesto mecanismos en base a lo observado en los casos por
    \emph{P. falciparum}, donde los procesos de invasión y adhesión
    parasitaria son relevantes. La respuesta celular contra la
    parasitemia, sumada a la citoadherencia de iRBC en estadio
    esquizonte a endotelio o RBC no infectados (proceso llamado
    ``rosetamiento''), desencadena una obstrucción microvascular que
    activa a las células endoteliales. En respuesta, estas liberan una
    mayor cantidad de citoquinas proinflamatorias, provocando la pérdida
    de perfusión y una consecuente disfunción microvascular que progresa
    por retroalimentación positiva.
  \item
    \textbf{Comparación}

    Históricamente, la malaria por \emph{P. vivax} ha sido considerada
    como ``benigna'' en comparación a \emph{P. falciparum} debido a su:
    (1) baja invasión parasitaria, sesgada a reticulocitos y rutas
    alternas de menor efectividad; y (2) pobre citoadhesión de sus iRBC,
    dada por la ausencia de protrusiones abastonadas o \emph{knob
    protrusions}, codificadas por genes \emph{var}. Sin embargo, la
    identificación de genes homólogos a \emph{Pf var} (\emph{Pv vir}) y
    evindencias \emph{post mortem} de iRBC en capilares pulmonares han
    sugerido la posibilidad de secuestramiento tisular por \emph{P.
    vivax} (\protect\hyperlink{ref-wassmer2015}{Wassmer et al. 2015}).
    En línea con esta evidencia, recientemente se ha demostrado que en
    los casos severos por \emph{P. vivax} la parasitemia periférica
    subestima la biomasa parasitaria total, siendo la fracción oculta la
    mayor contribuyente de citoquinas proinflamatorias, semejante a los
    casos por \emph{P. falciparum}
    (\protect\hyperlink{ref-barber2015}{Barber et al. 2015}).
  \end{enumerate}
\end{enumerate}

\hypertarget{respuesta-de-anticuerpos}{%
\subsubsection{Respuesta de
anticuerpos}\label{respuesta-de-anticuerpos}}

La concentración de anticuerpos contra antígenos de \emph{Plasmodium} es
un biomarcador sensible a la exposición de malaria
(\protect\hyperlink{ref-elliott2014}{Elliott et al. 2014}). Sin embargo,
las evidencias a favor de dianas para \emph{P. vivax} continúan estando
limitadas por la ausencia de cultivos \emph{in vitro} y modelos animales
de infección. Este problema se ve reflejado en su escaso número de
candidatos a vacuna (PvDBP y PvCSP) en comparación a los 23 de \emph{P.
falciparum} (\protect\hyperlink{ref-rainbow2016}{WHO, n.d.}). Por ello,
la principal fuente de evidencia a favor de biomarcadores para \emph{P.
vivax} está en la identificación de antígenos con respuestas de
anticuerpos asociadas a la adquisición natural de inmunidad en
poblaciones endémicas.

\begin{enumerate}
\def\labelenumi{\alph{enumi}.}
\item
  Marcadores

  \begin{enumerate}
  \def\labelenumii{\roman{enumii}.}
  \item
    \textbf{de exposición}

    Estudios longitudinales han sugerido que, en un inicio, la
    intensidad de la respuesta de anticuerpos actúa como marcador de
    exposición la cual, luego de una constante exposición, incrementa
    hasta superar un umbral de protección
    (\protect\hyperlink{ref-Stanisic2015}{Stanisic et al. 2015}). En
    este sentido, estudios seroepidemiológicos en \emph{P. vivax} han
    logrado asociar marcadores con exposición (PvCSP,
    PvMSP-1\textsubscript{19}, PvMSP-9\textsubscript{RIRII} y PvAMA-1) e
    inmunidad (PvMSP-1\textsubscript{19}, PvMSP-1\textsubscript{NT},
    PvMSP-3\(\alpha\) y PvMSP-9\textsubscript{NT})
    (\protect\hyperlink{ref-cutts2014meta}{Cutts et al. 2014}).
  \item
    \textbf{de inmunidad}

    En áreas de alta transmisión, los niños adquieren resistencia a la
    \emph{manifestación severa} de la malaria a la edad de cinco años,
    aproximadamente. Sin embargo, continúan siendo susceptibles a
    episodios no-complicados hasta la adolescencia, donde adquieren un
    estado resistente a la \emph{malaria sintomática}. A pesar de ello,
    no se ha demostrado que esta adquisición natural de inmunidad por la
    exposición acumulada a malaria otorgue una \emph{protección
    esterilizante} o de resistencia a la infección
    (\protect\hyperlink{ref-crompton2014rev}{Crompton et al. 2014}). Los
    detalles de los procesos involucrados en cada una de estas tres
    fases no están completamente entendidos, pero se tiene detalles de
    algunos componentes:

    \begin{enumerate}
    \def\labelenumiii{\arabic{enumiii}.}
    \item
      \textbf{contra la severidad}

      La adquisición de anticuerpos que bloqueen la adhesión de iRBC o
      invasión parasitaria son candidatos a otorgar una protección a
      este nivel (\protect\hyperlink{ref-wassmer2015}{Wassmer et al.
      2015}). Antígenos referenciales son (i) las proteínas VIR en
      \emph{P. vivax}, homólogas a VAR en \emph{P. falciparum},
      posiblemente secretadas en iRBC y causantes de adhesividad a
      endotelio o RBC no infectados, condición que desencadena el
      rosetamiento y posterior obstrucción microvascular
      (\protect\hyperlink{ref-portillo2001vir}{Portillo et al. 2001}), y
      (ii) las proteínas de la familia PvDBP y PvRBP (\emph{Duffy-} y
      \emph{Reticulocyte- binding proteins}), responsables de la
      invasión a reticulocitos por la ruta tradicional y alternativa,
      respectivamente (\protect\hyperlink{ref-galinski1992rbp}{Galinski
      et al. 1992}).
    \item
      \textbf{contra la enfermedad}

      En el caso de \emph{P. vivax}, su adquisición ocurre con una mayor
      rapidez, posiblemente facilitada por sus particularidades
      biológicas (\protect\hyperlink{ref-mueller2013}{Mueller et al.
      2013}). Los antígenos propuestos hasta el momento para esta
      especie son proteínas del micronema de merozoitos como DBP y AMA1,
      y proteínas de superficie como MSP1, MSP1-P (región C-terminal),
      MSP3 (PvMSP-3\(\alpha\), PvMSP-3\(\beta\)) y MSP9
      (\protect\hyperlink{ref-lopez2017}{López et al. 2017}).
    \item
      \textbf{contra la parasitemia}

      La adquisición de una inmunidad esterilizante solo ha sido
      comprobada mediante la inmunización con esporozoitos atenuados por
      radiación. Tanto en \emph{P. falciparum} como en \emph{P. vivax}
      se ha confirmado esta respuesta. Un reciente ensayo clínico de
      fase 2 en Colombia con \emph{P. vivax} encontró que luego de siete
      inmunizaciones a lo largo de 56 semanas y una reexposición con
      esporozoitos infecciosos, 5/12 voluntarios obtuvieron dicha
      protección y la respuesta de anticuerpos IgG1 anti-PvCSP estuvo
      asociada a ella
      (\protect\hyperlink{ref-arevalo2016spz}{Arévalo-Herrera et al.
      2016}).
    \end{enumerate}
  \end{enumerate}
\item
  Enfoques a larga escala

  El parásito de la malaria, al tener un ciclo de vida complejo con
  múltiples estadios de desarrollo en el humano, posee también un
  transcriptoma, proteoma y, por lo tanto, un \emph{inmunoma}
  estadio-específico que deriva de un total de \textasciitilde5300
  proteínas putativas. Por este motivo, tanto los esfuerzos de vacunas
  por subunidades como por organismos completos atenuados han fracasado,
  en contraste a su eficacia en otras infecciones como hepatitis B
  (HBsAg) y tuberculosis (BCG), respectivamente
  (\protect\hyperlink{ref-immunomics2016}{De Sousa and Doolan 2016}).

  \begin{enumerate}
  \def\labelenumii{\roman{enumii}.}
  \item
    \textbf{Inmunómica}

    En respuesta a las limitaciones de la \emph{vacunología reversa},
    basada en el uso inicial de la información genómica de patógenos, el
    enfoque \emph{inmunómico} hace uso de la información de ambos
    agentes interactuantes: patógeno y hospedero. Su estrategia
    selecciona inmunógenos tanto por (1) \emph{métodos empíricos},
    usando data transcriptómica y proteómica de patógenos dinámicos más
    la información clínica del hospedero, como por (2) \emph{métodos
    teóricos}, empleando algoritmos computacionales predictivos que
    tomen en cuenta la afinidad entre péptidos del patógeno y moléculas
    del complejo principal de histocompatibilidad (MHC) del hospedero.

    Particularmente, el primer método ha permitido obtener datos
    empíricos sobre diferencias en la amplitud, intensidad, cinética y
    longevidad de respuestas inmunes inducidas por patógenos. Además de
    mayor información sobre aspectos clave como el porcentaje del
    proteoma reconocido por el hospedero, la capacidad predictiva de
    anticuerpos sobre el estado de la enfermedad o de las secuencias de
    aminoácidos sobre su inmunogenicidad
    (\protect\hyperlink{ref-Davies2015Large}{Davies et al. 2015}).
  \end{enumerate}
\end{enumerate}

\hypertarget{microarreglos-de-proteuxednas}{%
\subsubsection{Microarreglos de
proteínas}\label{microarreglos-de-proteuxednas}}

Esta es una técnica empleada en la medición a larga escala de las
interacciones o actividades de proteínas capaz de reconstruir redes
biológicas entre diferentes clases de moléculas y estados celulares
(\protect\hyperlink{ref-uzoma2013interactome}{Uzoma and Zhu 2013}). Su
empleo en el contexto \emph{inmunómico}, con la cuantificación de las
interacciones antígeno-anticuerpo, posee características analíticas
similares a los microarreglos de ADN
(\protect\hyperlink{ref-sundaresh2006}{Sundaresh et al. 2006}), lo que
ha facilitado la adaptación de los componentes estandarizados para el
diseño de experimentos en esta tecnología
(\protect\hyperlink{ref-allison2006}{Allison et al. 2006}).

\begin{enumerate}
\def\labelenumi{\alph{enumi}.}
\item
  Componentes

  \begin{enumerate}
  \def\labelenumii{\roman{enumii}.}
  \item
    \textbf{Diseño}

    El desarrollo de un plan de experimentación tiene el objetivo de
    maximizar la calidad y cantidad de la información obtenida. Una de
    las secciones clave dentro del MIAME o ``información mínima sobre
    experimentos basados en microarreglos''
    (\protect\hyperlink{ref-brazma2001}{Brazma et al. 2001}) es el
    detalle de la información de los elementos a incluir (e.g.,
    secuencias de moléculas correspondiente a genes específicos). El
    primer diseño de microarreglos para \emph{P. vivax} y \emph{P.
    falciparum} fue estandarizado por Finney et al.~2014
    (\protect\hyperlink{ref-Finney2014}{Finney et al. 2014}) en base a
    la evidencia transcriptómica, proteómica y características
    inmunogénicas de proteínas predichas en ambos genomas.
    Posteriormente, en el 2015 el Centro Internacional de Excelencia
    para la Investigación de la Malaria (ICEMR) realizó una subselección
    empírica de este microarreglo con 142 muestras de 10 paises con
    malaria endémica, entre ellos Perú
    (\protect\hyperlink{ref-King2015FOC}{King et al. 2015}).
  \item
    \textbf{Pre procesamiento}

    Con el objetivo de remover variaciones sistemáticas en el
    experimento, se ejecutan en forma consecutiva los tres siguientes
    pasos:

    \begin{enumerate}
    \def\labelenumiii{\arabic{enumiii}.}
    \tightlist
    \item
      Normalización: Permite homogeneizar la variabilidad entre arreglos
      en base a controles, los cuales se asume que son invariantes entre
      muestras. Este procedimiento es ejecutado tradicionalmente a
      través de la sustracción de controles o \emph{fold over control}
      (FOC), obteniendo el cociente de las lecturas objetivo con
      respecto a la mediana de los controles negativos por muestra
      (\protect\hyperlink{ref-King2015FOC}{King et al. 2015});
    \item
      Transformación: Reduce la heterocedasticidad o heterogeneidad de
      varianzas entre las observaciones por errores aleatorios en el
      proceso de hibridación
      (\protect\hyperlink{ref-kreil2005bullet}{Kreil and Russell 2005}).
      Este fenómeno se evidencia en el incremento de la varianza
      directamente proporcional al incremento de su intensidad
      (\protect\hyperlink{ref-brown2001image}{Brown, Goodwin, and Sorger
      2001}). Dependiendo de la dispersión de los datos, se asumen
      modelos de error multiplicativo o aditivo, donde la transformación
      logarítmica o arcoseno hiperbólico es ejecutada sobre los valores
      normalizados.
    \item
      Filtrado: En experimentos a larga escala permite incrementar el
      poder de detección de elementos con expresión diferenciada
      (\protect\hyperlink{ref-bourgon2010filter}{Bourgon, Gentleman, and
      Huber 2010}). En estos, las estrategias de corrección posteriores
      a la comparación de múltiples hipótesis son sensibles a esta
      cantidad. En este sentido, este procedimiento retira de forma
      preliminar a los elementos con reducidas probabilidades de
      expresarse diferencialmente.
    \end{enumerate}
  \item
    \textbf{Inferencia}

    Su objetivo es poner a prueba hipótesis estadísticas relacionadas a
    la expression o reactividad diferencial de elementos entre grupos.
    Sin embargo, a esta escala se posee dos principales problemas: (1)
    presencia de varianzas artificialmente altas o bajas producto del
    bajo número de réplicas entre arreglos o muestras
    (\protect\hyperlink{ref-baldi2001cybert}{Baldi and Long 2001}) y (2)
    un amplio número de hipótesis puestas a prueba en forma simultánea,
    donde los test estadísticos tradicionales pueden generar un largo
    número de falsos positivos
    (\protect\hyperlink{ref-kayala2012cyber}{Kayala and Baldi 2012}).

    \begin{enumerate}
    \def\labelenumiii{\arabic{enumiii}.}
    \tightlist
    \item
      Test-t moderado: Un enfoque bayesiano empírico permite reducir (o
      moderar) las varianzas de todas las lecturas. Bajo el supuesto que
      las diferencias entre grupos generan cambios en la intensidad
      promedio del gen mas no en su varianza, se estima una varianza
      general como probabilidad previa para actualizar (corregir) todas
      las varianzas observadas en el experimento
      (\protect\hyperlink{ref-smyth2004ebayes}{Smyth and others 2004})
      Con ello, es posible obtener inferencias más estables que el
      test-t ordinario bajo un contexto limitado de réplicas
      (\protect\hyperlink{ref-kayala2012cyber}{Kayala and Baldi 2012}).
    \item
      Razón de falsos descubrimientos (FDR): Los métodos de corrección o
      ajuste de valores P permiten que el investigador pueda controlar
      la razón de falsos descubrimientos con respecto al total de
      hipótesis positivas (\protect\hyperlink{ref-brazma2001}{Brazma et
      al. 2001}). El método Benjamini-Hochberg determina un valor
      crítico dependiente del total de hipótesis puestas a prueba y el
      FDR que se desee tolerar, comúnmente del 5\%
      (\protect\hyperlink{ref-benjamini1995fdr}{Benjamini and Hochberg
      1995}).
    \end{enumerate}
  \item
    \textbf{Clasificación}

    Este proceso involucra la asignación de objetos (e.g., genes) en
    categorías pre-existentes (clasificación supervisada), o el
    desarrollo progresivo de un conjunto de categorías por
    características de los objetos. (clasificación no supervisada)
    (\protect\hyperlink{ref-allison2006}{Allison et al. 2006}). Ejemplos
    de esta técnica son \emph{support vector machines} (SVM) con
    validación por \emph{leave-one-out cross-validation} (LOOCV) y el
    agrupamiento jerárquico o \emph{hierarchical clustering} de acuerdo
    a la distancia Euclidiana entre elementos.
  \item
    \textbf{Parámetros adicionales}

    Además de la reactividad serológica, dos parámetros han demostrado
    ser de vital importancia para la caracterización de la respuesta
    humoral: la amplitud e intensidad de respuesta
    (\protect\hyperlink{ref-crompton2010}{Crompton et al. 2010};
    \protect\hyperlink{ref-Helb2015exposure}{Helb et al. 2015};
    \protect\hyperlink{ref-King2015FOC}{King et al. 2015}). El primero
    se obtiene calculando el número de antígenos reactivos por individuo
    y el segundo por el promedio de sus intensidades. Particularmente,
    la amplitud ha demostrado ser tan relevante como la reactividad de
    antígenos individuales en el desarrollo de inmunidad contra la
    malaria (\protect\hyperlink{ref-crompton2010}{Crompton et al.
    2010}).
  \end{enumerate}
\end{enumerate}

\hypertarget{definiciones-conceptuales}{%
\subsection{Definiciones conceptuales}\label{definiciones-conceptuales}}

\begin{enumerate}
\def\labelenumi{\alph{enumi}.}
\item
  Acrónimos

  \textbf{IVTT:} (Proteína) transcrita/traducida \emph{in vitro}.

  \textbf{MFI:} Unidad de lectura cruda expresada como intensidad
  fluorescente promedio de todos los píxeles de cada \emph{spot},
  normalizada localmente mediante la sustracción de la intensidad de
  fondo presente a su alrededor.

  \textbf{RTS:} Sistema rápido de traducción de proteínas recombinantes
  libre de células.
\item
  Términos

  \textbf{Antígeno-IVTT:} Antígeno objetivo o \emph{spot} con proteína
  IVTT a partir de un plásmido con ADN insertado correspondiente al
  polipéptido, segmento o exón de interés.

  \textbf{Control-IVTT:} Control negativo o \emph{spot} con mix de
  expresión RTS y plásmido sin ADN insertado, representante de la
  intensidad de fondo específica del paciente.

  \textbf{Proteína purificada:} Control de comparación o \emph{spot} con
  proteína de antigenicidad conocida expresada dentro de célula.

  \textbf{Intensidad de antígeno:} Lectura normalizada de cada
  antígeno-IVTT entre individuos. También llamada ``reactividad
  serológica.''

  \textbf{Antígeno reactivo:} Lectura transformada de cada antígeno-IVTT
  mayor o igual a dos veces la mediana de los control-IVTT por
  individuo.
\end{enumerate}

\hypertarget{meto}{%
\section{MATERIAL Y MÉTODOS}\label{meto}}

\hypertarget{material-bioluxf3gico-poblaciuxf3n-y-muestra}{%
\subsection{Material biológico (Población y
muestra)}\label{material-bioluxf3gico-poblaciuxf3n-y-muestra}}

\hypertarget{poblaciuxf3n}{%
\subsubsection{Población}\label{poblaciuxf3n}}

La población está delimitada por pacientes diagnosticados con malaria
por \emph{P. vivax} de la ciudad de Iquitos, Loreto - Perú, entre enero
del 2012 y junio del 2013 a lo largo de un estudio prospectivo ejecutado
en dos hospitales de referencia: Hospital de Apoyo y Hospital Regional.

\hypertarget{criterios-de-inclusiuxf3n-y-exclusiuxf3n}{%
\subsubsection{Criterios de inclusión y
exclusión}\label{criterios-de-inclusiuxf3n-y-exclusiuxf3n}}

\begin{enumerate}
\def\labelenumi{\alph{enumi}.}
\item
  Criterios de inclusión

  En el estudio prospectivo mencionado, 164 pacientes con malaria por
  \emph{P. vivax} fueron enrolados mediante vigilancia pasiva. Según la
  presencia de uno o más criterios clínicos y de laboratorio de la
  clasificación OMS para la Malaria Severa
  (\protect\hyperlink{ref-WHO2014severe}{WHO 2014}), los pacientes
  fueron estratificados como severos o no-severos.

  Los criterios clínicos de la clasificación OMS son: Shock circulatorio
  (presión sanguínea sistólica \textless{} 80 mmHg), Deterioro del nivel
  de conciencia (puntaje Glasgow \(\le\) 9/14), Daño del sistema
  nerviosos central (convulsión), Daño pulmonar (síndrome de dificultad
  respiratoria aguda o edema pulmonar).

  Los criterios de laboratorio de la clasificación OMS son: Hipoglicemia
  (glucosa \textless{} 40 mg/dL), Anemia severa (hemoglobina \textless{}
  7mg/dL), Daño renal (creatinina \textgreater{} 3mg/dl), e
  Hiperbilirrubina (bilirrubina sérica \textgreater{} 2.5 mg/dL).
\item
  Criterios de exclusión

  Dos criterios de exclusión fueron empleados: (i) Presencia de
  coinfecciones con especies de \emph{Plasmodium} distintas a \emph{P.
  vivax} determinadas por PCR, y (ii) presencia de coinfecciones como
  leptospirosis, dengue u otra arbovirosis determinada por técnicas de
  aislamiento viral e inmunofluorescencia.
\end{enumerate}

\hypertarget{selecciuxf3n-de-participantes}{%
\subsubsection{Selección de
participantes}\label{selecciuxf3n-de-participantes}}

A partir de esta estratificación, se ejecutará una selección arbitraria
de 30 pacientes representativos de malaria severa con uno o más
criterios de la clasificación OMS para el grupo de casos y una selección
aleatoria simple de 30 no-severos para el grupo control.

\begin{itemize}
\tightlist
\item
  Tipo de muestra: No probabilística para los casos y probabilística
  para los controles.\_\_ Los elementos del grupo caso provendrán de una
  selección arbitraria, mientras que el grupo control será seleccionado
  al azar, i.e.~con probabilidad de selección conocida.
\end{itemize}

\hypertarget{material-de-laboratorio}{%
\subsection{Material de laboratorio}\label{material-de-laboratorio}}

\hypertarget{instrumento-de-mediciuxf3n}{%
\subsubsection{Instrumento de
medición}\label{instrumento-de-mediciuxf3n}}

El presente estudio hará uso de microarreglos diseñados con 1014
fragmentos proteicos recombinantes (498 de \emph{P. falciparum} y 516 de
\emph{P. vivax}, expresado como Pf498/Pv516 o PfPv500) representando 873
proteínas predichas (427 de \emph{P. falciparum} y 446 de \emph{P.
vivax}, \textasciitilde8\% del total predicho para el genoma de \emph{P.
vivax} Sal1) seleccionadas luego de una extensiva evaluación serológica
con uno de mayor escala (Pf2208/Pv2233)
(\protect\hyperlink{ref-Finney2014}{Finney et al. 2014}). Este diseño ha
sido validado mediante el sondeo con plasma colectado de pacientes con
malaria y controles sanos alrededor del mundo
(\protect\hyperlink{ref-King2015FOC}{King et al. 2015}) y depositado en
la base de datos GEO
(\href{https://www.ncbi.nlm.nih.gov/geo/query/acc.cgi?acc=GPL18316}{GPL18316}).

\hypertarget{aplicaciuxf3n}{%
\paragraph{Aplicación}\label{aplicaciuxf3n}}

La aplicación del instrumento consiste en tres pasos: sondeo, escaneo y
análisis, tal como ha sido publicado previamente
(\protect\hyperlink{ref-Driguez2015}{Driguez et al. 2015}). Primero,
previo al sondeo los microarreglos se hidratan con buffer de bloqueo
dentro de cámaras de incubación. Paralelamente, el plasma de los
pacientes se diluye con aquel buffer en 1:100 y se pre-absorbe con
lisado de \emph{E. coli} en 10\%(w/v), con el objetivo de reducir ruido
de fondo en las mediciones tanto por uniones inespecíficas con el
sustrato como por antígenos bacterianos del sistema de expresión RTS.
Segundo, aspirado el buffer de bloqueo del microarreglo, se procede con
el sondeo al agregar el plasma pre-absorbido e incubar \emph{overnight}
en cámara húmeda a 4°C con leve agitación. Tercero, con lavados y
aspirados repetitivos antes y después de cada paso, se agrega la
solución con anticuerpos secundarios biotinilados y posteriormente la
solución con fluoróforos conjugados con estreptavidina. Cuarto, luego de
centrifugar las láminas, se procede al escaneo de las fluorescencias con
lectores de microarreglos de láser confocal (e.g., Genepix 4300A). Las
medidas crudas se obtienen al aplicar una normalización local mediante
la sustracción de la intensidad de fondo presente alrededor de cada
\emph{spot}, ejecutada en el software del lector (Genepix Pro 7).
Finalmente, se procede al análisis de datos detallado en la
\protect\hyperlink{anadata}{sección 4.4}.

\hypertarget{validez}{%
\paragraph{Validez}\label{validez}}

La validez o exactitud del experimento será evaluada mediante la
correlación lineal entre las lecturas de los antígenos-IVTT y sus
correspondientes proteínas purificadas por muestra, compuestas por
proteínas de inmunogenicidad conocida y expresadas por un sistema dentro
de célula (\protect\hyperlink{ref-crompton2010}{Crompton et al. 2010}).

\hypertarget{adquisiciuxf3n}{%
\paragraph{Adquisición}\label{adquisiciuxf3n}}

Todos los reactivos, incluidos los microarreglos, anticuerpos
secundarios biotinilados y fluoróforos conjugados con estreptavidina
serán adquiridos a la empresa Antigen Discovery, INC
(\protect\hyperlink{ref-adiinc}{Liang et al. 2004}).

\hypertarget{muxe9todos}{%
\subsection{Métodos:}\label{muxe9todos}}

\hypertarget{diseuxf1o}{%
\subsubsection{Diseño}\label{diseuxf1o}}

El diseño del estudio es de tipo Caso-Control. Su aplicación consistirá
en la selección de la población objetivo por presencia (casos) o
ausencia (controles) del evento en estudio. Además se fijará el número
de eventos a estudiar, así como el número de sujetos sin evento que se
incluirán en la población de comparación.

\hypertarget{tipo-de-investigaciuxf3n}{%
\paragraph{Tipo de investigación}\label{tipo-de-investigaciuxf3n}}

\begin{itemize}
\item
  Por la finalidad: Analítico. A diferencia de un descriptivo,
  realizaremos comparaciones entre grupos y evaluaremos una posible
  relación causal entre el factor y el efecto.
\item
  Por el control de la asignación: Observacional. A diferencia de un
  experimental, no controlaremos la asignación de los factores
  (severidad y episodios previos). Es decir, procederemos a observar el
  fenómeno, ejecutar la medición y analizar los resultados.
\item
  Por el seguimiento: Transversal. A diferencia de uno longitudinal, no
  ejecutaremos un seguimiento. Las variables se medirán una sola vez, en
  un mismo instante.
\item
  Por la relación cronológica: Prospectivo. A diferencia de un
  retrospectivo, recolectaremos los datos después de planificado el
  estudio.
\end{itemize}

\hypertarget{recolecciuxf3n-de-los-datos-e-instrumento}{%
\subsubsection{Recolección de los datos e
instrumento}\label{recolecciuxf3n-de-los-datos-e-instrumento}}

\hypertarget{tuxe9cnica-para-la-recolecciuxf3n-de-datos}{%
\paragraph{Técnica para la recolección de
datos}\label{tuxe9cnica-para-la-recolecciuxf3n-de-datos}}

Previo al tratamiento antimalárico del paciente, se extrajeron las
muestras de sangre tanto para el diagnóstico de malaria por \emph{P.
vivax} con la técnica de frotis como para las pruebas bioquímicas. Al
momento del diagnóstico positivo, bajo consentimiento informado y de
forma voluntaria, el plasma sanguíneo fue colectado y conservado a -80°C
hasta su uso.

\hypertarget{codificaciuxf3n-y-creaciuxf3n-del-archivo-de-datos}{%
\subsubsection{Codificación y creación del archivo de
datos}\label{codificaciuxf3n-y-creaciuxf3n-del-archivo-de-datos}}

Cada paciente del estudio estará identificado con un código de
estructura \texttt{LIM\#\#\#\#}, e.g.: \texttt{LIM1063}. Los antígenos
proteicos estarán identificados con el código asignado a sus genes en la
base de datos PlasmoDB, e.g.: \texttt{PF3D7\_0202500} o
\texttt{PVX\_091315}, ya sea un gen de \emph{P. falciparum} o \emph{P.
vivax}, respectivamente. En caso los genes posean múltiples exónes, se
amplificarán por separado y se extenderá el código de cada uno con su
número de orden y el total, e.g.: \texttt{\_1o2} exón 1 de un gen con 2
exónes. En caso los genes posean una longitud mayor a 3000 nucleótidos
(nt), se dividirán en segmentos sobrelapantes entre 300-3000nt y se
extenderá el código de cada uno con su respectivo número, e.g.:
\texttt{\_S1} para el primer segmento de un gen.

Los datos serán organizados en dos matrices: (i) archivo
\texttt{samples.csv} con los códigos de los pacientes y sus covariables
epidemiológicas y (ii) archivo \texttt{RawData.csv} con los códigos de
las proteínas y sus lecturas crudas en MFI por código de paciente.

\hypertarget{anadata}{%
\subsubsection{Análisis de datos}\label{anadata}}

Se hará uso del software de computación estadística R
(\protect\hyperlink{ref-R2016}{R Core Team 2016}), complementado con
funciones de distintos paquetes pertenecientes a dos principales
ambientes de análisis: Bioconductor
(\protect\hyperlink{ref-bioconductor2004}{Gentleman et al. 2004}) y
Tidyverse (\protect\hyperlink{ref-wickham2016r4ds}{Wickham and Grolemund
2016}).

\hypertarget{control-de-calidad-y-pre-procesamiento}{%
\paragraph{Control de calidad y pre
procesamiento}\label{control-de-calidad-y-pre-procesamiento}}

Preliminarmente, se describirá la distribución y proporción de las
covariables epidemiológicas, clínicas y bioquímicas de los pacientes de
la muestra. Luego, con las lecturas crudas en MFI se evaluará la validez
del ensayo mediante un test de asociación entre variables continuas con
la correlación de Pearson (\(r\)) o Spearman (\(\rho\)), de acuerdo a la
distribución, asumiendo un error del 5\%. Posteriormente, se procederá
con la normalización entre muestras de cada antígeno-IVTT con respecto a
la mediana de los control-IVTT por individuo, transformación a escala
logarítmica en base 2, y filtrado de todo antígeno que posea una
frecuencia de reactividad menor al 10\% de individuos en la muestra.
Para ello, un antígeno reactivo será definido operacionalmente como el
antígeno-IVTT con intensidad normalizada mayor o igual a 1. Por último,
se asociarán ambas matrices de datos en un \texttt{ExpressionSet}, a
través de los códigos de pacientes, empleando el paquete
\texttt{Biobase} (\protect\hyperlink{ref-Biobase}{Huber et al. 2015}).

\hypertarget{prueba-de-hipuxf3tesis}{%
\paragraph{Prueba de hipótesis}\label{prueba-de-hipuxf3tesis}}

Las dos hipótesis de diferencia entre grupos seguirán el siguiente
protocolo. Primero, se contrastará la amplitud e intensidad de respuesta
con un test de diferencias entre variables continuas y no pareadas de
dos grupos usando t-Student o Mann-Whitney, dependiendo de la
distribución, con un 5\% de error asumido. Segundo, se realizará un test
de reactividad diferenciada de anticuerpos entre dos grupos, usando el
test-t moderado (\protect\hyperlink{ref-smyth2004ebayes}{Smyth and
others 2004}) con corrección por comparación múltiple de la razón de
falsos descubrimientos (FDR) por el método de Benjamini-Hochberg,
disponible en el paquete \texttt{limma}
(\protect\hyperlink{ref-limma}{Ritchie et al. 2015}), con un 5\% de
error asumido. Tercero, se realizará un agrupamiento jerárquico o
\emph{hierarchical clustering} de los antígenos identificados en base a
su distancia euclidiana, disponible en el paquete \texttt{NMF}
(\protect\hyperlink{ref-Gaujoux2010NMF}{Gaujoux and Seoighe 2010}).
Finalmente, se describirá la presencia de dominios transmembrana,
péptido señal, número de ortólogos en \emph{Plasmodium}, ontología
génica, número de polimorfismos de nucleótido simple y la razón de
mutaciones no-sinónimas por sinónimas para el subgrupo de interés,
disponible en la base de datos PlasmoDB
(\protect\hyperlink{ref-plasmodb}{Aurrecoechea et al. 2009}).

\hypertarget{reporte-de-resultados}{%
\paragraph{Reporte de resultados}\label{reporte-de-resultados}}

Las correlaciones y pruebas de hipótesis serán reportadas con el valor
del estadístico de prueba y el valor P. Las distribuciones serán
visualizadas con diagramas de dispersión para la correlación entre
variables continuas, diagramas de cajas y barras para la comparación de
variables continuas y frecuencias, respectivamente. Las comparaciones
múltiples reportarán el valor P-ajustado y serán visualizadas con dos
gráficos: un diagrama tipo volcán o \emph{volcano plot} con el tamaño
del efecto o diferencia de reactividad entre grupos en escala
logarítmica contra sus respectivos valores P y un mapa de calor o
\emph{heat map} segmentado por racimos o \emph{clusters}. El informe
final consistirá de un reporte en formato \texttt{R\ Notebook} con
extensión \texttt{.Rmd} que integrará texto, código y resultados
(\protect\hyperlink{ref-knitr}{Xie 2013}), siguiendo principios de
reproducibilidad
(\protect\hyperlink{ref-CienciaReproducible2016}{Rodrı́guez-Sanchez et
al. 2016}).

\hypertarget{cronograma-de-trabajo}{%
\section{CRONOGRAMA DE TRABAJO}\label{cronograma-de-trabajo}}

El Proyecto de Tesis fue trabajado en el Departamento de Parasitología
del Centro de Enfermedades Tropicales de la Marina de los Estados Unidos
NAMRU-6, del 01 de enero del 2016 al 31 de diciembre del 2016, de
acuerdo al cronograma de la \autoref{tab:crono}.

\begin{table}[ht]
        \captionof{table}{Cronograma}
        \label{tab:crono}
        \vspace{-1mm}
\begin{center}
%\hspace*{-1cm}
\begin{tabular}{lcccccccccccc}
  \hline
  \textbf{ACTIVIDAD} & 
  \textbf{2016} & & & & & & & & & & &\\
  \hline
  & 
  ene & feb & mar & abr & may & jun & jul & ago & set & oct & nov & dic\\
  \cline{2-13}
  Recolección de datos & 
  x & & & & & & & & & & &\\
  \cline{2-13}
  Análisis de datos & 
  & x & x & x & x & x & & & & & &\\
  \cline{2-13}
  Interpretación de resultados & 
  & & & & & x & x & x & & & &\\
  %\cline{2-13}
  %Redacción de proyecto & 
  %& & & & & & & & & & &\\
  %\cline{2-13}
  %Redacción del Informe & 
  %& & & & & & & & & & &\\
  \cline{2-13}
  Redacción de discusión & 
  & & & & & & & x & x & x & &\\
  \cline{2-13}
  Redacción de conclusiones & 
  & & & & & & & & & x & x & x\\
  %\cline{2-13}
  %Correcciones & 
  %& & & & & & & & & & &\\
  %\cline{2-13}
  %Sustentación & 
  %& & & & & & & & & & &\\
  %\hline
  %d1 & 
  %& & & & & & & & & & & & 
  %& & & & & & & & & & &\\
  \hline
  % etc. ...
\end{tabular}
%\hspace*{-1cm}
\end{center}
\end{table}

\hypertarget{presupuesto-global}{%
\section{PRESUPUESTO GLOBAL}\label{presupuesto-global}}

El presupuesto detallado en la \autoref{tab:presup} fue ejecutado con
financiamiento interno de la institución.

\begin{longtable}[]{@{}lcc@{}}
\caption{Presupuesto \label{tab:presup}}\tabularnewline
\toprule
\begin{minipage}[b]{(\columnwidth - 2\tabcolsep) * \real{0.48}}\raggedright
\textbf{DESCRIPCIÓN}\strut
\end{minipage} &
\begin{minipage}[b]{(\columnwidth - 2\tabcolsep) * \real{0.25}}\centering
\textbf{MONTO (S/)}\strut
\end{minipage} &
\begin{minipage}[b]{(\columnwidth - 2\tabcolsep) * \real{0.25}}\centering
\textbf{PORCENTAJE (\%)}\strut
\end{minipage}\tabularnewline
\midrule
\endfirsthead
\toprule
\begin{minipage}[b]{(\columnwidth - 2\tabcolsep) * \real{0.48}}\raggedright
\textbf{DESCRIPCIÓN}\strut
\end{minipage} &
\begin{minipage}[b]{(\columnwidth - 2\tabcolsep) * \real{0.25}}\centering
\textbf{MONTO (S/)}\strut
\end{minipage} &
\begin{minipage}[b]{(\columnwidth - 2\tabcolsep) * \real{0.25}}\centering
\textbf{PORCENTAJE (\%)}\strut
\end{minipage}\tabularnewline
\midrule
\endhead
\begin{minipage}[t]{(\columnwidth - 2\tabcolsep) * \real{0.48}}\raggedright
\textbf{Bienes}\strut
\end{minipage} &
\begin{minipage}[t]{(\columnwidth - 2\tabcolsep) * \real{0.25}}\centering
\strut
\end{minipage} &
\begin{minipage}[t]{(\columnwidth - 2\tabcolsep) * \real{0.25}}\centering
\strut
\end{minipage}\tabularnewline
\begin{minipage}[t]{(\columnwidth - 2\tabcolsep) * \real{0.48}}\raggedright
Papelería, útiles y material de oficina\strut
\end{minipage} &
\begin{minipage}[t]{(\columnwidth - 2\tabcolsep) * \real{0.25}}\centering
50\strut
\end{minipage} &
\begin{minipage}[t]{(\columnwidth - 2\tabcolsep) * \real{0.25}}\centering
0.2\strut
\end{minipage}\tabularnewline
\begin{minipage}[t]{(\columnwidth - 2\tabcolsep) * \real{0.48}}\raggedright
Insumos e instrumental de laboratorio\strut
\end{minipage} &
\begin{minipage}[t]{(\columnwidth - 2\tabcolsep) * \real{0.25}}\centering
200\strut
\end{minipage} &
\begin{minipage}[t]{(\columnwidth - 2\tabcolsep) * \real{0.25}}\centering
0.6\strut
\end{minipage}\tabularnewline
\begin{minipage}[t]{(\columnwidth - 2\tabcolsep) * \real{0.48}}\raggedright
\textbf{Servicios}\strut
\end{minipage} &
\begin{minipage}[t]{(\columnwidth - 2\tabcolsep) * \real{0.25}}\centering
\strut
\end{minipage} &
\begin{minipage}[t]{(\columnwidth - 2\tabcolsep) * \real{0.25}}\centering
\strut
\end{minipage}\tabularnewline
\begin{minipage}[t]{(\columnwidth - 2\tabcolsep) * \real{0.48}}\raggedright
Compra, sondeo y lectura de microarreglos\strut
\end{minipage} &
\begin{minipage}[t]{(\columnwidth - 2\tabcolsep) * \real{0.25}}\centering
29610\strut
\end{minipage} &
\begin{minipage}[t]{(\columnwidth - 2\tabcolsep) * \real{0.25}}\centering
92.9\strut
\end{minipage}\tabularnewline
\begin{minipage}[t]{(\columnwidth - 2\tabcolsep) * \real{0.48}}\raggedright
Gastos en el transporte de muestras\strut
\end{minipage} &
\begin{minipage}[t]{(\columnwidth - 2\tabcolsep) * \real{0.25}}\centering
2000\strut
\end{minipage} &
\begin{minipage}[t]{(\columnwidth - 2\tabcolsep) * \real{0.25}}\centering
6.3\strut
\end{minipage}\tabularnewline
\begin{minipage}[t]{(\columnwidth - 2\tabcolsep) * \real{0.48}}\raggedright
\textbf{TOTAL}\strut
\end{minipage} &
\begin{minipage}[t]{(\columnwidth - 2\tabcolsep) * \real{0.25}}\centering
31910\strut
\end{minipage} &
\begin{minipage}[t]{(\columnwidth - 2\tabcolsep) * \real{0.25}}\centering
100\strut
\end{minipage}\tabularnewline
\bottomrule
\end{longtable}

\hypertarget{referencias-biobliogruxe1ficas}{%
\section{REFERENCIAS
BIOBLIOGRÁFICAS}\label{referencias-biobliogruxe1ficas}}

\hypertarget{refs}{}
\begin{CSLReferences}{1}{0}
\leavevmode\hypertarget{ref-allison2006}{}%
Allison, David B, Xiangqin Cui, Grier P Page, and Mahyar Sabripour.
2006. {``Microarray Data Analysis: From Disarray to Consolidation and
Consensus.''} \emph{Nature Reviews Genetics} 7 (1): 55--65.
\url{https://doi.org/10.1038/nrg1749}.

\leavevmode\hypertarget{ref-arevalo2014}{}%
Arévalo-Herrera, Myriam, David A Forero-Peña, Kelly Rubiano, José
Gómez-Hincapie, Nora L Martı́nez, Mary Lopez-Perez, Angélica Castellanos,
et al. 2014. {``Plasmodium Vivax Sporozoite Challenge in Malaria-Naive
and Semi-Immune Colombian Volunteers.''} \emph{PLoS One} 9 (6): 1--12.
\url{https://doi.org/10.1371/journal.pone.0099754}.

\leavevmode\hypertarget{ref-arevalo2016spz}{}%
Arévalo-Herrera, Myriam, Juan M Vásquez-Jiménez, Mary Lopez-Perez,
Andrés F Vallejo, Andrés B Amado-Garavito, Nora Céspedes, Angélica
Castellanos, et al. 2016. {``Protective Efficacy of Plasmodium Vivax
Radiation-Attenuated Sporozoites in Colombian Volunteers: A Randomized
Controlled Trial.''} \emph{PLoS Negl Trop Dis} 10 (10).
\url{https://doi.org/10.1371/journal.pntd.0005070}.

\leavevmode\hypertarget{ref-plasmodb}{}%
Aurrecoechea, Cristina, John Brestelli, Brian P Brunk, Jennifer Dommer,
Steve Fischer, Bindu Gajria, Xin Gao, et al. 2009. {``PlasmoDB: A
Functional Genomic Database for Malaria Parasites.''} \emph{Nucleic
Acids Research} 37 (suppl 1): D539--43.
\url{https://doi.org/10.1093/nar/gkn814}.

\leavevmode\hypertarget{ref-baird2009}{}%
Baird, J Kevin. 2009. {``Severe and Fatal Vivax Malaria Challenges
'Benign Tertian Malaria' Dogma.''} \emph{Annals of Tropical Paediatrics}
29 (4): 251--52. \url{https://doi.org/10.1179/027249309X12547917868808}.

\leavevmode\hypertarget{ref-baldevi2013}{}%
Baldeviano, G. Christian, Karina P. Leiva, Antonio M. Quispe, Julio
Ventocilla, L. Lorena Tapia, Salomon Durand, Meddly L. Santolalla, et
al. 2013. {``Serum Markers of Severe Clinical Complications During
Plasmodium Vivax Malaria Monoinfections in the {Peruvian Amazon}
Basin.''} In \emph{Abstract Book of the ASTMH 62nd Annual Meeting, Nov.
13--17, Washington {D.C.}, United States}, 340. 1120.
\url{http://www.astmh.org/ASTMH/media/Documents/AbstractBook2013Final.pdf}.

\leavevmode\hypertarget{ref-baldi2001cybert}{}%
Baldi, Pierre, and Anthony D Long. 2001. {``A Bayesian Framework for the
Analysis of Microarray Expression Data: Regularized t-Test and
Statistical Inferences of Gene Changes.''} \emph{Bioinformatics} 17 (6):
509--19. \url{https://doi.org/10.1093/bioinformatics/17.6.509}.

\leavevmode\hypertarget{ref-barber2015}{}%
Barber, Bridget E, Timothy William, Matthew J Grigg, Uma Parameswaran,
Kim A Piera, Ric N Price, Tsin W Yeo, and Nicholas M Anstey. 2015.
{``Parasite Biomass-Related Inflammation, Endothelial Activation,
Microvascular Dysfunction and Disease Severity in Vivax Malaria.''}
\emph{PLoS Pathog} 11 (1): 1--13.
\url{https://doi.org/10.1371/journal.ppat.1004558}.

\leavevmode\hypertarget{ref-benjamini1995fdr}{}%
Benjamini, Yoav, and Yosef Hochberg. 1995. {``Controlling the False
Discovery Rate: A Practical and Powerful Approach to Multiple
Testing.''} \emph{Journal of the Royal Statistical Society. Series B
(Methodological)}, 289--300. \url{https://doi.org/10.2307/2346101}.

\leavevmode\hypertarget{ref-bourgon2010filter}{}%
Bourgon, Richard, Robert Gentleman, and Wolfgang Huber. 2010.
{``Independent Filtering Increases Detection Power for High-Throughput
Experiments.''} \emph{Proceedings of the National Academy of Sciences}
107 (21): 9546--51. \url{https://doi.org/10.1073/pnas.0914005107}.

\leavevmode\hypertarget{ref-brazma2001}{}%
Brazma, Alvis, Pascal Hingamp, John Quackenbush, Gavin Sherlock, Paul
Spellman, Chris Stoeckert, John Aach, et al. 2001. {``Minimum
Information about a Microarray Experiment ({MIAME})---Toward Standards
for Microarray Data.''} \emph{Nature Genetics} 29 (4): 365--71.
\url{https://doi.org/10.1038/ng1201-365}.

\leavevmode\hypertarget{ref-brown2001image}{}%
Brown, Carl S, Paul C Goodwin, and Peter K Sorger. 2001. {``Image
Metrics in the Statistical Analysis of DNA Microarray Data.''}
\emph{Proceedings of the National Academy of Sciences} 98 (16):
8944--49. \url{https://doi.org/10.1073/pnas.161242998}.

\leavevmode\hypertarget{ref-leroch2009postmod}{}%
Chung, Duk-Won Doug, Nadia Ponts, Serena Cervantes, and Karine G Le
Roch. 2009. {``Post-Translational Modifications in Plasmodium: More Than
You Think!''} \emph{Molecular and Biochemical Parasitology} 168 (2):
123--34. \url{https://doi.org/10.1016/j.molbiopara.2009.08.001}.

\leavevmode\hypertarget{ref-chuquiyauri2015vivax}{}%
Chuquiyauri, Raul, Douglas M Molina, Eli L Moss, Ruobing Wang, Malcolm J
Gardner, Kimberly C Brouwer, Sonia Torres, et al. 2015. {``Genome-Scale
Protein Microarray Comparison of Human Antibody Responses in Plasmodium
Vivax Relapse and Reinfection.''} \emph{The American Journal of Tropical
Medicine and Hygiene} 93 (4): 801--9.
\url{https://doi.org/10.4269/ajtmh.15-0232}.

\leavevmode\hypertarget{ref-crompton2010}{}%
Crompton, Peter D, Matthew A Kayala, Boubacar Traore, Kassoum Kayentao,
Aissata Ongoiba, Greta E Weiss, Douglas M Molina, et al. 2010. {``A
Prospective Analysis of the Ab Response to Plasmodium Falciparum Before
and After a Malaria Season by Protein Microarray.''} \emph{Proceedings
of the National Academy of Sciences} 107 (15): 6958--63.
\url{https://doi.org/10.1073/pnas.1001323107}.

\leavevmode\hypertarget{ref-crompton2014rev}{}%
Crompton, Peter D, Jacqueline Moebius, Silvia Portugal, Michael
Waisberg, Geoffrey Hart, Lindsey S Garver, Louis H Miller, Carolina
Barillas-Mury, and Susan K Pierce. 2014. {``Malaria Immunity in Man and
Mosquito: Insights into Unsolved Mysteries of a Deadly Infectious
Disease.''} \emph{Annual Review of Immunology} 32: 157--87.
\url{https://doi.org/10.1146/annurev-iy-32-060414-200001}.

\leavevmode\hypertarget{ref-cutts2014meta}{}%
Cutts, Julia C, Rosanna Powell, Paul A Agius, James G Beeson, Julie A
Simpson, and Freya JI Fowkes. 2014. {``Immunological Markers of
Plasmodium Vivax Exposure and Immunity: A Systematic Review and
Meta-Analysis.''} \emph{BMC Medicine} 12 (1): 150.
\url{https://doi.org/10.1186/s12916-014-0150-1}.

\leavevmode\hypertarget{ref-Davies2015Large}{}%
Davies, D. Huw, Patrick Duffy, Jean-Luc Bodmer, Philip L. Felgner, and
Denise L. Doolan. 2015. {``Large Screen Approaches to Identify Novel
Malaria Vaccine Candidates.''} \emph{Vaccine} 33 (52): 7496--7505.
\url{https://doi.org/10.1016/j.vaccine.2015.09.059}.

\leavevmode\hypertarget{ref-immunomics2016}{}%
De Sousa, Karina P, and Denise L Doolan. 2016. {``Immunomics: A 21st
Century Approach to Vaccine Development for Complex Pathogens.''}
\emph{Parasitology} 143 (02): 236--44.
\url{https://doi.org/10.1017/S0031182015001079}.

\leavevmode\hypertarget{ref-Driguez2015}{}%
Driguez, Patrick, Denise L Doolan, Douglas M Molina, Alex Loukas, Angela
Trieu, Phil L Felgner, and Donald P McManus. 2015. {``Protein
Microarrays for Parasite Antigen Discovery.''} \emph{Parasite Genomics
Protocols}, 221--33. \url{https://doi.org/10.1007/978-1-4939-1438-8_13}.

\leavevmode\hypertarget{ref-elliott2014}{}%
Elliott, Salenna R, FJ Fowkes, Jack S Richards, Linda Reiling, Damien R
Drew, and James G Beeson. 2014. {``Research Priorities for the
Development and Implementation of Serological Tools for Malaria
Surveillance.''} \emph{F1000Prime Rep} 6: 100.
\url{https://doi.org/10.12703/P6-100}.

\leavevmode\hypertarget{ref-Finney2014}{}%
Finney, Olivia C., Samuel A. Danziger, Douglas M. Molina, Marissa
Vignali, Aki Takagi, Ming Ji, Danielle I. Stanisic, et al. 2014.
{``Predicting Antidisease Immunity Using Proteome Arrays and Sera from
Children Naturally Exposed to Malaria.''} \emph{Molecular \& Cellular
Proteomics} 13 (10): 2646--60.
\url{https://doi.org/10.1074/mcp.M113.036632}.

\leavevmode\hypertarget{ref-gagnon2002enso}{}%
Gagnon, Alexandre S, Karen E Smoyer-Tomic, and Andrew B Bush. 2002.
{``The {El Ni{ñ}o} Southern Oscillation and Malaria Epidemics in {South
America}.''} \emph{International Journal of Biometeorology} 46 (2):
81--89. \url{https://doi.org/10.1007/s00484-001-0119-6}.

\leavevmode\hypertarget{ref-galinski1992rbp}{}%
Galinski, Mary R, Claudia Corredor Medina, Paul Ingravallo, and John W
Barnwell. 1992. {``A Reticulocyte-Binding Protein Complex of Plasmodium
Vivax Merozoites.''} \emph{Cell} 69 (7): 1213--26.
\url{https://doi.org/10.1016/0092-8674(92)90642-P}.

\leavevmode\hypertarget{ref-Gaujoux2010NMF}{}%
Gaujoux, Renaud, and Cathal Seoighe. 2010. {``A Flexible {R} Package for
Nonnegative Matrix Factorization.''} \emph{{BMC} Bioinformatics} 11 (1):
367. \url{https://doi.org/10.1186/1471-2105-11-367}.

\leavevmode\hypertarget{ref-bioconductor2004}{}%
Gentleman, Robert C, Vincent J Carey, Douglas M Bates, Ben Bolstad,
Marcel Dettling, Sandrine Dudoit, Byron Ellis, et al. 2004.
{``Bioconductor: Open Software Development for Computational Biology and
Bioinformatics.''} \emph{Genome Biology} 5 (10): R80.
\url{https://doi.org/10.1186/gb-2004-5-10-r80}.

\leavevmode\hypertarget{ref-griffing2013history}{}%
Griffing, Sean M, Dionicia Gamboa, and Venkatachalam Udhayakumar. 2013.
{``The History of 20 Th Century Malaria Control in {Peru}.''}
\emph{Malaria Journal} 12 (1): 303.
\url{https://doi.org/10.1186/1475-2875-12-303}.

\leavevmode\hypertarget{ref-Helb2015exposure}{}%
Helb, Danica A., Kevin K. A. Tetteh, Philip L. Felgner, Jeff Skinner,
Alan Hubbard, Emmanuel Arinaitwe, Harriet Mayanja-Kizza, et al. 2015.
{``Novel Serologic Biomarkers Provide Accurate Estimates of Recent
Plasmodium Falciparum Exposure for Individuals and Communities.''}
\emph{Proceedings of the National Academy of Sciences} 112 (32):
E4438--47. \url{https://doi.org/10.1073/pnas.1501705112}.

\leavevmode\hypertarget{ref-howes2016global}{}%
Howes, Rosalind E, Katherine E Battle, Kamini N Mendis, David L Smith,
Richard E Cibulskis, J Kevin Baird, and Simon I Hay. 2016. {``Global
Epidemiology of Plasmodium Vivax.''} \emph{The American Journal of
Tropical Medicine and Hygiene} 95 (6 Suppl): 15--34.
\url{https://doi.org/10.4269/ajtmh.16-0141}.

\leavevmode\hypertarget{ref-Biobase}{}%
Huber, W., Carey, V. J., Gentleman, R., Anders, et al. 2015.
{``{O}rchestrating High-Throughput Genomic Analysis with
{B}ioconductor.''} \emph{Nature Methods} 12 (2): 115--21.
\url{https://doi.org/10.1038/nmeth.3252}.

\leavevmode\hypertarget{ref-kayala2012cyber}{}%
Kayala, Matthew A, and Pierre Baldi. 2012. {``Cyber-t Web Server:
Differential Analysis of High-Throughput Data.''} \emph{Nucleic Acids
Research} 40 (W1): W553--59. \url{https://doi.org/10.1093/nar/gks420}.

\leavevmode\hypertarget{ref-King2015FOC}{}%
King, C. L., D. H. Davies, P. Felgner, E. Baum, A. Jain, A. Randall, K.
Tetteh, C. J. Drakeley, and B. Greenhouse. 2015. {``Biosignatures of
Exposure/Transmission and Immunity.''} \emph{American Journal of
Tropical Medicine and Hygiene} 93 (3 Suppl): 16--27.
\url{https://doi.org/10.4269/ajtmh.15-0037}.

\leavevmode\hypertarget{ref-kreil2005bullet}{}%
Kreil, David P, and Roslin R Russell. 2005. {``Tutorial Section: There
Is No Silver Bullet---a Guide to Low-Level Data Transforms and
Normalisation Methods for Microarray Data.''} \emph{Briefings in
Bioinformatics} 6 (1): 86--97. \url{https://doi.org/10.1093/bib/6.1.86}.

\leavevmode\hypertarget{ref-adiinc}{}%
Liang, Xiaowu, Angela Yee, Joe Campo, Gary Hermanson, Arlo Randall, and
Philip L. Felgner. 2004. \emph{{Antigen Discovery Incorporated}}.
\href{https://www.antigendiscovery.com}{www.antigendiscovery.com}.

\leavevmode\hypertarget{ref-llanoschea2015}{}%
Llanos-Chea, Fiorella, Dalila Martínez, Angel Rosas, Frine Samalvides,
Joseph M. Vinetz, and Alejandro Llanos-Cuentas. 2015. {``Characteristics
of Travel-Related Severe Plasmodium Vivax and Plasmodium Falciparum
Malaria in Individuals Hospitalized at a Tertiary Referral Center in
Lima, {Peru}.''} \emph{The American Journal of Tropical Medicine and
Hygiene} 93 (6): 1249--53. \url{https://doi.org/10.4269/ajtmh.14-0652}.

\leavevmode\hypertarget{ref-lopez2017}{}%
López, Carolina, Yoelis Yepes-Pérez, Natalia Hincapié-Escobar, Diana
Dı́az-Arévalo, and Manuel A Patarroyo. 2017. {``What Is Known about the
Immune Response Induced by Plasmodium Vivax Malaria Vaccine
Candidates?''} \emph{Frontiers in Immunology} 8.
\url{https://doi.org/10.3389/fimmu.2017.00126}.

\leavevmode\hypertarget{ref-factores2001}{}%
MINSA. 2001. \emph{Factores de Riesgo de La Malaria Grave En El Per{ú}}.
{Ministerio de Salud}: Proyecto Vig{í}a (MINSA-USAID).
\url{http://bvs.minsa.gob.pe/local/minsa/1772.pdf}.

\leavevmode\hypertarget{ref-norma2001}{}%
---------. 2007. \emph{Norma técnica de Salud Para La Atención de La
Malaria y Malaria Grave En El Per{ú}}. {Ministerio de Salud}.
\url{http://www.minsa.gob.pe/portada/esnemo_normatividad.asp}.

\leavevmode\hypertarget{ref-mueller2013}{}%
Mueller, Ivo, Mary R Galinski, Takafumi Tsuboi, Myriam Arévalo-Herrera,
William E Collins, and Christopher L King. 2013. {``Natural Acquisition
of Immunity to Plasmodium Vivax: Epidemiological Observations and
Potential Targets.''} \emph{Adv Parasitol} 81: 77--131.
\url{https://doi.org/10.1016/B978-0-12-407826-0.00003-5}.

\leavevmode\hypertarget{ref-path2011}{}%
PATH. 2011. \emph{Staying the Course? Malaria Research and Development
in a Time of Economic Uncertainty}. Seattle, WA: PATH.
\href{https://www.malariavaccine.org/files/RD-report-June2011.pdf}{www.malariavaccine.org/files/RD-report-June2011.pdf}.

\leavevmode\hypertarget{ref-portillo2001vir}{}%
Portillo, Hernando A del, Carmen Fernandez-Becerra, Sharen Bowman, Karen
Oliver, Martin Preuss, Cecilia P Sanchez, Nick K Schneider, et al. 2001.
{``A Superfamily of Variant Genes Encoded in the Subtelomeric Region of
Plasmodium Vivax.''} \emph{Nature} 410 (6830): 839--42.
\url{https://doi.org/10.1038/35071118}.

\leavevmode\hypertarget{ref-accelerate2016}{}%
Quispe, Antonio M., Alejandro Llanos-Cuentas, Hugo Rodriguez, Martin
Clendenes, Cesar Cabezas, Luis M. Leon, Raul Chuquiyauri, et al. 2016.
{``Accelerating to Zero: Strategies to Eliminate Malaria in the
{Peruvian Amazon}.''} \emph{The American Journal of Tropical Medicine
and Hygiene} 94 (6): 1200--1207.
\url{https://doi.org/10.4269/ajtmh.15-0369}.

\leavevmode\hypertarget{ref-quispe2014}{}%
Quispe, Antonio M, Edwar Pozo, Edith Guerrero, Salomón Durand, G
Christian Baldeviano, Kimberly A Edgel, Paul CF Graf, and Andres G
Lescano. 2014. {``Plasmodium Vivax Hospitalizations in a Monoendemic
Malaria Region: Severe Vivax Malaria?''} \emph{The American Journal of
Tropical Medicine and Hygiene} 91 (1): 11--17.
\url{https://doi.org/10.4269/ajtmh.12-0610}.

\leavevmode\hypertarget{ref-R2016}{}%
R Core Team. 2016. \emph{R: A Language and Environment for Statistical
Computing}. Vienna, Austria: R Foundation for Statistical Computing.
\url{https://www.R-project.org/}.

\leavevmode\hypertarget{ref-reyburn2015}{}%
Reyburn, Hugh, Redempta Mbatia, Chris Drakeley, Jane Bruce, Ilona
Carneiro, Raimos Olomi, Jonathan Cox, et al. 2005. {``Association of
Transmission Intensity and Age with Clinical Manifestations and Case
Fatality of Severe Plasmodium Falciparum Malaria.''} \emph{JAMA} 293
(12): 1461--70. \url{https://doi.org/10.1001/jama.293.12.1461}.

\leavevmode\hypertarget{ref-limma}{}%
Ritchie, Matthew E, Belinda Phipson, Di Wu, Yifang Hu, Charity W Law,
Wei Shi, and Gordon K Smyth. 2015. {``{limma} Powers Differential
Expression Analyses for {RNA}-Sequencing and Microarray Studies.''}
\emph{Nucleic Acids Research} 43 (7): e47.
\url{https://doi.org/10.1093/nar/gkv007}.

\leavevmode\hypertarget{ref-CienciaReproducible2016}{}%
Rodrı́guez-Sanchez, Francisco, Antonio Jesús Pérez-Luque, Ignasi
Bartomeus, and Sara Varela. 2016. {``Ciencia Reproducible: {{}}Qu{é},
Por Qu{é}, c{ó}mo?''} \emph{{ECOS}} 25 (2): 83--92.
\url{https://doi.org/10.7818/ecos.2016.25-2.11}.

\leavevmode\hypertarget{ref-rogerson2007preg}{}%
Rogerson, Stephen J, Lars Hviid, Patrick E Duffy, Rose FG Leke, and
Diane W Taylor. 2007. {``Malaria in Pregnancy: Pathogenesis and
Immunity.''} \emph{The Lancet Infectious Diseases} 7 (2): 105--17.
\url{https://doi.org/10.1016/S1473-3099(07)70022-1}.

\leavevmode\hypertarget{ref-rosas2016peru}{}%
Rosas-Aguirre, Angel, Dionicia Gamboa, Paulo Manrique, Jan E Conn, Marta
Moreno, Andres G Lescano, Juan F Sanchez, et al. 2016. {``Epidemiology
of Plasmodium Vivax Malaria in {Peru}.''} \emph{The American Journal of
Tropical Medicine and Hygiene} 95 (6 Suppl): 133--44.
\url{https://doi.org/10.4269/ajtmh.16-0268}.

\leavevmode\hypertarget{ref-hotspots2015}{}%
Rosas-Aguirre, Angel, Niko Speybroeck, Alejandro Llanos-Cuentas, Anna
Rosanas-Urgell, Gabriel Carrasco-Escobar, Hugo Rodriguez, Dionicia
Gamboa, et al. 2015. {``Hotspots of Malaria Transmission in the
{Peruvian Amazon}: Rapid Assessment Through a Parasitological and
Serological Survey.''} \emph{PLOS ONE} 10 (9): 1--21.
\url{https://doi.org/10.1371/journal.pone.0137458}.

\leavevmode\hypertarget{ref-smith2013}{}%
Smith-Nuñez, Edward S., Salomon Durand, G. Christian Baldeviano, Antonio
M. Quispe, Frederique Jacquerioz, Moises Sihuincha, Juan C. Celis, et
al. 2013. {``WHO Criteria for Severe Malaria in Identifying Severe Vivax
Malaria: Preliminary Data from a Study in {Iquitos}, {Peru}.''} In
\emph{Abstract Book of the ASTMH 62nd Annual Meeting, Nov. 13--17,
Washington {D.C.}, United States}, 398.
\url{http://www.astmh.org/ASTMH/media/Documents/AbstractBook2013Final.pdf}.

\leavevmode\hypertarget{ref-smyth2004ebayes}{}%
Smyth, Gordon K, and others. 2004. {``Linear Models and Empirical Bayes
Methods for Assessing Differential Expression in Microarray
Experiments.''} \emph{Statistical Applications in Genetics and Molecular
Biology} 3 (1): 3. \url{https://doi.org/10.2202/1544-6115.1027}.

\leavevmode\hypertarget{ref-Stanisic2015}{}%
Stanisic, Danielle I., Freya J. I. Fowkes, Melanie Koinari, Sarah
Javati, Enmoore Lin, Benson Kiniboro, Jack S. Richards, et al. 2015.
{``Acquisition of Antibodies Against Plasmodium Falciparum Merozoites
and Malaria Immunity in Young Children and the Influence of Age, Force
of Infection, and Magnitude of Response.''} \emph{Infection and
Immunity} 83 (2): 646--60. \url{https://doi.org/10.1128/IAI.02398-14}.

\leavevmode\hypertarget{ref-sundaresh2006}{}%
Sundaresh, Suman, Denise L Doolan, Siddiqua Hirst, Yunxiang Mu, Berkay
Unal, D Huw Davies, Philip L Felgner, and Pierre Baldi. 2006.
{``Identification of Humoral Immune Responses in Protein Microarrays
Using DNA Microarray Data Analysis Techniques.''} \emph{Bioinformatics}
22 (14): 1760--66. \url{https://doi.org/10.1093/bioinformatics/btl162}.

\leavevmode\hypertarget{ref-Torres2014asymptomatic}{}%
Torres, K. J., C. E. Castrillon, E. L. Moss, M. Saito, R. Tenorio, D. M.
Molina, H. Davies, et al. 2014. {``Genome-Level Determination of
Plasmodium Falciparum Blood-Stage Targets of Malarial Clinical Immunity
in the {Peruvian Amazon}.''} \emph{Journal of Infectious Diseases},
November. \url{https://doi.org/10.1093/infdis/jiu614}.

\leavevmode\hypertarget{ref-uzoma2013interactome}{}%
Uzoma, Ijeoma, and Heng Zhu. 2013. {``Interactome Mapping: Using Protein
Microarray Technology to Reconstruct Diverse Protein Networks.''}
\emph{Genomics, Proteomics \& Bioinformatics} 11 (1): 18--28.
\url{https://doi.org/10.1016/j.gpb.2012.12.005}.

\leavevmode\hypertarget{ref-vigil2010}{}%
Vigil, Adam, D Huw Davies, and Philip L Felgner. 2010. {``Defining the
Humoral Immune Response to Infectious Agents Using High-Density Protein
Microarrays.''} \emph{Future Microbiology} 5 (2): 241--51.
\url{https://doi.org/10.2217/fmb.09.127}.

\leavevmode\hypertarget{ref-wassmer2015}{}%
Wassmer, Samuel C, Terrie E Taylor, Pradipsinh K Rathod, Saroj K Mishra,
Sanjib Mohanty, Myriam Arevalo-Herrera, Manoj T Duraisingh, and Joseph D
Smith. 2015. {``Investigating the Pathogenesis of Severe Malaria: A
Multidisciplinary and Cross-Geographical Approach.''} \emph{The American
Journal of Tropical Medicine and Hygiene} 93 (3 Suppl): 42--56.
\url{https://doi.org/10.4269/ajtmh.14-0841}.

\leavevmode\hypertarget{ref-WHO2014severe}{}%
WHO. 2014. {``Severe Malaria.''} \emph{Trop Med Int Health} 19
(September): 7--131. \url{https://doi.org/10.1111/tmi.12313_2}.

\leavevmode\hypertarget{ref-WHO2016world}{}%
---------. 2016. \emph{World Malaria Report 2016}. Vol. 13. Geneva:
World Health Organization.
\url{http://www.who.int/malaria/publications/world-malaria-report-2016/report/en/}.

\leavevmode\hypertarget{ref-world2020world}{}%
---------. 2020. {``World Malaria Report 2020: 20 Years of Global
Progress and Challenges.''}

\leavevmode\hypertarget{ref-rainbow2016}{}%
---------. n.d. {``Malaria Vaccine Rainbow Tables.''} In. World Health
Organization. Accessed: 15-june-2017.
\url{http://www.who.int/vaccine_research/links/Rainbow/en/index.html}.

\leavevmode\hypertarget{ref-wickham2016r4ds}{}%
Wickham, Hadley, and Garrett Grolemund. 2016. \emph{R for Data Science}.
Sebastopol, CA: O'Reilly. \url{http://r4ds.had.co.nz/}.

\leavevmode\hypertarget{ref-knitr}{}%
Xie, Yihui. 2013. {``Knitr: A General-Purpose Tool for Dynamic Report
Generation in r.''} \emph{R Package Version} 1 (1).
\url{http://yihui.name/knitr/}.

\end{CSLReferences}

\section{ANEXOS}\label{anexos}

\subsection{Matriz de consistencia}\label{matriz-de-consistencia}

Ver \autoref{tab:consis} al final del documento.


\afterpage{
    \clearpage
    \newgeometry{left=2.5cm,right=0.5cm,top=0.5cm,bottom=0.5cm}
    \thispagestyle{empty}
    \begin{landscape}
        \centering
\captionof{table}{Matriz de consistencia}
\label{tab:consis}
\begin{center}
\begin{tabular}{|m{3.2cm}m{3.2cm}m{3.2cm}m{3.2cm}m{3.2cm}m{3.2cm}m{3.2cm}|}
  \hline
  \textbf{Título:}
  &
  \multicolumn{6}{l|}{ %>{\centering}m{19cm}
  \begin{minipage}{19.2cm}
  Comparación de la respuesta de anticuerpos ante %la infección con 
  \textit{Plasmodium vivax}
  en pacientes de la Amazonía Peruana %ciudad de Iquitos (Loreto - Perú)
  según su severidad y episodios previos %exposición previa
  mediante microarreglos de proteínas  
  \end{minipage}  
  }\\
  \cline{1-7}
  \textbf{Problema} & \textbf{Objetivos} & \textbf{Hipótesis} & \textbf{Variables} & 
  \textbf{Diseño} & \textbf{Muestra} & %\textbf{Instrumentos} & 
  \textbf{Análisis}\\
  \hline
  \begin{minipage}{3.2cm} 
  %\textbf{Principal}\\
  1. ¿Cuáles son los antígenos con reactividad serológica diferenciada
  ante la infección por \textit{P. vivax} entre pacientes severos y no-severos\\
  \newline
  %\textbf{Secundario}\\
  2. ¿Cuáles son los antígenos con reactividad serológica diferenciada
  ante la infección por \textit{P. vivax} entre pacientes con y sin episodios previos\\
  \newline
  3. ¿Cuáles son las características proteicas de los antígenos
  con reactividad serológica diferenciada o predominante
  en pacientes con malaria por \textit{P. vivax}?
  \end{minipage} 
  & 
  \begin{minipage}{3.2cm} 
  %.\\
  \textbf{General}\\
  Identificar un subconjunto de antígenos con reactividad serológica 
  discriminante de condiciones clínicas relevantes ante la infección por \textit{P. vivax}.\\
  \newline
  \textbf{Específicos}\\
  1. Identificar antígenos de \textit{P. vivax} con reactividad serológica 
  diferenciada ante la infección entre pacientes 
  severos y no-severos.\\
  \newline
  2. Identificar antígenos de \textit{P. vivax} con reactividad serológica 
  diferenciada ante la infección entre pacientes 
  con y sin episodios previos.\\
  \newline
  3. Describir las características proteicas de los antígenos con reactividad 
  diferenciada o predominante.\\
  \end{minipage} 
  & 
  \begin{minipage}{3.2cm} 
  .\\
  \textbf{De diferencia}\\ \textbf{entre grupos:}\\
  1. Los pacientes con malaria no-severa poseen 
  mayor reactividad serológica contra antígenos de \textit{P. vivax}
  asociados a exposición, invasión o adhesión celular
  con respecto a los pacientes severos.\\
  \newline
  2. Los pacientes con episodios previos de malaria poseen
  mayor reactividad serológica contra antígenos de \textit{P. vivax}
  asociados a exposición
  con respecto a los pacientes sin episodios previos.\\
  \newline
  \textbf{Descriptiva:}\\
  3. Los antígenos con reactividad diferenciada o predominante
  se caracterizan por poseer una localización extracelular 
  y estar bajo presión selectiva por el sistema inmune.\\
  \end{minipage} 
  &
  \begin{minipage}{3.2cm} 
  \textbf{Dependiente}\\ Reactividad\\ serológica\\
  \newline 
  \textbf{Independiente}\\ Severidad\\
  \newline
  \textbf{Independiente}\\ Episodios previos\\
  \newline
  \underline{Instrumentos}:\\
  %\textbf{Reactividad serológica:}\\
  -Microarreglo de proteínas PfPv500.\\%Pf498/Pv516\\
  %\newline
  %\textbf{Severidad:}\\
  -Diagnóstico clínico y pruebas bioquímicas.\\%Criterios de la OMS para malaria severa % (criterio OMS)
  %\newline
  %\textbf{Episodios}\\ \textbf{previos:}\\
  -Encuesta.\\
  \newline
  \underline{Operacionalización}:\\
  -Ir a \autoref{tab:opera}
  \end{minipage} 
  &
  \begin{minipage}{3.2cm} 
  \textbf{Tipo:}\\
  Caso-Control.\\
  \newline
  \textbf{Clasificación:}\\
  -Por la finalidad: Analítico.\\
  %\newline
  -Por el control de la asignación:\\ Observacional.\\
  %\newline
  -Por el seguimiento: Transversal.\\
  %\newline
  -Por la relación cronológica:\\ Prospectivo.%\\
  %\newline
  %-Por la unidad de análisis:\\ Basado en individuo.
  \end{minipage}   
  &
  \begin{minipage}{3.2cm} 
  %\textbf{Universo teórico:}\\ 
  %Pacientes con malaria vivax 
  %de la cuenca amazónica del Perú.\\
  %\newline
  \textbf{Población}\\ %\textbf{Muestral:}\\
  Pacientes diagnosticados con malaria por \textit{P. vivax} de la ciudad de Iquitos, Loreto-Perú, 
  entre enero del 2012 y junio del 2013.\\
  \newline
  \textbf{Muestra:}\\
  Selección arbitraria
  de 30 pacientes con uno o más criterios de la clasificación OMS para malaria severa (casos) y 
  selección aleatoria simple de 30 no-severos (control).\\
  \newline
  \textbf{Tipo:}\\ No probabilística y Probabilística.
  \end{minipage}   
  &
%  \begin{minipage}{3.2cm} 
%  \textbf{Reactividad serológica:}\\
%  Microarreglos de\\proteínas Pf498/Pv516\\
%  \newline
%  \textbf{Severidad:}\\
%  Diagnóstico \\clínico \\y exámenes de \\laboratorio\\ según criterios de la OMS.\\
%  \newline
%  \textbf{Episodios}\\ \textbf{previos:}\\
%  Encuesta
%  \end{minipage}   
%  &
  \begin{minipage}{3.2cm} 
  \underline{Control de Calidad}:\\
  \newline
  \textbf{1. Validez:
  %y}\\ \textbf{Reproducibilidad:
  }\\
  correlación de Pearson o Spearman\\
  \newline
  \underline{Prueba de Hipótesis}:\\
  \newline
  \textbf{2. Amplitud}\\ \textbf{e intensidad de}\\ \textbf{respuesta:}\\
  prueba t-Student o Mann-Whitney\\
  \newline
  \textbf{3.}\\ \textbf{Reactividad}\\ \textbf{diferenciada:}\\%\textbf{3. Inferencia}\\
  test-t moderado con\\
  corrección del FDR\\por el método\\Benjamini-Hochberg\\
  \newline
  \textbf{4. Clasificación:}\\
  agrupamiento jerárquico o\\ \textit{hierarchical}\\ \textit{clustering}\\
  en base a la\\ distancia euclidiana.\\
  \newline
  \textbf{5. Descripción:}\\
  características disponibles en\\ PlasmoDB.
  \end{minipage}   
  \\
  %\cline{1-5}
  %y & z & m & n & y & z & m & n\\
  \hline
  % etc. ...
\end{tabular}

\end{center}
    \end{landscape}
    \restoregeometry
    \clearpage
}


\end{document}
