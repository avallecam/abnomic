\documentclass[a4paper]{article}
\usepackage{lmodern}
\usepackage{amssymb,amsmath}
\usepackage{ifxetex,ifluatex}
\usepackage{fixltx2e} % provides \textsubscript
\ifnum 0\ifxetex 1\fi\ifluatex 1\fi=0 % if pdftex
  \usepackage[T1]{fontenc}
  \usepackage[utf8]{inputenc}
\else % if luatex or xelatex
  \ifxetex
    \usepackage{mathspec}
  \else
    \usepackage{fontspec}
  \fi
  \defaultfontfeatures{Ligatures=TeX,Scale=MatchLowercase}
\fi
% use upquote if available, for straight quotes in verbatim environments
\IfFileExists{upquote.sty}{\usepackage{upquote}}{}
% use microtype if available
\IfFileExists{microtype.sty}{%
\usepackage{microtype}
\UseMicrotypeSet[protrusion]{basicmath} % disable protrusion for tt fonts
}{}
\usepackage[margin=1in]{geometry}
\usepackage{hyperref}
\PassOptionsToPackage{usenames,dvipsnames}{color} % color is loaded by hyperref
\hypersetup{unicode=true,
            colorlinks=true,
            linkcolor=Blue,
            citecolor=Blue,
            urlcolor=Blue,
            breaklinks=true}
\urlstyle{same}  % don't use monospace font for urls
\usepackage{longtable,booktabs}
\usepackage{graphicx,grffile}
\makeatletter
\def\maxwidth{\ifdim\Gin@nat@width>\linewidth\linewidth\else\Gin@nat@width\fi}
\def\maxheight{\ifdim\Gin@nat@height>\textheight\textheight\else\Gin@nat@height\fi}
\makeatother
% Scale images if necessary, so that they will not overflow the page
% margins by default, and it is still possible to overwrite the defaults
% using explicit options in \includegraphics[width, height, ...]{}
\setkeys{Gin}{width=\maxwidth,height=\maxheight,keepaspectratio}
\IfFileExists{parskip.sty}{%
\usepackage{parskip}
}{% else
\setlength{\parindent}{0pt}
\setlength{\parskip}{6pt plus 2pt minus 1pt}
}
\setlength{\emergencystretch}{3em}  % prevent overfull lines
\providecommand{\tightlist}{%
  \setlength{\itemsep}{0pt}\setlength{\parskip}{0pt}}
\setcounter{secnumdepth}{5}
% Redefines (sub)paragraphs to behave more like sections
\ifx\paragraph\undefined\else
\let\oldparagraph\paragraph
\renewcommand{\paragraph}[1]{\oldparagraph{#1}\mbox{}}
\fi
\ifx\subparagraph\undefined\else
\let\oldsubparagraph\subparagraph
\renewcommand{\subparagraph}[1]{\oldsubparagraph{#1}\mbox{}}
\fi

%%% Use protect on footnotes to avoid problems with footnotes in titles
\let\rmarkdownfootnote\footnote%
\def\footnote{\protect\rmarkdownfootnote}

%%% Change title format to be more compact
\usepackage{titling}

% Create subtitle command for use in maketitle
\newcommand{\subtitle}[1]{
  \posttitle{
    \begin{center}\large#1\end{center}
    }
}

\setlength{\droptitle}{-2em}
  \title{}
  \pretitle{\vspace{\droptitle}}
  \posttitle{}
  \author{}
  \preauthor{}\postauthor{}
  \date{}
  \predate{}\postdate{}

\usepackage{multirow}
\usepackage{pdflscape}
\usepackage{afterpage}
\usepackage{capt-of}
\usepackage{array}
\usepackage{color,soul}

\begin{document}

\renewcommand{\contentsname}{Índice General} 
\renewcommand{\tablename}{Tabla}
\renewcommand{\tableautorefname}{Tabla}

\pagenumbering{gobble}

\clearpage
\newgeometry{left=0.5cm,right=0.5cm,top=2cm,bottom=1cm}

\begin{centering}

\begin{figure}[!ht]
  \begin{center}
    \includegraphics[width=.8in]{figure/UNMSM_escudo-2000px.png}% en informe: 8
  \end{center}
\end{figure}

\Large %https://tex.stackexchange.com/questions/24599/what-point-pt-font-size-are-large-etc
UNIVERSIDAD NACIONAL MAYOR DE SAN MARCOS

\large
(Universidad del Perú, DECANA DE AMÉRICA)

\vspace{.3 cm}

\Large
FACULTAD DE CIENCIAS BIOLÓGICAS

\vspace{.3 cm}

\normalsize
ESCUELA PROFESIONAL DE

GENÉTICA Y BIOTECNOLOGÍA

\vspace{3 cm}

\Large
Comparación de la respuesta de anticuerpos ante %la %infección con 
\textit{Plasmodium vivax}
en \\pacientes de la Amazonía Peruana %ciudad de Iquitos (Loreto - Perú)
según su severidad y episodios \\previos %exposición previa
mediante microarreglos de proteínas
% aPerfil de anticuerpos en respuesta a la infección con malaria vivax
%mediante un enfoque inmunómico .
%% con sintomatología severa y no complicada/ pre-inmunes/ semi-inmunes

\vspace{3 cm}

\large
Proyecto de Tesis 

para optar al Título Profesional de Biólogo Genetista y Biotecnólogo

\vspace{.3 cm}

\large
Investigador

Bach. Andree Adolfo Valle Campos

\vspace{.3 cm}

Asesores

Interno: Dr. Juan Jiménez Chunga %Prof. Walter Cabrera-Febolá

Externo: PhD. G. Christian Baldeviano

\vspace{.3 cm}

Institución

Centro de Enfermedades Tropicales de la Marina de los Estados Unidos 
NAMRU-6

%\vspace{.3 cm}

%Duración: 5 meses

\vspace{1.7 cm}

\Large
Lima - Perú

%\vspace{.5 cm}

2017

\end{centering}

\vfill
\restoregeometry
\clearpage

\newpage

\tableofcontents

\newpage

\pagenumbering{arabic}

\section*{RESUMEN}\label{resumen}
\addcontentsline{toc}{section}{RESUMEN}

\begin{quote}
\emph{Plasmodium vivax} es responsable del 80\% de la malaria en el
Perú. Casos de enfermedad severa por mono-infecciones de \emph{P. vivax}
han sido reportados tanto en el Noreste amazónico como en Norte costero.
Sin embargo, dicha condición aún está subestimada en la Norma Técnica a
nivel nacional, incrementando el riesgo de posibles diagnósticos tardíos
o inapropiados por el personal de salud. Con el fin de identificar
biomarcadores de relevancia clínica contra la malaria severa por
\emph{P. vivax}, el presente proyecto de tesis propone comparar la
respuesta de anticuerpos ante la infección en pacientes severos y
no-severos provenientes de un estudio prospectivo caso-control ejecutado
en la ciudad de Iquitos, Loreto - Perú, empleando microarreglos de
proteínas. Los resultados permitirán proponer marcadores serológicos
discriminantes de severidad con el potencial de implementarse en
programas de serovigilancia en miras al control y eliminación de la
malaria por \emph{P. vivax} en la región.
\end{quote}

\section{PLANTEAMIENTO DEL PROBLEMA}\label{planteamiento-del-problema}

\subsection{Formulación del problema}\label{intro}

Malaria es una enfermedad parasitaria de importancia mundial, causada
por protozoarios del género \emph{Plasmodium} y transmitida por
mosquitos del género \emph{Anopheles}. En el 2015, se estimaron 212
millones de casos y 429,000 muertes atribuidas a esta
infección\textsuperscript{\protect\hyperlink{ref-WHO2016world}{1}}.
Aunque \emph{P. falciparum} representó el 96 y 99\% de estas cifras,
fuera de África se estimó que \emph{P. vivax} fue responsable del 41 y
86\%, respectivamente. Más aún, la región de las Américas tuvo la mayor
proporción de estos casos (69\%) donde Perú fue el tercer país con más
reportes (n=63,153), detrás de Brasil y Venezuela, atribuidos en un 80\%
a dicha
especie\textsuperscript{\protect\hyperlink{ref-rosas2016peru}{2}}.

Reportes recientes de malaria severa y fatal causados por \emph{P.
vivax} han desafiado su tradicional condición de enfermedad
benigna\textsuperscript{\protect\hyperlink{ref-baird2009}{3},\protect\hyperlink{ref-quispe2014}{4}}.
El incremento de estos casos puede ser consecuencia de un retraso en la
adquisición de inmunidad en la población por una reducción en la
intensidad de
transmisión\textsuperscript{\protect\hyperlink{ref-reyburn2015}{5}}. Por
ello, tanto su actual
subestimación\textsuperscript{\protect\hyperlink{ref-norma2001}{6}} como
la ausencia de marcadores serológicos de protección contra la severidad
en las actuales estrategias de
eliminación\textsuperscript{\protect\hyperlink{ref-accelerate2016}{7}}
podría incrementar su prevalencia a largo plazo. En contraste con dicha
hipótesis, pacientes de la Amazonía Peruana con malaria por \emph{P.
vivax} severa y no complicada mostraron un grado similar de exposición
previa, basado en la respuesta humoral contra un marcador
tradicional\textsuperscript{\protect\hyperlink{ref-baldevi2013}{8}}. Sin
embargo, recientes estudios a larga escala han desafiado su validez como
indicador de inmunidad o
exposición\textsuperscript{\protect\hyperlink{ref-crompton2010}{9},\protect\hyperlink{ref-Helb2015exposure}{10}}.

Por esta razón, el presente estudio tiene como objetivo comparar la
respuesta de anticuerpos ante la infección con \emph{P. vivax} contra
más de 500 antígenos del parásito en pacientes severos y no-severos de
la ciudad de Iquitos, empleando microarreglos de proteínas.
Hipotetizamos que los no-severos poseen mayor reactividad serológica
contra antígenos de exposición, invasión o adhesión celular, con
respecto a los severos. Primero, se identificarán los antígenos con
reactividad diferenciada entre ambos grupos. Luego, se determinarán los
antígenos de respuesta secundaria en pacientes con episodios previos de
esta infección. Finalmente, se describirán sus características proteicas
junto a los antígenos con mayor reactividad en toda la muestra con el
propósito de proponerlos como candidatos a vigilancia
seroepidemiológica.

\subsection{Preguntas de
investigación}\label{preguntas-de-investigacion}

\begin{enumerate}
\def\labelenumi{\arabic{enumi}.}
\item
  ¿Cuáles son los antígenos con reactividad serológica diferenciada ante
  la infección por \emph{P. vivax} entre pacientes severos y no-severos?
\item
  ¿Cuáles son los antígenos con reactividad serológica diferenciada ante
  la infección por \emph{P. vivax} entre pacientes con y sin episodios
  previos?
\item
  ¿Cuáles son las características proteicas de los antígenos con
  reactividad serológica diferenciada o predominante en pacientes con
  malaria por \emph{P. vivax}?
\end{enumerate}

\subsection{Objetivos}\label{objetivos}

\subsubsection{General}\label{general}

\begin{itemize}
\tightlist
\item
  Identificar un subconjunto de antígenos con reactividad serológica
  discriminante de condiciones clínicas relevantes ante la infección por
  \emph{P. vivax}.
\end{itemize}

\subsubsection{Específicos}\label{especificos}

\begin{itemize}
\item
  Identificar antígenos de \emph{P. vivax} con reactividad serológica
  diferenciada ante la infección entre pacientes severos y no-severos.
\item
  Identificar antígenos de \emph{P. vivax} con reactividad serológica
  diferenciada ante la infección entre pacientes con y sin episodios
  previos.
\item
  Describir las características proteicas de los antígenos con
  reactividad diferenciada o predominante.
\end{itemize}

\subsubsection{Exploratorio}\label{exploratorio}

\begin{itemize}
\tightlist
\item
  Comparar la amplitud e intensidad de respuesta de anticuerpos según la
  edad de los pacientes con malaria por \emph{P. vivax}.
\end{itemize}

\subsection{Justificación}\label{justif}

Ante la reemergencia repetitiva de la malaria en la Amazonía
Peruana\textsuperscript{\protect\hyperlink{ref-rosas2016peru}{2},\protect\hyperlink{ref-griffing2013history}{11}},
la implementación de vigilancias serológicas programáticas en nuestro
país es una
prioridad\textsuperscript{\protect\hyperlink{ref-hotspots2015}{12}}. En
zonas con baja transmisión, estos ensayos poseen una mayor sensibilidad
y representan un menor gasto económico en comparación a las estrategias
convencionales de monitoreo y
control\textsuperscript{\protect\hyperlink{ref-elliott2014}{13}}. Por
esta razón, el presente estudio se justifica en el descubrimiento de
antígenos potencialmente discriminantes de condiciones clínicas como
severidad o exposición. Estos permitirán optimizar y acelerar la
ejecución de vigilancias serológicas en los programas de salud pública
contra la malaria en el Perú, en miras a su control y posterior
eliminación\textsuperscript{\protect\hyperlink{ref-accelerate2016}{7}}.

\subsection{Limitaciones}\label{limit}

Resumimos tres limitaciones del estudio.

Primero, las muestras a evaluar provienen de pacientes con infecciones
no sincronizadas. El muestreo se ejecutó en la fase sintomática, momento
en el que acuden al Hospital, con un desconocimiento del inicio de la
infección por paciente. Sin embargo, infecciones experimentales han
estimado que dicha fase ocurre normalmente entre los 11 y 13 días de
infección\textsuperscript{\protect\hyperlink{ref-arevalo2014}{14}}.

Segundo, el estudio posee debilidades propias del diseño experimental a
emplear. El diseño de tipo Caso Control es susceptible a errores
sistemáticos de selección y clasificación, posee dificultad para
establecer relaciones causa-efecto y a través de él no se pueden
calcular prevalencias o incidencias. Además, la ausencia de un
seguimiento activo impide el registro de covariables relevantes para la
caracterización de la enfermedad.

Tercero, los microarreglos de proteínas poseen limitantes propias a su
fabricación\textsuperscript{\protect\hyperlink{ref-vigil2010}{15}}. Cada
paso (amplificación, clonamiento, expresión de genes a larga escala e
impresión del arreglo) posee una eficiencia límite que afectará la
calidad final de las proteínas. Además, el plegamiento proteico no será
posible de verificar a dicha escala. Por último, la identificación de
antígenos con modificaciones postranscriptacionales, particularmente
relevantes en
\emph{Plasmodium}\textsuperscript{\protect\hyperlink{ref-leroch2009postmod}{16}},
no serán posibles de reproducir en su integridad en el sistema de
expresión procarionte. A pesar de ello, se evaluará la validez del
ensayo con controles internos a detallar en la
\protect\hyperlink{validez}{sección 4.3.2.2}.

\subsection{Viabilidad}\label{viabilidad}

El proyecto es viable por la factibilidad de su ejecución, el interés de
sus resultados para el área, la novedad de su enfoque y la relevancia de
su método para la investigación biomédica.

Primero, su ejecución es factible por la disponibilidad inmediata de
muestras provenientes de un estudio prospectivo ya realizado, su
adecuado tamaño muestral (n=60), la objetividad de la pregunta planteada
y la experiencia técnica del investigador en la herramienta de análisis
R\textsuperscript{\protect\hyperlink{ref-R2016}{17}}/Bioconductor\textsuperscript{\protect\hyperlink{ref-bioconductor2004}{18}}.
Segundo, el problema es de interés para la comunidad por la posibilidad
de que sus resultados generen un cambio en la práctica médica. Tercero,
la novedad de su enfoque a larga escala permitirá corroborar, refutar o
extender el estado del conocimiento y brindar mayor evidencia con
respecto a la controversia mencionada. Cuarto, la relevancia de su
método de análisis reproducible con software libre permitirá
transparentar la generación de resultados y promover su puesta en
práctica en futuras investigaciones de esta u otras áreas.

\section{FORMULACIÓN DE HIPÓTESIS Y
VARIABLES}\label{formulacion-de-hipotesis-y-variables}

\subsection{Hipótesis}\label{hipotesis}

\subsubsection{De diferencia entre
grupos}\label{de-diferencia-entre-grupos}

\begin{enumerate}
\def\labelenumi{\arabic{enumi}.}
\item
  Los pacientes con malaria no-severa poseen mayor reactividad
  serológica contra antígenos de \emph{P. vivax} asociados a exposición,
  invasión o adhesión celular con respecto a los pacientes severos.
\item
  Los pacientes con episodios previos de malaria poseen mayor
  reactividad serológica contra antígenos de \emph{P. vivax} asociados a
  exposición con respecto a los pacientes sin episodios previos.
\end{enumerate}

\subsubsection{Descriptiva}\label{descriptiva}

\begin{enumerate}
\def\labelenumi{\arabic{enumi}.}
\setcounter{enumi}{2}
\tightlist
\item
  Los antígenos con reactividad diferenciada o predominante se
  caracterizan por poseer una localización extracelular y estar bajo
  presión selectiva por el sistema inmune.
\end{enumerate}

\subsection{Variables}\label{variables}

El presente estudio busca contrastar la respuesta de anticuerpos ante la
infección con \emph{P. vivax} entre pacientes clasificados según su
severidad y episodios previos. Para ello se medirá la variable de
reactividad serológica de antígenos de \emph{P. vivax} luego de sondear
plasma de pacientes en un microarreglo de proteínas. Además se
clasificará a los pacientes como severos o no-severos de acuerdo a
diagnóstico clínico y bioquímico, y como con-episodios o sin-episodios
previos de acuerdo a lo reportado en la encuesta con el médico tratante.

Primero, la reactividad serológica cuantificará la especificidad de los
anticuerpos de respuesta contra los antígenos del patógeno \emph{P.
vivax}. Esta variable medirá indirectamente la interacción entre
antígenos y anticuerpos, a través de la lectura de la intensidad
fluorescente producto de esta reacción.

Segundo, los pacientes severos se definirán por la presencia de
manifestaciones clínicas severas o complicaciones sistémicas. Esta
variable será asignada por la presencia de uno o más criterios clínicos
y de laboratorio para la malaria severa, según la clasificación
recomendada por la Organización Mundial de la Salud
(OMS)\textsuperscript{\protect\hyperlink{ref-WHO2014severe}{19}}.

Tercero, la clasificación por episodios previos reportados representará
la presencia de infecciones previas con malaria y, consecuentemente, la
presencia de una respuesta inmune humoral secundaria. Esta variable será
reportada por el paciente como el número de eventos previos a la actual
infección. Dada la incertidumbre de dicho dato, la variable será
dicotomizada con el fin de facilitar su inclusión.

\subsubsection{Operacionalización de
variables}\label{operacionalizacion-de-variables}

Ver \autoref{tab:opera}.

\begin{table}[ht]
        \captionof{table}{Operacionalización de variables}
        \label{tab:opera}
        \vspace{-1mm}
\begin{center}
\hspace*{-1cm}
\begin{tabular}{>{\centering}m{2.4cm} m{2.2cm}m{2.2cm}m{1.8cm}m{2cm}m{1.7cm}m{1.5cm}m{1.6cm} @{}m{0pt}@{} }
  
  \hline
  \multirow{2}{*}{Variable}
  & 
  \multicolumn{2}{c}{Definición} 
  %&
  %\begin{minipage}{2.2cm}
  %Definición\\conceptual
  %\end{minipage}
  %&
  %\begin{minipage}{2.2cm}
  %Definición\\operacional
  %\end{minipage}
  & 
  \multirow{2}{*}{
  \begin{minipage}{1.8cm}
  Instrumento\\de medición
  \end{minipage}
  }
  &
  \multirow{2}{*}{
  \begin{minipage}{2cm}
  Criterios\\de medición
  \end{minipage}
  }
  &
  \multirow{2}{*}{
  \begin{minipage}{1.7cm}
  Tipo de\\variable
  \end{minipage}
  }
  &
  \multirow{2}{*}{
  \begin{minipage}{1.5cm}
  Escala de \\medición
  \end{minipage}
  }
  &
  \multirow{2}{*}{
  Fuente
  } &\\[0ex]
  %\hline
  \cline{2-3}
  
  &
  Conceptual
  &
  Operacional
  & 
  &
  &
  & &\\[1ex]
  \hline
  
  \textbf{Dependiente} Reactividad serológica
  & 
  % esCONCEPTUAL: 
  \begin{minipage}{2.2cm} 
  Especificidad \\de anticuerpos \\de respuesta contra un antígeno
  \end{minipage} 
  &
  % aOPERACIONAL: 
  \begin{minipage}{2.2cm} 
  Medida \\indirecta de \\la reacción antígeno-anticuerpo
  \end{minipage} 
  % aDETALLES: medida indirecta de la reacción antígeno-anticuerpo 
  % mediante la lectura de la reacción fluorescente entre 
  % anticuerpo secundario y fluoroforo por spot
  & 
  \begin{minipage}{2.2cm} 
  %Lector\\ óptico de\\
  Microarreglo\\de proteínas
  \end{minipage}
  & 
  \begin{minipage}{2cm} 
  \textbf{0-6000} MFI o intensidad\\
  fluorescente \\promedio.
  \end{minipage} 
  &
  Numérica contínua
  & 
  Razón
  &
  Plasma sanguíneo &\\[13ex]
  \hline

  \textbf{Independiente} Severidad
  & 
  % aCONCEPTUAL: 
  Presencia de manifestaciones clínicas severas o complicaciones sistémicas
  &
  % aOPERACIONAL:
  Número de criterios de la clasificación OMS para malaria severa
  & 
  \begin{minipage}{2.2cm} 
  Diagnóstico \\clínico \\y pruebas \\bioquímicas 
  \end{minipage}
  & 
  \begin{minipage}{2cm} 
  \textbf{No-severa:} 0 criterios\\
  \textbf{Severa:} 1 o más criterios.
  \end{minipage}
  &
  Categórica dicotómica
  & 
  Nominal
  &
  Historia clínica y muestra de sangre &\\[15ex]
  \hline
  
  \textbf{Independiente} Episodios previos
  & 
  % aCONCEPTUAL: 
  Exposición a la infección de malaria en el pasado
  &
  % aOPERACIONAL:
  Número de episodios previos reportados 
  & 
  Encuesta
  & 
  \begin{minipage}{2.1cm} 
  \textbf{Sin:} 0 episodios\\
  \textbf{Con:} 1 o más episodios
  \end{minipage}
  &
  Categórica dicotómica
  & 
  Nominal
  &
  Historia clínica &\\[10ex]
  \hline

%  \textbf{Interviniente} Edad %Confusora
%  & 
%  % aCONCEPTUAL: 
%  Edad del paciente
%  &
%  % aOPERACIONAL:
%  Años de vida reportados
%  & 
%  Encuesta
%  & 
%  \begin{minipage}{2.2cm} 
%  \textbf{0-90} años.
%  \end{minipage}
%  &
%  Numérica discreta
%  & 
%  Razón
%  &
%  Historia clínica &\\[10ex]
%  \hline


  % etc. ...
\end{tabular}
\hspace*{-1cm}
\end{center}
\end{table}

\section{MARCO TEÓRICO}\label{marco-teorico}

\subsection{Antecedentes de la
investigación}\label{antecedentes-de-la-investigacion}

\begin{enumerate}
\def\labelenumi{\alph{enumi}.}
\item
  En África

  Un hito en la aplicación de microarreglos de proteínas para el estudio
  de la respuesta humoral a escala epidemiológica fue la publicación de
  Crompton et al.
  2010\textsuperscript{\protect\hyperlink{ref-crompton2010}{9}}. Este
  estudio comparó la respuesta de anticuerpos contra el 23\% del
  proteoma de \emph{P. falciparum} antes y después de la temporada de
  malaria en 220 individuos de Mali, en dos grupos poblacionales: 2-10 y
  18-25 años. Dentro del grupo de niños entre 8 y 10 años se
  identificaron 49 proteínas con mayor reactividad serológica en el
  grupo de infectados asintomáticos, en comparación a los sintomáticos.
  Cinco de los principales candidatos a vacuna (CSP, LSA-3, MSP1, MSP2,
  AMA1) no lograron discriminar ambos grupos. Sin embargo, cuatro
  candidatos secundarios (STARP, LSA-1, RESA, antígeno 332), con
  expresión en diferentes estadios del ciclo biológico, sí lograron tal
  discriminación.

  Un segundo hito de interés representa el trabajo de Helb et al.
  2015\textsuperscript{\protect\hyperlink{ref-Helb2015exposure}{10}}.
  Ellos reportaron una estrategia para identificar combinaciones de
  respuestas de anticuerpos contra más de un antígeno que maximicen la
  información de la exposición reciente al nivel de individuos. Para
  ello emplearon modelos basados en el aprendizaje automático o
  \emph{machine learning} para el análisis de las respuestas contra 865
  antígenos de \emph{P. falciparum} en 186 niños (3-6 años) de Uganda en
  base al registro activo y pasivo de sus historias clínicas a los largo
  de un año. En contraste a los marcadores tradicionalmente empleados
  para evaluar exposición a nivel poblacional (CSP, MSP1, MSP2,
  AMA1)\textsuperscript{\protect\hyperlink{ref-elliott2014}{13}}, se
  identificaron marcadores más informativos como hyp2, GEXP18, EMP1,
  ETRAMP4, HSP40-II y PF70. La validación de este método permitirá
  optimizar la selección tradicional de marcadores. 
\item
  En Perú

  El primer estudio en publicarse fue de Torres et al.
  2014\textsuperscript{\protect\hyperlink{ref-Torres2014asymptomatic}{20}}
  donde reportaron marcadores de inmunidad clínica -no esterilizante- de
  adquisición natural a la malaria por \emph{P. falciparum}. Además de
  ello, el resultado con mayor implicancias fue que estos marcadores
  presentaron un enriquecimiento de polimorfismos no-sinónimos,
  indicativo de una presión de selección positiva por parte del sistema
  inmune, a pesar de provenir de una zona con baja transmisión. Las 51
  proteínas con mayor reactividad en 14 pacientes infectados y
  asintomáticos se obtuvieron al comparar las respuestas con 24 paciente
  sintomáticos provenientes de la Amazonía Peruana (Departamento de
  Loreto, Provincia de Maynas y Requena) contra 824 fragmentos (699
  proteínas) de \emph{P. falciparum}.

  El segundo y último estudio en ser publicado ha sido el de Chuquiyauri
  et al.
  2015\textsuperscript{\protect\hyperlink{ref-chuquiyauri2015vivax}{21}}.
  Ellos compararon la respuesta contra \emph{P. vivax} de pacientes con
  relapsos y reinfección, sin encontrar diferencia alguna entre ambos
  grupos. Sin embargo, al igual que el anterior estudio, identificaron
  un enriquecimiento de proteínas con polimorfismos no-sinónimos en el
  grupo de antígenos reactivos. Además, dentro del grupo con mayor
  reactividad en toda la muestra, resaltaron a PvMSP-10 como potencial
  candidato a vacuna al presentar la expresión más consistente y validar
  lo reportado por dos estudios previos con enfoque tradicional donde
  emplean ambos sistemas de expresión: eucariota y procariota. El
  estudio empleó un arreglo con 2233 fragmentos (1936 proteínas) en 106
  individuos de la ciudad de Maynas, Loreto - Perú.
\end{enumerate}

\subsection{Bases teóricas}\label{bases-teoricas}

\subsubsection{\texorpdfstring{Malaria por \emph{Plasmodium
vivax}}{Malaria por Plasmodium vivax}}\label{malaria-por-plasmodium-vivax}

\begin{enumerate}
\def\labelenumi{\alph{enumi}.}
\item
  Epidemiología

  \begin{enumerate}
  \def\labelenumii{\roman{enumii}.}
  \item
    \textbf{A nivel mundial}

    \emph{P. vivax} y \emph{P. falciparum} son los principales
    responsables de los casos de malaria en humanos. Ambas especies
    exponen aproximadamente a 2.5 mil millones de personas en riesgo de
    infección\textsuperscript{\protect\hyperlink{ref-howes2016global}{22}}.
    Sin embargo, \emph{P. vivax} es el parásito dominante en las
    regiones fuera del África subsahariana, en su mayoría densamente
    pobladas y empobrecidas. Entre ellas, Etiopía, India, Indonesia y
    Pakistán acumularon el 78\% de casos de \emph{P. vivax} a nivel
    mundial. A su vez, la región de las Américas tuvo la mayor
    proporción de estos, con un
    69\%\textsuperscript{\protect\hyperlink{ref-WHO2016world}{1}}. A
    pesar de ello, hasta el momento la mayoría de la investigación y
    financiamiento está destinado a la prevención, tratamiento y control
    de \emph{P.
    falciparum}\textsuperscript{\protect\hyperlink{ref-path2011}{23}}.
  \item
    \textbf{En el Perú}

    En el 2015, Perú fue el tercer país con más casos reportados en
    Latinoamérica (19\%), detrás de Brasil (24\%) y Venezuela
    (30\%)\textsuperscript{\protect\hyperlink{ref-WHO2016world}{1}}. El
    80\% fueron causados por \emph{P. vivax} (63,153 en total), en
    regiones con endemismo y transmisión
    heterogénea\textsuperscript{\protect\hyperlink{ref-rosas2016peru}{2}}.
    El 95\% pertenecieron al Noreste amazónico (con una razón Pv/Pf de
    4/1) y el resto al Norte costero y la región minera del Suroeste.
    Notablemente, comunidades en Madre de Dios no han registrado caso
    por \emph{P. falciparum} en una década
    (2001-2012)\textsuperscript{\protect\hyperlink{ref-rosas2016peru}{2}}.
    Con respecto a factores ambientales, en el año 1998 durante el
    fenómeno El Niño-Oscilación Sur (ENSO) se produjo el mayor pico de
    casos anuales en la historia (200,000 casos), donde la región del
    Norte costero representó el 48\% de los casos a nivel
    nacional\textsuperscript{\protect\hyperlink{ref-gagnon2002enso}{24}}.
  \end{enumerate}
\item
  Biología

  \begin{enumerate}
  \def\labelenumii{\roman{enumii}.}
  \item
    \textbf{Ciclo de vida}

    La ecología natural del parásito de la malaria involucra a dos
    hospederos, el humano y el mosquito, y tres ciclos de vida:

    \begin{enumerate}
    \def\labelenumiii{\arabic{enumiii}.}
    \tightlist
    \item
      El \emph{ciclo hepático o exo-eritrocítico} inicia con la picadura
      del mosquito infectado, la liberación de los esporozoitos al
      fluido sanguíneo del humano y su ingreso a las células hepáticas.
      Aquí se da su desarrollo asexual donde se multiplican hasta formar
      esquizontes, los cuales egresan al torrente sanguíneo en la forma
      de merozoitos.
    \item
      El \emph{ciclo eritrocítico} se inicia con la invasión de glóbulos
      rojos (RBC) y el desarrollo consecutivo de trofozoitos inmaduros ,
      maduros y esquizontes, los cuales a su ruptura liberan nuevos
      merozoitos que reinfectan a más RBC. Por motivos aún no
      determinados, a partir de los trofozoitos inmaduros se inicia el
      desarrollo de gametocitos diferenciándose sexualmente dentro del
      torrente sanguíneo del humano.
    \item
      El \emph{ciclo esporogónico} o fase sexual se inicia en el
      mosquito hembra mediante la ingestión accidental de gametocitos en
      la alimentación sanguínea, con la intención inicial de proveer de
      nutrientes a sus huevos. En su tracto digestivo se generan cigotos
      mótiles que invaden las paredes para formar un oocisto donde los
      esporozoitos se desarrollan y replican hasta su ruptura y
      liberación. Finalmente migran a las glándulas salivares para
      continuar la transmisión en la siguiente alimentación del
      mosquito.
    \end{enumerate}
  \item
    \textbf{Particularidades}

    \emph{P. vivax} posee importantes variantes biológicas que
    caracterizan su epidemiología y dinámica de
    infección\textsuperscript{\protect\hyperlink{ref-howes2016global}{22}}.
    Tres de las más importantes son:

    \begin{enumerate}
    \def\labelenumiii{\arabic{enumiii}.}
    \tightlist
    \item
      Presencia de relapsos por la activación de hipnozoitos, el cual es
      un estado de latencia o dormacia en el \emph{ciclo hepático},
      condición que puede generar tanto una liberación prolongada y
      lenta de merozoitos como reservorios de infección que prolongan su
      transmisión;
    \item
      Tropismo hacia reticulocitos o RBC inmaduros (1-2\% de RBC
      circulantes) en el \emph{ciclo eritrocítico}, condición que genera
      bajas parasitemias en comparación a \emph{P. falciparum}; y
    \item
      Dependencia del antígeno \emph{Duffy} para la invasión de RBC,
      cuya ausencia en individuos con ancestría africana ha sido
      considerada como un factor de resistencia a \emph{P. vivax}. 
    \end{enumerate}
  \end{enumerate}
\end{enumerate}

\subsubsection{Malaria Severa}\label{malaria-severa}

\begin{enumerate}
\def\labelenumi{\alph{enumi}.}
\item
  Definición

  Ante la ausencia de una descripción especie-específica, la malaria
  severa por \emph{P. vivax} está definida por los criterios para
  \emph{P. falciparum} otorgados por la OMS en el
  2014\textsuperscript{\protect\hyperlink{ref-WHO2014severe}{19}}, el
  cual incluye una o más de las siguientes (todas registradas en
  mono-infecciones por \emph{P. vivax}):

  \begin{enumerate}
  \def\labelenumii{\arabic{enumii}.}
  \tightlist
  \item
    Condición neurológica: coma, mareo, pérdida conciencia;
  \item
    Condición hematológica: anemia/trombocitopenia severa;
  \item
    Síntomas sistémicos: shock circulatorio; y
  \item
    Fallo de órganos vitales: disfución respiratoria, estrés
    respiratorio agudo, daño renal agudo, ruptura esplénica, disfunción
    hepática e ictericia (hiperbilirrubina).
  \end{enumerate}
\item
  Epidemiología

  \begin{enumerate}
  \def\labelenumii{\roman{enumii}.}
  \item
    \textbf{A nivel mundial}

    Entre el 2005 y 2015, 1-3\% de casos no-complicados fueron asumidos
    como malaria severa. Esta causó la muerte de 429,000
    personas\textsuperscript{\protect\hyperlink{ref-WHO2016world}{1}},
    en su mayoría (90\%) niños menores de 5 años en
    África\textsuperscript{\protect\hyperlink{ref-wassmer2015}{25}}. La
    vulnerabilidad a esta manifestación ha sido asociada con la
    intensidad de transmisión y el desarrollo de la inmunidad
    dependiente a la
    edad\textsuperscript{\protect\hyperlink{ref-reyburn2015}{5}}.

    \begin{enumerate}
    \def\labelenumiii{\arabic{enumiii}.}
    \tightlist
    \item
      En zonas de \emph{alta transmisión} como en el África
      subsahariana, las poblaciones más vulnerables son: niños menores
      de 5 años con un desarrollo incompleto de inmunidad parcial contra
      la
      malaria\textsuperscript{\protect\hyperlink{ref-Stanisic2015}{26}},
      mujeres embarazadas en parte a la adhesión placentaria de glóbulos
      rojos infectados
      (iRBC)\textsuperscript{\protect\hyperlink{ref-rogerson2007preg}{27}},
      y viajeros o migrantes sin inmunidad provenientes de áreas con
      baja o ninguna transmisión de malaria.
    \item
      En zonas de \emph{baja transmisión} como en Asia y América Latina,
      al haber una menor exposición a la infección, la mayoría de la
      población llega a la adultez sin haber desarrollado una inmunidad
      protectiva. Como consecuencia, la población adolescente y adulta
      joven es la más susceptible a desarrollar esta
      patología\textsuperscript{\protect\hyperlink{ref-llanoschea2015}{28}},
      comúnmente al iniciar trabajos a campo abierto, e.g.~actividades
      madereras o mineras, en zonas de alto riesgo de contacto con
      mosquitos
      infectados\textsuperscript{\protect\hyperlink{ref-factores2001}{29}}.
    \end{enumerate}
  \item
    \textbf{En el Perú}

    En el Norte costero, Quispe et al.
    2014\textsuperscript{\protect\hyperlink{ref-quispe2014}{4}} mediante
    un estudio retrospectivo (2008-2009) identificaron 81/6502 casos de
    malaria severa por \emph{P. vivax} con anemia severa (57\%), shock
    circulatorio (25\%), hiperbilirrubina (25\%), daño pulmonar (21\%),
    daño renal agudo (14\%) y Glasgow \(\le\) 9/14 (11\%). Comparados
    con los pacientes no complicados, los severos fueron mayores (38 vs
    26 años, P\textless{}0.001).

    En el Noreste amazónico, un reciente estudio prospectivo dirigido
    por El Centro de Enfermedades Tropicales de la Marina de los Estados
    Unidos NAMRU-6
    (2012-2013)\textsuperscript{\protect\hyperlink{ref-smith2013}{30}}
    identificó 55/164 casos de malaria severa por \emph{P. vivax} con
    shock circulatorio (27.6\%), daño pulmonar (24.5\%) hiperbilirubina
    (18.1\%), daño renal agudo (2\%), Glasgow \(\le\) 9/14 (2\%) y
    anemia severa (1.7\%). No se encontraron diferencias con respecto a
    la edad entre no-severos y severos (33 vs 29 años, P=0.497). Sin
    embargo, sí se identificó una mayor proporción de malaria severa en
    mujeres (52 vs 35\%, P=0.039).
  \end{enumerate}
\end{enumerate}

\begin{enumerate}
\def\labelenumi{\alph{enumi}.}
\setcounter{enumi}{2}
\item
  Biología

  \begin{enumerate}
  \def\labelenumii{\roman{enumii}.}
  \item
    \textbf{Patogénesis}

    Se han propuesto mecanismos en base a lo observado en los casos por
    \emph{P. falciparum}, donde los procesos de invasión y adhesión
    parasitaria son relevantes. La respuesta celular contra la
    parasitemia, sumada a la citoadherencia de iRBC en estadio
    esquizonte a endotelio o RBC no infectados (proceso llamado
    ``rosetamiento''), desencadena una obstrucción microvascular que
    activa a las células endoteliales. En respuesta, estas liberan una
    mayor cantidad de citoquinas proinflamatorias, provocando la pérdida
    de perfusión y una consecuente disfunción microvascular que progresa
    por retroalimentación positiva.
  \item
    \textbf{Comparación}

    Históricamente, la malaria por \emph{P. vivax} ha sido considerada
    como ``benigna'' en comparación a \emph{P. falciparum} debido a su:
    (1) baja invasión parasitaria, sesgada a reticulocitos y rutas
    alternas de menor efectividad; y (2) pobre citoadhesión de sus iRBC,
    dada por la ausencia de protrusiones abastonadas o \emph{knob
    protrusions}, codificadas por genes \emph{var}. Sin embargo, la
    identificación de genes homólogos a \emph{Pf var} (\emph{Pv vir}) y
    evindencias \emph{post mortem} de iRBC en capilares pulmonares han
    sugerido la posibilidad de secuestramiento tisular por \emph{P.
    vivax}\textsuperscript{\protect\hyperlink{ref-wassmer2015}{25}}. En
    línea con esta evidencia, recientemente se ha demostrado que en los
    casos severos por \emph{P. vivax} la parasitemia periférica
    subestima la biomasa parasitaria total, siendo la fracción oculta la
    mayor contribuyente de citoquinas proinflamatorias, semejante a los
    casos por \emph{P.
    falciparum}\textsuperscript{\protect\hyperlink{ref-barber2015}{31}}.
  \end{enumerate}
\end{enumerate}

\subsubsection{Respuesta de anticuerpos}\label{respuesta-de-anticuerpos}

La concentración de anticuerpos contra antígenos de \emph{Plasmodium} es
un biomarcador sensible a la exposición de
malaria\textsuperscript{\protect\hyperlink{ref-elliott2014}{13}}. Sin
embargo, las evidencias a favor de dianas para \emph{P. vivax} continúan
estando limitadas por la ausencia de cultivos \emph{in vitro} y modelos
animales de infección. Este problema se ve reflejado en su escaso número
de candidatos a vacuna (PvDBP y PvCSP) en comparación a los 23 de
\emph{P.
falciparum}\textsuperscript{\protect\hyperlink{ref-rainbow2016}{32}}.
Por ello, la principal fuente de evidencia a favor de biomarcadores para
\emph{P. vivax} está en la identificación de antígenos con respuestas de
anticuerpos asociadas a la adquisición natural de inmunidad en
poblaciones endémicas.

\begin{enumerate}
\def\labelenumi{\alph{enumi}.}
\item
  Marcadores

  \begin{enumerate}
  \def\labelenumii{\roman{enumii}.}
  \item
    \textbf{de exposición}

    Estudios longitudinales han sugerido que, en un inicio, la
    intensidad de la respuesta de anticuerpos actúa como marcador de
    exposición la cual, luego de una constante exposición, incrementa
    hasta superar un umbral de
    protección\textsuperscript{\protect\hyperlink{ref-Stanisic2015}{26}}.
    En este sentido, estudios seroepidemiológicos en \emph{P. vivax} han
    logrado asociar marcadores con exposición (PvCSP,
    PvMSP-1\textsubscript{19}, PvMSP-9\textsubscript{RIRII} y PvAMA-1) e
    inmunidad (PvMSP-1\textsubscript{19}, PvMSP-1\textsubscript{NT},
    PvMSP-3\(\alpha\) y
    PvMSP-9\textsubscript{NT})\textsuperscript{\protect\hyperlink{ref-cutts2014meta}{33}}.
  \item
    \textbf{de inmunidad}

    En áreas de alta transmisión, los niños adquieren resistencia a la
    \emph{manifestación severa} de la malaria a la edad de cinco años,
    aproximadamente. Sin embargo, continúan siendo susceptibles a
    episodios no-complicados hasta la adolescencia, donde adquieren un
    estado resistente a la \emph{malaria sintomática}. A pesar de ello,
    no se ha demostrado que esta adquisición natural de inmunidad por la
    exposición acumulada a malaria otorgue una \emph{protección
    esterilizante} o de resistencia a la
    infección\textsuperscript{\protect\hyperlink{ref-crompton2014rev}{34}}.
    Los detalles de los procesos involucrados en cada una de estas tres
    fases no están completamente entendidos, pero se tiene detalles de
    algunos componentes:

    \begin{enumerate}
    \def\labelenumiii{\arabic{enumiii}.}
    \item
      \textbf{contra la severidad}

      La adquisición de anticuerpos que bloqueen la adhesión de iRBC o
      invasión parasitaria son candidatos a otorgar una protección a
      este
      nivel\textsuperscript{\protect\hyperlink{ref-wassmer2015}{25}}.
      Antígenos referenciales son (i) las proteínas VIR en \emph{P.
      vivax}, homólogas a VAR en \emph{P. falciparum}, posiblemente
      secretadas en iRBC y causantes de adhesividad a endotelio o RBC no
      infectados, condición que desencadena el rosetamiento y posterior
      obstrucción
      microvascular\textsuperscript{\protect\hyperlink{ref-portillo2001vir}{35}},
      y (ii) las proteínas de la familia PvDBP y PvRBP (\emph{Duffy-} y
      \emph{Reticulocyte- binding proteins}), responsables de la
      invasión a reticulocitos por la ruta tradicional y alternativa,
      respectivamente\textsuperscript{\protect\hyperlink{ref-galinski1992rbp}{36}}.
    \item
      \textbf{contra la enfermedad}

      En el caso de \emph{P. vivax}, su adquisición ocurre con una mayor
      rapidez, posiblemente facilitada por sus particularidades
      biológicas\textsuperscript{\protect\hyperlink{ref-mueller2013}{37}}.
      Los antígenos propuestos hasta el momento para esta especie son
      proteínas del micronema de merozoitos como DBP y AMA1, y proteínas
      de superficie como MSP1, MSP1-P (región C-terminal), MSP3
      (PvMSP-3\(\alpha\), PvMSP-3\(\beta\)) y
      MSP9\textsuperscript{\protect\hyperlink{ref-lopez2017}{38}}. 
    \item
      \textbf{contra la parasitemia}

      La adquisición de una inmunidad esterilizante solo ha sido
      comprobada mediante la inmunización con esporozoitos atenuados por
      radiación. Tanto en \emph{P. falciparum} como en \emph{P. vivax}
      se ha confirmado esta respuesta. Un reciente ensayo clínico de
      fase 2 en Colombia con \emph{P. vivax} encontró que luego de siete
      inmunizaciones a lo largo de 56 semanas y una reexposición con
      esporozoitos infecciosos, 5/12 voluntarios obtuvieron dicha
      protección y la respuesta de anticuerpos IgG1 anti-PvCSP estuvo
      asociada a
      ella\textsuperscript{\protect\hyperlink{ref-arevalo2016spz}{39}}. 
    \end{enumerate}
  \end{enumerate}
\item
  Enfoques a larga escala

  El parásito de la malaria, al tener un ciclo de vida complejo con
  múltiples estadios de desarrollo en el humano, posee también un
  transcriptoma, proteoma y, por lo tanto, un \emph{inmunoma}
  estadio-específico que deriva de un total de \textasciitilde{}5300
  proteínas putativas. Por este motivo, tanto los esfuerzos de vacunas
  por subunidades como por organismos completos atenuados han fracasado,
  en contraste a su eficacia en otras infecciones como hepatitis B
  (HBsAg) y tuberculosis (BCG),
  respectivamente\textsuperscript{\protect\hyperlink{ref-immunomics2016}{40}}.

  \begin{enumerate}
  \def\labelenumii{\roman{enumii}.}
  \item
    \textbf{Inmunómica}

    En respuesta a las limitaciones de la \emph{vacunología reversa},
    basada en el uso inicial de la información genómica de patógenos, el
    enfoque \emph{inmunómico} hace uso de la información de ambos
    agentes interactuantes: patógeno y hospedero. Su estrategia
    selecciona inmunógenos tanto por (1) \emph{métodos empíricos},
    usando data transcriptómica y proteómica de patógenos dinámicos más
    la información clínica del hospedero, como por (2) \emph{métodos
    teóricos}, empleando algoritmos computacionales predictivos que
    tomen en cuenta la afinidad entre péptidos del patógeno y moléculas
    del complejo principal de histocompatibilidad (MHC) del hospedero.

    Particularmente, el primer método ha permitido obtener datos
    empíricos sobre diferencias en la amplitud, intensidad, cinética y
    longevidad de respuestas inmunes inducidas por patógenos. Además de
    mayor información sobre aspectos clave como el porcentaje del
    proteoma reconocido por el hospedero, la capacidad predictiva de
    anticuerpos sobre el estado de la enfermedad o de las secuencias de
    aminoácidos sobre su
    inmunogenicidad\textsuperscript{\protect\hyperlink{ref-Davies2015Large}{41}}.
  \end{enumerate}
\end{enumerate}

\subsubsection{Microarreglos de
proteínas}\label{microarreglos-de-proteinas}

Esta es una técnica empleada en la medición a larga escala de las
interacciones o actividades de proteínas capaz de reconstruir redes
biológicas entre diferentes clases de moléculas y estados
celulares\textsuperscript{\protect\hyperlink{ref-uzoma2013interactome}{42}}.
Su empleo en el contexto \emph{inmunómico}, con la cuantificación de las
interacciones antígeno-anticuerpo, posee características analíticas
similares a los microarreglos de
ADN\textsuperscript{\protect\hyperlink{ref-sundaresh2006}{43}}, lo que
ha facilitado la adaptación de los componentes estandarizados para el
diseño de experimentos en esta
tecnología\textsuperscript{\protect\hyperlink{ref-allison2006}{44}}.

\begin{enumerate}
\def\labelenumi{\alph{enumi}.}
\item
  Componentes

  \begin{enumerate}
  \def\labelenumii{\roman{enumii}.}
  \item
    \textbf{Diseño}

    El desarrollo de un plan de experimentación tiene el objetivo de
    maximizar la calidad y cantidad de la información obtenida. Una de
    las secciones clave dentro del MIAME o ``información mínima sobre
    experimentos basados en
    microarreglos''\textsuperscript{\protect\hyperlink{ref-brazma2001}{45}}
    es el detalle de la información de los elementos a incluir (e.g.,
    secuencias de moléculas correspondiente a genes específicos). El
    primer diseño de microarreglos para \emph{P. vivax} y \emph{P.
    falciparum} fue estandarizado por Finney et al.
    2014\textsuperscript{\protect\hyperlink{ref-Finney2014}{46}} en base
    a la evidencia transcriptómica, proteómica y características
    inmunogénicas de proteínas predichas en ambos genomas.
    Posteriormente, en el 2015 el Centro Internacional de Excelencia
    para la Investigación de la Malaria (ICEMR) realizó una subselección
    empírica de este microarreglo con 142 muestras de 10 paises con
    malaria endémica, entre ellos
    Perú\textsuperscript{\protect\hyperlink{ref-King2015FOC}{47}}.
  \item
    \textbf{Pre procesamiento}

    Con el objetivo de remover variaciones sistemáticas en el
    experimento, se ejecutan en forma consecutiva los tres siguientes
    pasos:

    \begin{enumerate}
    \def\labelenumiii{\arabic{enumiii}.}
    \tightlist
    \item
      Normalización: Permite homogeneizar la variabilidad entre arreglos
      en base a controles, los cuales se asume que son invariantes entre
      muestras. Este procedimiento es ejecutado tradicionalmente a
      través de la sustracción de controles o \emph{fold over control}
      (FOC), obteniendo el cociente de las lecturas objetivo con
      respecto a la mediana de los controles negativos por
      muestra\textsuperscript{\protect\hyperlink{ref-King2015FOC}{47}}; 
    \item
      Transformación: Reduce la heterocedasticidad o heterogeneidad de
      varianzas entre las observaciones por errores aleatorios en el
      proceso de
      hibridación\textsuperscript{\protect\hyperlink{ref-kreil2005bullet}{48}}.
      Este fenómeno se evidencia en el incremento de la varianza
      directamente proporcional al incremento de su
      intensidad\textsuperscript{\protect\hyperlink{ref-brown2001image}{49}}.
      Dependiendo de la dispersión de los datos, se asumen modelos de
      error multiplicativo o aditivo, donde la transformación
      logarítmica o arcoseno hiperbólico es ejecutada sobre los valores
      normalizados. 
    \item
      Filtrado: En experimentos a larga escala permite incrementar el
      poder de detección de elementos con expresión
      diferenciada\textsuperscript{\protect\hyperlink{ref-bourgon2010filter}{50}}.
      En estos, las estrategias de corrección posteriores a la
      comparación de múltiples hipótesis son sensibles a esta cantidad.
      En este sentido, este procedimiento retira de forma preliminar a
      los elementos con reducidas probabilidades de expresarse
      diferencialmente .
    \end{enumerate}
  \item
    \textbf{Inferencia}

    Su objetivo es poner a prueba hipótesis estadísticas relacionadas a
    la expression o reactividad diferencial de elementos entre grupos.
    Sin embargo, a esta escala se posee dos principales problemas: (1)
    presencia de varianzas artificialmente altas o bajas producto del
    bajo número de réplicas entre arreglos o
    muestras\textsuperscript{\protect\hyperlink{ref-baldi2001cybert}{51}};
    y (2) un amplio número de hipótesis puestas a prueba en forma
    simultánea, donde los test estadísticos tradicionales pueden generar
    un largo número de falsos
    positivos\textsuperscript{\protect\hyperlink{ref-kayala2012cyber}{52}}.

    \begin{enumerate}
    \def\labelenumiii{\arabic{enumiii}.}
    \tightlist
    \item
      Test-t moderado: Un enfoque bayesiano empírico permite reducir (o
      moderar) las varianzas de todas las lecturas. Bajo el supuesto que
      las diferencias entre grupos generan cambios en la intensidad
      promedio del gen mas no en su varianza, se estima una varianza
      general como probabilidad previa para actualizar (corregir) todas
      las varianzas observadas en el
      experimento\textsuperscript{\protect\hyperlink{ref-smyth2004ebayes}{53}}
      Con ello, es posible obtener inferencias más estables que el
      test-t ordinario bajo un contexto limitado de
      réplicas\textsuperscript{\protect\hyperlink{ref-kayala2012cyber}{52}}.
    \item
      Razón de falsos descubrimientos (FDR): Los métodos de corrección o
      ajuste de valores P permiten que el investigador pueda controlar
      la razón de falsos descubrimientos con respecto al total de
      hipótesis
      positivas\textsuperscript{\protect\hyperlink{ref-brazma2001}{45}}.
      El método Benjamini-Hochberg determina un valor crítico
      dependiente del total de hipótesis puestas a prueba y el FDR que
      se desee tolerar, comúnmente del
      5\%\textsuperscript{\protect\hyperlink{ref-benjamini1995fdr}{54}}.
    \end{enumerate}
  \item
    \textbf{Clasificación}

    Este proceso involucra la asignación de objetos (e.g., genes) en
    categorías pre-existentes (clasificación supervisada), o el
    desarrollo progresivo de un conjunto de categorías por
    características de los objetos. (clasificación no
    supervisada)\textsuperscript{\protect\hyperlink{ref-allison2006}{44}}.
    Ejemplos de esta técnica son \emph{support vector machines} (SVM)
    con validación por \emph{leave-one-out cross-validation} (LOOCV) y
    el agrupamiento jerárquico o \emph{hierarchical clustering} de
    acuerdo a la distancia Euclidiana entre elementos.
  \item
    \textbf{Parámetros adicionales}

    Además de la reactividad serológica, dos parámetros han demostrado
    ser de vital importancia para la caracterización de la respuesta
    humoral: la amplitud e intensidad de
    respuesta\textsuperscript{\protect\hyperlink{ref-crompton2010}{9},\protect\hyperlink{ref-Helb2015exposure}{10},\protect\hyperlink{ref-King2015FOC}{47}}.
    El primero se obtiene calculando el número de antígenos reactivos
    por individuo y el segundo por el promedio de sus intensidades.
    Particularmente, la amplitud ha demostrado ser tan relevante como la
    reactividad de antígenos individuales en el desarrollo de inmunidad
    contra la
    malaria\textsuperscript{\protect\hyperlink{ref-crompton2010}{9}}. 
  \end{enumerate}
\end{enumerate}

\subsection{Definiciones conceptuales}\label{definiciones-conceptuales}

\begin{enumerate}
\def\labelenumi{\alph{enumi}.}
\item
  Acrónimos

  \textbf{IVTT:} (Proteína) transcrita/traducida \emph{in vitro}.

  \textbf{MFI:} Unidad de lectura cruda expresada como intensidad
  fluorescente promedio de todos los píxeles de cada \emph{spot},
  normalizada localmente mediante la sustracción de la intensidad de
  fondo presente a su alrededor.

  \textbf{RTS:} Sistema rápido de traducción de proteínas recombinantes
  libre de células.
\item
  Términos

  \textbf{Antígeno-IVTT:} Antígeno objetivo o \emph{spot} con proteína
  IVTT a partir de un plásmido con ADN insertado correspondiente al
  polipéptido, segmento o exón de interés.

  \textbf{Control-IVTT:} Control negativo o \emph{spot} con mix de
  expresión RTS y plásmido sin ADN insertado, representante de la
  intensidad de fondo específica del paciente.

  \textbf{Proteína purificada:} Control de comparación o \emph{spot} con
  proteína de antigenicidad conocida expresada dentro de célula.

  \textbf{Intensidad de antígeno:} Lectura normalizada de cada
  antígeno-IVTT entre individuos. También llamada ``reactividad
  serológica''.

  \textbf{Antígeno reactivo:} Lectura transformada de cada antígeno-IVTT
  mayor o igual a dos veces la mediana de los control-IVTT por
  individuo.
\end{enumerate}

\section{METODOLOGÍA}\label{meto}

\subsection{Diseño}\label{diseno}

El diseño del estudio es de tipo Caso-Control. Su aplicación consistirá
en la selección de la población objetivo por presencia (casos) o
ausencia (controles) del evento en estudio. Además se fijará el número
de eventos a estudiar, así como el número de sujetos sin evento que se
incluirán en la población de comparación.

\subsubsection{Tipo de investigación}\label{tipo-de-investigacion}

\textbf{Por la finalidad: Analítico.} A diferencia de un descriptivo,
realizaremos comparaciones entre grupos y evaluaremos una posible
relación causal entre el factor y el efecto.

\textbf{Por el control de la asignación: Observacional.} A diferencia de
un experimental, no controlaremos la asignación de los factores
(severidad y episodios previos). Es decir, procederemos a observar el
fenómeno, ejecutar la medición y analizar los resultados.

\textbf{Por el seguimiento: Transversal.} A diferencia de uno
longitudinal, no ejecutaremos un seguimiento. Las variables se medirán
una sola vez, en un mismo instante.

\textbf{Por la relación cronológica: Prospectivo.} A diferencia de un
retrospectivo, recolectaremos los datos después de planificado el
estudio.

\subsection{Población y muestra}\label{poblacion-y-muestra}

\subsubsection{Población}\label{poblacion}

La población está delimitada por pacientes diagnosticados con malaria
por \emph{P. vivax} de la ciudad de Iquitos, Loreto - Perú, entre enero
del 2012 y junio del 2013 a lo largo de un estudio prospectivo ejecutado
en dos hospitales de referencia: Hospital de Apoyo y Hospital Regional.

\subsubsection{Criterios de inclusión y
exclusión}\label{criterios-de-inclusion-y-exclusion}

\begin{enumerate}
\def\labelenumi{\alph{enumi}.}
\item
  Criterios de inclusión

  En el estudio prospectivo mencionado, 164 pacientes con malaria por
  \emph{P. vivax} fueron enrolados mediante vigilancia pasiva. Según la
  presencia de uno o más criterios clínicos y de laboratorio de la
  clasificación OMS para la Malaria
  Severa\textsuperscript{\protect\hyperlink{ref-WHO2014severe}{19}}, los
  pacientes fueron estratificados como severos o no-severos.

  Los criterios clínicos de la clasificación OMS son: Shock circulatorio
  (presión sanguínea sistólica \textless{} 80 mmHg), Deterioro del nivel
  de conciencia (puntaje Glasgow \(\le\) 9/14), Daño del sistema
  nerviosos central (convulsión), Daño pulmonar (síndrome de dificultad
  respiratoria aguda o edema pulmonar).

  Los criterios de laboratorio de la clasificación OMS son: Hipoglicemia
  (glucosa \textless{} 40 mg/dL), Anemia severa (hemoglobina \textless{}
  7mg/dL), Daño renal (creatinina \textgreater{} 3mg/dl), e
  Hiperbilirrubina (bilirrubina sérica \textgreater{} 2.5 mg/dL). 
\item
  Criterios de exclusión

  Dos criterios de exclusión fueron empleados: (i) Presencia de
  coinfecciones con especies de \emph{Plasmodium} distintas a \emph{P.
  vivax} determinadas por PCR, y (ii) presencia de coinfecciones como
  leptospirosis, dengue u otra arbovirosis determinada por técnicas de
  aislamiento viral e inmunofluorescencia. 
\end{enumerate}

\subsubsection{Selección de
participantes}\label{seleccion-de-participantes}

A partir de esta estratificación, se ejecutará una selección arbitraria
de 30 pacientes representativos de malaria severa con uno o más
criterios de la clasificación OMS para el grupo de casos y una selección
aleatoria simple de 30 no-severos para el grupo control.

\textbf{Tipo de muestra: No probabilística y Probabilística.} Los
elementos del grupo caso provendrán de una selección arbitraria,
mientras que el grupo control será seleccionado al azar, i.e.~con
probabilidad de selección conocida.

\subsection{Recolección de los datos e
instrumento}\label{recoleccion-de-los-datos-e-instrumento}

\subsubsection{Técnica para la recolección de
datos}\label{tecnica-para-la-recoleccion-de-datos}

Previo al tratamiento antimalárico del paciente, se extrajeron las
muestras de sangre tanto para el diagnóstico de malaria por \emph{P.
vivax} con la técnica de frotis como para las pruebas bioquímicas. Al
momento del diagnóstico positivo, bajo consentimiento informado y de
forma voluntaria, el plasma sanguíneo fue colectado y conservado a -80°C
hasta su uso.

\subsubsection{Instrumento de medición}\label{instrumento-de-medicion}

El presente estudio hará uso de microarreglos diseñados con 1014
fragmentos proteicos recombinantes (498 de \emph{P. falciparum} y 516 de
\emph{P. vivax}, expresado como Pf498/Pv516 o PfPv500) representando 873
proteínas predichas (427 de \emph{P. falciparum} y 446 de \emph{P.
vivax}, \textasciitilde{}8\% del total predicho para el genoma de
\emph{P. vivax} Sal1) seleccionadas luego de una extensiva evaluación
serológica con uno de mayor escala
(Pf2208/Pv2233)\textsuperscript{\protect\hyperlink{ref-Finney2014}{46}}.
Este diseño ha sido validado mediante el sondeo con plasma colectado de
pacientes con malaria y controles sanos alrededor del
mundo\textsuperscript{\protect\hyperlink{ref-King2015FOC}{47}} y
depositado en la base de datos GEO
(\href{https://www.ncbi.nlm.nih.gov/geo/query/acc.cgi?acc=GPL18316}{GPL18316}).

\paragraph{Aplicación}\label{aplicacion}

La aplicación del instrumento consiste en tres pasos: sondeo, escaneo y
análisis, tal como ha sido publicado
previamente\textsuperscript{\protect\hyperlink{ref-Driguez2015}{55}}.
Primero, previo al sondeo los microarreglos se hidratan con buffer de
bloqueo dentro de cámaras de incubación. Paralelamente, el plasma de los
pacientes se diluye con aquel buffer en 1:100 y se pre-absorbe con
lisado de \emph{E. coli} en 10\%(w/v), con el objetivo de reducir ruido
de fondo en las mediciones tanto por uniones inespecíficas con el
sustrato como por antígenos bacterianos del sistema de expresión RTS.
Segundo, aspirado el buffer de bloqueo del microarreglo, se procede con
el sondeo al agregar el plasma pre-absorbido e incubar \emph{overnight}
en cámara húmeda a 4°C con leve agitación. Tercero, con lavados y
aspirados repetitivos antes y después de cada paso, se agrega la
solución con anticuerpos secundarios biotinilados y posteriormente la
solución con fluoróforos conjugados con estreptavidina. Cuarto, luego de
centrifugar las láminas, se procede al escaneo de las fluorescencias con
lectores de microarreglos de láser confocal (e.g., Genepix 4300A). Las
medidas crudas se obtienen al aplicar una normalización local mediante
la sustracción de la intensidad de fondo presente alrededor de cada
\emph{spot}, ejecutada en el software del lector (Genepix Pro 7).
Finalmente, se procede al análisis de datos detallado en la
\protect\hyperlink{anadata}{sección 4.4}.

\hypertarget{validez}{\paragraph{Validez}\label{validez}}

La validez o exactitud del experimento será evaluada mediante la
correlación lineal entre las lecturas de los antígenos-IVTT y sus
correspondientes proteínas purificadas por muestra, compuestas por
proteínas de inmunogenicidad conocida y expresadas por un sistema dentro
de célula\textsuperscript{\protect\hyperlink{ref-crompton2010}{9}}.

\subsubsection{Codificación y creación del archivo de
datos}\label{codificacion-y-creacion-del-archivo-de-datos}

Cada paciente del estudio estará identificado con un código de
estructura \texttt{LIM\#\#\#\#}, e.g.: \texttt{LIM1063}. Los antígenos
proteicos estarán identificados con el código asignado a sus genes en la
base de datos PlasmoDB, e.g.: \texttt{PF3D7\_0202500} o
\texttt{PVX\_091315}, ya sea un gen de \emph{P. falciparum} o \emph{P.
vivax}, respectivamente. En caso los genes posean múltiples exónes, se
amplificarán por separado y se extenderá el código de cada uno con su
número de orden y el total, e.g.: \texttt{\_1o2} exón 1 de un gen con 2
exónes. En caso los genes posean una longitud mayor a 3000 nucleótidos
(nt), se dividirán en segmentos sobrelapantes entre 300-3000nt y se
extenderá el código de cada uno con su respectivo número, e.g.:
\texttt{\_S1} para el primer segmento de un gen.

Los datos serán organizados en dos matrices: (i) archivo
\texttt{samples.csv} con los códigos de los pacientes y sus covariables
epidemiológicas y (ii) archivo \texttt{RawData.csv} con los códigos de
las proteínas y sus lecturas crudas en MFI por código de paciente.

\hypertarget{anadata}{\subsection{Análisis de datos}\label{anadata}}

Se hará uso del software de computación estadística
R\textsuperscript{\protect\hyperlink{ref-R2016}{17}}, complementado con
funciones de distintos paquetes pertenecientes a dos principales
ambientes de análisis:
Bioconductor\textsuperscript{\protect\hyperlink{ref-bioconductor2004}{18}}
y
Tidyverse\textsuperscript{\protect\hyperlink{ref-wickham2016r4ds}{56}}.

\subsubsection{Control de calidad y pre
procesamiento}\label{control-de-calidad-y-pre-procesamiento}

Preliminarmente, se describirá la distribución y proporción de las
covariables epidemiológicas, clínicas y bioquímicas de los pacientes de
la muestra. Luego, con las lecturas crudas en MFI se evaluará la validez
del ensayo mediante un test de asociación entre variables continuas con
la correlación de Pearson (\(r\)) o Spearman (\(\rho\)), de acuerdo a la
distribución, asumiendo un error del 5\%. Posteriormente, se procederá
con la normalización entre muestras de cada antígeno-IVTT con respecto a
la mediana de los control-IVTT por individuo, transformación a escala
logarítmica en base 2, y filtrado de todo antígeno que posea una
frecuencia de reactividad menor al 10\% de individuos en la muestra.
Para ello, un antígeno reactivo será definido operacionalmente como el
antígeno-IVTT con intensidad normalizada mayor o igual a 1. Por último,
se asociarán ambas matrices de datos en un \texttt{ExpressionSet}, a
través de los códigos de pacientes, empleando el paquete
\texttt{Biobase}\textsuperscript{\protect\hyperlink{ref-Biobase}{57}}.

\subsubsection{Prueba de hipótesis}\label{prueba-de-hipotesis}

Las dos hipótesis de diferencia entre grupos seguirán el siguiente
protocolo. Primero, se contrastará la amplitud e intensidad de respuesta
con un test de diferencias entre variables continuas y no pareadas de
dos grupos usando t-Student o Mann-Whitney, dependiendo de la
distribución, con un 5\% de error asumido. Segundo, se realizará un test
de reactividad diferenciada de anticuerpos entre dos grupos, usando el
test-t
moderado\textsuperscript{\protect\hyperlink{ref-smyth2004ebayes}{53}}
con corrección por comparación múltiple de la razón de falsos
descubrimientos (FDR) por el método de Benjamini-Hochberg, disponible en
el paquete
\texttt{limma}\textsuperscript{\protect\hyperlink{ref-limma}{58}}, con
un 5\% de error asumido. Tercero, se realizará un agrupamiento
jerárquico o \emph{hierarchical clustering} de los antígenos
identificados en base a su distancia euclidiana, disponible en el
paquete
\texttt{NMF}\textsuperscript{\protect\hyperlink{ref-Gaujoux2010NMF}{59}}.
Finalmente, se describirá la presencia de dominios transmembrana,
péptido señal, número de ortólogos en \emph{Plasmodium}, ontología
génica, número de polimorfismos de nucleótido simple y la razón de
mutaciones no-sinónimas por sinónimas para el subgrupo de interés,
disponible en la base de datos
PlasmoDB\textsuperscript{\protect\hyperlink{ref-plasmodb}{60}}.

\subsubsection{Reporte de resultados}\label{reporte-de-resultados}

Las correlaciones y pruebas de hipótesis serán reportadas con el valor
del estadístico de prueba y el valor P. Las distribuciones serán
visualizadas con diagramas de dispersión para la correlación entre
variables continuas, diagramas de cajas y barras para la comparación de
variables continuas y frecuencias, respectivamente. Las comparaciones
múltiples reportarán el valor P-ajustado y serán visualizadas con dos
gráficos: un diagrama tipo volcán o \emph{volcano plot} con el tamaño
del efecto o diferencia de reactividad entre grupos en escala
logarítmica contra sus respectivos valores P y un mapa de calor o
\emph{heat map} segmentado por racimos o \emph{clusters}. El informe
final consistirá de un reporte en formato \texttt{R\ Notebook} con
extensión \texttt{.Rmd} que integrará texto, código y
resultados\textsuperscript{\protect\hyperlink{ref-knitr}{61}}, siguiendo
principios de
reproducibilidad\textsuperscript{\protect\hyperlink{ref-CienciaReproducible2016}{62}}.

\section{ASPECTOS ADMINISTRATIVOS}\label{aspectos-administrativos}

\subsection{Cronograma de actividades}\label{cronograma-de-actividades}

El Proyecto de Tesis fue trabajado en el Departamento de Parasitología
del Centro de Enfermedades Tropicales de la Marina de los Estados Unidos
NAMRU-6, del 01 de enero del 2016 al 31 de diciembre del 2016, de
acuerdo al cronograma de la \autoref{tab:crono}.

\begin{table}[ht]
        \captionof{table}{Cronograma}
        \label{tab:crono}
        \vspace{2mm}
\begin{center}
%\hspace*{-1cm}
\begin{tabular}{lcccccccccccc}
  \hline
  \textbf{ACTIVIDAD} & 
  \textbf{2016} & & & & & & & & & & &\\
  \hline
  & 
  ene & feb & mar & abr & may & jun & jul & ago & set & oct & nov & dic\\
  \cline{2-13}
  Recolección de datos & 
  x & & & & & & & & & & &\\
  \cline{2-13}
  Análisis de datos & 
  & x & x & x & x & x & & & & & &\\
  \cline{2-13}
  Interpretación de resultados & 
  & & & & & x & x & x & & & &\\
  %\cline{2-13}
  %Redacción de proyecto & 
  %& & & & & & & & & & &\\
  %\cline{2-13}
  %Redacción del Informe & 
  %& & & & & & & & & & &\\
  \cline{2-13}
  Redacción de discusión & 
  & & & & & & & x & x & x & &\\
  \cline{2-13}
  Redacción de conclusiones & 
  & & & & & & & & & x & x & x\\
  %\cline{2-13}
  %Correcciones & 
  %& & & & & & & & & & &\\
  %\cline{2-13}
  %Sustentación & 
  %& & & & & & & & & & &\\
  %\hline
  %d1 & 
  %& & & & & & & & & & & & 
  %& & & & & & & & & & &\\
  \hline
  % etc. ...
\end{tabular}
%\hspace*{-1cm}
\end{center}
\end{table}

\subsection{Presupuesto y
financiamiento}\label{presupuesto-y-financiamiento}

El presupuesto que se detalla en la \autoref{tab:presup} fue ejecutado
con financiamiento interno de la institución.

\begin{longtable}[]{@{}lcc@{}}
\caption{Presupuesto \label{tab:presup}}\tabularnewline
\toprule
\begin{minipage}[b]{0.44\columnwidth}\raggedright\strut
\textbf{DESCRIPCIÓN}\strut
\end{minipage} & \begin{minipage}[b]{0.23\columnwidth}\centering\strut
\textbf{MONTO (S/)}\strut
\end{minipage} & \begin{minipage}[b]{0.23\columnwidth}\centering\strut
\textbf{PORCENTAJE (\%)}\strut
\end{minipage}\tabularnewline
\midrule
\endfirsthead
\toprule
\begin{minipage}[b]{0.44\columnwidth}\raggedright\strut
\textbf{DESCRIPCIÓN}\strut
\end{minipage} & \begin{minipage}[b]{0.23\columnwidth}\centering\strut
\textbf{MONTO (S/)}\strut
\end{minipage} & \begin{minipage}[b]{0.23\columnwidth}\centering\strut
\textbf{PORCENTAJE (\%)}\strut
\end{minipage}\tabularnewline
\midrule
\endhead
\begin{minipage}[t]{0.44\columnwidth}\raggedright\strut
\textbf{Bienes}\strut
\end{minipage} & \begin{minipage}[t]{0.23\columnwidth}\centering\strut
\strut
\end{minipage} & \begin{minipage}[t]{0.23\columnwidth}\centering\strut
\strut
\end{minipage}\tabularnewline
\begin{minipage}[t]{0.44\columnwidth}\raggedright\strut
Papelería, útiles y material de oficina\strut
\end{minipage} & \begin{minipage}[t]{0.23\columnwidth}\centering\strut
50\strut
\end{minipage} & \begin{minipage}[t]{0.23\columnwidth}\centering\strut
0.2\strut
\end{minipage}\tabularnewline
\begin{minipage}[t]{0.44\columnwidth}\raggedright\strut
Insumos e instrumental de laboratorio\strut
\end{minipage} & \begin{minipage}[t]{0.23\columnwidth}\centering\strut
200\strut
\end{minipage} & \begin{minipage}[t]{0.23\columnwidth}\centering\strut
0.6\strut
\end{minipage}\tabularnewline
\begin{minipage}[t]{0.44\columnwidth}\raggedright\strut
\textbf{Servicios}\strut
\end{minipage} & \begin{minipage}[t]{0.23\columnwidth}\centering\strut
\strut
\end{minipage} & \begin{minipage}[t]{0.23\columnwidth}\centering\strut
\strut
\end{minipage}\tabularnewline
\begin{minipage}[t]{0.44\columnwidth}\raggedright\strut
Compra, sondeo y lectura de microarreglos\strut
\end{minipage} & \begin{minipage}[t]{0.23\columnwidth}\centering\strut
29610\strut
\end{minipage} & \begin{minipage}[t]{0.23\columnwidth}\centering\strut
92.9\strut
\end{minipage}\tabularnewline
\begin{minipage}[t]{0.44\columnwidth}\raggedright\strut
Gastos en el transporte de muestras\strut
\end{minipage} & \begin{minipage}[t]{0.23\columnwidth}\centering\strut
2000\strut
\end{minipage} & \begin{minipage}[t]{0.23\columnwidth}\centering\strut
6.3\strut
\end{minipage}\tabularnewline
\begin{minipage}[t]{0.44\columnwidth}\raggedright\strut
\textbf{TOTAL}\strut
\end{minipage} & \begin{minipage}[t]{0.23\columnwidth}\centering\strut
31910\strut
\end{minipage} & \begin{minipage}[t]{0.23\columnwidth}\centering\strut
100\strut
\end{minipage}\tabularnewline
\bottomrule
\end{longtable}

\section{BIBLIOGRAFÍA}\label{bibliografia}

\hypertarget{refs}{}
\hypertarget{ref-WHO2016world}{}
1. WHO. \emph{World Malaria Report 2016}. Vol 13. Geneva: World Health
Organization; 2016.
\url{http://www.who.int/malaria/publications/world-malaria-report-2016/report/en/}.

\hypertarget{ref-rosas2016peru}{}
2. Rosas-Aguirre A, Gamboa D, Manrique P, et al. Epidemiology of
plasmodium vivax malaria in Peru. \emph{The American Journal of Tropical
Medicine and Hygiene}. 2016;95(6 Suppl):133-144.
doi:\href{https://doi.org/10.4269/ajtmh.16-0268}{10.4269/ajtmh.16-0268}.

\hypertarget{ref-baird2009}{}
3. Baird JK. Severe and fatal vivax malaria challenges 'benign tertian
malaria' dogma. \emph{Annals of tropical paediatrics}.
2009;29(4):251-252.
doi:\href{https://doi.org/10.1179/027249309X12547917868808}{10.1179/027249309X12547917868808}.

\hypertarget{ref-quispe2014}{}
4. Quispe AM, Pozo E, Guerrero E, et al. Plasmodium vivax
hospitalizations in a monoendemic malaria region: Severe vivax malaria?
\emph{The American journal of tropical medicine and hygiene}.
2014;91(1):11-17.
doi:\href{https://doi.org/10.4269/ajtmh.12-0610}{10.4269/ajtmh.12-0610}.

\hypertarget{ref-reyburn2015}{}
5. Reyburn H, Mbatia R, Drakeley C, et al. Association of transmission
intensity and age with clinical manifestations and case fatality of
severe plasmodium falciparum malaria. \emph{JAMA}.
2005;293(12):1461-1470.
doi:\href{https://doi.org/10.1001/jama.293.12.1461}{10.1001/jama.293.12.1461}.

\hypertarget{ref-norma2001}{}
6. MINSA. \emph{Norma Técnica de Salud Para La Atención de La Malaria Y
Malaria Grave En El Perú}. Ministerio de Salud; 2007.
\url{http://www.minsa.gob.pe/portada/esnemo_normatividad.asp}.

\hypertarget{ref-accelerate2016}{}
7. Quispe AM, Llanos-Cuentas A, Rodriguez H, et al. Accelerating to
zero: Strategies to eliminate malaria in the Peruvian Amazon. \emph{The
American Journal of Tropical Medicine and Hygiene}.
2016;94(6):1200-1207.
doi:\href{https://doi.org/10.4269/ajtmh.15-0369}{10.4269/ajtmh.15-0369}.

\hypertarget{ref-baldevi2013}{}
8. Baldeviano GC, Leiva KP, Quispe AM, et al. Serum markers of severe
clinical complications during plasmodium vivax malaria monoinfections in
the Peruvian Amazon basin. In: \emph{Abstract Book of the Astmh 62nd
Annual Meeting, Nov. 13--17, Washington D.C., United States}.; 2013:340.
\url{http://www.astmh.org/ASTMH/media/Documents/AbstractBook2013Final.pdf}.

\hypertarget{ref-crompton2010}{}
9. Crompton PD, Kayala MA, Traore B, et al. A prospective analysis of
the ab response to plasmodium falciparum before and after a malaria
season by protein microarray. \emph{Proceedings of the National Academy
of Sciences}. 2010;107(15):6958-6963.
doi:\href{https://doi.org/10.1073/pnas.1001323107}{10.1073/pnas.1001323107}.

\hypertarget{ref-Helb2015exposure}{}
10. Helb DA, Tetteh KKA, Felgner PL, et al. Novel serologic biomarkers
provide accurate estimates of recent plasmodium falciparum exposure for
individuals and communities. \emph{Proceedings of the National Academy
of Sciences}. 2015;112(32):E4438-E4447.
doi:\href{https://doi.org/10.1073/pnas.1501705112}{10.1073/pnas.1501705112}.

\hypertarget{ref-griffing2013history}{}
11. Griffing SM, Gamboa D, Udhayakumar V. The history of 20 th century
malaria control in Peru. \emph{Malaria journal}. 2013;12(1):303.
doi:\href{https://doi.org/10.1186/1475-2875-12-303}{10.1186/1475-2875-12-303}.

\hypertarget{ref-hotspots2015}{}
12. Rosas-Aguirre A, Speybroeck N, Llanos-Cuentas A, et al. Hotspots of
malaria transmission in the Peruvian Amazon: Rapid assessment through a
parasitological and serological survey. \emph{PLOS ONE}.
2015;10(9):1-21.
doi:\href{https://doi.org/10.1371/journal.pone.0137458}{10.1371/journal.pone.0137458}.

\hypertarget{ref-elliott2014}{}
13. Elliott SR, Fowkes F, Richards JS, Reiling L, Drew DR, Beeson JG.
Research priorities for the development and implementation of
serological tools for malaria surveillance. \emph{F1000Prime Rep}.
2014;6:100. doi:\href{https://doi.org/10.12703/P6-100}{10.12703/P6-100}.

\hypertarget{ref-arevalo2014}{}
14. Arévalo-Herrera M, Forero-Peña DA, Rubiano K, et al. Plasmodium
vivax sporozoite challenge in malaria-naive and semi-immune colombian
volunteers. \emph{PLoS One}. 2014;9(6):e99754.
doi:\href{https://doi.org/10.1371/journal.pone.0099754}{10.1371/journal.pone.0099754}.

\hypertarget{ref-vigil2010}{}
15. Vigil A, Davies DH, Felgner PL. Defining the humoral immune response
to infectious agents using high-density protein microarrays.
\emph{Future microbiology}. 2010;5(2):241-251.
doi:\href{https://doi.org/10.2217/fmb.09.127}{10.2217/fmb.09.127}.

\hypertarget{ref-leroch2009postmod}{}
16. Chung D-WD, Ponts N, Cervantes S, Le Roch KG. Post-translational
modifications in plasmodium: More than you think! \emph{Molecular and
biochemical parasitology}. 2009;168(2):123-134.
doi:\href{https://doi.org/10.1016/j.molbiopara.2009.08.001}{10.1016/j.molbiopara.2009.08.001}.

\hypertarget{ref-R2016}{}
17. R Core Team. \emph{R: A Language and Environment for Statistical
Computing}. Vienna, Austria: R Foundation for Statistical Computing;
2016. \url{https://www.R-project.org/}.

\hypertarget{ref-bioconductor2004}{}
18. Gentleman RC, Carey VJ, Bates DM, et al. Bioconductor: Open software
development for computational biology and bioinformatics. \emph{Genome
biology}. 2004;5(10):R80.
doi:\href{https://doi.org/10.1186/gb-2004-5-10-r80}{10.1186/gb-2004-5-10-r80}.

\hypertarget{ref-WHO2014severe}{}
19. WHO. Severe malaria. \emph{Trop Med Int Health}. 2014;19:7-131.
doi:\href{https://doi.org/10.1111/tmi.12313_2}{10.1111/tmi.12313\_2}.

\hypertarget{ref-Torres2014asymptomatic}{}
20. Torres KJ, Castrillon CE, Moss EL, et al. Genome-level determination
of plasmodium falciparum blood-stage targets of malarial clinical
immunity in the Peruvian Amazon. \emph{Journal of Infectious Diseases}.
November 2014.
doi:\href{https://doi.org/10.1093/infdis/jiu614}{10.1093/infdis/jiu614}.

\hypertarget{ref-chuquiyauri2015vivax}{}
21. Chuquiyauri R, Molina DM, Moss EL, et al. Genome-scale protein
microarray comparison of human antibody responses in plasmodium vivax
relapse and reinfection. \emph{The American journal of tropical medicine
and hygiene}. 2015;93(4):801-809.
doi:\href{https://doi.org/10.4269/ajtmh.15-0232}{10.4269/ajtmh.15-0232}.

\hypertarget{ref-howes2016global}{}
22. Howes RE, Battle KE, Mendis KN, et al. Global epidemiology of
plasmodium vivax. \emph{The American Journal of Tropical Medicine and
Hygiene}. 2016;95(6 Suppl):15-34.
doi:\href{https://doi.org/10.4269/ajtmh.16-0141}{10.4269/ajtmh.16-0141}.

\hypertarget{ref-path2011}{}
23. PATH. \emph{Staying the Course? Malaria Research and Development in
a Time of Economic Uncertainty}. Seattle, WA: PATH; 2011.
\url{www.malariavaccine.org/files/RD-report-June2011.pdf}.

\hypertarget{ref-gagnon2002enso}{}
24. Gagnon AS, Smoyer-Tomic KE, Bush AB. The El Niño southern
oscillation and malaria epidemics in South America. \emph{International
Journal of Biometeorology}. 2002;46(2):81-89.
doi:\href{https://doi.org/10.1007/s00484-001-0119-6}{10.1007/s00484-001-0119-6}.

\hypertarget{ref-wassmer2015}{}
25. Wassmer SC, Taylor TE, Rathod PK, et al. Investigating the
pathogenesis of severe malaria: A multidisciplinary and
cross-geographical approach. \emph{The American journal of tropical
medicine and hygiene}. 2015;93(3 Suppl):42-56.
doi:\href{https://doi.org/10.4269/ajtmh.14-0841}{10.4269/ajtmh.14-0841}.

\hypertarget{ref-Stanisic2015}{}
26. Stanisic DI, Fowkes FJI, Koinari M, et al. Acquisition of antibodies
against plasmodium falciparum merozoites and malaria immunity in young
children and the influence of age, force of infection, and magnitude of
response. \emph{Infection and Immunity}. 2015;83(2):646-660.
doi:\href{https://doi.org/10.1128/IAI.02398-14}{10.1128/IAI.02398-14}.

\hypertarget{ref-rogerson2007preg}{}
27. Rogerson SJ, Hviid L, Duffy PE, Leke RF, Taylor DW. Malaria in
pregnancy: Pathogenesis and immunity. \emph{The Lancet infectious
diseases}. 2007;7(2):105-117.
doi:\href{https://doi.org/10.1016/S1473-3099(07)70022-1}{10.1016/S1473-3099(07)70022-1}.

\hypertarget{ref-llanoschea2015}{}
28. Llanos-Chea F, Martínez D, Rosas A, Samalvides F, Vinetz JM,
Llanos-Cuentas A. Characteristics of travel-related severe plasmodium
vivax and plasmodium falciparum malaria in individuals hospitalized at a
tertiary referral center in lima, Peru. \emph{The American Journal of
Tropical Medicine and Hygiene}. 2015;93(6):1249-1253.
doi:\href{https://doi.org/10.4269/ajtmh.14-0652}{10.4269/ajtmh.14-0652}.

\hypertarget{ref-factores2001}{}
29. MINSA. \emph{Factores de Riesgo de La Malaria Grave En El Perú}.
Ministerio de Salud: Proyecto Vigía (MINSA-USAID); 2001.
\url{http://bvs.minsa.gob.pe/local/minsa/1772.pdf}.

\hypertarget{ref-smith2013}{}
30. Smith-Nuñez ES, Durand S, Baldeviano GC, et al. WHO criteria for
severe malaria in identifying severe vivax malaria: Preliminary data
from a study in Iquitos, Peru. In: \emph{Abstract Book of the Astmh 62nd
Annual Meeting, Nov. 13--17, Washington D.C., United States}.; 2013:398.
\url{http://www.astmh.org/ASTMH/media/Documents/AbstractBook2013Final.pdf}.

\hypertarget{ref-barber2015}{}
31. Barber BE, William T, Grigg MJ, et al. Parasite biomass-related
inflammation, endothelial activation, microvascular dysfunction and
disease severity in vivax malaria. \emph{PLoS Pathog}. 2015;11(1):1-13.
doi:\href{https://doi.org/10.1371/journal.ppat.1004558}{10.1371/journal.ppat.1004558}.

\hypertarget{ref-rainbow2016}{}
32. WHO. Malaria vaccine rainbow tables. In: World Health Organization.
Accessed: 15-june-2017.
\url{http://www.who.int/vaccine_research/links/Rainbow/en/index.html}.

\hypertarget{ref-cutts2014meta}{}
33. Cutts JC, Powell R, Agius PA, Beeson JG, Simpson JA, Fowkes FJ.
Immunological markers of plasmodium vivax exposure and immunity: A
systematic review and meta-analysis. \emph{BMC medicine}.
2014;12(1):150.
doi:\href{https://doi.org/10.1186/s12916-014-0150-1}{10.1186/s12916-014-0150-1}.

\hypertarget{ref-crompton2014rev}{}
34. Crompton PD, Moebius J, Portugal S, et al. Malaria immunity in man
and mosquito: Insights into unsolved mysteries of a deadly infectious
disease. \emph{Annual review of immunology}. 2014;32:157-187.
doi:\href{https://doi.org/10.1146/annurev-iy-32-060414-200001}{10.1146/annurev-iy-32-060414-200001}.

\hypertarget{ref-portillo2001vir}{}
35. Portillo HA del, Fernandez-Becerra C, Bowman S, et al. A superfamily
of variant genes encoded in the subtelomeric region of plasmodium vivax.
\emph{Nature}. 2001;410(6830):839-842.
doi:\href{https://doi.org/10.1038/35071118}{10.1038/35071118}.

\hypertarget{ref-galinski1992rbp}{}
36. Galinski MR, Medina CC, Ingravallo P, Barnwell JW. A
reticulocyte-binding protein complex of plasmodium vivax merozoites.
\emph{Cell}. 1992;69(7):1213-1226.
doi:\href{https://doi.org/10.1016/0092-8674(92)90642-P}{10.1016/0092-8674(92)90642-P}.

\hypertarget{ref-mueller2013}{}
37. Mueller I, Galinski MR, Tsuboi T, Arévalo-Herrera M, Collins WE,
King CL. Natural acquisition of immunity to plasmodium vivax:
Epidemiological observations and potential targets. \emph{Adv
Parasitol}. 2013;81:77-131.
doi:\href{https://doi.org/10.1016/B978-0-12-407826-0.00003-5}{10.1016/B978-0-12-407826-0.00003-5}.

\hypertarget{ref-lopez2017}{}
38. López C, Yepes-Pérez Y, Hincapié-Escobar N, Díaz-Arévalo D,
Patarroyo MA. What is known about the immune response induced by
plasmodium vivax malaria vaccine candidates? \emph{Frontiers in
immunology}. 2017;8.
doi:\href{https://doi.org/10.3389/fimmu.2017.00126}{10.3389/fimmu.2017.00126}.

\hypertarget{ref-arevalo2016spz}{}
39. Arévalo-Herrera M, Vásquez-Jiménez JM, Lopez-Perez M, et al.
Protective efficacy of plasmodium vivax radiation-attenuated sporozoites
in colombian volunteers: A randomized controlled trial. \emph{PLoS Negl
Trop Dis}. 2016;10(10).
doi:\href{https://doi.org/10.1371/journal.pntd.0005070}{10.1371/journal.pntd.0005070}.

\hypertarget{ref-immunomics2016}{}
40. De Sousa KP, Doolan DL. Immunomics: A 21st century approach to
vaccine development for complex pathogens. \emph{Parasitology}.
2016;143(02):236-244.
doi:\href{https://doi.org/10.1017/S0031182015001079}{10.1017/S0031182015001079}.

\hypertarget{ref-Davies2015Large}{}
41. Davies DH, Duffy P, Bodmer J-L, Felgner PL, Doolan DL. Large screen
approaches to identify novel malaria vaccine candidates. \emph{Vaccine}.
2015;33(52):7496-7505.
doi:\href{https://doi.org/10.1016/j.vaccine.2015.09.059}{10.1016/j.vaccine.2015.09.059}.

\hypertarget{ref-uzoma2013interactome}{}
42. Uzoma I, Zhu H. Interactome mapping: Using protein microarray
technology to reconstruct diverse protein networks. \emph{Genomics,
proteomics \& bioinformatics}. 2013;11(1):18-28.
doi:\href{https://doi.org/10.1016/j.gpb.2012.12.005}{10.1016/j.gpb.2012.12.005}.

\hypertarget{ref-sundaresh2006}{}
43. Sundaresh S, Doolan DL, Hirst S, et al. Identification of humoral
immune responses in protein microarrays using dna microarray data
analysis techniques. \emph{Bioinformatics}. 2006;22(14):1760-1766.
doi:\href{https://doi.org/10.1093/bioinformatics/btl162}{10.1093/bioinformatics/btl162}.

\hypertarget{ref-allison2006}{}
44. Allison DB, Cui X, Page GP, Sabripour M. Microarray data analysis:
From disarray to consolidation and consensus. \emph{Nature reviews
genetics}. 2006;7(1):55-65.
doi:\href{https://doi.org/10.1038/nrg1749}{10.1038/nrg1749}.

\hypertarget{ref-brazma2001}{}
45. Brazma A, Hingamp P, Quackenbush J, et al. Minimum information about
a microarray experiment (MIAME)---toward standards for microarray data.
\emph{Nature genetics}. 2001;29(4):365-371.
doi:\href{https://doi.org/10.1038/ng1201-365}{10.1038/ng1201-365}.

\hypertarget{ref-Finney2014}{}
46. Finney OC, Danziger SA, Molina DM, et al. Predicting antidisease
immunity using proteome arrays and sera from children naturally exposed
to malaria. \emph{Molecular \& Cellular Proteomics}.
2014;13(10):2646-2660.
doi:\href{https://doi.org/10.1074/mcp.M113.036632}{10.1074/mcp.M113.036632}.

\hypertarget{ref-King2015FOC}{}
47. King CL, Davies DH, Felgner P, et al. Biosignatures of
exposure/transmission and immunity. \emph{American Journal of Tropical
Medicine and Hygiene}. 2015;93(3 Suppl):16-27.
doi:\href{https://doi.org/10.4269/ajtmh.15-0037}{10.4269/ajtmh.15-0037}.

\hypertarget{ref-kreil2005bullet}{}
48. Kreil DP, Russell RR. Tutorial section: There is no silver
bullet---a guide to low-level data transforms and normalisation methods
for microarray data. \emph{Briefings in bioinformatics}.
2005;6(1):86-97.
doi:\href{https://doi.org/10.1093/bib/6.1.86}{10.1093/bib/6.1.86}.

\hypertarget{ref-brown2001image}{}
49. Brown CS, Goodwin PC, Sorger PK. Image metrics in the statistical
analysis of dna microarray data. \emph{Proceedings of the National
Academy of Sciences}. 2001;98(16):8944-8949.
doi:\href{https://doi.org/10.1073/pnas.161242998}{10.1073/pnas.161242998}.

\hypertarget{ref-bourgon2010filter}{}
50. Bourgon R, Gentleman R, Huber W. Independent filtering increases
detection power for high-throughput experiments. \emph{Proceedings of
the National Academy of Sciences}. 2010;107(21):9546-9551.
doi:\href{https://doi.org/10.1073/pnas.0914005107}{10.1073/pnas.0914005107}.

\hypertarget{ref-baldi2001cybert}{}
51. Baldi P, Long AD. A bayesian framework for the analysis of
microarray expression data: Regularized t-test and statistical
inferences of gene changes. \emph{Bioinformatics}. 2001;17(6):509-519.
doi:\href{https://doi.org/10.1093/bioinformatics/17.6.509}{10.1093/bioinformatics/17.6.509}.

\hypertarget{ref-kayala2012cyber}{}
52. Kayala MA, Baldi P. Cyber-t web server: Differential analysis of
high-throughput data. \emph{Nucleic acids research}.
2012;40(W1):W553-W559.
doi:\href{https://doi.org/10.1093/nar/gks420}{10.1093/nar/gks420}.

\hypertarget{ref-smyth2004ebayes}{}
53. Smyth GK, others. Linear models and empirical bayes methods for
assessing differential expression in microarray experiments.
\emph{Statistical Applications in Genetics and Molecular Biology}.
2004;3(1):3.
doi:\href{https://doi.org/10.2202/1544-6115.1027}{10.2202/1544-6115.1027}.

\hypertarget{ref-benjamini1995fdr}{}
54. Benjamini Y, Hochberg Y. Controlling the false discovery rate: A
practical and powerful approach to multiple testing. \emph{Journal of
the royal statistical society Series B (Methodological)}. 1995:289-300.
doi:\href{https://doi.org/10.2307/2346101}{10.2307/2346101}.

\hypertarget{ref-Driguez2015}{}
55. Driguez P, Doolan DL, Molina DM, et al. Protein microarrays for
parasite antigen discovery. \emph{Parasite genomics protocols}.
2015:221-233.
doi:\href{https://doi.org/10.1007/978-1-4939-1438-8_13}{10.1007/978-1-4939-1438-8\_13}.

\hypertarget{ref-wickham2016r4ds}{}
56. Wickham H, Grolemund G. \emph{R for Data Science}. Sebastopol, CA:
O'Reilly.; 2016. \url{http://r4ds.had.co.nz/}.

\hypertarget{ref-Biobase}{}
57. Huber, W., Carey, et al. Orchestrating high-throughput genomic
analysis with Bioconductor. \emph{Nature Methods}. 2015;12(2):115-121.
doi:\href{https://doi.org/10.1038/nmeth.3252}{10.1038/nmeth.3252}.

\hypertarget{ref-limma}{}
58. Ritchie ME, Phipson B, Wu D, et al. limma powers differential
expression analyses for RNA-sequencing and microarray studies.
\emph{Nucleic Acids Research}. 2015;43(7):e47.
doi:\href{https://doi.org/10.1093/nar/gkv007}{10.1093/nar/gkv007}.

\hypertarget{ref-Gaujoux2010NMF}{}
59. Gaujoux R, Seoighe C. A flexible R package for nonnegative matrix
factorization. \emph{BMC Bioinformatics}. 2010;11(1):367.
doi:\href{https://doi.org/10.1186/1471-2105-11-367}{10.1186/1471-2105-11-367}.

\hypertarget{ref-plasmodb}{}
60. Aurrecoechea C, Brestelli J, Brunk BP, et al. PlasmoDB: A functional
genomic database for malaria parasites. \emph{Nucleic acids research}.
2009;37(suppl 1):D539-D543.
doi:\href{https://doi.org/10.1093/nar/gkn814}{10.1093/nar/gkn814}.

\hypertarget{ref-knitr}{}
61. Xie Y. Knitr: A general-purpose tool for dynamic report generation
in r. \emph{R package version}. 2013;1(1).
\url{http://yihui.name/knitr/}.

\hypertarget{ref-CienciaReproducible2016}{}
62. Rodríguez-Sanchez F, Pérez-Luque AJ, Bartomeus I, Varela S. Ciencia
reproducible: qué, por qué, cómo? \emph{ECOS}. 2016;25(2):83-92.
doi:\href{https://doi.org/10.7818/ecos.2016.25-2.11}{10.7818/ecos.2016.25-2.11}.

\section{ANEXOS}\label{anexos}

\subsection{Matriz de consistencia}\label{matriz-de-consistencia}

Ver \autoref{tab:consis} al final del documento.


\afterpage{
    \clearpage
    \newgeometry{left=2.5cm,right=0.5cm,top=0.5cm,bottom=0.5cm}
    \thispagestyle{empty}
    \begin{landscape}
        \centering
\captionof{table}{Matriz de consistencia}
\label{tab:consis}
\begin{center}
\begin{tabular}{|m{3.2cm}m{3.2cm}m{3.2cm}m{3.2cm}m{3.2cm}m{3.2cm}m{3.2cm}|}
  \hline
  \textbf{Título:}
  &
  \multicolumn{6}{l|}{ %>{\centering}m{19cm}
  \begin{minipage}{19.2cm}
  Comparación de la respuesta de anticuerpos ante %la infección con 
  \textit{Plasmodium vivax}
  en pacientes de la Amazonía Peruana %ciudad de Iquitos (Loreto - Perú)
  según su severidad y episodios previos %exposición previa
  mediante microarreglos de proteínas  
  \end{minipage}  
  }\\
  \cline{1-7}
  \textbf{Problema} & \textbf{Objetivos} & \textbf{Hipótesis} & \textbf{Variables} & 
  \textbf{Diseño} & \textbf{Muestra} & %\textbf{Instrumentos} & 
  \textbf{Análisis}\\
  \hline
  \begin{minipage}{3.2cm} 
  %\textbf{Principal}\\
  1. ¿Cuáles son los antígenos con reactividad serológica diferenciada
  ante la infección por \textit{P. vivax} entre pacientes severos y no-severos\\
  \newline
  %\textbf{Secundario}\\
  2. ¿Cuáles son los antígenos con reactividad serológica diferenciada
  ante la infección por \textit{P. vivax} entre pacientes con y sin episodios previos\\
  \newline
  3. ¿Cuáles son las características proteicas de los antígenos
  con reactividad serológica diferenciada o predominante
  en pacientes con malaria por \textit{P. vivax}?
  \end{minipage} 
  & 
  \begin{minipage}{3.2cm} 
  %.\\
  \textbf{General}\\
  Identificar un subconjunto de antígenos con reactividad serológica 
  discriminante de condiciones clínicas relevantes ante la infección por \textit{P. vivax}.\\
  \newline
  \textbf{Específicos}\\
  1. Identificar antígenos de \textit{P. vivax} con reactividad serológica 
  diferenciada ante la infección entre pacientes 
  severos y no-severos.\\
  \newline
  2. Identificar antígenos de \textit{P. vivax} con reactividad serológica 
  diferenciada ante la infección entre pacientes 
  con y sin episodios previos.\\
  \newline
  3. Describir las características proteicas de los antígenos con reactividad 
  diferenciada o predominante.\\
  \end{minipage} 
  & 
  \begin{minipage}{3.2cm} 
  .\\
  \textbf{De diferencia}\\ \textbf{entre grupos:}\\
  1. Los pacientes con malaria no-severa poseen 
  mayor reactividad serológica contra antígenos de \textit{P. vivax}
  asociados a exposición, invasión o adhesión celular
  con respecto a los pacientes severos.\\
  \newline
  2. Los pacientes con episodios previos de malaria poseen
  mayor reactividad serológica contra antígenos de \textit{P. vivax}
  asociados a exposición
  con respecto a los pacientes sin episodios previos.\\
  \newline
  \textbf{Descriptiva:}\\
  3. Los antígenos con reactividad diferenciada o predominante
  se caracterizan por poseer una localización extracelular 
  y estar bajo presión selectiva por el sistema inmune.\\
  \end{minipage} 
  &
  \begin{minipage}{3.2cm} 
  \textbf{Dependiente}\\ Reactividad\\ serológica\\
  \newline 
  \textbf{Independiente}\\ Severidad\\
  \newline
  \textbf{Independiente}\\ Episodios previos\\
  \newline
  \underline{Instrumentos}:\\
  %\textbf{Reactividad serológica:}\\
  -Microarreglo de proteínas PfPv500.\\%Pf498/Pv516\\
  %\newline
  %\textbf{Severidad:}\\
  -Diagnóstico clínico y pruebas bioquímicas.\\%Criterios de la OMS para malaria severa % (criterio OMS)
  %\newline
  %\textbf{Episodios}\\ \textbf{previos:}\\
  -Encuesta.\\
  \newline
  \underline{Operacionalización}:\\
  -Ir a \autoref{tab:opera}
  \end{minipage} 
  &
  \begin{minipage}{3.2cm} 
  \textbf{Tipo:}\\
  Caso-Control.\\
  \newline
  \textbf{Clasificación:}\\
  -Por la finalidad: Analítico.\\
  %\newline
  -Por el control de la asignación:\\ Observacional.\\
  %\newline
  -Por el seguimiento: Transversal.\\
  %\newline
  -Por la relación cronológica:\\ Prospectivo.%\\
  %\newline
  %-Por la unidad de análisis:\\ Basado en individuo.
  \end{minipage}   
  &
  \begin{minipage}{3.2cm} 
  %\textbf{Universo teórico:}\\ 
  %Pacientes con malaria vivax 
  %de la cuenca amazónica del Perú.\\
  %\newline
  \textbf{Población}\\ %\textbf{Muestral:}\\
  Pacientes diagnosticados con malaria por \textit{P. vivax} de la ciudad de Iquitos, Loreto-Perú, 
  entre enero del 2012 y junio del 2013.\\
  \newline
  \textbf{Muestra:}\\
  Selección arbitraria
  de 30 pacientes con uno o más criterios de la clasificación OMS para malaria severa (casos) y 
  selección aleatoria simple de 30 no-severos (control).\\
  \newline
  \textbf{Tipo:}\\ No probabilística y Probabilística.
  \end{minipage}   
  &
%  \begin{minipage}{3.2cm} 
%  \textbf{Reactividad serológica:}\\
%  Microarreglos de\\proteínas Pf498/Pv516\\
%  \newline
%  \textbf{Severidad:}\\
%  Diagnóstico \\clínico \\y exámenes de \\laboratorio\\ según criterios de la OMS.\\
%  \newline
%  \textbf{Episodios}\\ \textbf{previos:}\\
%  Encuesta
%  \end{minipage}   
%  &
  \begin{minipage}{3.2cm} 
  \underline{Control de Calidad}:\\
  \newline
  \textbf{1. Validez:
  %y}\\ \textbf{Reproducibilidad:
  }\\
  correlación de Pearson o Spearman\\
  \newline
  \underline{Prueba de Hipótesis}:\\
  \newline
  \textbf{2. Amplitud}\\ \textbf{e intensidad de}\\ \textbf{respuesta:}\\
  prueba t-Student o Mann-Whitney\\
  \newline
  \textbf{3.}\\ \textbf{Reactividad}\\ \textbf{diferenciada:}\\%\textbf{3. Inferencia}\\
  test-t moderado con\\
  corrección del FDR\\por el método\\Benjamini-Hochberg\\
  \newline
  \textbf{4. Clasificación:}\\
  agrupamiento jerárquico o\\ \textit{hierarchical}\\ \textit{clustering}\\
  en base a la\\ distancia euclidiana.\\
  \newline
  \textbf{5. Descripción:}\\
  características disponibles en\\ PlasmoDB.
  \end{minipage}   
  \\
  %\cline{1-5}
  %y & z & m & n & y & z & m & n\\
  \hline
  % etc. ...
\end{tabular}

\end{center}
    \end{landscape}
    \restoregeometry
    \clearpage
}


\end{document}
