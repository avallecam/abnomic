\documentclass[]{article}
\usepackage{lmodern}
\usepackage{amssymb,amsmath}
\usepackage{ifxetex,ifluatex}
\usepackage{fixltx2e} % provides \textsubscript
\ifnum 0\ifxetex 1\fi\ifluatex 1\fi=0 % if pdftex
  \usepackage[T1]{fontenc}
  \usepackage[utf8]{inputenc}
\else % if luatex or xelatex
  \ifxetex
    \usepackage{mathspec}
  \else
    \usepackage{fontspec}
  \fi
  \defaultfontfeatures{Ligatures=TeX,Scale=MatchLowercase}
\fi
% use upquote if available, for straight quotes in verbatim environments
\IfFileExists{upquote.sty}{\usepackage{upquote}}{}
% use microtype if available
\IfFileExists{microtype.sty}{%
\usepackage{microtype}
\UseMicrotypeSet[protrusion]{basicmath} % disable protrusion for tt fonts
}{}
\usepackage[margin=1in]{geometry}
\usepackage{hyperref}
\PassOptionsToPackage{usenames,dvipsnames}{color} % color is loaded by hyperref
\hypersetup{unicode=true,
            colorlinks=true,
            linkcolor=Blue,
            citecolor=Blue,
            urlcolor=Blue,
            breaklinks=true}
\urlstyle{same}  % don't use monospace font for urls
\usepackage{longtable,booktabs}
\usepackage{graphicx,grffile}
\makeatletter
\def\maxwidth{\ifdim\Gin@nat@width>\linewidth\linewidth\else\Gin@nat@width\fi}
\def\maxheight{\ifdim\Gin@nat@height>\textheight\textheight\else\Gin@nat@height\fi}
\makeatother
% Scale images if necessary, so that they will not overflow the page
% margins by default, and it is still possible to overwrite the defaults
% using explicit options in \includegraphics[width, height, ...]{}
\setkeys{Gin}{width=\maxwidth,height=\maxheight,keepaspectratio}
\IfFileExists{parskip.sty}{%
\usepackage{parskip}
}{% else
\setlength{\parindent}{0pt}
\setlength{\parskip}{6pt plus 2pt minus 1pt}
}
\setlength{\emergencystretch}{3em}  % prevent overfull lines
\providecommand{\tightlist}{%
  \setlength{\itemsep}{0pt}\setlength{\parskip}{0pt}}
\setcounter{secnumdepth}{5}
% Redefines (sub)paragraphs to behave more like sections
\ifx\paragraph\undefined\else
\let\oldparagraph\paragraph
\renewcommand{\paragraph}[1]{\oldparagraph{#1}\mbox{}}
\fi
\ifx\subparagraph\undefined\else
\let\oldsubparagraph\subparagraph
\renewcommand{\subparagraph}[1]{\oldsubparagraph{#1}\mbox{}}
\fi

%%% Use protect on footnotes to avoid problems with footnotes in titles
\let\rmarkdownfootnote\footnote%
\def\footnote{\protect\rmarkdownfootnote}

%%% Change title format to be more compact
\usepackage{titling}

% Create subtitle command for use in maketitle
\newcommand{\subtitle}[1]{
  \posttitle{
    \begin{center}\large#1\end{center}
    }
}

\setlength{\droptitle}{-2em}
  \title{}
  \pretitle{\vspace{\droptitle}}
  \posttitle{}
  \author{}
  \preauthor{}\postauthor{}
  \date{}
  \predate{}\postdate{}

\usepackage{multirow}
\usepackage{pdflscape}
\usepackage{afterpage}
\usepackage{capt-of}
\usepackage{array}
\usepackage{color,soul}

\begin{document}

\renewcommand{\contentsname}{Índice General} 
\renewcommand{\tablename}{Tabla}
\renewcommand{\tableautorefname}{Tabla}

\pagenumbering{gobble}

\clearpage
\newgeometry{left=0.5cm,right=0.5cm,top=1.8cm,bottom=1cm}

\begin{centering}

\begin{figure}[!ht]
  \begin{center}
    \includegraphics[width=.8in]{figure/UNMSM_escudo-2000px.png}% en informe: 8
  \end{center}
\end{figure}

\Large %https://tex.stackexchange.com/questions/24599/what-point-pt-font-size-are-large-etc
UNIVERSIDAD NACIONAL MAYOR DE SAN MARCOS

\large
(Universidad del Perú, DECANA DE AMÉRICA)

\vspace{.3 cm}

\Large
FACULTAD DE CIENCIAS BIOLÓGICAS

\vspace{.3 cm}

\normalsize
ESCUELA ACADÉMICO PROFESIONAL DE

GENÉTICA Y BIOTECNOLOGÍA

\vspace{2.3 cm}

\Large
Comparación de la respuesta de anticuerpos ante la %infección con 
malaria vivax 
en \\pacientes de la Amazonía peruana %ciudad de Iquitos (Loreto - Perú)
según su severidad y episodios \\previos %exposición previa
mediante microarreglos de proteínas
% aPerfil de anticuerpos en respuesta a la infección con malaria vivax
%mediante un enfoque inmunómico .
%% con sintomatología severa y no complicada/ pre-inmunes/ semi-inmunes

\vspace{2.3 cm}

\large
Proyecto de Tesis 

para optar al Título Profesional de Biólogo Genetista y Biotecnólogo

\vspace{.3 cm}

\large
Investigador

Bach. Andree Adolfo Valle Campos

\vspace{.3 cm}

Asesores

Interno: Prof. Walter Cabrera-Febolá

Externo: PhD. G. Christian Baldeviano

\vspace{.3 cm}

Institución

Centro de Enfermedades Tropicales de la Marina de los Estados Unidos 
NAMRU-6

\vspace{.3 cm}

Duración: 6 meses

\vspace{1 cm}

\Large
Lima - Perú

%\vspace{.5 cm}

2017

\end{centering}

\vfill
\restoregeometry
\clearpage

\newpage

\tableofcontents

\newpage

\pagenumbering{arabic}

\section*{RESUMEN}\label{resumen}
\addcontentsline{toc}{section}{RESUMEN}

\begin{quote}
\emph{Plasmodium vivax} es responsable del 80\% de la malaria en el
Perú. Casos de enfermedad severa por mono-infecciones de \emph{P. vivax}
han sido reportados tanto en el noreste amazónico como en norte costero.
Sin embargo, dicha condición aún está subestimada en las normas técnicas
a nivel nacional, incrementando el riesgo de posibles diagnósticos
tardíos o inapropiados por el personal de salud. Con el fin de
identificar biomarcadores de relevancia clínica contra la malaria vivax
severa, el presente proyecto de tesis propone comparar la respuesta de
anticuerpos ante la infección en pacientes severos y no-severos
provenientes de un estudio prospectivo caso-control ejecutado en la
ciudad de Iquitos, Loreto - Perú, empleando microarreglos de proteínas.
Los resultados permitirán proponer marcadores serológicos discriminantes
de severidad con el potencial de implementarse en programas de
serovigilancia en miras al control y eliminación de la malaria en la
región.
\end{quote}

\section{PLANTEAMIENTO DEL PROBLEMA}\label{planteamiento-del-problema}

\subsection{Formulación del problema}\label{intro}

Malaria es una enfermedad parasitaria de importancia mundial, causada
por protozoarios del género \emph{Plasmodium} y transmitida por
mosquitos del género \emph{Anopheles}. En el 2015, se estimaron 212
millones de casos y 429,000 muertes atribuidas a esta
infección\textsuperscript{\protect\hyperlink{ref-WHO2016world}{1}}.
Aunque \emph{P. falciparum} representó el 96 y 99\% de estas cifras,
fuera de África se estimó que \emph{P. vivax} fue responsable del 41 y
86\%, respectivamente. Más aún, la región de las Américas tuvo la mayor
proporción de estos casos (69\%) donde Perú fue el tercer país con más
reportes (n=63,153), detrás de Brasil y Venezuela, atribuidos en un 80\%
a dicha
especie\textsuperscript{\protect\hyperlink{ref-rosas2016peru}{2}}.

Reportes recientes de malaria severa y fatal causados por \emph{P.
vivax} han desafiado su tradicional condición de enfermedad
benigna\textsuperscript{\protect\hyperlink{ref-baird2009}{3},\protect\hyperlink{ref-quispe2014}{4}}.
El incremento de estos casos puede ser consecuencia de un retraso en la
adquisición de inmunidad en la población por una reducción en la
intensidad de
transmisión\textsuperscript{\protect\hyperlink{ref-reyburn2015}{5}}. Por
ello, tanto su actual
subestimación\textsuperscript{\protect\hyperlink{ref-norma2001}{6}} como
la ausencia de marcadores serológicos de protección contra la severidad
en las actuales estrategias de
eliminación\textsuperscript{\protect\hyperlink{ref-accelerate2016}{7}}
podría incrementar su prevalencia a largo plazo. En contraste con dicha
hipótesis, pacientes de la Amazonía peruana con malaria vivax severa y
no complicada mostraron un grado similar de exposición previa, basado en
la respuesta humoral contra un marcador
tradicional\textsuperscript{\protect\hyperlink{ref-baldevi2013}{8}}. Sin
embargo, recientes estudios a larga escala han desafiado su validez como
indicador de inmunidad o
exposición\textsuperscript{\protect\hyperlink{ref-crompton2010}{9},\protect\hyperlink{ref-Helb2015exposure}{10}}.

Por esta razón, el presente estudio tiene como objetivo comparar la
respuesta de anticuerpos ante la infección con malaria vivax contra más
de 500 antígenos de \emph{P. vivax} en pacientes severos y no-severos de
la ciudad de Iquitos, empleando microarreglos de proteínas.
Hipotetizamos que los no-severos poseen mayor reactividad serológica
contra antígenos de exposición, invasión o adhesión celular, con
respecto a los severos. Primero, se identificarán los antígenos con
reactividad diferenciada entre ambos grupos. Luego, se determinarán los
antígenos de respuesta secundaria en pacientes con episodios previos de
esta infección. Finalmente, se describirán sus características proteicas
junto a los antígenos con mayor reactividad en toda la muestra con el
propósito de proponerlos como candidatos a vigilancia
seroepidemiológica.

\subsection{Preguntas de
investigación}\label{preguntas-de-investigacion}

\begin{enumerate}
\def\labelenumi{\arabic{enumi}.}
\item
  ¿Cuáles son los antígenos de \emph{P. vivax} con reactividad
  serológica diferenciada ante la infección con malaria vivax entre
  pacientes severos y no-severos?
\item
  ¿Cuáles son los antígenos de \emph{P. vivax} con reactividad
  serológica diferenciada ante la infección con malaria vivax entre
  pacientes con y sin episodios previos?
\item
  ¿Cuáles son las características proteicas de los antígenos de \emph{P.
  vivax} con reactividad diferenciada o predominante en pacientes con
  malaria vivax?
\end{enumerate}

\subsection{Objetivos}\label{objetivos}

\subsubsection{General}\label{general}

\begin{itemize}
\tightlist
\item
  Identificar un subconjunto de antígenos de \emph{P. vivax} con
  reactividad serológica discriminante de condiciones clínicas
  relevantes ante la infección con malaria vivax.
\end{itemize}

\subsubsection{Específicos}\label{especificos}

\begin{itemize}
\item
  Identificar antígenos de \emph{P. vivax} con reactividad serológica
  diferenciada ante la infección con malaria vivax entre pacientes
  severos y no-severos.
\item
  Identificar antígenos de \emph{P. vivax} con reactividad serológica
  diferenciada ante la infección con malaria vivax entre pacientes con y
  sin episodios previos.
\item
  Describir las características proteicas de los antígenos con
  reactividad diferenciada o predominante.
\end{itemize}

\subsubsection{Exploratorio}\label{exploratorio}

\begin{itemize}
\tightlist
\item
  Comparar la amplitud e intensidad de respuesta de anticuerpos según la
  edad de los pacientes con malaria vivax.
\end{itemize}

\subsection{Justificación}\label{justif}

Ante la reemergencia repetitiva de la malaria en la Amazonía
peruana\textsuperscript{\protect\hyperlink{ref-rosas2016peru}{2},\protect\hyperlink{ref-griffing2013history}{11}},
la implementación de vigilancias serológicas programáticas en nuestro
país es una
prioridad\textsuperscript{\protect\hyperlink{ref-hotspots2015}{12}}. En
zonas con baja transmisión, estos ensayos poseen una mayor sensibilidad
y representan un menor gasto económico en comparación a las estrategias
convencionales de monitoreo y
control\textsuperscript{\protect\hyperlink{ref-elliott2014}{13}}. Por
esta razón, el presente estudio se justifica en el descubrimiento de
antígenos potencialmente discriminantes de condiciones clínicas como
severidad o exposición que permitirán optimizar y acelerar su ejecución
en programas de salud pública contra la malaria en el Perú en miras a su
control y posterior
eliminación\textsuperscript{\protect\hyperlink{ref-accelerate2016}{7}}.

\subsection{Limitaciones}\label{limit}

Resumimos tres limitaciones del estudio.

Primero, las muestras a evaluar provienen de pacientes con infecciones
no sincronizadas. El muestreo se ejecutó en la fase sintomática, momento
en el que acuden al Hospital, con un desconocimiento del inicio de la
infección por paciente. Sin embargo, infecciones experimentales han
estimado que dicha fase ocurre normalmente entre los 11 y 13 días de
infección\textsuperscript{\protect\hyperlink{ref-arevalo2014}{14}}.

Segundo, el estudio posee debilidades propias del diseño experimental a
emplear. El diseño de tipo Caso Control es susceptible a errores
sistemáticos de selección y clasificación, posee dificultad para
establecer relaciones causa-efecto y a través de él no se pueden
calcular prevalencias o incidencias. Además, la ausencia de un
seguimiento activo impide el registro de covariables relevantes para la
caracterización de la enfermedad.

Tercero, los microarreglos de proteínas poseen limitantes propias a su
fabricación\textsuperscript{\protect\hyperlink{ref-vigil2010}{15}}. Cada
paso (amplificación, clonamiento, expresión de genes a larga escala e
impresión del arreglo) posee una eficiencia límite que afectará la
calidad final de las proteínas. Además, el plegamiento proteico no será
posible de verificar a dicha escala. Por último, la identificación de
antígenos con modificaciones postranscriptacionales, particularmente
relevantes en
\emph{Plasmodium}\textsuperscript{\protect\hyperlink{ref-leroch2009postmod}{16}},
no serán posibles de reproducir en su integridad en el sistema de
expresión procarionte. A pesar de ello, se evaluará la validez y
reproducibilidad del ensayo con controles internos a detallar en la
\protect\hyperlink{validez}{sección 4.3.2.2}.

\subsection{Viabilidad}\label{viabilidad}

El proyecto es viable por la factibilidad de su ejecución, el interés de
sus resultados para el área, la novedad de su enfoque y la relevancia de
su método para la investigación biomédica.

Primero, su ejecución es factible por la disponibilidad inmediata de
muestras provenientes de un estudio prospectivo ya realizado, su
adecuado tamaño muestral (n=60), la objetividad de la pregunta planteada
y la experiencia técnica del investigador en la herramienta de análisis
R\textsuperscript{\protect\hyperlink{ref-R2016}{17}}/Bioconductor\textsuperscript{\protect\hyperlink{ref-bioconductor2004}{18}}.
Segundo, el problema es de interés para la comunidad por la posibilidad
de que sus resultados generen un cambio en la práctica médica. Tercero,
la novedad de su enfoque a larga escala permitirá corroborar, refutar o
extender el estado del conocimiento y brindar mayor evidencia con
respecto a la controversia mencionada. Cuarto, la relevancia de su
método de análisis reproducible con software libre permitirá
transparentar la generación de resultados y promover su puesta en
práctica en futuras investigaciones de esta u otras áreas.

\section{FORMULACIÓN DE HIPÓTESIS Y
VARIABLES}\label{formulacion-de-hipotesis-y-variables}

\subsection{Hipótesis}\label{hipotesis}

\subsubsection{De diferencia entre
grupos}\label{de-diferencia-entre-grupos}

\begin{enumerate}
\def\labelenumi{\arabic{enumi}.}
\item
  Los pacientes con malaria vivax no-severa poseen mayor reactividad
  serológica contra antígenos de \emph{P. vivax} asociados a exposición,
  invasión o adhesión celular con respecto a los pacientes severos.
\item
  Los pacientes con episodios previos de malaria poseen mayor
  reactividad serológica contra antígenos de \emph{P. vivax} asociados a
  exposición con respecto a los pacientes sin episodios previos.
\end{enumerate}

\subsubsection{Descriptiva}\label{descriptiva}

\begin{enumerate}
\def\labelenumi{\arabic{enumi}.}
\setcounter{enumi}{2}
\tightlist
\item
  Los antígenos con reactividad diferenciada o predominante se
  caracterizan por poseer una localización extracelular y estar bajo
  presión selectiva por el sistema inmune.
\end{enumerate}

\subsection{Variables}\label{variables}

El presente estudio busca contrastar la respuesta de anticuerpos ante la
infección con malaria vivax entre pacientes clasificados según su
severidad y episodios previos. Para ello se medirá la variable de
reactividad serológica de antígenos de \emph{P. vivax} luego de sondear
plasma de pacientes en un microarreglo de proteínas. Además se
clasificará a los pacientes como severos o no-severos de acuerdo a
diagnóstico clínico y bioquímico, y como con-episodios o sin-episodios
previos de acuerdo a lo reportado en la encuesta con el médico tratante.

Primero, la reactividad serológica cuantificará la especificidad de los
anticuerpos de respuesta contra los antígenos del patógeno \emph{P.
vivax}. Esta variable medirá indirectamente la interacción entre
antígenos y anticuerpos, a través de la lectura de la intensidad
fluorescente producto de esta reacción.

Segundo, los pacientes severos se definirán por la presencia de
manifestaciones clínicas severas o complicaciones sistémicas. Esta
variable será asignada por la presencia de uno o más parámetros clínicos
y de laboratorio para la malaria severa, según el criterio recomendado
por la Organización Mundial de la Salud
(OMS)\textsuperscript{\protect\hyperlink{ref-WHO2014severe}{19}}.

Tercero, la clasificación por episodios previos reportados representará
la presencia de infecciones previas con malaria y, consecuentemente, la
presencia de una respuesta inmune humoral secundaria. Esta variable será
reportada por el paciente como el número de eventos previos a la actual
infección. Dada la incertidumbre de dicho dato, la variable será
dicotomizada con el fin de facilitar su inclusión.

\subsubsection{Operacionalización de
variables}\label{operacionalizacion-de-variables}

Ver \autoref{tab:opera}.

\begin{table}[ht]
\begin{center}
\hspace*{-1cm}
\begin{tabular}{>{\centering}m{2.4cm} m{2.2cm}m{2.2cm}m{2cm}m{2.2cm}m{1.7cm}m{1.5cm}m{1.6cm} @{}m{0pt}@{} }
  
  \hline
  \multirow{2}{*}{Variable}
  & 
  \multicolumn{2}{c}{Definición} 
  %&
  %\begin{minipage}{2.2cm}
  %Definición\\conceptual
  %\end{minipage}
  %&
  %\begin{minipage}{2.2cm}
  %Definición\\operacional
  %\end{minipage}
  & 
  \multirow{2}{*}{
  \begin{minipage}{2.2cm}
  Instrumento\\de medición
  \end{minipage}
  }
  &
  \multirow{2}{*}{
  \begin{minipage}{2.2cm}
  Criterios\\de medición
  \end{minipage}
  }
  &
  \multirow{2}{*}{
  \begin{minipage}{1.7cm}
  Tipo de\\variable
  \end{minipage}
  }
  &
  \multirow{2}{*}{
  \begin{minipage}{1.5cm}
  Escala de \\medición
  \end{minipage}
  }
  &
  \multirow{2}{*}{
  Fuente
  } &\\[0ex]
  %\hline
  \cline{2-3}
  
  &
  Conceptual
  &
  Operacional
  & 
  &
  &
  & &\\[1ex]
  \hline
  
  \textbf{Dependiente} Reactividad serológica
  & 
  % esCONCEPTUAL: 
  \begin{minipage}{2.2cm} 
  Especificidad \\de anticuerpos \\de respuesta contra un antígeno
  \end{minipage} 
  &
  % aOPERACIONAL: 
  \begin{minipage}{2.2cm} 
  Medida \\indirecta de \\la reacción antígeno-anticuerpo
  \end{minipage} 
  % aDETALLES: medida indirecta de la reacción antígeno-anticuerpo 
  % mediante la lectura de la reacción fluorescente entre 
  % anticuerpo secundario y fluoroforo por spot
  & 
  \begin{minipage}{2.2cm} 
  Microarreglo\\
  de proteínas
  \end{minipage}
  & 
  \begin{minipage}{2.2cm} 
  \textbf{0-6000} MFI o intensidad\\
  fluorescente \\promedio.
  \end{minipage} 
  &
  Numérica contínua
  & 
  Razón
  &
  Plasma sanguíneo &\\[13ex]
  \hline

  \textbf{Independiente} Severidad
  & 
  % aCONCEPTUAL: 
  Presencia de manifestaciones clínicas severas o complicaciones sistémicas
  &
  % aOPERACIONAL:
  Número de parámetros del criterio OMS para malaria severa
  & 
  \begin{minipage}{2.2cm} 
  Diagnóstico \\clínico \\y pruebas \\bioquímicas 
  \end{minipage}
  & 
  \begin{minipage}{2.2cm} 
  \textbf{No-severa:} 0 parámetros.\\
  \textbf{Severa:} 1 o más parámetros.
  \end{minipage}
  &
  Categórica dicotómica
  & 
  Nominal
  &
  Historia clínica y muestra de sangre &\\[15ex]
  \hline
  
  \textbf{Independiente} Episodios previos
  & 
  % aCONCEPTUAL: 
  Exposición a la infección de malaria en el pasado
  &
  % aOPERACIONAL:
  Número de episodios previos reportados 
  & 
  Encuesta
  & 
  \begin{minipage}{2.2cm} 
  \textbf{Sin:} 0 episodios.\\
  \textbf{Con:} 1 o más episodios.
  \end{minipage}
  &
  Categórica dicotómica
  & 
  Nominal
  &
  Historia clínica &\\[10ex]
  \hline

%  \textbf{Interviniente} Edad %Confusora
%  & 
%  % aCONCEPTUAL: 
%  Edad del paciente
%  &
%  % aOPERACIONAL:
%  Años de vida reportados
%  & 
%  Encuesta
%  & 
%  \begin{minipage}{2.2cm} 
%  \textbf{0-90} años.
%  \end{minipage}
%  &
%  Numérica discreta
%  & 
%  Razón
%  &
%  Historia clínica &\\[10ex]
%  \hline


  % etc. ...
\end{tabular}
\hspace*{-1cm}
\end{center}
        \captionof{table}{Operacionalización de variables}
        \label{tab:opera}
\end{table}

\section{MARCO TEÓRICO}\label{marco-teorico}

\subsection{Antecedentes de la
investigación}\label{antecedentes-de-la-investigacion}

\begin{enumerate}
\def\labelenumi{\alph{enumi}.}
\item
  En África

  Un hito en la aplicación de microarreglos de proteínas para el estudio
  de la respuesta humoral a escala epidemiológica fue la publicación de
  Crompton et al.
  2010\textsuperscript{\protect\hyperlink{ref-crompton2010}{9}}. Este
  estudio comparó la respuesta de anticuerpos contra el 23\% del
  proteoma de \emph{P. falciparum} antes y después de la temporada de
  malaria en 220 individuos de Mali, en dos grupos poblacionales: 2-10 y
  18-25 años. Dentro del grupo de niños entre 8 y 10 años se
  identificaron 49 proteínas con mayor reactividad serológica en el
  grupo de infectados asintomáticos, en comparación a los sintomáticos.
  Cinco de los principales candidatos a vacuna (CSP, LSA-3, MSP1, MSP2,
  AMA1) no lograron discriminar ambos grupos. Sin embargo, cuatro
  candidatos secundarios (STARP, LSA-1, RESA, antígeno 332), con
  expresión en diferentes estadios del ciclo biológico, sí lograron tal
  discriminación.

  Un segundo hito de interés representa el trabajo de Helb et al.
  2015\textsuperscript{\protect\hyperlink{ref-Helb2015exposure}{10}}.
  Ellos reportaron una estrategia para identificar combinaciones de
  respuestas de anticuerpos contra más de un antígeno que maximicen la
  información de la exposición reciente al nivel de individuos. Para
  ello emplearon modelos basados en el aprendizaje automático o
  \emph{machine learning} para el análisis de las respuestas contra 865
  antígenos de \emph{P. falciparum} en 186 niños (3-6 años) de Uganda en
  base al registro activo y pasivo de sus historias clínicas a los largo
  de un año. En contraste a los marcadores tradicionalmente empleados
  para evaluar exposición a nivel poblacional (CSP, MSP1, MSP2,
  AMA1)\textsuperscript{\protect\hyperlink{ref-elliott2014}{13}}, se
  identificaron marcadores más informativos como hyp2, GEXP18, EMP1,
  ETRAMP4, HSP40-II y PF70. La validación de este método permitirá
  seleccionar mejores marcadores para la vigilancia seroepidemiológica,
  relevante para el guiado y evaluación de los programas de control y
  eliminación de malaria a nivel mundial.
\item
  En Perú

  El primer estudio en publicarse fue de Torres et al.
  2014\textsuperscript{\protect\hyperlink{ref-Torres2014asymptomatic}{20}}
  donde reportaron marcadores de inmunidad clínica -no esterilizante- de
  adquisición natural a la malaria falciparum. Además de ello, el
  resultado con mayor implicancias fue que estos marcadores presentaron
  un enriquecimiento de polimorfismos no-sinónimos, indicativo de una
  presión de selección positiva por parte del sistema inmune, a pesar de
  provenir de una zona con baja transmisión. Las 51 proteínas con mayor
  reactividad en 14 pacientes infectados y asintomáticos se obtuvieron
  al comparar las respuestas con 24 paciente sintomáticos provenientes
  de la Amazonía peruana (Departamento de Loreto, Provincia de Maynas y
  Requena) contra 824 fragmentos (699 proteínas) de \emph{P.
  falciparum}.

  El segundo y último estudio en ser publicado ha sido el de Chuquiyauri
  et al.
  2015\textsuperscript{\protect\hyperlink{ref-chuquiyauri2015vivax}{21}}.
  Ellos compararon la respuesta contra \emph{P. vivax} de pacientes con
  relapsos y reinfección, sin encontrar diferencia alguna entre ambos
  grupos. Sin embargo, al igual que el anterior estudio, identificaron
  un enriquecimiento de proteínas con polimorfismos no-sinónimos en el
  grupo de antígenos reactivos. Además, dentro del grupo con mayor
  reactividad en toda la muestra, resaltaron a PvMSP-10 como potencial
  candidato a vacuna al presentar la expresión más consistente y validar
  lo reportado por dos estudios previos con enfoque tradicional donde
  emplean ambos sistemas de expresión: eucariota y procariota. El
  estudio empleó un arreglo con 2233 fragmentos (1936 proteínas) en 106
  individuos de la ciudad de Maynas, Loreto - Perú.
\end{enumerate}

\subsection{Bases teóricas}\label{bases-teoricas}

\subsubsection{Malaria Vivax}\label{malaria-vivax}

\begin{enumerate}
\def\labelenumi{\alph{enumi}.}
\item
  Epidemiología

  \begin{enumerate}
  \def\labelenumii{\roman{enumii}.}
  \item
    \textbf{A nivel mundial}

    \emph{P. vivax} y \emph{P. falciparum} son los principales
    responsables de los casos de malaria en humanos. Ambas especies
    exponen aproximadamente a 2.5 mil millones de personas en riesgo de
    infección\textsuperscript{\protect\hyperlink{ref-howes2016global}{22}}.
    Sin embargo, \emph{P. vivax} es el parásito dominante en las
    regiones fuera del África subsahariana, en su mayoría densamente
    pobladas y empobrecidas. Entre ellas, Etiopía, India, Indonesia y
    Pakistán acumularon el 78\% de casos de \emph{P. vivax} a nivel
    mundial. A su vez, la región de las Américas tuvo la mayor
    proporción de estos, con un
    69\%\textsuperscript{\protect\hyperlink{ref-WHO2016world}{1}}. A
    pesar de ello, hasta el momento la mayoría de la investigación y
    financiamiento está destinado a la prevención, tratamiento y control
    de \emph{P.
    falciparum}\textsuperscript{\protect\hyperlink{ref-path2011}{23}}.
  \item
    \textbf{En el Perú}

    En el 2015, Perú fue el tercer país con más casos reportados en
    Latinoamérica (19\%), detrás de Brasil (24\%) y Venezuela
    (30\%)\textsuperscript{\protect\hyperlink{ref-WHO2016world}{1}}. El
    80\% fueron causados por \emph{P. vivax} (63,153 en total), en
    regiones con endemismo y transmisión
    heterogénea\textsuperscript{\protect\hyperlink{ref-rosas2016peru}{2}}.
    El 95\% pertenecieron al noreste amazónico (con una razón Pv/Pf de
    4/1) y el resto al norte costero y la región minera del suroeste.
    Notablemente, comunidades en Madre de Dios no han registrado caso de
    malaria falciparum en una década (2001-2012). Con respecto a
    factores ambientales, en el año 1998 durante el fenómeno El
    Niño-Oscilación Sur (ENSO) se produjo el mayor pico de casos anuales
    en la historia (200,000 casos), con mayor efecto en el norte
    costero\textsuperscript{\protect\hyperlink{ref-gagnon2002enso}{24}}.
    Los casos anuales provenientes de Tumbes, Piura y Lambayeque a nivel
    nacional pasaron de representar menos del 7\% en 1996, a un 48\%
    entre 1998 y 1999.
  \end{enumerate}
\item
  Biología

  \begin{enumerate}
  \def\labelenumii{\roman{enumii}.}
  \item
    \textbf{Ciclo de vida}

    La ecología natural del parásito de la malaria involucra a dos
    hospederos: el humano y el mosquito. A través de la picadura del
    mosquito infectado, los esporozoitos son librerados al fluido
    sanguíneo del hospedero e ingresan a las células hepáticas
    iniciándose su desarrollo asexual, llamado estadio hepático. Ahí se
    multiplican hasta formar esquizontes, los cuales egresan al torrente
    sanguíneo en la forma de merozoitos. El segundo estadio, llamado
    eritrocítico, se inicia al invadir glóbulos rojos (RBC) y
    desarrollar en forma consecutiva trofozoitos inmaduros (en forma de
    anillo), maduros y esquizontes, los cuales a su ruptura liberan
    nuevos merozoitos que reinfectan a más glóbulos rojos. Por motivos
    aún desconocidos, a partir de los trofozoitos inmaduros se inicia el
    desarrollo de gametocitos diferenciándose sexualmente dentro del
    torrente sanguíneo del humano. La fase sexual en el mosquito se
    inicia mediante la ingestión accidental de gametocitos en la
    alimentación sanguínea de mosquitos hembra, con la intención inicial
    de proveer nutrientes a sus huevos. En su tracto digestivo, se
    desarrollan los esporozoitos, los cuales migran a las glándulas
    salivares para continuar la transmisión en la siguiente alimentación
    de los mosquitos hembra.
  \item
    \textbf{Particularidades}

    \emph{P. vivax} posee importantes variantes biológicas que
    caracterizan su epidemiología y dinámica de
    infección\textsuperscript{\protect\hyperlink{ref-howes2016global}{22}}.
    Tres de las más importantes son (i) la presencia de relapsos por la
    activación de hipnozoitos, el cual es un estado de latencia o
    dormacia en las células hepáticas, condición que puede generar tanto
    una liberación prolongada y lenta de merozoitos como reservorios de
    infección que prolongan su transmisión; (ii) el tropismo hacia
    reticulocitos o RBC inmaduros (1-2\% de RBC circulantes), condición
    que genera bajas parasitemias en comparación a \emph{P. falciparum};
    y (iii) la dependencia del antígeno \emph{Duffy} para la invasión de
    RBC, cuya ausencia en individuos con ancestría africana ha sido
    considerada como un factor de resistencia a \emph{P. vivax}. 
  \end{enumerate}
\end{enumerate}

\subsubsection{Malaria Severa}\label{malaria-severa}

\begin{enumerate}
\def\labelenumi{\alph{enumi}.}
\item
  General

  \begin{enumerate}
  \def\labelenumii{\roman{enumii}.}
  \item
    \textbf{Definición}

    Ante la ausencia de una descripción especie-específica, la Malaria
    Vivax Severa está definida por el criterio para \emph{P. falciparum}
    otorgado por la OMS en el
    2014\textsuperscript{\protect\hyperlink{ref-WHO2014severe}{19}}, el
    cual incluye una o más de las siguientes (todas registradas en
    mono-infecciones por \emph{P. vivax}):

    \begin{enumerate}
    \def\labelenumiii{\arabic{enumiii}.}
    \tightlist
    \item
      condición neurológica: coma, mareo, pérdida conciencia;
    \item
      condición hematológica: anemia/trombocitopenia severa;
    \item
      síntomas sistémicos: shock circulatorio; y
    \item
      fallo de órganos vitales: disfución respiratoria, estrés
      respiratorio agudo, daño renal agudo, ruptura esplénica,
      disfunción hepática e ictericia (hiperbilirrubina).
    \end{enumerate}
  \item
    \textbf{Riesgo}

    La vulnerabilidad a la Malaria Severa ha sido asociada a la
    intensidad de transmisión y el desarrollo de la inmunidad
    independiente a la
    edad\textsuperscript{\protect\hyperlink{ref-reyburn2015}{5}}. Por
    ejemplo, en zonas de alta transmisión como en el África
    subsahariana, las poblaciones más vulnerables son: niños menores de
    5 años con un desarrollo incompleto de inmunidad parcial contra la
    malaria\textsuperscript{\protect\hyperlink{ref-Stanisic2015}{25}},
    mujeres embarazadas en parte a la adhesión placentaria de glóbulos
    rojos infectados
    (iRBC)\textsuperscript{\protect\hyperlink{ref-rogerson2007preg}{26}},
    y viajeros o migrantes sin inmunidad provenientes de áreas con baja
    o ninguna transmisión de malaria. Por otro lado, en zonas de baja
    transmisión como en Asia y América Latina, al haber una menor
    exposición a la infección, la mayoría de la población llega a la
    adultez sin haber desarrollado una inmunidad protectiva. Como
    consecuencia, tanto en la Amazonía peruana como en el Norte costero
    la población adolescente y adulta joven es la más susceptible a
    desarrollar esta
    patología\textsuperscript{\protect\hyperlink{ref-quispe2014}{4},\protect\hyperlink{ref-llanoschea2015}{27}},
    comúnmente al iniciar trabajos a campo abierto, e.g.~actividades
    madereras o mineras, en zonas de alto riesgo de contacto con
    mosquitos
    infectados\textsuperscript{\protect\hyperlink{ref-factores2001}{28}}.

    Cabe agregar que la presencia de comorbilidades, coinfecciones y
    malnutrición son reconocidos como factores contribuyentes a la
    Malaria Severa, los cuales deben ser adecuadamente registrados. Sin
    embargo, se ha sugerido que la coinfección de \emph{Plasmodium} con
    Dengue no está asociada con una peor enfermedad, asemejándose a una
    mono-infección con Dengue tanto en frecuencia de síntomas como en el
    nivel de marcadores
    inmunológicos\textsuperscript{\protect\hyperlink{ref-baldevi2016}{29}}.
    \newpage
  \end{enumerate}
\item
  Epidemiología

  \begin{enumerate}
  \def\labelenumii{\roman{enumii}.}
  \item
    \textbf{A nivel mundial}

    Entre el 2005 y 2015, 1-3\% de casos no-complicados fueron asumidos
    como malaria severa, causante de la muerte de 429,000
    personas\textsuperscript{\protect\hyperlink{ref-WHO2016world}{1}} y
    en un 90\% niños menores de 5 años en
    África\textsuperscript{\protect\hyperlink{ref-wassmer2015}{30}}. De
    las estrategias preventivas desarrolladas hasta el momento (control
    vectorial, quimioprevención, vacunas e indicadores), la única vacuna
    en completar un ensayo de fase 3, RTS,S/AS01, solo redujo en un 39\%
    la incidencia de enfermedad y en 31.5\% la de malaria severa en
    niños entre 5-17
    meses\textsuperscript{\protect\hyperlink{ref-rts2015}{31}}.

    Con respecto a la malaria vivax severa, un reciente meta-análisis
    con 46,411 casos registrados desde el año 1900 identificó patrones
    geográficos en la prevalencia de manifestaciones
    clínicas\textsuperscript{\protect\hyperlink{ref-rahimi2014meta}{32}}.
    Concluyeron que en las zonas de alta transmisión (e.g., Papúa Nueva
    Guinea) la anemia severa fue el síntoma con mayor prevalencia en
    niños jóvenes. Por otro lado, en zonas con baja transmisión (e.g.,
    la región de las Américas) los adultos eran los más susceptibles a
    la severidad con una mayor variedad de disfunciones orgánicas. Por
    ejemplo, en Brasil un estudio retrospectivo identifico 17/11,251
    casos de malaria vivax severa, entre 28-80 años de edad, con
    hiperbilirrubina (59\%), anemia severa (29\%), daño renal agudo
    (12\%), daño pulmonar (12\%) y shock circulatorio
    (6\%)\textsuperscript{\protect\hyperlink{ref-alexandre2010}{33}}.
  \item
    \textbf{En el Perú}

    En el norte costero, Quispe et al.
    2014\textsuperscript{\protect\hyperlink{ref-quispe2014}{4}} mediante
    un estudio retrospectivo (2008-2009) identificó 81/6502 casos de
    malaria vivax severa con anemia severa (57\%), shock circulatorio
    (25\%), hiperbilirrubina (25\%), daño pulmonar (21\%), daño renal
    agudo (14\%) y malaria cerebral (11\%). Comparados con los pacientes
    no complicados, los severos fueron mayores (38 vs 26 años,
    P\textless{}0.001).

    En el noreste amazónico, un reciente estudio prospectivo dirigido
    por El Centro de Enfermedades Tropicales de la Marina de los Estados
    Unidos NAMRU-6
    (2012-2013)\textsuperscript{\protect\hyperlink{ref-smith2013}{34}}
    identificó 55/164 casos de malaria vivax severa con postración
    (83\%), confusión (32\%), shock (27.6\%), daño pulmonar (24.5\%)
    hiperbilirubina (18.1\%), daño renal (2\%) y anemia severa (1.7\%).
    No se encontraron diferencias con respecto a la edad entre
    no-severos y severos (33 vs 29 años, P=0.497). Sin embargo, sí se
    identificó una mayor proporción de pacientes con malaria severa en
    mujeres (52 vs 35\%, P=0.039).
  \end{enumerate}
\item
  Biología

  \begin{enumerate}
  \def\labelenumii{\roman{enumii}.}
  \item
    \textbf{Comparación}

    Históricamente, la malaria vivax ha sido considerada como
    ``benigna'' en comparación a \emph{P. falciparum} debido a su: (i)
    baja invasión parasitaria, sesgada a reticulocitos y rutas alternas
    de menor efectividad; y (ii) pobre citoadhesión de sus iRBC, dada
    por la ausencia de protrusiones abastonadas o \emph{knob
    protrusions}. Sin embargo, la presencia de genes homólogos a Pf
    \emph{var}, o genes Pv \emph{vir}, y la baja proporción de
    esquizontes tardíos sugieren la presencia de secuestramiento
    tisular, corroborado en necropsias mas no experimentalmente en
    sistemas de cultivo
    continuo\textsuperscript{\protect\hyperlink{ref-wassmer2015}{30}}.
    En línea con esta evidencia, recientemente se ha demostrado que en
    los casos severos por \emph{P. vivax} la parasitemia periférica
    subestima la biomasa parasitaria
    total\textsuperscript{\protect\hyperlink{ref-barber2015}{35}}. En
    este estudio, la parasitemia oculta fue la mayor contribuyente de
    citoquinas proinflamatorias, semejante a lo observado con \emph{P.
    falciparum}, mas no de la activación endotelial, sugiriendo una
    posible acumulación parasitaria en partes de órganos sin endotelio.
  \item
    \textbf{Patogénesis}

    Se han propuesto mecanismos en base a lo observado en los casos de
    malaria cerebral por \emph{P. falciparum}, donde los procesos de
    invasión y adhesión parasitaria son relevantes. La respuesta celular
    contra la parasitemia, dependiente de su tasa de multiplicación,
    sumada a la citoadherencia de iRBC en estadio esquizonte a endotelio
    o RBC no infectados (rosetamiento), desencadena una obstrucción
    microvascular que activa a sus células endoteliales. En respuesta,
    estas liberan una mayor cantidad de citoquinas proinflamatorias,
    provocando la pérdida de perfusión y una consecuente disfunción
    microvascular que progresa por retroalimentación positiva.
  \end{enumerate}
\end{enumerate}

\subsubsection{Anticuerpos}\label{anticuerpos}

\begin{enumerate}
\def\labelenumi{\alph{enumi}.}
\item
  Relevancia

  La concentración de anticuerpos contra antígenos de \emph{Plasmodium}
  es un marcador sensible a la exposición de malaria a nivel
  poblacional\textsuperscript{\protect\hyperlink{ref-elliott2014}{13}}.
  Con métodos analíticos aplicables a escala epidemiológica, es posible
  inferir su desarrollo e identificar focos o \emph{hotspots} de
  transmisión\textsuperscript{\protect\hyperlink{ref-hotspots2015}{12}}.
  Además, su aplicación en salud pública tiene el potencial de
  determinar el impacto de intervenciones, confirmar los estados de
  control o eliminación, y monitorear
  reemergencias\textsuperscript{\protect\hyperlink{ref-sepulveda2015}{36}}.

  Sin embargo, las evidencias a favor de marcadores en \emph{P. vivax}
  continúan estando limitadas por la ausencia de cultivos \emph{in
  vitro} y modelos animales de infección. Esto se ve reflejado en sus
  dos candidatos a vacuna (PvDBP en ensayos clínicos fase 1 y PvCSP en
  estudios pre-clínicos) contra los 23 de \emph{P.
  falciparum}\textsuperscript{\protect\hyperlink{ref-rainbow2016}{37}}.
  Por ello, la principal fuente de evidencia en aquella especie está en
  la identificación de antígenos con respuestas asociadas a la
  adquisición natural de inmunidad en poblaciones endémicas.
\item
  Marcadores

  \begin{enumerate}
  \def\labelenumii{\roman{enumii}.}
  \item
    \textbf{de exposición}

    Su presencia es un paso previo a la adquisición de inmunidad.
    Estudios longitudinales han sugerido que en un inicio la intensidad
    de la respuesta de anticuerpos actúa como marcador de exposición la
    cual, luego de una exposición constante, incrementa hasta superar un
    umbral de
    protección\textsuperscript{\protect\hyperlink{ref-Stanisic2015}{25}}.

    En este sentido, estudios seroepidemiológicos en \emph{P. vivax} han
    logrado asociar marcadores con exposición (PvCSP,
    PvMSP-1\textsubscript{19}, PvMSP-9\textsubscript{RIRII} y PvAMA-1) e
    inmunidad (PvMSP-1\textsubscript{19}, PvMSP-1\textsubscript{NT},
    PvMSP-3\(\alpha\) y
    PvMSP-9\textsubscript{NT})\textsuperscript{\protect\hyperlink{ref-cutts2014meta}{38}}.
    Sin embargo, su principal limitante está en la carencia de
    evaluaciones en su cinética de adquisición, desarrollo y
    mantenimiento en cohortes longitudinales. 
  \item
    \textbf{de inmunidad}

    En áreas de alta transmisión, los niños adquieren resistencia a la
    manifestación severa de la malaria a la edad de cinco años,
    aproximadamente. Sin embargo, continúan siendo susceptibles a
    episodios no-complicados hasta la adolescencia, donde adquieren un
    estado resistente a la malaria sintomática. A pesar de ello, no se
    ha demostrado que esta adquisición natural de inmunidad por la
    exposición acumulada a malaria otorgue una protección esterilizante
    o de resistencia a la
    infección\textsuperscript{\protect\hyperlink{ref-crompton2014rev}{39}}.
    Los detalles de los procesos involucrados en cada una de estas fases
    no están completamente entendidos, pero se tiene detalles de algunos
    componentes:

    \begin{enumerate}
    \def\labelenumiii{\arabic{enumiii}.}
    \item
      \textbf{contra la severidad}

      Se conoce que la respuesta contra esta manifestación está guiada
      por una excesiva respuesta inflamatoria que incluye la producción
      de las citoquinas proinflamatorias TNF-\(\alpha\), INF-\(\alpha\),
      IL-1\(\beta\) e
      IL-6\textsuperscript{\protect\hyperlink{ref-baird2013}{40}} junto
      a una reducción en la producción de la citoquina antiinflamatoria
      IL-10, un regulador central en la respuesta contra la
      malaria\textsuperscript{\protect\hyperlink{ref-jagannathan2014}{41}}.
      Es por ello que esta primera fase implicaría la adquisición de
      inmunidad a componentes parasitarios que gatillan una respuesta
      inflamatoria innata. Una hipótesis está en anticuerpos que
      interfieran la unión de glucoproteínas proinflamatorias con
      receptores
      Toll\textsuperscript{\protect\hyperlink{ref-schofield2006toll}{42},\protect\hyperlink{ref-coban2005toll}{43}}.

      Por otro lado, la adquisición de anticuerpos que bloqueen la
      adhesión de iRBC o invasión parasitaria también son candidatos a
      otorgar una protección a este
      nivel\textsuperscript{\protect\hyperlink{ref-wassmer2015}{30}}.
      Antígenos referenciales son las proteínas VIR en \emph{P. vivax},
      homólogas a VAR en \emph{P. falciparum}, posiblemente secretadas
      en iRBC y causantes de adhesividad a endotelio o RBC no
      infectados, condición gatilladora del rosetamiento y posterior
      obstrucción
      microvascular\textsuperscript{\protect\hyperlink{ref-portillo2001vir}{44}},
      y las proteínas de la familia PvDBP y PvRBP (\emph{Duffy-} y
      \emph{Reticulocyte- binding proteins}), responsables de la
      invasión a reticulocitos por la ruta tradicional y alternativa,
      respectivamente\textsuperscript{\protect\hyperlink{ref-galinski1992rbp}{45}}.
      \newpage
    \item
      \textbf{contra la enfermedad}

      Los antígenos del estadio eritrocítico (merozoitos) fueron los
      primeros marcadores en ser propuestos como importantes para la
      protección a este nivel, tal como lo sugirió en 1961 la terapia
      efectiva por transferencia pasiva de anticuerpos donados por
      adultos con inmunidad adquirida de forma
      natural\textsuperscript{\protect\hyperlink{ref-cohen1961}{46}}.

      En el caso de \emph{P. vivax}, su adquisición ocurre con una mayor
      rapidez, posiblemente facilitada por sus particularidades
      biológicas\textsuperscript{\protect\hyperlink{ref-mueller2013}{47}}.
      Los antígenos propuestos hasta el momento para esta especie son
      proteínas del micronema de merozoitos como DBP y AMA1, y proteínas
      de superficie como MSP1, MSP1-P (región C-terminal), MSP3
      (PvMSP-3\(\alpha\), PvMSP-3\(\beta\)) y
      MSP9\textsuperscript{\protect\hyperlink{ref-lopez2017}{48}}. 
    \item
      \textbf{contra la parasitemia}

      La adquisición de una inmunidad esterilizante solo ha sido
      comprobada mediante la inmunización con esporozoitos atenuados por
      radiación. Tanto en \emph{P. falciparum} como en \emph{P. vivax}
      se ha confirmado esta respuesta. Un reciente ensayo clínico de
      fase 2 en Colombia con \emph{P. vivax} encontró que luego de siete
      inmunizaciones a lo largo de 56 semanas y una reexposición con
      esporozoitos infecciosos, 5/12 voluntarios obtuvieron dicha
      protección y las respuestas de anticuerpos IgG1 anti-PvCSP
      estuvieron asociadas a
      ella\textsuperscript{\protect\hyperlink{ref-arevalo2016spz}{49}}. 
    \end{enumerate}
  \end{enumerate}
\item
  Enfoque a larga escala

  El parásito de la malaria, al tener un ciclo de vida complejo con
  múltiples estadios de desarrollo en el humano, posee también un
  transcriptoma, proteoma y, por lo tanto, un inmunoma
  estadio-específico que deriva de un total de \textasciitilde{}5300
  proteínas putativas. Por este motivo, tanto los esfuerzos de vacunas
  por subunidades como por organismos completos atenuados han fracasado,
  en contraste a la eficacia observada de dichas estrategias en otras
  infecciones como la HBsAg contra la hepatitis B y BCG contra la
  tuberculosis,
  respectivamente\textsuperscript{\protect\hyperlink{ref-immunomics2016}{50}}.
  Incluso, ensayos clínicos de fase 3 han demostrado que los actuales
  candidatos poseen una pobre inmunogenicidad y protección contra la
  malaria (e.g.,
  RTS,S)\textsuperscript{\protect\hyperlink{ref-rts2015}{31}}.

  \begin{enumerate}
  \def\labelenumii{\roman{enumii}.}
  \item
    \textbf{Inmunómica}

    La era genómica permitió un cambio de perspectiva en la
    investigación inmunológica: se pasó de una enfocada en plataformas
    de vacunación a una en antígenos de vacunación. Es por ello que,
    actualmente, el objetivo de mayor importancia está en la
    identificación de subgrupos de antígenos capaces de inducir
    respuestas inmunes
    protectivas\textsuperscript{\protect\hyperlink{ref-Davies2015Large}{51}}.

    El primer enfoque en respuesta a esta problemática fue llamado
    \emph{Vacunología Reversa}, el cual hace uso de la información
    genómica de los patógenos para la predicción bioinformática de
    antígenos candidatos a vacuna a partir de características protéticas
    asociadas a inmunogenicidad y efectividad en vacunación. La
    principal limitación de este enfoque radica en el que el genoma es
    estático en el tiempo, no aplicable en organismos con ciclos de vida
    complejos\textsuperscript{\protect\hyperlink{ref-immunomics2016}{50}}.

    Un segundo enfoque fue llamado \emph{Inmunómica}, el cual hace uso
    de la información de ambos agentes interaccionantes. La selección de
    inmunógenos es realizada por métodos empíricos usando data
    transcriptómica y proteómica de patógenos dinámicos más la
    información clínica del hospedero, o métodos teóricos empleando
    algoritmos computacionales predictivos que tomen en cuenta la
    afinidad entre péptidos del patógeno y moléculas del complejo
    principal de histocompatibilidad (MHC) del hospedero. La principal
    ventaja del primer método está en la capacidad de priorizar la
    selección de antígenos en base a criterios clínicamente relevantes,
    como el de individuos con inmunidad por adquisición
    natural\textsuperscript{\protect\hyperlink{ref-immunomics2016}{50}}.

    Este enfoque ha permitido obtener datos empíricos sobre diferencias
    en la amplitud, intensidad, cinética y longevidad de respuestas
    inmunes inducidas por patógenos. Además de mayor información sobre
    aspectos clave como el porcentaje del proteoma reconocido por el
    hospedero, la capacidad predictora de anticuerpos sobre el estado de
    la enfermedad o de las secuencias de aminoácidos sobre su
    inmunogenicidad\textsuperscript{\protect\hyperlink{ref-Davies2015Large}{51}}.
    Dos plataformas representativas son los microarreglos de proteínas y
    péptidos, este último con mayor resolución en el mapeo de
    epítopes\textsuperscript{\protect\hyperlink{ref-carmona2015peptide}{52}}.
  \end{enumerate}
\end{enumerate}

\subsubsection{Microarreglos}\label{microarreglos}

\begin{enumerate}
\def\labelenumi{\alph{enumi}.}
\item
  De proteínas

  Esta es una técnica empleada en la medición a larga escala de las
  interacciones o actividades de proteínas capaz de reconstruir redes
  biológicas entre diferentes clases de moléculas y estados
  celulares\textsuperscript{\protect\hyperlink{ref-uzoma2013interactome}{53}}.
  En el contexto inmunómico, cuantificando la interacción
  antígeno-anticuerpo, el tamaño y complejidad de los genomas de
  \emph{Plasmodium} desafían su ejecución a una escala de proteoma
  completo. Por esta razón, su diseño requiere de procesos de selección
  en base a características proteicas y conocimiento empírico. De esta
  manera se ha logrado desarrollar plataformas costo-efectivas y rápidas
  de emplear, capaces de facilitar el descubrimiento de antígenos para
  vacunas o serovigilancia con el menor trabajo en laboratorio e
  independiente de algoritmos predictivos.

  \begin{enumerate}
  \def\labelenumii{\roman{enumii}.}
  \item
    \textbf{Primera generación}

    Finney et al.
    2014\textsuperscript{\protect\hyperlink{ref-Finney2014}{54}}
    diseñaron microarreglos de \emph{P. vivax} y \emph{P. falciparum} en
    base a (i) proteínas predichas para el estadio eritrocítico con
    capacidad de ser expuestas a superficie, (ii) evidencia en
    microarreglos, proteomas, EST, o secretoma predicho para \emph{P.
    falciparum}, (iii) proteínas únicas y ortólogas con presencia de
    péptidos señal (SP) o dominios transmembrana (TMD) para \emph{P.
    vivax}, (iv) proteínas putativas citoplasmáticas carentes de SP o
    TMD en base a filtros de punto isoeléctrico por especie, (v)
    exclusión de proteínas con variabilidad antigénica (e.g., PfEMP1,
    RIFIN, surfin, stevors; Pv \emph{var} y pseudogenes), e (vi)
    inclusión de genes simples y multi-exónicos.
  \item
    \textbf{Segunda generación}

    En el 2015, el Centro Internacional de Excelencia para la
    Investigación de la Malaria ICEMR realizó una subselección empírica
    del microarreglo anteriormente
    detallado\textsuperscript{\protect\hyperlink{ref-King2015FOC}{55}}.
    Para ello sondearon los microarreglos de \emph{P. falciparum} con 20
    muestras de Papúa Nueva Guinea, 20 de Kenya, 20 de Mali y 10
    controles norteamericanos, y los de \emph{P. vivax} con 15 de Papúa
    Nueva Guinea, 15 de China, 22 de Perú, 10 de Tailandia y 10
    controles. Luego se seleccionó a los antígenos seroreactivos por
    país, cumpliendo la condición de ser mayores a dos veces la
    desviación estándar de la media de la reactividad serológica en los
    controles. Finalmente, se seleccionó el top 500 para ambas especies
    empleando un filtrado jerárquico, dándole prioridad a los antígenos
    con seroreactividad en todos los países y en las posiciones
    restantes los antígenos en orden descendiente a su reactividad
    promedio. Este diseño ha sido depositado en la base de datos GEO con
    el código
    \href{https://www.ncbi.nlm.nih.gov/geo/query/acc.cgi?acc=GPL18316}{GPL18316}.
  \end{enumerate}
\item
  Análisis de datos

  Sundaresh et al. 2006 demostró que las características de las
  mediciones en microarreglos de proteínas permitían la adaptación y
  aplicación de las técnicas de análisis desarrolladas para
  microarreglos de
  ADN\textsuperscript{\protect\hyperlink{ref-sundaresh2006}{56}}. Ambas
  plataformas buscan resolver el mismo problema: realizar un análisis
  diferencial de señales de luz derivadas de eventos de unión molecular.
  Por este motivo, su análisis debe de incluir a los componentes
  estandarizados para el análisis de
  microarreglos\textsuperscript{\protect\hyperlink{ref-allison2006}{57}},
  además de parámetros de relevancia exclusiva para el análisis de
  respuestas humorales. Finalmente, se detallarán las ventajas del uso
  de software libre para su ejecución, con respecto a los privativos y
  servidores web.

  \begin{enumerate}
  \def\labelenumii{\roman{enumii}.}
  \item
    \textbf{Componentes}

    \begin{enumerate}
    \def\labelenumiii{\arabic{enumiii}.}
    \item
      \textbf{Diseño}

      De acuerdo al MIAME o información mínima sobre experimentos
      basados en
      microarreglos\textsuperscript{\protect\hyperlink{ref-brazma2001}{58}},
      se requiere definir 6 secciones a ser detalladas en el
      \protect\hyperlink{meto}{capítulo 4}:

      \begin{itemize}
      \tightlist
      \item
        Diseño experimental: descripción del conjunto total de
        experimentos a ejecutar;
      \item
        Diseño de arreglos: detalle de la información de los elementos a
        incluir en los \emph{spots};
      \item
        Muestras: descripción de la fuente de la muestra y el criterio
        para su clasificación;
      \item
        Hibridación: descripción de las condiciones de laboratorio bajo
        las cuales se realizará;
      \item
        Mediciones: descripción del progreso de las imágenes escaneadas
        a la matriz de datos; y
      \item
        Controles de normalización: descripción de los elementos
        conocidos o invariantes.
      \end{itemize}
    \item
      \textbf{Pre procesamiento}

      Incluye la transformación, normalización y filtrado de la matriz
      de datos.

      Primero, la transformación tiene el objetivo reducir de la
      heterocedasticidad o heterogeneidad de varianzas entre las
      observaciones producto de errores aleatorios en proceso de
      hibridación, mas no del
      instrumento\textsuperscript{\protect\hyperlink{ref-kreil2005bullet}{59}}.
      Este fenómeno se evidencia en el incremento de la varianza de las
      medidas directamente proporcional al incremento de su
      intensidad\textsuperscript{\protect\hyperlink{ref-brown2001image}{60}}.
      Para ello, dos principales transformaciones han sido definidas:

      \begin{itemize}
      \tightlist
      \item
        logarítmica, la cual asume un error multiplicativo que corrige
        las varianzas de alta intensidad pero puede inflar a las de baja
        intensidad; y
      \item
        asinh, la cual asume un error aditivo y multiplicativo que
        corrige a las varianzas de alta y baja intensidad.
      \end{itemize}

      Segundo, la normalización busca homogeneizar la variabilidad entre
      arreglos en base a controles (blanco, negativos o positivos), los
      cuales deben ser invariantes entre muestras. De acuerdo al diseño
      de arreglos, este procedimiento es ejecutado de tres formas:

      \begin{itemize}
      \tightlist
      \item
        sustracción de controles, aplicado luego de ejecutar una
        transformación logarítmica y llamado \emph{fold over control}
        (FOC) en microarreglos de
        proteínas\textsuperscript{\protect\hyperlink{ref-King2015FOC}{55}};
        y
      \item
        estabilización de varianzas o VSN, aplicado en forma paralela a
        la transformación asinh con respecto a controles \emph{spike-in}
        de intensidad invariante entre
        muestras\textsuperscript{\protect\hyperlink{ref-huber2002vsn}{61}}.
      \item
        modelo robusto lineal o RLM, con la capacidad de controlar la
        variabilidad entre arreglos (i.e., muestras) observada en los
        diseños experimentales de microarreglos de
        proteínas\textsuperscript{\protect\hyperlink{ref-sboner2009rlm}{62}}.
      \end{itemize}

      Tercero, el filtrado en experimentos a larga escala permite
      incrementar el poder de detección de elementos con expresión
      diferenciada\textsuperscript{\protect\hyperlink{ref-bourgon2010filter}{63}}.
      Las estrategias de corrección posteriores a la comparación
      múltiple e independiente de varios elementos son sensibles a esta
      cantidad. En este sentido, este procedimiento retira de forma
      preliminar a los elementos con reducidas probabilidades de
      expresarse diferencialmente sin la necesidad de realizar la prueba
      de hipótesis.
    \item
      \textbf{Inferencia}

      Los objetivos de la inferencia estadística de microarreglos son
      (i) poner a prueba la hipótesis nula de igualdad de medias en las
      expresión de elementos por grupo y (ii) ranquear a los elemento en
      orden a la evidencia contra la hipótesis nula, incluso luego de
      fallar en el rechazo de dicha
      hipótesis\textsuperscript{\protect\hyperlink{ref-smyth2004ebayes}{64}}.
      Sin embargo, a esta escala existen dos principales problemas:

      \begin{itemize}
      \tightlist
      \item
        presencia de varianzas artificialmente altas o bajas producto
        del bajo número de réplicas entre arreglos o
        muestras\textsuperscript{\protect\hyperlink{ref-baldi2001cybert}{65}};
        y un
      \item
        amplio número de hipótesis puestas a prueba en forma simultánea,
        donde los test estadísticos tradicionales pueden generar un
        largo número de falsos
        positivos\textsuperscript{\protect\hyperlink{ref-kayala2012cyber}{66}}.
      \end{itemize}

      Ante el primer problema, un enfoque bayesiano empírico permite
      reducir (o moderar) las varianzas de todas las lecturas. Una
      primera estrategia corregía la varianza observada de cada gen
      tomando en cuenta la varianza de genes vecinos con medias de
      expresión
      similar\textsuperscript{\protect\hyperlink{ref-baldi2001cybert}{65}}.
      Posteriormente, esta se extendió y optimizó mediante la estimación
      de una varianza general previa para la actualización de todas las
      varianzas
      observadas\textsuperscript{\protect\hyperlink{ref-smyth2004ebayes}{64}}.
      Con ello, la ejecución de un test-t con varianzas moderadas (o
      test-t moderado) es capaz de obtener inferencias más estables que
      el test-t ordinario con un número limitado de
      réplicas\textsuperscript{\protect\hyperlink{ref-kayala2012cyber}{66}}.

      Ante el segundo problema, los métodos de corrección o ajuste de
      valores p permiten que el investigador pueda controlar la razón de
      falsos positivos con respecto a todos los descubrimientos o razón
      de falsos descubrimientos
      (FDR)\textsuperscript{\protect\hyperlink{ref-brazma2001}{58}}. Dos
      principales enfoques han sido propuestos. El enfoque
      probabilístico emplea modelos mixtos sobre la distribución de
      valores p para diferenciar a las poblaciones de hipótesis nulas y
      alternativas\textsuperscript{\protect\hyperlink{ref-allison2002mmm}{67}}.
      El enfoque frecuentista emplea el método Benjamini-Hochberg basado
      en la determinación de un valor crítico de valores p dependiente
      del total de hipótesis puestas a prueba y el valor de FDR
      deseado\textsuperscript{\protect\hyperlink{ref-benjamini1995fdr}{68}}.
    \item
      \textbf{Clasificación}

      Una aplicación importante de los microarreglos de proteínas está
      en la identificación de un conjunto de antígenos de
      serodiagnóstico que puedan detectar patrones y clasificar perfiles
      de respuesta
      humoral\textsuperscript{\protect\hyperlink{ref-sundaresh2007}{69}}.
      Por ello, el esquema de ranqueo de elementos comúnmente ha sido
      complementado con técnicas de clasificación. Este proceso
      involucra la asignación de objetos (e.g., genes) en categorías
      pre-existentes (clasificación supervisada), o el desarrollo
      progresivo de un conjunto de categorías por características de los
      objetos. (clasificación no
      supervisada)\textsuperscript{\protect\hyperlink{ref-allison2006}{57}}.

      La clasificación supervisada o predicción de clase, define \emph{a
      priori} el número de categorías, el conjunto de entrenamiento y el
      de prueba. Requiere del uso de procedimientos de validación
      cruzada y tomar en cuenta sesgos de selección. Por otro lado, su
      precisión predictiva es afectada por la sobreparametrización, más
      aún con bajo número de muestras y amplia cantidad de elementos u
      objetos. Ejemplos de esta técnica son \emph{support vector
      machines} (SVM), \emph{k-nearest neighbors} (k-NN) y validación
      por \emph{leave-one-out cross-validation} (LOOCV).

      La clasificación no-supervisada o descubrimiento de clase, agrupa
      objetos en base a métricas de similaridad de los objetos a lo
      largo de las muestras. Para ello requiere definir 2 tipos de
      algoritmos: de distancia y agrupamiento. Ejemplos del primero son
      la distancia Euclidiana, en la que se prioriza la magnitud sobre
      la forma de la distribución de señales, y la correlación de
      Pearson, en la que se prioriza la forma sobre la magnitud. Ejemplo
      del segundo es el agrupamiento jerárquico o \emph{hierarchical
      clustering}, en la que los racimos o \emph{clusters} se forman
      iterativamente al calcular la distancia entre elementos y racimos
      en formación. Otros son el \emph{k-mean clustering},
      \emph{self-organizing maps} (SOM) y \emph{principal component
      analysis} (PCA). Las principales desventajas de la técnica están
      en el requerimiento de decisiones entre varios algoritmos sin
      consenso y su escasa reproducibilidad entre experimentos.
    \end{enumerate}
  \item
    \textbf{Parámetros adicionales}

    Además de la reactividad serológica, dos parámetros han demostrado
    ser de vital importancia para la caracterización de la respuesta
    humoral: la amplitud e intensidad de
    respuesta\textsuperscript{\protect\hyperlink{ref-crompton2010}{9},\protect\hyperlink{ref-Helb2015exposure}{10},\protect\hyperlink{ref-King2015FOC}{55}}.
    El primero se obtiene calculando la proporción o porcentaje de
    antígenos reactivos por individuo y el segundo por el promedio de
    sus intensidades. Particularmente, la amplitud ha demostrado ser tan
    relevante como la reactividad de antígenos individuales en el
    desarrollo de inmunidad contra la
    malaria\textsuperscript{\protect\hyperlink{ref-crompton2010}{9}}. 
  \item
    \textbf{Software}

    \begin{enumerate}
    \def\labelenumiii{\arabic{enumiii}.}
    \item
      \textbf{Privativo}

      La amplia variedad de herramientas desarrolladas para la solución
      de cada uno de los componentes mencionados ha generado una alta
      heterogeneidad de metodologías observadas en la literatura. Este
      problema limita la reproducibilidad de estos estudios debido al
      reporte excesivo o escaso de parámetros empleados, además de
      desacelerar el progreso del área. Ejemplos de ello son Molina et
      al. 2012 (Cyber-T y GraphPad
      Prism)\textsuperscript{\protect\hyperlink{ref-molina2012}{70}},
      Baum et al. 2013, 2015 (SAM, MeV, JMP9.0 y
      Systat11)\textsuperscript{\protect\hyperlink{ref-baum2013}{71},\protect\hyperlink{ref-baum2015}{72}},
      Lu et al. 2014 (GraphPad, SigmaPlot10.0, Cluster3.0, TreeView y
      pearsoncorrelation.com)\textsuperscript{\protect\hyperlink{ref-lu2014rama}{73}},
      y Chen et al. 2015 (SAS, MULTTEST y TIGR
      MeV)\textsuperscript{\protect\hyperlink{ref-chen2015immunomics}{74}}.
    \item
      \textbf{Libre}

      En respuesta a esta problemática, el uso de lenguajes de
      programación de alto nivel como
      R\textsuperscript{\protect\hyperlink{ref-R2016}{17}} junto a
      ambientes de análisis como
      Bioconductor\textsuperscript{\protect\hyperlink{ref-bioconductor2004}{18}}
      han reducido las barreras de acceso a la investigación
      interdisciplinaria para el análisis transparente de datos
      moleculares de larga escala. Su flexibilidad a la creación de
      \emph{scripts} de análisis o \emph{workflows} combinando
      diferentes tipos de data, herramientas estadísticas y
      visualizaciones\textsuperscript{\protect\hyperlink{ref-Biobase}{75}}
      dentro de reportes
      dinámicos\textsuperscript{\protect\hyperlink{ref-knitr}{76}} con
      registro explícito de todas las ediciones a data cruda y
      parámetros empleados, permite seguir los principios de Ciencia
      Reproducible\textsuperscript{\protect\hyperlink{ref-CienciaReproducible2016}{77}}.
      \newpage
    \item
      \textbf{Servidores}

      Desventajas otorgadas a los software privativos por sus limitados
      métodos estadísticos y capacidades visuales, y a las alternativas
      libres por las habilidades computacionales requeridas para su uso,
      han estimulado el desarrollo de servidores web como
      GMine\textsuperscript{\protect\hyperlink{ref-gmine2016}{78}} para
      el análisis integral y exploratorio. Sin embargo, la falta de
      reporte o registro de los de parámetros empleados nuevamente cae
      en una escasa o dilatada reproducibilidad no automatizada.
    \end{enumerate}
  \end{enumerate}
\end{enumerate}

\subsection{Definiciones conceptuales}\label{definiciones-conceptuales}

\begin{enumerate}
\def\labelenumi{\alph{enumi}.}
\item
  Acrónimos

  \textbf{IVTT:} (Proteína) transcrita/traducida \emph{in vitro}.

  \textbf{MFI:} Unidad de lectura cruda expresada como intensidad
  fluorescente promedio de todos los pixeles de cada \emph{spot},
  normalizada localmente mediante la sustracción de la intensidad de
  fondo presente a su alrededor.

  \textbf{RTS:} Sistema rápido de traducción de proteínas recombinantes
  libre de células.
\item
  Términos

  \begin{enumerate}
  \def\labelenumii{\roman{enumii}.}
  \item
    Generales

    \textbf{Antígeno.-} Molécula que se une con productos de la
    respuesta inmune, como anticuerpos o receptores de linfocitos T o
    B\textsuperscript{\protect\hyperlink{ref-abbas2012}{79}}.

    \textbf{Inmunógeno.-} Un antígeno que induce una respuesta
    inmunitaria\textsuperscript{\protect\hyperlink{ref-abbas2012}{79}}.

    \textbf{Inmunoma.-} Conjunto de todos los inmunógenos que
    interaccionan con el sistema inmune de determinado
    hospedero\textsuperscript{\protect\hyperlink{ref-immunomics2016}{50},\protect\hyperlink{ref-sette2005}{80}}.

    \textbf{Inmunómica.-} Estudio del
    inmunoma\textsuperscript{\protect\hyperlink{ref-immunomics2016}{50}}.

    \textbf{Anticuerpo.-} Tipo de molécula glucoproteica, también
    llamada inmunoglobulina (Ig), producida por los linfocitos B, que se
    une a antígenos con un grado alto de especificidad y
    afinidad\textsuperscript{\protect\hyperlink{ref-abbas2012}{79}}.

    \textbf{Anticuerpo secundario.-} Anticuerpo de unión específica a la
    fracción constante de los anticuerpos de un hospedero, comúnmente
    conjugado con biotina.

    \textbf{Fluoróforo.-} Componente de una molécula que brinda la
    cualidad de fluorescencia, comúnmente conjugada con estreptavidina
    para la detección de moléculas biotiniladas.
  \item
    Específicos

    \textbf{Antígeno-IVTT:} Antígeno objetivo o \emph{spot} con proteína
    IVTT a partir de un plásmido con ADN insertado correspondiente al
    polipéptido, segmento o exón de interés.

    \textbf{Control-IVTT:} Control negativo o \emph{spot} con mix de
    expresión RTS y plásmido sin ADN insertado, representante de la
    intensidad de fondo específica del paciente.

    \textbf{Proteína purificada:} Control de comparación o \emph{spot}
    con proteína de antigenicidad conocida expresada dentro de célula.

    \textbf{Intensidad de antígeno:} Lectura normalizada de cada
    antígeno-IVTT entre individuos. También llamada ``reactividad
    serológica''.

    \textbf{Antígeno reactivo:} Lectura transformada de cada
    antígeno-IVTT mayor o igual a dos veces la mediana de los
    control-IVTT por individuo.

    \textbf{Frecuencia del antígeno:} Porcentaje de individuos con
    antígeno reactivo por antígeno-IVTT.
  \end{enumerate}
\end{enumerate}

\hypertarget{meto}{\section{METODOLOGÍA}\label{meto}}

\subsection{Diseño}\label{diseno}

El diseño del estudio es de tipo Caso-Control. Su aplicación consistirá
en la selección de la población objetivo por presencia (casos) o
ausencia (controles) del evento en estudio. Además se fijará el número
de eventos a estudiar, así como el número de sujetos sin evento que se
incluirán en la población de comparación.

\subsubsection{Tipo de investigación}\label{tipo-de-investigacion}

\textbf{Por la finalidad: Analítico.} A diferencia de un descriptivo,
realizaremos comparaciones entre grupos y evaluaremos una posible
relación causal entre el factor y el efecto.

\textbf{Por el control de la asignación: Observacional.} A diferencia de
un experimental, no controlaremos la asignación de los factores
(severidad y episodios previos). Es decir, procederemos a observar el
fenómeno, ejecutar la medición y analizar los resultados.

\textbf{Por el seguimiento: Transversal.} A diferencia de uno
longitudinal, no ejecutaremos un seguimiento. Las variables se medirán
una sola vez, en un mismo instante.

\textbf{Por la relación cronológica: Prospectivo.} A diferencia de un
retrospectivo, recolectaremos los datos después de planificado el
estudio.

\textbf{Por la unidad de análisis: Basado en individuo.} A diferencia de
uno ecológico, la unidad de observación son individuos o pacientes.

\subsection{Población y muestra}\label{poblacion-y-muestra}

\subsubsection{Población}\label{poblacion}

El universo teórico está representado por los pacientes con malaria
vivax de la cuenca amazónica del Perú.

El marco muestral está delimitado por pacientes diagnosticados con
malaria vivax de la ciudad de Iquitos, Loreto - Perú, entre enero del
2012 y junio del 2013 a lo largo de un estudio prospectivo ejecutado en
dos hospitales de referencia: Hospital de Apoyo y Hospital Regional.

\subsubsection{Criterios de inclusión y
exclusión}\label{criterios-de-inclusion-y-exclusion}

\begin{enumerate}
\def\labelenumi{\alph{enumi}.}
\item
  Criterios de inclusión

  En el estudio prospectivo mencionado, 164 pacientes con malaria vivax
  fueron enrolados mediante vigilancia pasiva. Según la presencia de uno
  o más parámetros clínicos y de laboratorio del criterio OMS para la
  Malaria
  Severa\textsuperscript{\protect\hyperlink{ref-WHO2014severe}{19}}, los
  pacientes fueron clasificados como severos o no-severos.

  Los parámetros clínicos del criterio OMS son: Shock circulatorio
  (presión sanguínea sistólica \textless{} 80 mmHg), Deterioro del nivel
  de conciencia (puntaje Glasgow \(\le\) 9/14), Daño del sistema
  nerviosos central (convulsión, postración, coma, confusión), Daño
  pulmonar (disnea, taquipnea, infiltración, edema), y el Síndrome de
  dificultad respiratoria aguda o SDRA.

  Los parámetros de laboratorio del criterio OMS son: Hipoglicemia
  (glucosa \textless{} 40 mg/dL), Anemia severa (hemoglobina \textless{}
  7mg/dL), Daño renal (creatinina \textgreater{} 3mg/dl), e
  Hiperbilirrubina (bilirrubina sérica \textgreater{} 2.5 mg/dL). 
\item
  Criterios de exclusión

  Dos criterios de exclusión fueron empleados: Presencia de
  mono-infecciones de \emph{P. vivax} determinadas por PCR y ausencia de
  coinfecciones como leptospirosis, dengue u otra arbovirosis por las
  técnicas de aislamiento viral e inmunofluorescencia.
\end{enumerate}

\subsubsection{Selección de
participantes}\label{seleccion-de-participantes}

A partir de esta clasificación, se ejecutará una selección aleatoria
simple de 30 pacientes con uno o más parámetros del criterio OMS de
malaria severa para el grupo de casos y 30 no-severos para el grupo
control.

\textbf{Tipo de muestra: Probabilística.} Cada elemento de la muestra
será seleccionado al azar, con probabilidad de selección conocida.

\subsection{Recolección de los datos e
instrumento}\label{recoleccion-de-los-datos-e-instrumento}

\subsubsection{Técnica para la recolección de
datos}\label{tecnica-para-la-recoleccion-de-datos}

Previo al tratamiento antimalárico del paciente, se extrajeron las
muestras de sangre tanto para el diagnóstico de malaria vivax por la
técnica de frotis como para las pruebas bioquímicas. Al momento del
diagnóstico positivo, bajo consentimiento informado y de forma
voluntaria, el plasma sanguíneo fue colectado y conservado a -80°C hasta
su uso.

\subsubsection{Instrumento de medición}\label{instrumento-de-medicion}

El presente estudio hará uso de microarreglos diseñados con 1014
fragmentos proteicos recombinantes (498 de \emph{P. falciparum} y 516 de
\emph{P. vivax}, expresado como Pf498/Pv516 o PfPv500) representando 873
proteínas predichas (427 de \emph{P. falciparum} y 446 de \emph{P.
vivax}, \textasciitilde{}8\% del total predicho para el genoma de
\emph{P. vivax} Sal1) seleccionadas luego de una extensiva evaluación
serológica con uno de mayor escala
(Pf2208/Pv2233)\textsuperscript{\protect\hyperlink{ref-Finney2014}{54}}.
Las proteínas son transcritas/traducidas \emph{in vitro} empleando un
sistema de expresión de \emph{Escherichia coli} libre de células e
impresas en bloques de nitrocelulosa sobre una lámina portaobjetos con 8
bloques paralelos, cada uno con 4 arreglos paralelos compuestos por
cuadrículas de 17x17 \emph{spots}. Este diseño ha sido validado mediante
el sondeo con plasma colectado de pacientes con malaria y controles
sanos alrededor del
mundo\textsuperscript{\protect\hyperlink{ref-King2015FOC}{55}}.

\paragraph{Aplicación}\label{aplicacion}

La aplicación del instrumento consiste en tres pasos: sondeo, escaneo y
análisis, tal como ha sido publicado
previamente\textsuperscript{\protect\hyperlink{ref-Driguez2015}{81}}.
Primero, previo al sondeo los microarreglos se hidratan con buffer de
bloqueo dentro de cámaras de incubación. Paralelamente, el plasma de los
pacientes se diluye con aquel buffer en 1:100 y se pre-absorbe con
lisado de \emph{E. coli} en 10\%(w/v), con el objetivo de reducir ruido
de fondo en las mediciones tanto por uniones inespecíficas con el
sustrato como por antígenos bacterianos del sistema de expresión RTS.
Segundo, aspirado el buffer de bloqueo del microarreglo, se procede con
el sondeo al agregar el plasma pre-absorbido e incubar \emph{overnight}
en cámara húmeda a 4°C con leve agitación. Tercero, con lavados y
aspirados repetitivos antes y después de cada paso, se agrega la
solución con anticuerpos secundarios biotinilados y posteriormente la
solución con fluoróforos conjugados con estreptavidina. Cuarto, luego de
centrifugar las láminas, se procede al escaneo de las fluorescencias con
lectores de microarreglos de láser confocal (e.g., Genepix 4300A). Las
medidas crudas se obtienen al aplicar una normalización local mediante
la sustracción de la intensidad de fondo presente alrededor de cada
\emph{spot}, ejecutada en el software del lector (Genepix Pro 7).
Finalmente, se procede al análisis de los datos el cual será detallado
en la \protect\hyperlink{anadata}{sección 4.4}.

\hypertarget{validez}{\paragraph{Validez y
confiabilidad}\label{validez}}

Ambos parámetros serán evaluados siguiendo la metodología empleada en
estudios
previos\textsuperscript{\protect\hyperlink{ref-crompton2010}{9}} y que
estará incluida en la \protect\hyperlink{anadata}{sección 4.4}. Primero,
la validez o exactitud del experimento será evaluada mediante la
correlación lineal entre las lecturas de los antígenos-IVTT y sus
correspondientes proteínas purificadas por muestra, compuestas por
proteínas de inmunogenicidad conocida y expresadas por un sistema dentro
de célula. Segundo, la confiabilidad o reproducibilidad será evaluada
mediante la correlación lineal entre controles positivos agregados a las
lecturas del primer y segundo día de sondeo de las muestras.

\subsubsection{Codificación y creación del archivo de
datos}\label{codificacion-y-creacion-del-archivo-de-datos}

Cada paciente del estudio estará identificado con un código de
estructura \texttt{LIM\#\#\#\#}, e.g.: \texttt{LIM1063}. Los antígenos
proteicos estarán identificados con el código asignado a sus genes en la
base de datos PlasmoDB, e.g.: \texttt{PF3D7\_0202500} o
\texttt{PVX\_091315}, ya sea un gen de \emph{P. falciparum} o \emph{P.
vivax}, respectivamente. En caso los genes posean múltiples exónes, se
amplificarán por separado y se extenderá el código de cada uno con su
número de orden y el total, e.g.: \texttt{\_1o2} exón 1 de un gen con 2
exónes. En caso los genes posean una longitud mayor a 3000 nucleótidos
(nt), se dividirán en segmentos sobrelapantes entre 300-3000nt y se
extenderá el código de cada uno con su respectivo número, e.g.:
\texttt{\_S1} para el primer segmento de un gen.

Los datos serán organizados en dos matrices: (i) archivo
\texttt{samples.csv} con los códigos de los pacientes y sus covariables
epidemiológicas y (ii) archivo \texttt{RawData.csv} con los códigos de
las proteínas y sus lecturas crudas en MFI por código de paciente.

\hypertarget{anadata}{\subsection{Análisis de datos}\label{anadata}}

Se hará uso del software de computación estadística
R\textsuperscript{\protect\hyperlink{ref-R2016}{17}}, complementado con
funciones de distintos paquetes pertenecientes a dos principales
ambientes de análisis:
Bioconductor\textsuperscript{\protect\hyperlink{ref-bioconductor2004}{18}}
y
Tidyverse\textsuperscript{\protect\hyperlink{ref-wickham2016r4ds}{82}}.

\subsubsection{Control de calidad y pre
procesamiento}\label{control-de-calidad-y-pre-procesamiento}

Preliminarmente, se describirá la distribución y proporción de las
covariables epidemiológicas, clínicas y bioquímicas de los pacientes de
la muestra. Luego, con las lecturas crudas en MFI se evaluará la validez
y reproducibilidad del ensayo mediante un test de asociación entre
variables continuas, usando correlación de Pearson (\(r\)) o Spearman
(\(\rho\)), dependiendo de la distribución, con un 0.1\% de error
asumido. Posteriormente, se procederá con su transformación a escala
logarítmica en base 2, normalización entre muestras por sustracción de
la mediana de los control-IVTT a cada antígeno-IVTT por individuo, y
filtrado de todo antígeno que posea una frecuencia de reactividad menor
al 10\%. Para ello, un antígeno reactivo será definido operacionalmente
como el antígeno-IVTT con intensidad normalizada mayor o igual a 1. Por
último, se asociarán ambas matrices de datos en un
\texttt{ExpressionSet}, a través de los códigos de pacientes, empleando
el paquete
\texttt{Biobase}\textsuperscript{\protect\hyperlink{ref-Biobase}{75}}.

\subsubsection{Prueba de hipótesis}\label{prueba-de-hipotesis}

Las dos hipótesis de diferencia entre grupos seguirán el siguiente
protocolo. Primero, se contrastará la amplitud e intensidad de respuesta
con un test de diferencias entre variables continuas y no pareadas de
dos grupos usando t-Student o Mann-Whitney, dependiendo de la
distribución, con un 5\% de error asumido. Segundo, se realizará un test
de reactividad diferenciada de anticuerpos entre dos grupos, usando el
test-t
moderado\textsuperscript{\protect\hyperlink{ref-smyth2004ebayes}{64}}
con corrección por comparación múltiple de la razón de falsos
descubrimientos (FDR) por el método de Benjamini-Hochberg, disponible en
el paquete
\texttt{limma}\textsuperscript{\protect\hyperlink{ref-limma}{83}}, con
un 0.1\% de error asumido. Tercero, se realizará un agrupamiento
jerárquico o \emph{hierarchical clustering} de los antígenos
identificados en base a su distancia euclidiana, disponible en el
paquete
\texttt{NMF}\textsuperscript{\protect\hyperlink{ref-Gaujoux2010NMF}{84}}.
Finalmente, se describirá la presencia de dominios transmembrana,
péptido señal, número de ortólogos en \emph{Plasmodium}, ontología
génica y razón de mutaciones no-sinónimas sobre sinónimas para el
subgrupo de interés, disponible en la base de datos
PlasmoDB\textsuperscript{\protect\hyperlink{ref-plasmodb}{85}}.

\subsubsection{Reporte de resultados}\label{reporte-de-resultados}

Las correlaciones y pruebas de hipótesis serán reportadas con el valor
del estadístico de prueba y el valor p.~Las distribuciones serán
visualizadas con diagramas de dispersión para la correlación entre
variables continuas, diagramas de cajas y barras para la comparación de
variables continuas y frecuencias, respectivamente. Las comparaciones
múltiples reportarán el valor p-ajustado y serán visualizadas con dos
gráficos: un diagrama tipo volcán o \emph{volcano plot} con la
diferencia de reactividad entre grupos en escala logarítmica contra sus
respectivos valores p y un mapa de calor o \emph{heat map} segmentado
por racimos o \emph{clusters}. El informe final consistirá de un reporte
en formato \texttt{R\ Notebook} con extensión \texttt{.Rmd} que
integrará texto, código y
resultados\textsuperscript{\protect\hyperlink{ref-knitr}{76}}, siguiendo
principios de
reproducibilidad\textsuperscript{\protect\hyperlink{ref-CienciaReproducible2016}{77}}.

\section{ASPECTOS ADMINISTRATIVOS}\label{aspectos-administrativos}

\subsection{Cronograma de actividades}\label{cronograma-de-actividades}

\begin{longtable}[]{@{}lllllll@{}}
\toprule
\textbf{ACTIVIDAD PROGRAMADA} & & & & & &\tabularnewline
\midrule
\endhead
& \textbf{Jul} & \textbf{Ago} & \textbf{Sep} & \textbf{Oct} &
\textbf{Nov} & \textbf{Dic}\tabularnewline
Recolección de datos & X & & & & &\tabularnewline
Análisis de datos & & X & & & &\tabularnewline
Interpretación de resultados & & & X & & &\tabularnewline
Redacción final & & & & X & &\tabularnewline
Correcciones & & & & & X &\tabularnewline
\textbf{Sustentación} & & & & & & X\tabularnewline
\bottomrule
\end{longtable}

\subsection{Presupuesto y
financiamiento}\label{presupuesto-y-financiamiento}

El proyecto será realizado en El Centro de Enfermedades Tropicales de la
Marina de los Estados Unidos NAMRU-6 y financiado por The Global
Emerging Infectious Surveillance and Response System (GEIS) section at
the Armed Forces Health Surveillance Branch (AFHSC), The Military
Infectious Diseases Research Program (MIDRP) y el NIH Training grant
2D43TW007393.

\begin{longtable}[]{@{}lcc@{}}
\toprule
\begin{minipage}[b]{0.46\columnwidth}\raggedright\strut
\textbf{DESCRIPCIÓN}\strut
\end{minipage} & \begin{minipage}[b]{0.22\columnwidth}\centering\strut
\textbf{MONTO (S/)}\strut
\end{minipage} & \begin{minipage}[b]{0.22\columnwidth}\centering\strut
\textbf{PORCENTAJE (\%)}\strut
\end{minipage}\tabularnewline
\midrule
\endhead
\begin{minipage}[t]{0.46\columnwidth}\raggedright\strut
\textbf{Bienes}\strut
\end{minipage} & \begin{minipage}[t]{0.22\columnwidth}\centering\strut
\strut
\end{minipage} & \begin{minipage}[t]{0.22\columnwidth}\centering\strut
\strut
\end{minipage}\tabularnewline
\begin{minipage}[t]{0.46\columnwidth}\raggedright\strut
Papelería, útiles y material de oficina\strut
\end{minipage} & \begin{minipage}[t]{0.22\columnwidth}\centering\strut
50\strut
\end{minipage} & \begin{minipage}[t]{0.22\columnwidth}\centering\strut
0.16\strut
\end{minipage}\tabularnewline
\begin{minipage}[t]{0.46\columnwidth}\raggedright\strut
Insumos, instrumental y accesorios de laboratorio\strut
\end{minipage} & \begin{minipage}[t]{0.22\columnwidth}\centering\strut
200\strut
\end{minipage} & \begin{minipage}[t]{0.22\columnwidth}\centering\strut
0.63\strut
\end{minipage}\tabularnewline
\begin{minipage}[t]{0.46\columnwidth}\raggedright\strut
Productos químicos\strut
\end{minipage} & \begin{minipage}[t]{0.22\columnwidth}\centering\strut
25\strut
\end{minipage} & \begin{minipage}[t]{0.22\columnwidth}\centering\strut
0.08\strut
\end{minipage}\tabularnewline
\begin{minipage}[t]{0.46\columnwidth}\raggedright\strut
\textbf{Servicios}\strut
\end{minipage} & \begin{minipage}[t]{0.22\columnwidth}\centering\strut
\strut
\end{minipage} & \begin{minipage}[t]{0.22\columnwidth}\centering\strut
\strut
\end{minipage}\tabularnewline
\begin{minipage}[t]{0.46\columnwidth}\raggedright\strut
Compra, sondeo y lectura de microarreglos\strut
\end{minipage} & \begin{minipage}[t]{0.22\columnwidth}\centering\strut
29610\strut
\end{minipage} & \begin{minipage}[t]{0.22\columnwidth}\centering\strut
92.79\strut
\end{minipage}\tabularnewline
\begin{minipage}[t]{0.46\columnwidth}\raggedright\strut
Gastos en el transporte de muestras\strut
\end{minipage} & \begin{minipage}[t]{0.22\columnwidth}\centering\strut
2000\strut
\end{minipage} & \begin{minipage}[t]{0.22\columnwidth}\centering\strut
6.27\strut
\end{minipage}\tabularnewline
\begin{minipage}[t]{0.46\columnwidth}\raggedright\strut
Impresiones, encuadernación y empastado\strut
\end{minipage} & \begin{minipage}[t]{0.22\columnwidth}\centering\strut
25\strut
\end{minipage} & \begin{minipage}[t]{0.22\columnwidth}\centering\strut
0.08\strut
\end{minipage}\tabularnewline
\begin{minipage}[t]{0.46\columnwidth}\raggedright\strut
\textbf{TOTAL}\strut
\end{minipage} & \begin{minipage}[t]{0.22\columnwidth}\centering\strut
31910\strut
\end{minipage} & \begin{minipage}[t]{0.22\columnwidth}\centering\strut
100\strut
\end{minipage}\tabularnewline
\bottomrule
\end{longtable}

\section{BIBLIOGRAFÍA}\label{bibliografia}

\hypertarget{refs}{}
\hypertarget{ref-WHO2016world}{}
1. WHO. \emph{World Malaria Report 2016}. Vol 13. Geneva: World Health
Organization; 2016.
\url{http://www.who.int/malaria/publications/world-malaria-report-2016/report/en/}.

\hypertarget{ref-rosas2016peru}{}
2. Rosas-Aguirre A, Gamboa D, Manrique P, et al. Epidemiology of
plasmodium vivax malaria in Peru. \emph{The American Journal of Tropical
Medicine and Hygiene}. 2016;95(6 Suppl):133-144.
doi:\href{https://doi.org/10.4269/ajtmh.16-0268}{10.4269/ajtmh.16-0268}.

\hypertarget{ref-baird2009}{}
3. Baird JK. Severe and fatal vivax malaria challenges 'benign tertian
malaria' dogma. \emph{Annals of tropical paediatrics}.
2009;29(4):251-252.
doi:\href{https://doi.org/10.1179/027249309X12547917868808}{10.1179/027249309X12547917868808}.

\hypertarget{ref-quispe2014}{}
4. Quispe AM, Pozo E, Guerrero E, et al. Plasmodium vivax
hospitalizations in a monoendemic malaria region: Severe vivax malaria?
\emph{The American journal of tropical medicine and hygiene}.
2014;91(1):11-17.
doi:\href{https://doi.org/10.4269/ajtmh.12-0610}{10.4269/ajtmh.12-0610}.

\hypertarget{ref-reyburn2015}{}
5. Reyburn H, Mbatia R, Drakeley C, et al. Association of transmission
intensity and age with clinical manifestations and case fatality of
severe plasmodium falciparum malaria. \emph{JAMA}.
2005;293(12):1461-1470.
doi:\href{https://doi.org/10.1001/jama.293.12.1461}{10.1001/jama.293.12.1461}.

\hypertarget{ref-norma2001}{}
6. MINSA. \emph{Norma Técnica de Salud Para La Atención de La Malaria Y
Malaria Grave En El Perú}. Ministerio de Salud; 2007.
\url{http://www.minsa.gob.pe/portada/esnemo_normatividad.asp}.

\hypertarget{ref-accelerate2016}{}
7. Quispe AM, Llanos-Cuentas A, Rodriguez H, et al. Accelerating to
zero: Strategies to eliminate malaria in the Peruvian Amazon. \emph{The
American Journal of Tropical Medicine and Hygiene}.
2016;94(6):1200-1207.
doi:\href{https://doi.org/10.4269/ajtmh.15-0369}{10.4269/ajtmh.15-0369}.

\hypertarget{ref-baldevi2013}{}
8. Baldeviano GC, Leiva KP, Quispe AM, et al. Serum markers of severe
clinical complications during plasmodium vivax malaria monoinfections in
the Peruvian Amazon basin. In: \emph{Abstract Book of the Astmh 62nd
Annual Meeting, Nov. 13--17, Washington D.C., United States}.; 2013:340.
\url{http://www.astmh.org/ASTMH/media/Documents/AbstractBook2013Final.pdf}.

\hypertarget{ref-crompton2010}{}
9. Crompton PD, Kayala MA, Traore B, et al. A prospective analysis of
the ab response to plasmodium falciparum before and after a malaria
season by protein microarray. \emph{Proceedings of the National Academy
of Sciences}. 2010;107(15):6958-6963.
doi:\href{https://doi.org/10.1073/pnas.1001323107}{10.1073/pnas.1001323107}.

\hypertarget{ref-Helb2015exposure}{}
10. Helb DA, Tetteh KKA, Felgner PL, et al. Novel serologic biomarkers
provide accurate estimates of recent plasmodium falciparum exposure for
individuals and communities. \emph{Proceedings of the National Academy
of Sciences}. 2015;112(32):E4438-E4447.
doi:\href{https://doi.org/10.1073/pnas.1501705112}{10.1073/pnas.1501705112}.

\hypertarget{ref-griffing2013history}{}
11. Griffing SM, Gamboa D, Udhayakumar V. The history of 20 th century
malaria control in Peru. \emph{Malaria journal}. 2013;12(1):303.
doi:\href{https://doi.org/10.1186/1475-2875-12-303}{10.1186/1475-2875-12-303}.

\hypertarget{ref-hotspots2015}{}
12. Rosas-Aguirre A, Speybroeck N, Llanos-Cuentas A, et al. Hotspots of
malaria transmission in the Peruvian Amazon: Rapid assessment through a
parasitological and serological survey. \emph{PLOS ONE}.
2015;10(9):1-21.
doi:\href{https://doi.org/10.1371/journal.pone.0137458}{10.1371/journal.pone.0137458}.

\hypertarget{ref-elliott2014}{}
13. Elliott SR, Fowkes F, Richards JS, Reiling L, Drew DR, Beeson JG.
Research priorities for the development and implementation of
serological tools for malaria surveillance. \emph{F1000Prime Rep}.
2014;6:100. doi:\href{https://doi.org/10.12703/P6-100}{10.12703/P6-100}.

\hypertarget{ref-arevalo2014}{}
14. Arévalo-Herrera M, Forero-Peña DA, Rubiano K, et al. Plasmodium
vivax sporozoite challenge in malaria-naive and semi-immune colombian
volunteers. \emph{PLoS One}. 2014;9(6):e99754.
doi:\href{https://doi.org/10.1371/journal.pone.0099754}{10.1371/journal.pone.0099754}.

\hypertarget{ref-vigil2010}{}
15. Vigil A, Davies DH, Felgner PL. Defining the humoral immune response
to infectious agents using high-density protein microarrays.
\emph{Future microbiology}. 2010;5(2):241-251.
doi:\href{https://doi.org/10.2217/fmb.09.127}{10.2217/fmb.09.127}.

\hypertarget{ref-leroch2009postmod}{}
16. Chung D-WD, Ponts N, Cervantes S, Le Roch KG. Post-translational
modifications in plasmodium: More than you think! \emph{Molecular and
biochemical parasitology}. 2009;168(2):123-134.
doi:\href{https://doi.org/10.1016/j.molbiopara.2009.08.001}{10.1016/j.molbiopara.2009.08.001}.

\hypertarget{ref-R2016}{}
17. R Core Team. \emph{R: A Language and Environment for Statistical
Computing}. Vienna, Austria: R Foundation for Statistical Computing;
2016. \url{https://www.R-project.org/}.

\hypertarget{ref-bioconductor2004}{}
18. Gentleman RC, Carey VJ, Bates DM, et al. Bioconductor: Open software
development for computational biology and bioinformatics. \emph{Genome
biology}. 2004;5(10):R80.
doi:\href{https://doi.org/10.1186/gb-2004-5-10-r80}{10.1186/gb-2004-5-10-r80}.

\hypertarget{ref-WHO2014severe}{}
19. WHO. Severe malaria. \emph{Trop Med Int Health}. 2014;19:7-131.
doi:\href{https://doi.org/10.1111/tmi.12313_2}{10.1111/tmi.12313\_2}.

\hypertarget{ref-Torres2014asymptomatic}{}
20. Torres KJ, Castrillon CE, Moss EL, et al. Genome-level determination
of plasmodium falciparum blood-stage targets of malarial clinical
immunity in the Peruvian Amazon. \emph{Journal of Infectious Diseases}.
November 2014.
doi:\href{https://doi.org/10.1093/infdis/jiu614}{10.1093/infdis/jiu614}.

\hypertarget{ref-chuquiyauri2015vivax}{}
21. Chuquiyauri R, Molina DM, Moss EL, et al. Genome-scale protein
microarray comparison of human antibody responses in plasmodium vivax
relapse and reinfection. \emph{The American journal of tropical medicine
and hygiene}. 2015;93(4):801-809.
doi:\href{https://doi.org/10.4269/ajtmh.15-0232}{10.4269/ajtmh.15-0232}.

\hypertarget{ref-howes2016global}{}
22. Howes RE, Battle KE, Mendis KN, et al. Global epidemiology of
plasmodium vivax. \emph{The American Journal of Tropical Medicine and
Hygiene}. 2016;95(6 Suppl):15-34.
doi:\href{https://doi.org/10.4269/ajtmh.16-0141}{10.4269/ajtmh.16-0141}.

\hypertarget{ref-path2011}{}
23. PATH. \emph{Staying the Course? Malaria Research and Development in
a Time of Economic Uncertainty}. Seattle, WA: PATH; 2011.
\url{www.malariavaccine.org/files/RD-report-June2011.pdf}.

\hypertarget{ref-gagnon2002enso}{}
24. Gagnon AS, Smoyer-Tomic KE, Bush AB. The El Niño southern
oscillation and malaria epidemics in South America. \emph{International
Journal of Biometeorology}. 2002;46(2):81-89.
doi:\href{https://doi.org/10.1007/s00484-001-0119-6}{10.1007/s00484-001-0119-6}.

\hypertarget{ref-Stanisic2015}{}
25. Stanisic DI, Fowkes FJI, Koinari M, et al. Acquisition of antibodies
against plasmodium falciparum merozoites and malaria immunity in young
children and the influence of age, force of infection, and magnitude of
response. \emph{Infection and Immunity}. 2015;83(2):646-660.
doi:\href{https://doi.org/10.1128/IAI.02398-14}{10.1128/IAI.02398-14}.

\hypertarget{ref-rogerson2007preg}{}
26. Rogerson SJ, Hviid L, Duffy PE, Leke RF, Taylor DW. Malaria in
pregnancy: Pathogenesis and immunity. \emph{The Lancet infectious
diseases}. 2007;7(2):105-117.
doi:\href{https://doi.org/10.1016/S1473-3099(07)70022-1}{10.1016/S1473-3099(07)70022-1}.

\hypertarget{ref-llanoschea2015}{}
27. Llanos-Chea F, Martínez D, Rosas A, Samalvides F, Vinetz JM,
Llanos-Cuentas A. Characteristics of travel-related severe plasmodium
vivax and plasmodium falciparum malaria in individuals hospitalized at a
tertiary referral center in lima, Peru. \emph{The American Journal of
Tropical Medicine and Hygiene}. 2015;93(6):1249-1253.
doi:\href{https://doi.org/10.4269/ajtmh.14-0652}{10.4269/ajtmh.14-0652}.

\hypertarget{ref-factores2001}{}
28. MINSA. \emph{Factores de Riesgo de La Malaria Grave En El Perú}.
Ministerio de Salud: Proyecto Vigía (MINSA-USAID); 2001.
\url{http://bvs.minsa.gob.pe/local/minsa/1772.pdf}.

\hypertarget{ref-baldevi2016}{}
29. Halsey ES, Baldeviano GC, Edgel KA, Vilcarromero S, Sihuincha M,
Lescano AG. Symptoms and immune markers in plasmodium/dengue virus
co-infection compared with mono-infection with either in Peru.
\emph{PLOS Neglected Tropical Diseases}. 2016;10(4):1-16.
doi:\href{https://doi.org/10.1371/journal.pntd.0004646}{10.1371/journal.pntd.0004646}.

\hypertarget{ref-wassmer2015}{}
30. Wassmer SC, Taylor TE, Rathod PK, et al. Investigating the
pathogenesis of severe malaria: A multidisciplinary and
cross-geographical approach. \emph{The American journal of tropical
medicine and hygiene}. 2015;93(3 Suppl):42-56.
doi:\href{https://doi.org/10.4269/ajtmh.14-0841}{10.4269/ajtmh.14-0841}.

\hypertarget{ref-rts2015}{}
31. RTS,S Clinical Trials Partnership. Efficacy and safety of RTS,S/AS01
malaria vaccine with or without a booster dose in infants and children
in africa: Final results of a phase 3, individually randomised,
controlled trial. \emph{The Lancet}. 2015;386(9988):31-45.
doi:\href{https://doi.org/10.1016/S0140-6736(15)60721-8}{10.1016/S0140-6736(15)60721-8}.

\hypertarget{ref-rahimi2014meta}{}
32. Rahimi BA, Thakkinstian A, White NJ, Sirivichayakul C, Dondorp AM,
Chokejindachai W. Severe vivax malaria: A systematic review and
meta-analysis of clinical studies since 1900. \emph{Malaria journal}.
2014;13(1):481.
doi:\href{https://doi.org/10.1186/1475-2875-13-481}{10.1186/1475-2875-13-481}.

\hypertarget{ref-alexandre2010}{}
33. Alexandre MA, Ferreira CO, Siqueira AM, et al. Severe plasmodium
vivax malaria, Brazilian Amazon. \emph{Emerging infectious diseases}.
2010;16(10):1611.
doi:\href{https://doi.org/10.3201/eid1610.100685}{10.3201/eid1610.100685}.

\hypertarget{ref-smith2013}{}
34. Smith-Nuñez ES, Durand S, Baldeviano GC, et al. WHO criteria for
severe malaria in identifying severe vivax malaria: Preliminary data
from a study in Iquitos, Peru. In: \emph{Abstract Book of the Astmh 62nd
Annual Meeting, Nov. 13--17, Washington D.C., United States}.; 2013:398.
\url{http://www.astmh.org/ASTMH/media/Documents/AbstractBook2013Final.pdf}.

\hypertarget{ref-barber2015}{}
35. Barber BE, William T, Grigg MJ, et al. Parasite biomass-related
inflammation, endothelial activation, microvascular dysfunction and
disease severity in vivax malaria. \emph{PLoS Pathog}. 2015;11(1):1-13.
doi:\href{https://doi.org/10.1371/journal.ppat.1004558}{10.1371/journal.ppat.1004558}.

\hypertarget{ref-sepulveda2015}{}
36. Sepúlveda N, Stresman G, White MT, Drakeley CJ. Current mathematical
models for analyzing anti-malarial antibody data with an eye to malaria
elimination and eradication. \emph{Journal of Immunology Research}.
2015;2015:1-21.
doi:\href{https://doi.org/10.1155/2015/738030}{10.1155/2015/738030}.

\hypertarget{ref-rainbow2016}{}
37. WHO. Malaria vaccine rainbow tables. In: World Health Organization.
Accessed: 15-june-2017.
\url{http://www.who.int/vaccine_research/links/Rainbow/en/index.html}.

\hypertarget{ref-cutts2014meta}{}
38. Cutts JC, Powell R, Agius PA, Beeson JG, Simpson JA, Fowkes FJ.
Immunological markers of plasmodium vivax exposure and immunity: A
systematic review and meta-analysis. \emph{BMC medicine}.
2014;12(1):150.
doi:\href{https://doi.org/10.1186/s12916-014-0150-1}{10.1186/s12916-014-0150-1}.

\hypertarget{ref-crompton2014rev}{}
39. Crompton PD, Moebius J, Portugal S, et al. Malaria immunity in man
and mosquito: Insights into unsolved mysteries of a deadly infectious
disease. \emph{Annual review of immunology}. 2014;32:157-187.
doi:\href{https://doi.org/10.1146/annurev-iy-32-060414-200001}{10.1146/annurev-iy-32-060414-200001}.

\hypertarget{ref-baird2013}{}
40. Baird JK. Evidence and implications of mortality associated with
acute plasmodium vivax malaria. \emph{Clinical Microbiology Reviews}.
2013;26(1):36-57.
doi:\href{https://doi.org/10.1128/CMR.00074-12}{10.1128/CMR.00074-12}.

\hypertarget{ref-jagannathan2014}{}
41. Jagannathan P, Eccles-James I, Bowen K, et al. IFN\(\gamma\)/il-10
co-producing cells dominate the cd4 response to malaria in highly
exposed children. \emph{PLoS Pathog}. 2014;10(1):e1003864.
doi:\href{https://doi.org/10.1371/journal.ppat.1003864}{10.1371/journal.ppat.1003864}.

\hypertarget{ref-schofield2006toll}{}
42. Schofield L, Mueller I. Clinical immunity to malaria. \emph{Current
molecular medicine}. 2006;6(2):205-221.
doi:\href{https://doi.org/10.2174/156652406776055221}{10.2174/156652406776055221}.

\hypertarget{ref-coban2005toll}{}
43. Coban C, Ishii KJ, Kawai T, et al. Toll-like receptor 9 mediates
innate immune activation by the malaria pigment hemozoin. \emph{Journal
of Experimental Medicine}. 2005;201(1):19-25.
doi:\href{https://doi.org/10.1084/jem.20041836}{10.1084/jem.20041836}.

\hypertarget{ref-portillo2001vir}{}
44. Portillo HA del, Fernandez-Becerra C, Bowman S, et al. A superfamily
of variant genes encoded in the subtelomeric region of plasmodium vivax.
\emph{Nature}. 2001;410(6830):839-842.
doi:\href{https://doi.org/10.1038/35071118}{10.1038/35071118}.

\hypertarget{ref-galinski1992rbp}{}
45. Galinski MR, Medina CC, Ingravallo P, Barnwell JW. A
reticulocyte-binding protein complex of plasmodium vivax merozoites.
\emph{Cell}. 1992;69(7):1213-1226.
doi:\href{https://doi.org/10.1016/0092-8674(92)90642-P}{10.1016/0092-8674(92)90642-P}.

\hypertarget{ref-cohen1961}{}
46. Cohen S, McGregor I, Carrington S. Gamma-globulin and acquired
immunity to human malaria. \emph{Nature}. 1961;192(4804):733-737.
doi:\href{https://doi.org/10.1038/192733a0}{10.1038/192733a0}.

\hypertarget{ref-mueller2013}{}
47. Mueller I, Galinski MR, Tsuboi T, Arévalo-Herrera M, Collins WE,
King CL. Natural acquisition of immunity to plasmodium vivax:
Epidemiological observations and potential targets. \emph{Adv
Parasitol}. 2013;81:77-131.
doi:\href{https://doi.org/10.1016/B978-0-12-407826-0.00003-5}{10.1016/B978-0-12-407826-0.00003-5}.

\hypertarget{ref-lopez2017}{}
48. López C, Yepes-Pérez Y, Hincapié-Escobar N, Díaz-Arévalo D,
Patarroyo MA. What is known about the immune response induced by
plasmodium vivax malaria vaccine candidates? \emph{Frontiers in
immunology}. 2017;8.
doi:\href{https://doi.org/10.3389/fimmu.2017.00126}{10.3389/fimmu.2017.00126}.

\hypertarget{ref-arevalo2016spz}{}
49. Arévalo-Herrera M, Vásquez-Jiménez JM, Lopez-Perez M, et al.
Protective efficacy of plasmodium vivax radiation-attenuated sporozoites
in colombian volunteers: A randomized controlled trial. \emph{PLoS Negl
Trop Dis}. 2016;10(10).
doi:\href{https://doi.org/10.1371/journal.pntd.0005070}{10.1371/journal.pntd.0005070}.

\hypertarget{ref-immunomics2016}{}
50. De Sousa KP, Doolan DL. Immunomics: A 21st century approach to
vaccine development for complex pathogens. \emph{Parasitology}.
2016;143(02):236-244.
doi:\href{https://doi.org/10.1017/S0031182015001079}{10.1017/S0031182015001079}.

\hypertarget{ref-Davies2015Large}{}
51. Davies DH, Duffy P, Bodmer J-L, Felgner PL, Doolan DL. Large screen
approaches to identify novel malaria vaccine candidates. \emph{Vaccine}.
2015;33(52):7496-7505.
doi:\href{https://doi.org/10.1016/j.vaccine.2015.09.059}{10.1016/j.vaccine.2015.09.059}.

\hypertarget{ref-carmona2015peptide}{}
52. Carmona SJ, Nielsen M, Schafer-Nielsen C, et al. Towards
high-throughput immunomics for infectious diseases: Use of
next-generation peptide microarrays for rapid discovery and mapping of
antigenic determinants. \emph{Molecular \& Cellular Proteomics}.
2015;14(7):1871-1884.
doi:\href{https://doi.org/10.1074/mcp.M114.045906}{10.1074/mcp.M114.045906}.

\hypertarget{ref-uzoma2013interactome}{}
53. Uzoma I, Zhu H. Interactome mapping: Using protein microarray
technology to reconstruct diverse protein networks. \emph{Genomics,
proteomics \& bioinformatics}. 2013;11(1):18-28.
doi:\href{https://doi.org/10.1016/j.gpb.2012.12.005}{10.1016/j.gpb.2012.12.005}.

\hypertarget{ref-Finney2014}{}
54. Finney OC, Danziger SA, Molina DM, et al. Predicting antidisease
immunity using proteome arrays and sera from children naturally exposed
to malaria. \emph{Molecular \& Cellular Proteomics}.
2014;13(10):2646-2660.
doi:\href{https://doi.org/10.1074/mcp.M113.036632}{10.1074/mcp.M113.036632}.

\hypertarget{ref-King2015FOC}{}
55. King CL, Davies DH, Felgner P, et al. Biosignatures of
exposure/transmission and immunity. \emph{American Journal of Tropical
Medicine and Hygiene}. 2015;93(3 Suppl):16-27.
doi:\href{https://doi.org/10.4269/ajtmh.15-0037}{10.4269/ajtmh.15-0037}.

\hypertarget{ref-sundaresh2006}{}
56. Sundaresh S, Doolan DL, Hirst S, et al. Identification of humoral
immune responses in protein microarrays using dna microarray data
analysis techniques. \emph{Bioinformatics}. 2006;22(14):1760-1766.
doi:\href{https://doi.org/10.1093/bioinformatics/btl162}{10.1093/bioinformatics/btl162}.

\hypertarget{ref-allison2006}{}
57. Allison DB, Cui X, Page GP, Sabripour M. Microarray data analysis:
From disarray to consolidation and consensus. \emph{Nature reviews
genetics}. 2006;7(1):55-65.
doi:\href{https://doi.org/10.1038/nrg1749}{10.1038/nrg1749}.

\hypertarget{ref-brazma2001}{}
58. Brazma A, Hingamp P, Quackenbush J, et al. Minimum information about
a microarray experiment (MIAME)---toward standards for microarray data.
\emph{Nature genetics}. 2001;29(4):365-371.
doi:\href{https://doi.org/10.1038/ng1201-365}{10.1038/ng1201-365}.

\hypertarget{ref-kreil2005bullet}{}
59. Kreil DP, Russell RR. Tutorial section: There is no silver
bullet---a guide to low-level data transforms and normalisation methods
for microarray data. \emph{Briefings in bioinformatics}.
2005;6(1):86-97.
doi:\href{https://doi.org/10.1093/bib/6.1.86}{10.1093/bib/6.1.86}.

\hypertarget{ref-brown2001image}{}
60. Brown CS, Goodwin PC, Sorger PK. Image metrics in the statistical
analysis of dna microarray data. \emph{Proceedings of the National
Academy of Sciences}. 2001;98(16):8944-8949.
doi:\href{https://doi.org/10.1073/pnas.161242998}{10.1073/pnas.161242998}.

\hypertarget{ref-huber2002vsn}{}
61. Huber W, Von Heydebreck A, Sültmann H, Poustka A, Vingron M.
Variance stabilization applied to microarray data calibration and to the
quantification of differential expression. \emph{Bioinformatics}.
2002;18(suppl 1):S96-S104.
doi:\href{https://doi.org/10.1093/bioinformatics/18.suppl_1.S96}{10.1093/bioinformatics/18.suppl\_1.S96}.

\hypertarget{ref-sboner2009rlm}{}
62. Sboner A, Karpikov A, Chen G, et al. Robust-linear-model
normalization to reduce technical variability in functional protein
microarrays. \emph{Journal of proteome research}. 2009;8(12):5451-5464.
doi:\href{https://doi.org/10.1021/pr900412k}{10.1021/pr900412k}.

\hypertarget{ref-bourgon2010filter}{}
63. Bourgon R, Gentleman R, Huber W. Independent filtering increases
detection power for high-throughput experiments. \emph{Proceedings of
the National Academy of Sciences}. 2010;107(21):9546-9551.
doi:\href{https://doi.org/10.1073/pnas.0914005107}{10.1073/pnas.0914005107}.

\hypertarget{ref-smyth2004ebayes}{}
64. Smyth GK, others. Linear models and empirical bayes methods for
assessing differential expression in microarray experiments.
\emph{Statistical Applications in Genetics and Molecular Biology}.
2004;3(1):3.
doi:\href{https://doi.org/10.2202/1544-6115.1027}{10.2202/1544-6115.1027}.

\hypertarget{ref-baldi2001cybert}{}
65. Baldi P, Long AD. A bayesian framework for the analysis of
microarray expression data: Regularized t-test and statistical
inferences of gene changes. \emph{Bioinformatics}. 2001;17(6):509-519.
doi:\href{https://doi.org/10.1093/bioinformatics/17.6.509}{10.1093/bioinformatics/17.6.509}.

\hypertarget{ref-kayala2012cyber}{}
66. Kayala MA, Baldi P. Cyber-t web server: Differential analysis of
high-throughput data. \emph{Nucleic acids research}.
2012;40(W1):W553-W559.
doi:\href{https://doi.org/10.1093/nar/gks420}{10.1093/nar/gks420}.

\hypertarget{ref-allison2002mmm}{}
67. Allison DB, Gadbury GL, Heo M, et al. A mixture model approach for
the analysis of microarray gene expression data. \emph{Computational
Statistics \& Data Analysis}. 2002;39(1):1-20.
doi:\href{https://doi.org/10.1016/S0167-9473(01)00046-9}{10.1016/S0167-9473(01)00046-9}.

\hypertarget{ref-benjamini1995fdr}{}
68. Benjamini Y, Hochberg Y. Controlling the false discovery rate: A
practical and powerful approach to multiple testing. \emph{Journal of
the royal statistical society Series B (Methodological)}. 1995:289-300.
doi:\href{https://doi.org/10.2307/2346101}{10.2307/2346101}.

\hypertarget{ref-sundaresh2007}{}
69. Sundaresh S, Randall A, Unal B, et al. From protein microarrays to
diagnostic antigen discovery: A study of the pathogen francisella
tularensis. \emph{Bioinformatics}. 2007;23(13):i508-i518.
doi:\href{https://doi.org/10.1093/bioinformatics/btm207}{10.1093/bioinformatics/btm207}.

\hypertarget{ref-molina2012}{}
70. Molina DM, Finney OC, Arevalo-Herrera M, et al. Plasmodium vivax
pre-erythrocytic--Stage antigen discovery: Exploiting naturally acquired
humoral responses. \emph{The American Journal of Tropical Medicine and
Hygiene}. 2012;87(3):460-469.
doi:\href{https://doi.org/10.4269/ajtmh.2012.12-0222}{10.4269/ajtmh.2012.12-0222}.

\hypertarget{ref-baum2013}{}
71. Baum E, Badu K, Molina DM, Liang X, Felgner PL, Yan G. Protein
microarray analysis of antibody responses to plasmodium falciparum in
western kenyan highland sites with differing transmission levels.
\emph{PLOS one}. 2013;8(12):e82246.
doi:\href{https://doi.org/10.1371/journal.pone.0082246}{10.1371/journal.pone.0082246}.

\hypertarget{ref-baum2015}{}
72. Baum E, Sattabongkot J, Sirichaisinthop J, et al. Submicroscopic and
asymptomatic plasmodium falciparum and plasmodium vivax infections are
common in western thailand-molecular and serological evidence.
\emph{Malaria journal}. 2015;14(1):95.
doi:\href{https://doi.org/10.1186/s12936-015-0611-9}{10.1186/s12936-015-0611-9}.

\hypertarget{ref-lu2014rama}{}
73. Lu F, Li J, Wang B, et al. Profiling the humoral immune responses to
plasmodium vivax infection and identification of candidate immunogenic
rhoptry-associated membrane antigen (RAMA). \emph{Journal of
proteomics}. 2014;102:66-82.
doi:\href{https://doi.org/10.1016/j.jprot.2014.02.029}{10.1016/j.jprot.2014.02.029}.

\hypertarget{ref-chen2015immunomics}{}
74. Chen J-H, Chen S-B, Wang Y, et al. An immunomics approach for the
analysis of natural antibody responses to plasmodium vivax infection.
\emph{Molecular BioSystems}. 2015;11(8):2354-2363.
doi:\href{https://doi.org/10.1039/C5MB00330J}{10.1039/C5MB00330J}.

\hypertarget{ref-Biobase}{}
75. Huber, W., Carey, et al. Orchestrating high-throughput genomic
analysis with Bioconductor. \emph{Nature Methods}. 2015;12(2):115-121.
doi:\href{https://doi.org/10.1038/nmeth.3252}{10.1038/nmeth.3252}.

\hypertarget{ref-knitr}{}
76. Xie Y. Knitr: A general-purpose tool for dynamic report generation
in r. \emph{R package version}. 2013;1(1).
\url{http://yihui.name/knitr/}.

\hypertarget{ref-CienciaReproducible2016}{}
77. Rodríguez-Sanchez F, Pérez-Luque AJ, Bartomeus I, Varela S. Ciencia
reproducible: qué, por qué, cómo? \emph{ECOS}. 2016;25(2):83-92.
doi:\href{https://doi.org/10.7818/ecos.2016.25-2.11}{10.7818/ecos.2016.25-2.11}.

\hypertarget{ref-gmine2016}{}
78. Proietti C, Zakrzewski M, Watkins TS, et al. Mining, visualizing and
comparing multidimensional biomolecular data using the Genomics Data
Miner (GMine) Web-Server. \emph{Scientific Reports}. 2016;6.
doi:\href{https://doi.org/10.1038/srep38178}{10.1038/srep38178}.

\hypertarget{ref-abbas2012}{}
79. Abbas A, Lichtman A, Pillai S. \emph{Inmunología Celular Y
Molecular}. 7th ed. Elsevier España(ES); 2012.

\hypertarget{ref-sette2005}{}
80. Sette A, Fleri W, Peters B, Sathiamurthy M, Bui H-H, Wilson S. A
roadmap for the immunomics of category A--C pathogens. \emph{Immunity}.
2005;22(2):155-161.
doi:\href{https://doi.org/10.1016/j.immuni.2005.01.009}{10.1016/j.immuni.2005.01.009}.

\hypertarget{ref-Driguez2015}{}
81. Driguez P, Doolan DL, Molina DM, et al. Protein microarrays for
parasite antigen discovery. \emph{Parasite genomics protocols}.
2015:221-233.
doi:\href{https://doi.org/10.1007/978-1-4939-1438-8_13}{10.1007/978-1-4939-1438-8\_13}.

\hypertarget{ref-wickham2016r4ds}{}
82. Wickham H, Grolemund G. \emph{R for Data Science}. Sebastopol, CA:
O'Reilly.; 2016. \url{http://r4ds.had.co.nz/}.

\hypertarget{ref-limma}{}
83. Ritchie ME, Phipson B, Wu D, et al. limma powers differential
expression analyses for RNA-sequencing and microarray studies.
\emph{Nucleic Acids Research}. 2015;43(7):e47.
doi:\href{https://doi.org/10.1093/nar/gkv007}{10.1093/nar/gkv007}.

\hypertarget{ref-Gaujoux2010NMF}{}
84. Gaujoux R, Seoighe C. A flexible R package for nonnegative matrix
factorization. \emph{BMC Bioinformatics}. 2010;11(1):367.
doi:\href{https://doi.org/10.1186/1471-2105-11-367}{10.1186/1471-2105-11-367}.

\hypertarget{ref-plasmodb}{}
85. Aurrecoechea C, Brestelli J, Brunk BP, et al. PlasmoDB: A functional
genomic database for malaria parasites. \emph{Nucleic acids research}.
2009;37(suppl 1):D539-D543.
doi:\href{https://doi.org/10.1093/nar/gkn814}{10.1093/nar/gkn814}.

\section{ANEXOS}\label{anexos}

\subsection{Matriz de consistencia}\label{matriz-de-consistencia}

Ver \autoref{tab:consis} al final del documento.


\afterpage{
    \clearpage
    \newgeometry{left=2.5cm,right=0.5cm,top=0.5cm,bottom=0.5cm}
    \thispagestyle{empty}
    \begin{landscape}
        \centering
\begin{center}

\begin{tabular}{|m{3.2cm}m{3.2cm}m{3.2cm}m{3.2cm}m{3.2cm}m{3.2cm}m{3.2cm}|}
  \hline
  \textbf{Título:}
  &
  \multicolumn{6}{l|}{ %>{\centering}m{19cm}
  \begin{minipage}{19.2cm}
  Comparación de la respuesta de anticuerpos ante la %infección con 
  malaria vivax
  en pacientes de la Amazonía peruana %ciudad de Iquitos (Loreto - Perú)
  según su severidad y episodios previos %exposición previa
  mediante microarreglos de proteínas  
  \end{minipage}  
  }\\
  \cline{1-7}
  \textbf{Problema} & \textbf{Objetivos} & \textbf{Hipótesis} & \textbf{Variables} & 
  \textbf{Diseño} & \textbf{Muestra} & %\textbf{Instrumentos} & 
  \textbf{Análisis}\\
  \hline
  \begin{minipage}{3.2cm} 
  \textbf{Principal}\\
  1. ¿Cuáles son los antígenos de \textit{P. vivax} con reactividad serológica diferenciada
  ante la infección con malaria vivax entre pacientes severos y no-severos?\\
  \newline
  \textbf{Secundario}\\
  2. ¿Cuáles son los antígenos de \textit{P. vivax}  con reactividad serológica diferenciada
  ante la infección con malaria vivax entre pacientes con y sin episodios previos?\\
  \newline
  3. ¿Cuáles son las características proteicas de los antígenos de \textit{P. vivax} 
  con reactividad diferenciada o predominante
  en pacientes con malaria vivax?
  \end{minipage} 
  & 
  \begin{minipage}{3.2cm} 
  \textbf{Principal}\\
  1. Identificar antígenos de \textit{P. vivax} con reactividad serológica 
  diferenciada ante la infección con malaria vivax entre pacientes 
  severos y no-severos.\\
  \newline
  \textbf{Secundario}\\
  2. Identificar antígenos de \textit{P. vivax} con reactividad serológica 
  diferenciada ante la infección con malaria vivax entre pacientes 
  con y sin episodios previos.\\
  \newline
  3. Describir las características proteicas de los antígenos con reactividad 
  diferenciada o predominante.
  \end{minipage} 
  & 
  \begin{minipage}{3.2cm} 
  .\\
  \textbf{De diferencia}\\ \textbf{entre grupos:}\\
  1. Los pacientes con malaria vivax no-severa poseen 
  mayor reactividad serológica contra antígenos de \textit{P. vivax}
  asociados a exposición, invasión o adhesión celular
  con respecto a los pacientes severos.\\
  \newline
  2. Los pacientes con episodios previos de malaria poseen
  mayor reactividad serológica contra antígenos de \textit{P. vivax}
  asociados a exposición
  con respecto a los pacientes sin episodios previos.\\
  \newline
  \textbf{Descriptiva:}\\
  3. Los antígenos con reactividad diferenciada o predominante
  se caracterizan por poseer una localización extracelular 
  y estar bajo presión selectiva por el sistema inmune.\\
  \end{minipage} 
  &
  \begin{minipage}{3.2cm} 
  \textbf{Dependiente}\\ Reactividad\\ serológica\\
  \newline 
  \textbf{Independiente}\\ Severidad\\
  \newline
  \textbf{Independiente}\\ Episodios previos\\
  \newline
  \underline{Instrumentos}:\\
  %\textbf{Reactividad serológica:}\\
  -Microarreglo de proteínas PfPv500.\\%Pf498/Pv516\\
  %\newline
  %\textbf{Severidad:}\\
  -Diagnóstico clínico y pruebas bioquímicas.\\%Criterios de la OMS para malaria severa % (criterio OMS)
  %\newline
  %\textbf{Episodios}\\ \textbf{previos:}\\
  -Encuesta.\\
  \newline
  \underline{Operacionalización}:\\
  -Ir a \autoref{tab:opera}
  \end{minipage} 
  &
  \begin{minipage}{3.2cm} 
  \textbf{Tipo:}\\
  Caso-Control.\\
  \newline
  \textbf{Clasificación:}\\
  -Por la finalidad: Analítico.\\
  %\newline
  -Por el control de la asignación:\\ Observacional.\\
  %\newline
  -Por el seguimiento: Transversal.\\
  %\newline
  -Por la relación cronológica:\\ Prospectivo.\\
  %\newline
  -Por la unidad de análisis:\\ Basado en individuo.
  \end{minipage}   
  &
  \begin{minipage}{3.2cm} 
  \textbf{Universo teórico:}\\ 
  Pacientes con malaria vivax 
  de la cuenca amazónica del Perú.\\
  \newline
  \textbf{Marco}\\ \textbf{Muestral:}\\
  Pacientes diagnosticados con malaria vivax de la ciudad de Iquitos, Loreto-Perú, 
  entre enero del 2012 y junio del 2013.\\
  \newline
  \textbf{Muestra:}\\
  Selección aleatoria simple
  de 30 pacientes con uno o más parámetros del criterio OMS para malaria severa (casos) y 
  30 no-severos (control).\\
  \newline
  \textbf{Tipo:}\\ Probabilística.
  \end{minipage}   
  &
%  \begin{minipage}{3.2cm} 
%  \textbf{Reactividad serológica:}\\
%  Microarreglos de\\proteínas Pf498/Pv516\\
%  \newline
%  \textbf{Severidad:}\\
%  Diagnóstico \\clínico \\y exámenes de \\laboratorio\\ según criterios de la OMS.\\
%  \newline
%  \textbf{Episodios}\\ \textbf{previos:}\\
%  Encuesta
%  \end{minipage}   
%  &
  \begin{minipage}{3.2cm} 
  \underline{Control de Calidad}:\\
  \newline
  \textbf{1. Validez y}\\ \textbf{Reproducibilidad:}\\
  correlación de Pearson o Spearman\\
  \newline
  \underline{Prueba de Hipótesis}:\\
  \newline
  \textbf{2. Amplitud}\\ \textbf{e intensidad de}\\ \textbf{respuesta:}\\
  prueba t-Student o Mann-Whitney\\
  \newline
  \textbf{3.}\\ \textbf{Reactividad}\\ \textbf{diferenciada:}\\%\textbf{3. Inferencia}\\
  test-t moderado con\\
  corrección del FDR\\por el método\\Benjamini-Hochberg\\
  \newline
  \textbf{4. Clasificación:}\\
  agrupamiento jerárquico o\\ \textit{hierarchical}\\ \textit{clustering}\\
  en base a la\\ distancia euclidiana.\\
  \newline
  \textbf{5. Descripción:}\\
  características disponibles en\\ PlasmoDB.
  \end{minipage}   
  \\
  %\cline{1-5}
  %y & z & m & n & y & z & m & n\\
  \hline
  % etc. ...
\end{tabular}

\end{center}
        \captionof{table}{Matriz de consistencia}
        \label{tab:consis}
    \end{landscape}
    \restoregeometry
    \clearpage
}


\end{document}
