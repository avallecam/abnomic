\documentclass[]{article}
\usepackage{lmodern}
\usepackage{amssymb,amsmath}
\usepackage{ifxetex,ifluatex}
\usepackage{fixltx2e} % provides \textsubscript
\ifnum 0\ifxetex 1\fi\ifluatex 1\fi=0 % if pdftex
  \usepackage[T1]{fontenc}
  \usepackage[utf8]{inputenc}
\else % if luatex or xelatex
  \ifxetex
    \usepackage{mathspec}
  \else
    \usepackage{fontspec}
  \fi
  \defaultfontfeatures{Ligatures=TeX,Scale=MatchLowercase}
\fi
% use upquote if available, for straight quotes in verbatim environments
\IfFileExists{upquote.sty}{\usepackage{upquote}}{}
% use microtype if available
\IfFileExists{microtype.sty}{%
\usepackage{microtype}
\UseMicrotypeSet[protrusion]{basicmath} % disable protrusion for tt fonts
}{}
\usepackage[margin=1in]{geometry}
\usepackage{hyperref}
\hypersetup{unicode=true,
            pdfborder={0 0 0},
            breaklinks=true}
\urlstyle{same}  % don't use monospace font for urls
\usepackage{longtable,booktabs}
\usepackage{graphicx,grffile}
\makeatletter
\def\maxwidth{\ifdim\Gin@nat@width>\linewidth\linewidth\else\Gin@nat@width\fi}
\def\maxheight{\ifdim\Gin@nat@height>\textheight\textheight\else\Gin@nat@height\fi}
\makeatother
% Scale images if necessary, so that they will not overflow the page
% margins by default, and it is still possible to overwrite the defaults
% using explicit options in \includegraphics[width, height, ...]{}
\setkeys{Gin}{width=\maxwidth,height=\maxheight,keepaspectratio}
\IfFileExists{parskip.sty}{%
\usepackage{parskip}
}{% else
\setlength{\parindent}{0pt}
\setlength{\parskip}{6pt plus 2pt minus 1pt}
}
\setlength{\emergencystretch}{3em}  % prevent overfull lines
\providecommand{\tightlist}{%
  \setlength{\itemsep}{0pt}\setlength{\parskip}{0pt}}
\setcounter{secnumdepth}{5}
% Redefines (sub)paragraphs to behave more like sections
\ifx\paragraph\undefined\else
\let\oldparagraph\paragraph
\renewcommand{\paragraph}[1]{\oldparagraph{#1}\mbox{}}
\fi
\ifx\subparagraph\undefined\else
\let\oldsubparagraph\subparagraph
\renewcommand{\subparagraph}[1]{\oldsubparagraph{#1}\mbox{}}
\fi

%%% Use protect on footnotes to avoid problems with footnotes in titles
\let\rmarkdownfootnote\footnote%
\def\footnote{\protect\rmarkdownfootnote}

%%% Change title format to be more compact
\usepackage{titling}

% Create subtitle command for use in maketitle
\newcommand{\subtitle}[1]{
  \posttitle{
    \begin{center}\large#1\end{center}
    }
}

\setlength{\droptitle}{-2em}
  \title{}
  \pretitle{\vspace{\droptitle}}
  \posttitle{}
  \author{}
  \preauthor{}\postauthor{}
  \date{}
  \predate{}\postdate{}

\usepackage{multirow}
\usepackage{pdflscape}
\usepackage{afterpage}
\usepackage{capt-of}
\usepackage{array}

\begin{document}

\renewcommand{\contentsname}{Índice General} 
\renewcommand{\tablename}{Tabla}
\renewcommand{\tableautorefname}{Tabla}

\pagenumbering{gobble}

\clearpage
\newgeometry{top=1cm,bottom=1cm} \vspace*{\fill}

\begin{centering}

\begin{figure}[!ht]
  \begin{center}
    \includegraphics[width=.8in]{figure/UNMSM_escudo-2000px.png}
  \end{center}
\end{figure}

\Large
UNIVERSIDAD NACIONAL MAYOR DE SAN MARCOS

\large
(Universidad del Perú, DECANA DE AMÉRICA)

\vspace{.5 cm}

\Large
FACULTAD DE CIENCIAS BIOLÓGICAS

\vspace{.5 cm}

\normalsize
ESCUELA ACADÉMICO PROFESIONAL DE

GENÉTICA Y BIOTECNOLOGÍA

\vspace{1.5 cm}

\Large
Comparación de la respuesta de anticuerpos ante la %infección con 
malaria vivax 
en pacientes de la ciudad de Iquitos (Loreto - Perú)
según su severidad y exposición previa
mediante microarreglos de proteínas
% aPerfil de anticuerpos en respuesta a la infección con malaria vivax
%mediante un enfoque inmunómico .
%% con sintomatología severa y no complicada/ pre-inmunes/ semi-inmunes

\vspace{1.5 cm}

\Large
Proyecto de Tesis para obtar al Título Profesional de 

Biólogo Genetista y Biotecnólogo

\vspace{1 cm}

\Large
Bach. Andree Adolfo Valle Campos

\vspace{1 cm}

\Large
Asesores

Prof. Walter Cabrera-Febolá

PhD. G. Christian Baldeviano


\vspace{1 cm}

Lima - Perú

\vspace{.5 cm}

2017

\end{centering}

\vfill
\restoregeometry
\clearpage

\newpage

\tableofcontents

\newpage

\pagenumbering{arabic}

\section*{RESUMEN EJECUTIVO}\label{resumen-ejecutivo}
\addcontentsline{toc}{section}{RESUMEN EJECUTIVO}

\begin{quote}
RESUMEN A ESCRIBIR Y CORREGIR AL FINAL!!!! El presente proyecto tiene
como objetivo identificar antígenos con relevancia clínica contra la
malaria vivax. Para ello, se realizará un perfil a larga escala de
anticuerpos en pacientes sintomáticos a través de microarreglos de 500
proteínas de \emph{Plasmodium vivax} y 500 de \emph{P. falciparum}. Se
contrastarán parámetros generales de la respuesta humoral y evaluará la
reactividad diferenciada de antígenos en pacientes estratificados por
exposición previa y severidad. Finalmente, se propondrán nuevos
antígenos de \emph{P. vivax} con posible aplicación en la vigilancia
sero-epidemiológica. Finalmente, se propondrán nuevos antígenos de
\emph{P. vivax} con posible aplicación en la vigilancia
sero-epidemiológica.
\end{quote}

\section{PLANTEAMIENTO DEL PROBLEMA}\label{planteamiento-del-problema}

\subsection{Formulación del problema}\label{intro}

Malaria es una enfermedad infecciosa de importancia mundial, causada por
protozoarios parásitos del género \emph{Plasmodium} y transmitida por
mosquitos del género \emph{Anopheles}. En el 2015, se estimaron 212
millones de casos y 429,000 muertes atribuidas a esta infección a nivel
mundial.\textsuperscript{\protect\hyperlink{ref-WHO2016world}{1}} Aunque
\emph{P. falciparum} representó el 96 y 99\% de estas cifras, fuera de
África se estimó que \emph{P. vivax} fue responsable del 41 y 86\%,
respectivamente. Más aún, la región de las Américas tuvo la mayor
proporción de estos casos (69\%) donde Perú fue el tercer país con más
reportes (19\%), detrás de Brasil y Venezuela, atribuidos en un 80\%
(62,220 en total) a dicha
especie.\textsuperscript{\protect\hyperlink{ref-rosas2016peru}{2}}

La malaria por \emph{P. vivax} ha desafiado su tradicional condición de
enfermedad benigna debido a reportes recientes de enfermedad severa y
fatal.\textsuperscript{\protect\hyperlink{ref-quispe2014}{3}} A pesar de
este reconocimiento, factores condicionantes como la respuesta del
hospedero aún no han sido explorados. Por un lado, modelos
experimentales han sugerido un rol de la inmunidad en la reducción de la
severidad.\textsuperscript{\protect\hyperlink{ref-Moreno2013}{4}} Sin
embargo, estudios preliminares basados en la respuesta de anticuerpos
contra un solo antígeno han sugerido un grado similar de exposición
previa entre pacientes con malaria vivax severa y no
complicada.\textsuperscript{\protect\hyperlink{ref-baldevi2013}{5}}
Hasta el momento, dicha hipótesis no se ha puesto a prueba con un
enfoque a larga escala, cuya aplicación ha descubierto nuevos antígenos
con mayor resolución para la discriminación de condiciones clínicas
relevantes como inmunidad y
exposición.\textsuperscript{\protect\hyperlink{ref-crompton2010}{6},\protect\hyperlink{ref-Helb2015exposure}{7}}

Por esta razón, el presente estudio tiene como objetivo comparar la
respuesta de anticuerpos ante la infección con malaria vivax contra más
de 500 antígenos de \emph{P. vivax} en pacientes severos y no-severos
provenientes de la ciudad de Iquitos, empleando microarreglos de
proteínas. Primero, se identificarán los antígenos con reactividad
serológica diferenciada entre ambos grupos. Luego, se discriminarán las
respuestas secundarias de las primarias determinando los antígenos con
mayor reactividad en pacientes con exposición previa a esta infección.
Finalmente, se describirán sus características protéicas junto a los
antígenos con mayor reactividad en toda la muestra con el propósito de
proponerlos como candidatos a vigilancia sero-epidemiológica.

\subsection{Preguntas de
investigación}\label{preguntas-de-investigacion}

\begin{enumerate}
\def\labelenumi{\arabic{enumi}.}
\item
  ¿Cuáles son los antígenos de \emph{P. vivax} con reactividad
  serológica diferenciada ante la infección con malaria vivax entre
  pacientes severos y no-severos, de la ciudad de Iquitos (Loreto-Perú)?
\item
  ¿Cuáles son los antígenos de \emph{P. vivax} con reactividad
  serológica diferenciada ante la infección con malaria vivax entre
  pacientes con y sin exposición previa, de la ciudad de Iquitos
  (Loreto-Perú)?
\item
  ¿Cuáles son las características protéicas de los antígenos de \emph{P.
  vivax} con reactividad diferenciada o predominante en pacientes con
  malaria vivax?
\end{enumerate}

\subsection{Objetivos}\label{objetivos}

\subsubsection{General}\label{general}

\begin{itemize}
\tightlist
\item
  Identificar un subconjunto de antígenos de \emph{P. vivax} con
  reactividad serológica predominante o discriminante de condiciones
  clínicas ante infecciones con malaria vivax.
\end{itemize}

\subsubsection{Específicos}\label{especificos}

\begin{itemize}
\item
  Identificar antígenos con reactividad diferenciada según la severidad
  y exposición previa de los pacientes con malaria vivax.
\item
  Identificar los antígenos con mayor reactividad en los pacientes con
  malaria vivax sintomática.
\item
  Describir las características protéicas de los antígenos identificados
  según parámetros disponibles en la base de datos PlasmoDB.
\end{itemize}

\subsubsection{Exploratorios}\label{exploratorios}

\begin{itemize}
\item
  Comparar la amplitud e intensidad de respuesta de anticuerpos según la
  exposición y severidad de los pacientes con malaria vivax.
\item
  Comparar la amplitud e intensidad de respuesta de anticuerpos por
  rangos de edad de los pacientes con malaria vivax.
\item
  Comparar la amplitud e intensidad de respuesta de anticuerpos según la
  exposición y severidad por rangos de edad de los pacientes con malaria
  vivax.
\end{itemize}

\subsection{Justificación}\label{justif}

Ante la reemergencia repetitiva de la malaria en la amazonía peruana, la
implementación de vigilancias serológicas programáticas en nuestro país
es una
prioridad.\textsuperscript{\protect\hyperlink{ref-hotspots2015}{8}} En
zonas con baja transmisión, estos ensayos poseen una mayor sensibilidad
y representan un menor gasto económico en comparación a las estrategias
convensionales de monitoreo y
control.\textsuperscript{\protect\hyperlink{ref-elliott2014}{9}} Por
esta razón, el descubrimiento de antígenos potencialmente discriminantes
de condiciones clínicas como severidad o exposición permitirá optimizar
su ejecución en programas de salud pública contra la malaria en el Perú
en miras a su
eliminación.\textsuperscript{\protect\hyperlink{ref-accelerate2016}{10}}

En segundo lugar, un aspecto novedoso del presente estudio estará en el
desarrollo de un flujo de trabajo reproducible empleando el lenguaje de
programación y software libre
R\textsuperscript{\protect\hyperlink{ref-R}{11}} para el análisis de
respuestas serológicas a larga escala. Su implementación otorgará
ventajas en su desarrollo, corrección, reporte final y reutilización,
estimulando así la contribución de otros grupos y acelerando el progreso
en esta área de
investigación.\textsuperscript{\protect\hyperlink{ref-CienciaReproducible2016}{12}}
Este enfoque de análisis permitirá reducir las brechas técnicas que
continúan limitando el desarrollo de esta tecnología en nuestro país.

\subsection{Limitaciones}\label{limit}

Diseño Caso Control Entre sus debilidades está su suceptibilidad a
errores sistemáticos de selección y clasificación, su dificultad para
establecer relaciones causa-efecto y la imposibilidad de calcular
prevalencias o incidencias.

Infecciones no sincronizadas. Evaluados en etapa sintomática,
manifestada por fiebre, momento en el que acuden al centro de salud.

La primera limitante es inherente a la herramienta a emplear en el
estudio.\textsuperscript{\protect\hyperlink{ref-vigil2010}{13}} Cada
paso en la fabricación del microareglo de proteínas (amplificación,
clonamiento y expresión de genes a larga escala) posee una eficiencia
límite que afectará la expresión completa de las proteínas inicialmente
planificadas en el diseño. Además, el plegamiento y formación de
complejos multiméricos no será posible de verificar a dicha escala. Por
último, la identificación de antígenos con modificaciones
post-transcriptacionales, particularmente relevantes en
\emph{Plasmodium}\textsuperscript{\protect\hyperlink{ref-leroch2009postmod}{14}},
y antígenos no-protéicos, como polisacáridos y glucolípidos, no serán
posibles de reproducir en su integridad en el sistema de expresión
procarionte \emph{in vitro} a emplear.

Por otro lado, una última limitante está en el diseño experimental y las
variables recolectadas del estudio en el que se obtuvieron las muestras.
La baja incidencia de la malaria vivax severa justificó el empleo de una
vigilancia pasiva para ejecución de un estudio transversal. Esto implicó
la auscencia de un seguimiento activo o el registro de la historia
clínica de los pacientes con variables relevantes para la
caracterización de la severidad en base a la exposición de los pacientes
a malaria durante el último año.

\subsection{Viabilidad}\label{viabilidad}

El diseño del estudio es de tipo Caso-Control, ideal para el estudio de
casos infrecuentes, la evaluación de multiples factores para el mismo
resultado y eficiente en recursos por tiempo.

pequeños volumenes de suero, posibilidad de incremental tamaño muestral
a escala epidemiológica

\section{FORMULACIÓN DE HIPÓTESIS Y
VARIABLES}\label{formulacion-de-hipotesis-y-variables}

\subsection{Hipótesis}\label{hipotesis}

\subsubsection{De diferencia entre
grupos}\label{de-diferencia-entre-grupos}

\begin{enumerate}
\def\labelenumi{\arabic{enumi}.}
\item
  Los pacientes con malaria vivax no-severa poseen mayor reactividad
  serológica contra un subconjunto de antígenos de \emph{P. vivax} ante
  la infección que los pacientes con malatia vivax severa.
\item
  Los pacientes con exposición previa a malaria poseen mayor reactividad
  serológica contra un subconjunto de antígenos de \emph{P. vivax} ante
  la infección que los pacientes sin exposición previa.
\end{enumerate}

\subsubsection{Descriptiva}\label{descriptiva}

\begin{enumerate}
\def\labelenumi{\arabic{enumi}.}
\setcounter{enumi}{2}
\tightlist
\item
  Los antígenos con reactividad diferenciada o predominante en pacientes
  con malaria vivax poseen características propias de proteínas con
  localización extracelular y bajo presión selectiva por el sistema
  inmune.
\end{enumerate}

\subsection{Variables}\label{variables}

La reactividad serológica cuantificará la interacción específica de
antígeno y anticuerpo. El producto será una lectura cruda, con MFI como
unidades, la cual será previamente procesada para el análisis. Esta
cuantificación permitirá conocer las dianas de los anticuerpos
secretados por células plasmáticas en respuesta a la infección.

Los pacientes serán clasificados según episodios previos reportados o
criterios OMS de malaria severa, y ambas variables por presencia o
auscencia de dichos criterios. Con esta estratificación se podrá a
prueba la hipótesis de un perfil de anticuerpos dependiente de los
eventos previos y posteriormente de un perfil dependiente de la
severidad.

\subsubsection{Operacionalización de
variables}\label{operacionalizacion-de-variables}

Ver \autoref{tab:opera}.

\begin{table}[ht]
\begin{center}
\hspace*{-1cm}
\begin{tabular}{>{\centering}m{2.4cm} m{2.2cm}m{2.2cm}m{2cm}m{2.2cm}m{1.7cm}m{1.5cm}m{1.6cm} @{}m{0pt}@{} }
  
  \hline
  \multirow{2}{*}{Variable}
  & 
  \multicolumn{2}{c}{Definición} 
  %&
  %\begin{minipage}{2.2cm}
  %Definición\\conceptual
  %\end{minipage}
  %&
  %\begin{minipage}{2.2cm}
  %Definición\\operacional
  %\end{minipage}
  & 
  \multirow{2}{*}{
  \begin{minipage}{2.2cm}
  Instrumento\\de medición
  \end{minipage}
  }
  &
  \multirow{2}{*}{
  \begin{minipage}{2.2cm}
  Criterios\\de medición
  \end{minipage}
  }
  &
  \multirow{2}{*}{
  \begin{minipage}{1.7cm}
  Tipo de\\variable
  \end{minipage}
  }
  &
  \multirow{2}{*}{
  \begin{minipage}{1.5cm}
  Escala de \\medición
  \end{minipage}
  }
  &
  \multirow{2}{*}{
  Fuente
  } &\\[0ex]
  %\hline
  \cline{2-3}
  
  &
  Conceptual
  &
  Operacional
  & 
  &
  &
  & &\\[1ex]
  \hline
  
  \textbf{Dependiente} Reactividad serológica
  & 
  % esCONCEPTUAL: 
  \begin{minipage}{2.2cm} 
  Especificidad \\de anticuerpos \\de respuesta contra un antígeno
  \end{minipage} 
  &
  % aOPERACIONAL: 
  \begin{minipage}{2.2cm} 
  Medida \\indirecta de \\la reacción antígeno-anticuerpo
  \end{minipage} 
  % aDETALLES: medida indirecta de la reacción antígeno-anticuerpo 
  % mediante la lectura de la reacción fluorescente entre 
  % anticuerpo secundario y fluoroforo por spot
  & 
  \begin{minipage}{2.2cm} 
  Lector de\\
  microarreglos
  \end{minipage}
  & 
  \begin{minipage}{2.2cm} 
  \textbf{0-6000} MFI o intensidad\\
  fluorescente \\promedio.
  \end{minipage} 
  &
  Numérica contínua
  & 
  Razón
  &
  Plasma sanguíneo &\\[13ex]
  \hline

  \textbf{Independiente} Severidad
  & 
  % aCONCEPTUAL: 
  Presencia de manifestaciones clínicas severas o complicaciones sistémicas
  &
  % aOPERACIONAL:
  Número de criterios OMS para malaria severa
  & 
  \begin{minipage}{2.2cm} 
  Diagnóstico \\clínico \\y exámenes de \\laboratorio 
  \end{minipage}
  & 
  \begin{minipage}{2.2cm} 
  \textbf{No-severa:} 0 criterios.\\
  \textbf{Severa:} 1 o más criterios.
  \end{minipage}
  &
  Categórica dicotómica
  & 
  Nominal
  &
  Historia clínica y muestra de sangre &\\[15ex]
  \hline
  
  \textbf{Independiente} Exposición previa
  & 
  % aCONCEPTUAL: 
  Presencia de infecciones de malaria en el pasado
  &
  % aOPERACIONAL:
  Número de eventos previos reportados 
  & 
  Encuesta
  & 
  \begin{minipage}{2.2cm} 
  \textbf{Sin:} 0 eventos.\\
  \textbf{Con:} 1 o más eventos.
  \end{minipage}
  &
  Categórica dicotómica
  & 
  Nominal
  &
  Historia clínica &\\[10ex]
  \hline

%  \textbf{Interviniente} Edad %Confusora
%  & 
%  % aCONCEPTUAL: 
%  Edad del paciente
%  &
%  % aOPERACIONAL:
%  Años de vida reportados
%  & 
%  Encuesta
%  & 
%  \begin{minipage}{2.2cm} 
%  \textbf{0-90} años.
%  \end{minipage}
%  &
%  Numérica discreta
%  & 
%  Razón
%  &
%  Historia clínica &\\[10ex]
%  \hline


  % etc. ...
\end{tabular}
\hspace*{-1cm}
\end{center}
        \captionof{table}{Operacionalización de variables}
        \label{tab:opera}
\end{table}

\section{MARCO TEÓRICO}\label{marco-teorico}

\subsection{Antecedentes de la
investigación}\label{antecedentes-de-la-investigacion}

\subsection{Bases teóricas}\label{bases-teoricas}

\subsection{Definiciones conceptuales}\label{definiciones-conceptuales}

\subsubsection{Acrónimos}\label{acronimos}

\textbf{RTS:} Sistema rápido de expresión de proteínas recombinantes
líbre de células.

\textbf{IVTT:} (Proteína) transcrita/traducida \emph{in vitro}.

\textbf{MFI:} Unidad de lectura cruda expresada como \emph{Mean
Fluorescence Intensity} o intensidad fluorescente promedio de todos los
pixeles de cada \emph{spot}, normalizada localmente mediante la
sustracción de la intensidad de fondo presente a su alrededor.

\subsubsection{Términos}\label{terminos}

\begin{enumerate}
\def\labelenumi{\alph{enumi}.}
\tightlist
\item
  Generales
\end{enumerate}

\textbf{Antígeno.-} Molécula que se une con productos de la respuesta
inmune, como anticuerpos o receptores de lifocitos T o
B.\textsuperscript{\protect\hyperlink{ref-abbas2012}{15}}

\textbf{Inmunógeno.-} Un antígeno que induce una respuesta
inmunitaria.\textsuperscript{\protect\hyperlink{ref-abbas2012}{15}}

\textbf{Inmunoma.-} Conjunto de todos los inmunógenos que interactúan
con el sistema inmune de determinado
hospedero\textsuperscript{\protect\hyperlink{ref-immunomics2016}{16},\protect\hyperlink{ref-sette2005}{17}}

\textbf{Inmunómica.-} Estudio del
inmunoma.\textsuperscript{\protect\hyperlink{ref-immunomics2016}{16}}

\textbf{Anticuerpo.-} Tipo de molécula glucoprotéica, también llamada
inmunoglobulina (Ig), producida por los linfocitos B, que se une a
antígenos con un grado alto de especificidad y
afinidad.\textsuperscript{\protect\hyperlink{ref-abbas2012}{15}}

\textbf{Anticuerpo secundario.-} Anticuerpo de unión específica a la
fracción constante de los anticuerpos de un hospedero y, en este caso,
conjugado con biotina.

\textbf{Fluoróforo.-} Componente de una molécula que brinda la cualidad
de fluorescencia y, en este caso, conjugado con estreptavidina.

\begin{enumerate}
\def\labelenumi{\alph{enumi}.}
\setcounter{enumi}{1}
\tightlist
\item
  Específicos
\end{enumerate}

\textbf{Antígeno-IVTT:} Antígeno objetivo o \emph{spot} con proteína
IVTT a partir de un plásmido con DNA insertado del polipéptido, segmento
o exón de interés.

\textbf{Control-IVTT:} Control negativo o \emph{spot} con mix de
expresión RTS y plásmido sin DNA insertado, representante de la
intensidad de fondo específica del paciente.

\textbf{Proteína purificada:} Control de comparación o \emph{spot} con
proteína de antigenicidad conocida expresada dentro de célula.

\textbf{Transformación:} Logaritmo en base dos de la lectura cruda en
MFI de antígenos-IVTT y control-IVTT.

\textbf{Normalización:} Sustracción de la mediana de los control-IVTT a
cada antígeno-IVTT por individuo.

\textbf{Intensidad de antígeno:} Lectura normalizada de cada
antígeno-IVTT entre individuos. También llamada ``Reactividad de
anticuerpos''.

\textbf{Antígeno reactivo:} Antígenos-IVTT con una lectura transformada
mayor o igual a dos veces la mediana de los control-IVTT por individuo.

\textbf{Frecuencia del antígeno:} Porcentaje de individuos con antígeno
reactivo por antígeno-IVTT.

\textbf{Filtrado:} Remoción de todo antígeno que posea una frecuencia
menor al 10\% de todas las muestras.

\textbf{Intensidad de respuesta:} Promedio de intensidades de antígenos
reactivos por individuo

\textbf{Amplitud de respuesta:} Porcentaje de antígenos reactivos por
individuo.

\section{METODOLOGÍA}\label{metodologia}

\subsection{Diseño}\label{diseno}

El diseño del estudio es de tipo Caso-Control. Su aplicación consistirá
en la selección de la población objetivo por presencia (casos) o
auscencia (controles) del evento en estudio. Además se fijará el número
de eventos a estudiar, así como el número de sujetos sin evento que se
incluirán en la población de comparación.

\subsubsection{Tipo de investigación}\label{tipo-de-investigacion}

\textbf{Por la finalidad: Analítico.} A diferencia de un descriptivo,
realizaremos compraciones entre grupos y evaluaremos una posible
relación causal entre el factor y el efecto.

\textbf{Por el Control de la asignación: Observacional.} A diferencia de
un experimental, no controlaremos la asignación de los factores
(severidad y exposición previa). Es decir, se procederemos a observar el
fenómeno, ejecutar la medición y analizar los resultados.

\textbf{Por el seguimiento: Transversal.} A diferencia de uno
longitudinal, no ejecutaremos un seguimiento. Las variables se medirán
una sola vez, en un mismo instante.

\textbf{Por la relación cronológica: Prospectivo.} A diferencia de un
retrospectivo, recolectaremos los datos después de planificado el
estudio.

\textbf{Por la unidad de análisis: Basado en individuo.} A diferencia de
uno ecológico, la unidad de observación son individuos, llamados aquí
pacientes.

\subsection{Población y muestra}\label{poblacion-y-muestra}

\subsubsection{Población}\label{poblacion}

El universo teórico está representado por los pacientes con malaria
vivax de la cuenca amazónica del Perú.

El marco muestral está delimitado por pacientes diagnosticados con
malaria vivax de la ciudad de Iquitos, Loreto - Perú, entre enero del
2012 y junio del 2013 a lo largo de un estudio prospectivo ejecutado en
dos hospitales de referencia: Hospital de Apoyo y Hospital Regional.

\subsubsection{Criterios de inclusión}\label{criterios-de-inclusion}

En el estudio prospectivo mencionado, 164 pacientes con malaria vivax
fueron enrolados mediante vigilancia pasiva. Bajo los criterios clínicos
y de laboratorio de la Organización Mundial de la Salud (OMS o WHO por
sus siglas en inglés) para la Malaria
Severa\textsuperscript{\protect\hyperlink{ref-WHO2014severe}{18}}, 67
fueron clasificados como severos y 97 como no-severos.

Los criterios clínicos empleados fueron: Shock circulatorio (presión
sanguínea sistólica \textless{} 80 mmHg), Deterioro del nivel de
conciencia (puntaje Glasgow \(\le\) 9/14), Daño del sistema nerviosos
central (convulsión, postración, coma, confusión), Daño pulmonar
(disnea, taquipnea, infiltracion, edema), Síndrome de dificultad
respiratoria aguda o SDRA.

Los criterios de laboratorio empleados fueron: Hipoglicemia (glucosa
\textless{} 40 mg/dL), Anemia severa (hemoglobina \textless{} 7mg/dL),
Daño renal (creatinina \textgreater{} 3mg/dl), Hiperbilirrubina
(bilirrubina sérica \textgreater{} 2.5 mg/dL),

\subsubsection{Criterios de exclusión}\label{criterios-de-exclusion}

Dos criterios de exclusión fueron empleados: Presencia de
mono-infecciones de \emph{P. vivax} determinadas por PCR y auscencia de
co-infecciones como leptospirosis, dengue u otra arbovirosis por las
técnicas de aislamiento viral e inmunofluorescencia.

\subsubsection{Selección de
participantes}\label{seleccion-de-participantes}

A partir de esta clasificación, se ejecutará una selección aleatoria
simple de 30 pacientes con más de 2 criterios de malaria severa para el
grupo de casos y 30 no-severos para el grupo control.

\textbf{Tipo de muestra: Probabilística.} Cada elemento de la muestra
será seleccionado al azar, con probabilidad de selección conocida.

\subsection{Recolección de los datos e
Instrumento}\label{recoleccion-de-los-datos-e-instrumento}

\subsubsection{Técnica para la para recolección de
datos}\label{tecnica-para-la-para-recoleccion-de-datos}

Previo al tratamiento antimalárico del paciente, se extrajeron las
muestras de sangre tanto para el diagnóstico de malaria vivax por la
técnica de frotís como para las pruebas bioquímicas. Al momento del
diagnóstico positivo, bajo consentimiento informado y de forma
voluntaria, el plasma sanguíneo fue colectado y conservado a -80°C hasta
su uso.

\subsubsection{Instrumento de medición}\label{instrumento-de-medicion}

El microarreglo de proteínas es una herramienta que permite medir a
larga escala los anticuerpos reactivos a determinados antígenos de un
patógeno presentes en suero. Cada microarreglo ha sido diseñado con 1014
fragmentos protéicos recombinantes (498 de \emph{P. falciparum} y 516 de
\emph{P. vivax}, o expresado como Pf498/Pv516) representando 873
proteínas predichas (427 de \emph{P. falciparum} y 446 de \emph{P.
vivax}, \textasciitilde{}8\% del total predicho para el genoma de
\emph{P. vivax} Sal1) seleccionadas luego de una extensiva evaluación
serológica con un microarreglo de mayor escala (Pf2208/Pv2233). Las
proteínas del presente microarreglo han sido transcritas/traducidas
\emph{in vitro} empleando un sistema de expresión libre de células de
\emph{E. coli} e impresas en bloques de nitrocelulosa sobre una lámina
portaobjetos con 8 bloques paralelos, cada uno con 4 arreglos paralelos
compuestos por cuadrículas de 17x17 \emph{spots}. Este microarreglo ha
sido validado mediante el sondeo con suero colectado de pacientes con
malaria y controles sanos alrededor del
mundo.\textsuperscript{\protect\hyperlink{ref-King2015FOC}{19}}

\paragraph{Aplicación}\label{aplicacion}

La aplicación del instrumento consiste en tres pasos: sondeo, escaneo y
análisis, tal como ha sido publicado
previamente.\textsuperscript{\protect\hyperlink{ref-Driguez2015}{20}}
Primero, dentro de una cámara de incubación, los microarreglos son
hidratados con buffer de bloqueo. Segundo, el suero se diluye con buffer
de bloqueo en 1:100 y se pre-absorbe con lisado de \emph{E. coli} en
10\%(w/v), con el objetivo de reducir ruido de fondo producto de
anticuerpos reactivos con antígenos bacterianos presentes en el extracto
protéico empleado en el sistema de expresión. Tercero, se aspira el
buffer de bloqueo del microarreglo y se agrega el suero pre-absorbido.
Luego de una incubación \emph{overnight} en cámara húmeda a 4°C y en
leve agitación, se agrega la solución con anticuerpos secundarios
conjugados con biotina y posteriormente la solución con fluoróforo
conjugada con estreptavidina, con lavados y aspirado repetitivos entre
cada paso. A continuación, ya centrifugadas las láminas son escaneadas
con lectores de microarreglos de laser confocal (e.g., Genepix 4300A) y
sus medidas finales obtenidas luego de normalizarlas localmente mediante
la sustracción de la intensidad de fondo presente alrededor de cada
\emph{spot}. Finalmente, se procede al análisis de los datos el cual
será detallado \protect\hyperlink{procanal}{líneas abajo}.

\paragraph{Validez y confiabilidad}\label{validez}

Los métodos aquí empleados seguirán la metodología previamente reportada
para el presente intrumento de
medición.\textsuperscript{\protect\hyperlink{ref-crompton2010}{6}}

En primer lugar, la validez o exactitud del experimento será evaluada
mediante la correlación entre las lecturas de los antígenos-IVTT y sus
correspondientes proteínas purificadas por muestra, compuestas por
proteínas de inmunogenicidad conocida y expresadas por un sistema dentro
de célula.

En segundo lugar, la confiabilidad o reproducibilidad será evaluada
mediante la correlación entre controles positivos agregados a las
lecturas del primer y segundo día de sondeo de las muestras.

\subsubsection{Codificación y creación del archivo de
datos}\label{codificacion-y-creacion-del-archivo-de-datos}

Cada paciente del estudio estará identificado con un código de
estructura \texttt{LIM\#\#\#\#}, e.g.: \texttt{LIM1063}. Los antígenos
protéicos estarán identificados con el código asignado a sus genes en la
base de datos PlasmoDB, e.g.: \texttt{PF3D7\_0202500} o
\texttt{PVX\_091315}, ya sea un gen de \emph{P. falciparum} o \emph{P.
vivax}, respectivamente. En caso los genes poseean multiples exones, se
amplificarán por separado y se extenderá el código de cada uno con el
número del exón y el total de exones, e.g.: \texttt{\_1o2} exón 1 de un
gen con 2 exones. En caso los genes poseean una longitud mayor a 3000
nucleótidos, se dividirán en segmentos sobrelapantes entre 300 y 3000 nt
y se extenderá el código de cada uno con su respectivo número, e.g.:
\texttt{\_S1} para el primer segmento de un gen.

Los datos serán organizados en dos matrices: (i) archivo
\texttt{samples.csv} con los códigos de los pacientes y sus covariables
epidemiológicas y (ii) archivo \texttt{RawData.csv} con los códigos de
las proteínas y sus lecturas crudas en MFI por código de paciente.

\hypertarget{procanal}{\subsection{Procesamiento y Análisis de
datos}\label{procanal}}

Todo el análisis estadístico se realizará en el software de computación
estadística R\textsuperscript{\protect\hyperlink{ref-R}{11}}, el cual se
complementará con funciones provenientes de distintos paquetes.

\subsubsection{Procesamiento}\label{procesamiento}

Preliminarmente, describirá la distribución y proporción de las
covariables epidemiológicas, clínicas y de laboratorio de los pacientes
de la muestra. Luego, con las lecturas crudas en MFI se evaluará la
validez y reproducibilidad del ensayo mediante un test de asociación
entre variables contínuas, usando correlación de Pearson (\(r\)) o
Spearman (\(\rho\)), dependiendo de la distribución. Posteriormente, se
procederá con su transformación a escala logarítmica, normalización
entre muestras y filtrado en base al punto de corte establecido. Por
último, se asociarán ambas matrices de datos en un
\texttt{ExpressionSet}, a través de los códigos de pacientes, empleando
el paquete
\texttt{Biobase}.\textsuperscript{\protect\hyperlink{ref-Biobase}{21}}

\subsubsection{Confirmación de
hipótesis}\label{confirmacion-de-hipotesis}

Se pondrán a prueba las hipótesis con el siguiente protocolo. Primero,
se contrastará la amplitud e intensidad de respuesta con un test de
diferencias entre variables contínuas y no pareadas de dos grupos,
usando t-Student o Mann-Whitney, dependiendo de la distribución.
Segundo, se realizará un test de reactividad diferenciada de anticuerpos
entre dos grupos, usando el test t-moderado empírico de Bayes o
eBayes\textsuperscript{\protect\hyperlink{ref-smyth2004ebayes}{22}} con
corrección por comparación múltiple de la razón de falsos
descubrimientos por el método de Benjamini-Hochberg, disponible en el
paquete
\texttt{limma}.\textsuperscript{\protect\hyperlink{ref-limma}{23}}
Tercero, se realizará un \emph{clustering} jerárquico para agrupar a los
antígenos identificados en base a su distancia euclidiana, disponible en
el paquete
\texttt{NMF}.\textsuperscript{\protect\hyperlink{ref-Gaujoux2010NMF}{24}}
Finalmente, se mostrarán las siguiente características protéicas de los
antígenos identificados: presencia de dominios transmembrana, péptido
señal, número de ortólogos en \emph{Plasmodium}, ontología génica y
razón de mutaciones no-sinónimas sobre sinónimas, disponibles en la base
de datos PlasmoDB.\textsuperscript{\protect\hyperlink{ref-plasmodb}{25}}

\subsubsection{Visualización de
resultados}\label{visualizacion-de-resultados}

Los resultados se visualizarán mediante cinco tipos de gráficos.
Diagrama de dispersión para mostrar la correlación entre dos variables
contínuas, Diagramas de cajas para variables contínuas, Diagrama de
barras para frecuencias, \emph{Volcano plots} para mostrar la
reactividad diferenciada de antígenos entre grupos contra el valor P no
ajustado y resaltar los antígenos con valor P ajustado significativo, y
\emph{Heatmaps} para el perfil serológico en escala de dos colores
segmentado por clusters.

\section{ASPECTOS ADMINISTRATIVOS}\label{aspectos-administrativos}

\subsection{Cronograma de actividades}\label{cronograma-de-actividades}

Dado que las muestras ya han sido colectadas, solo se procederá a
recolectar lo datos a través del instrumento de medición descrito.

\begin{longtable}[]{@{}lllllll@{}}
\toprule
\textbf{ACTIVIDAD PROGRAMADA} & & & & & &\tabularnewline
\midrule
\endhead
& \textbf{Jun} & \textbf{Jul} & \textbf{Ago} & \textbf{Sep} &
\textbf{Oct} & \textbf{Nov}\tabularnewline
Aprovación del proyecto de tesis & X & & & & &\tabularnewline
Procesamiento de datos & & X & & & &\tabularnewline
Interpretación de resultados & & & X & & &\tabularnewline
Redacción final & & & & X & &\tabularnewline
Correcciones & & & & & X &\tabularnewline
\textbf{Sustentación} & & & & & & X\tabularnewline
\bottomrule
\end{longtable}

\subsection{Presupuesto y
financiamiento}\label{presupuesto-y-financiamiento}

El proyecto será realizado en El Centro de Enfermedades Tropicales de la
Marina de los Estados Unidos NAMRU-6 y financiado por ****.

\begin{longtable}[]{@{}lcc@{}}
\toprule
\begin{minipage}[b]{0.46\columnwidth}\raggedright\strut
\textbf{DESCRIPCIÓN}\strut
\end{minipage} & \begin{minipage}[b]{0.22\columnwidth}\centering\strut
\textbf{MONTO (S/)}\strut
\end{minipage} & \begin{minipage}[b]{0.22\columnwidth}\centering\strut
\textbf{PORCENTAJE (\%)}\strut
\end{minipage}\tabularnewline
\midrule
\endhead
\begin{minipage}[t]{0.46\columnwidth}\raggedright\strut
\textbf{Bienes}\strut
\end{minipage} & \begin{minipage}[t]{0.22\columnwidth}\centering\strut
\strut
\end{minipage} & \begin{minipage}[t]{0.22\columnwidth}\centering\strut
\strut
\end{minipage}\tabularnewline
\begin{minipage}[t]{0.46\columnwidth}\raggedright\strut
Papelería, útiles y material de oficina\strut
\end{minipage} & \begin{minipage}[t]{0.22\columnwidth}\centering\strut
50\strut
\end{minipage} & \begin{minipage}[t]{0.22\columnwidth}\centering\strut
0.16\strut
\end{minipage}\tabularnewline
\begin{minipage}[t]{0.46\columnwidth}\raggedright\strut
Insumos, instrumental y accesorios de laboratorio\strut
\end{minipage} & \begin{minipage}[t]{0.22\columnwidth}\centering\strut
200\strut
\end{minipage} & \begin{minipage}[t]{0.22\columnwidth}\centering\strut
0.63\strut
\end{minipage}\tabularnewline
\begin{minipage}[t]{0.46\columnwidth}\raggedright\strut
Productos químicos\strut
\end{minipage} & \begin{minipage}[t]{0.22\columnwidth}\centering\strut
25\strut
\end{minipage} & \begin{minipage}[t]{0.22\columnwidth}\centering\strut
0.08\strut
\end{minipage}\tabularnewline
\begin{minipage}[t]{0.46\columnwidth}\raggedright\strut
\textbf{Servicios}\strut
\end{minipage} & \begin{minipage}[t]{0.22\columnwidth}\centering\strut
\strut
\end{minipage} & \begin{minipage}[t]{0.22\columnwidth}\centering\strut
\strut
\end{minipage}\tabularnewline
\begin{minipage}[t]{0.46\columnwidth}\raggedright\strut
Compra, sondeo y lectura de microarreglos\strut
\end{minipage} & \begin{minipage}[t]{0.22\columnwidth}\centering\strut
29610\strut
\end{minipage} & \begin{minipage}[t]{0.22\columnwidth}\centering\strut
92.79\strut
\end{minipage}\tabularnewline
\begin{minipage}[t]{0.46\columnwidth}\raggedright\strut
Gastos en el transporte de muestras\strut
\end{minipage} & \begin{minipage}[t]{0.22\columnwidth}\centering\strut
2000\strut
\end{minipage} & \begin{minipage}[t]{0.22\columnwidth}\centering\strut
6.27\strut
\end{minipage}\tabularnewline
\begin{minipage}[t]{0.46\columnwidth}\raggedright\strut
Impresiones, encuadernación y empastado\strut
\end{minipage} & \begin{minipage}[t]{0.22\columnwidth}\centering\strut
25\strut
\end{minipage} & \begin{minipage}[t]{0.22\columnwidth}\centering\strut
0.08\strut
\end{minipage}\tabularnewline
\begin{minipage}[t]{0.46\columnwidth}\raggedright\strut
\textbf{TOTAL}\strut
\end{minipage} & \begin{minipage}[t]{0.22\columnwidth}\centering\strut
31910\strut
\end{minipage} & \begin{minipage}[t]{0.22\columnwidth}\centering\strut
100\strut
\end{minipage}\tabularnewline
\bottomrule
\end{longtable}

\section{BIBLIOGRAFÍA}\label{bibliografia}

\textbf{De acuerdo al orden en el que han sido citadas:}

\hypertarget{refs}{}
\hypertarget{ref-WHO2016world}{}
1. WHO. World malaria report 2016. \emph{Geneva: WHO}. 2016;13.

\hypertarget{ref-rosas2016peru}{}
2. Rosas-Aguirre A, Gamboa D, Manrique P, et al. Epidemiology of
plasmodium vivax malaria in peru. \emph{The American Journal of Tropical
Medicine and Hygiene}. 2016;95(6 Suppl):133-144.
doi:\href{https://doi.org/https://doi.org/10.4269/ajtmh.16-0268}{https://doi.org/10.4269/ajtmh.16-0268}.

\hypertarget{ref-quispe2014}{}
3. Quispe AM, Pozo E, Guerrero E, et al. Plasmodium vivax
hospitalizations in a monoendemic malaria region: Severe vivax malaria?
\emph{The American journal of tropical medicine and hygiene}.
2014;91(1):11-17.
doi:\href{https://doi.org/https://doi.org/10.4269/ajtmh.12-0610}{https://doi.org/10.4269/ajtmh.12-0610}.

\hypertarget{ref-Moreno2013}{}
4. Moreno A, Cabrera-Mora M, Garcia A, et al. Plasmodium coatneyi in
rhesus macaques replicates the multisystemic dysfunction of severe
malaria in humans. \emph{Infection and Immunity}. 2013;81(6):1889-1904.
doi:\href{https://doi.org/10.1128/IAI.00027-13}{10.1128/IAI.00027-13}.

\hypertarget{ref-baldevi2013}{}
5. Baldeviano GC, Leiva KP, Quispe AM, et al. Serum markers of severe
clinical complications during plasmodium vivax malaria monoinfections in
the peruvian amazon basin. In: \emph{Abstract Book of the Astmh 62nd
Annual Meeting, Nov. 13--17, Washington d.C., United States}.; 2013:340.
\url{http://www.astmh.org/ASTMH/media/Documents/AbstractBook2013Final.pdf}.

\hypertarget{ref-crompton2010}{}
6. Crompton PD, Kayala MA, Traore B, et al. A prospective analysis of
the ab response to plasmodium falciparum before and after a malaria
season by protein microarray. \emph{Proceedings of the National Academy
of Sciences}. 2010;107(15):6958-6963.

\hypertarget{ref-Helb2015exposure}{}
7. Helb DA, Tetteh KKA, Felgner PL, et al. Novel serologic biomarkers
provide accurate estimates of recent plasmodium falciparum exposure for
individuals and communities. \emph{Proceedings of the National Academy
of Sciences}. 2015;112(32):E4438-E4447.
doi:\href{https://doi.org/10.1073/pnas.1501705112}{10.1073/pnas.1501705112}.

\hypertarget{ref-hotspots2015}{}
8. Rosas-Aguirre NAL-C Angel AND Speybroeck. Hotspots of malaria
transmission in the peruvian amazon: Rapid assessment through a
parasitological and serological survey. \emph{PLOS ONE}.
2015;10(9):1-21.
doi:\href{https://doi.org/10.1371/journal.pone.0137458}{10.1371/journal.pone.0137458}.

\hypertarget{ref-elliott2014}{}
9. Elliott SR, Fowkes F, Richards JS, Reiling L, Drew DR, Beeson JG.
Research priorities for the development and implementation of
serological tools for malaria surveillance. \emph{F1000Prime Rep}.
2014;6:100.

\hypertarget{ref-accelerate2016}{}
10. Quispe AM, Llanos-Cuentas A, Rodriguez H, et al. Accelerating to
zero: Strategies to eliminate malaria in the peruvian amazon. \emph{The
American Journal of Tropical Medicine and Hygiene}.
2016;94(6):1200-1207.
doi:\href{https://doi.org/https://doi.org/10.4269/ajtmh.15-0369}{https://doi.org/10.4269/ajtmh.15-0369}.

\hypertarget{ref-R}{}
11. R Core Team. \emph{R: A Language and Environment for Statistical
Computing}. Vienna, Austria: R Foundation for Statistical Computing;
2016. \url{https://www.R-project.org/}.

\hypertarget{ref-CienciaReproducible2016}{}
12. Rodríguez-Sanchez F, Pérez-Luque AJ, Bartomeus I, Varela S. Ciencia
reproducible: qué, por qué, cómo? \emph{ECOS}. 2016;25(2):83-92.
doi:\href{https://doi.org/10.7818/ecos.2016.25-2.11}{10.7818/ecos.2016.25-2.11}.

\hypertarget{ref-vigil2010}{}
13. Vigil A, Davies DH, Felgner PL. Defining the humoral immune response
to infectious agents using high-density protein microarrays.
\emph{Future microbiology}. 2010;5(2):241-251.

\hypertarget{ref-leroch2009postmod}{}
14. Chung D-WD, Ponts N, Cervantes S, Le Roch KG. Post-translational
modifications in plasmodium: More than you think! \emph{Molecular and
biochemical parasitology}. 2009;168(2):123-134.

\hypertarget{ref-abbas2012}{}
15. Abbas A, Lichtman A, Pillai S. \emph{Inmunología Celular Y Molecular
+ Student Consult}. Elsevier Health Sciences Spain; 2012.
\url{https://books.google.es/books?id=iQCupVehIXQC}.

\hypertarget{ref-immunomics2016}{}
16. De Sousa KP, Doolan DL. Immunomics: A 21st century approach to
vaccine development for complex pathogens. \emph{Parasitology}.
2016;143(02):236-244.

\hypertarget{ref-sette2005}{}
17. Sette A, Fleri W, Peters B, Sathiamurthy M, Bui H-H, Wilson S. A
roadmap for the immunomics of category a--C pathogens. \emph{Immunity}.
2005;22(2):155-161.

\hypertarget{ref-WHO2014severe}{}
18. WHO. Severe malaria. \emph{Trop Med Int Health}. 2014;19:7-131.
doi:\href{https://doi.org/10.1111/tmi.12313_2}{10.1111/tmi.12313\_2}.

\hypertarget{ref-King2015FOC}{}
19. King CL, Davies DH, Felgner P, et al. Biosignatures of
exposure/transmission and immunity. \emph{American Journal of Tropical
Medicine and Hygiene}. 2015;93(3 Suppl):16-27.
doi:\href{https://doi.org/10.4269/ajtmh.15-0037}{10.4269/ajtmh.15-0037}.

\hypertarget{ref-Driguez2015}{}
20. Driguez P, Doolan DL, Molina DM, et al. Protein microarrays for
parasite antigen discovery. In: Peacock C, ed. \emph{Parasite Genomics
Protocols}. New York, NY: Springer New York; 2015:221-233.
doi:\href{https://doi.org/10.1007/978-1-4939-1438-8_13}{10.1007/978-1-4939-1438-8\_13}.

\hypertarget{ref-Biobase}{}
21. Huber, W., Carey, et al. Orchestrating high-throughput genomic
analysis with Bioconductor. \emph{Nature Methods}. 2015;12(2):115-121.
\url{http://www.nature.com/nmeth/journal/v12/n2/full/nmeth.3252.html}.

\hypertarget{ref-smyth2004ebayes}{}
22. Smyth GK, others. Linear models and empirical bayes methods for
assessing differential expression in microarray experiments. \emph{Stat
Appl Genet Mol Biol}. 2004;3(1):3.

\hypertarget{ref-limma}{}
23. Ritchie ME, Phipson B, Wu D, et al. limma powers differential
expression analyses for RNA-sequencing and microarray studies.
\emph{Nucleic Acids Research}. 2015;43(7):e47.

\hypertarget{ref-Gaujoux2010NMF}{}
24. Gaujoux R, Seoighe C. A flexible r package for nonnegative matrix
factorization. \emph{BMC Bioinformatics}. 2010;11(1):367.
doi:\href{https://doi.org/10.1186/1471-2105-11-367}{10.1186/1471-2105-11-367}.

\hypertarget{ref-plasmodb}{}
25. Aurrecoechea C, Brestelli J, Brunk BP, et al. PlasmoDB: A functional
genomic database for malaria parasites. \emph{Nucleic acids research}.
2009;37(suppl 1):D539-D543.


\end{document}
