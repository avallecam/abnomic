\documentclass[]{article}
\usepackage{lmodern}
\usepackage{amssymb,amsmath}
\usepackage{ifxetex,ifluatex}
\usepackage{fixltx2e} % provides \textsubscript
\ifnum 0\ifxetex 1\fi\ifluatex 1\fi=0 % if pdftex
  \usepackage[T1]{fontenc}
  \usepackage[utf8]{inputenc}
\else % if luatex or xelatex
  \ifxetex
    \usepackage{mathspec}
  \else
    \usepackage{fontspec}
  \fi
  \defaultfontfeatures{Ligatures=TeX,Scale=MatchLowercase}
\fi
% use upquote if available, for straight quotes in verbatim environments
\IfFileExists{upquote.sty}{\usepackage{upquote}}{}
% use microtype if available
\IfFileExists{microtype.sty}{%
\usepackage{microtype}
\UseMicrotypeSet[protrusion]{basicmath} % disable protrusion for tt fonts
}{}
\usepackage[margin=1in]{geometry}
\usepackage{hyperref}
\hypersetup{unicode=true,
            pdfborder={0 0 0},
            breaklinks=true}
\urlstyle{same}  % don't use monospace font for urls
\usepackage{longtable,booktabs}
\usepackage{graphicx,grffile}
\makeatletter
\def\maxwidth{\ifdim\Gin@nat@width>\linewidth\linewidth\else\Gin@nat@width\fi}
\def\maxheight{\ifdim\Gin@nat@height>\textheight\textheight\else\Gin@nat@height\fi}
\makeatother
% Scale images if necessary, so that they will not overflow the page
% margins by default, and it is still possible to overwrite the defaults
% using explicit options in \includegraphics[width, height, ...]{}
\setkeys{Gin}{width=\maxwidth,height=\maxheight,keepaspectratio}
\IfFileExists{parskip.sty}{%
\usepackage{parskip}
}{% else
\setlength{\parindent}{0pt}
\setlength{\parskip}{6pt plus 2pt minus 1pt}
}
\setlength{\emergencystretch}{3em}  % prevent overfull lines
\providecommand{\tightlist}{%
  \setlength{\itemsep}{0pt}\setlength{\parskip}{0pt}}
\setcounter{secnumdepth}{5}
% Redefines (sub)paragraphs to behave more like sections
\ifx\paragraph\undefined\else
\let\oldparagraph\paragraph
\renewcommand{\paragraph}[1]{\oldparagraph{#1}\mbox{}}
\fi
\ifx\subparagraph\undefined\else
\let\oldsubparagraph\subparagraph
\renewcommand{\subparagraph}[1]{\oldsubparagraph{#1}\mbox{}}
\fi

%%% Use protect on footnotes to avoid problems with footnotes in titles
\let\rmarkdownfootnote\footnote%
\def\footnote{\protect\rmarkdownfootnote}

%%% Change title format to be more compact
\usepackage{titling}

% Create subtitle command for use in maketitle
\newcommand{\subtitle}[1]{
  \posttitle{
    \begin{center}\large#1\end{center}
    }
}

\setlength{\droptitle}{-2em}
  \title{}
  \pretitle{\vspace{\droptitle}}
  \posttitle{}
  \author{}
  \preauthor{}\postauthor{}
  \date{}
  \predate{}\postdate{}

\usepackage{multirow}
\usepackage{pdflscape}
\usepackage{afterpage}
\usepackage{capt-of}
\usepackage{array}

\begin{document}

\renewcommand{\contentsname}{Índice General} 
\renewcommand{\tablename}{Tabla}
\renewcommand{\tableautorefname}{Tabla}

\pagenumbering{gobble}

\clearpage
\newgeometry{top=1cm,bottom=1cm} \vspace*{\fill}

\begin{centering}

\begin{figure}[!ht]
  \begin{center}
    \includegraphics[width=.8in]{figure/UNMSM_escudo-2000px.png}
  \end{center}
\end{figure}

\Large
UNIVERSIDAD NACIONAL MAYOR DE SAN MARCOS

\large
(Universidad del Perú, DECANA DE AMÉRICA)

\vspace{.5 cm}

\Large
FACULTAD DE CIENCIAS BIOLÓGICAS

\vspace{.5 cm}

\normalsize
ESCUELA ACADÉMICO PROFESIONAL DE

GENÉTICA Y BIOTECNOLOGÍA

\vspace{1.5 cm}

\Large
Comparación de la respuesta de anticuerpos ante la %infección con 
malaria vivax 
en pacientes de la ciudad de Iquitos (Loreto - Perú)
según su severidad y exposición previa
mediante microarreglos de proteínas
% aPerfil de anticuerpos en respuesta a la infección con malaria vivax
%mediante un enfoque inmunómico .
%% con sintomatología severa y no complicada/ pre-inmunes/ semi-inmunes

\vspace{1.5 cm}

\Large
Proyecto de Tesis para obtar al Título Profesional de 

Biólogo Genetista y Biotecnólogo

\vspace{1 cm}

\Large
Bach. Andree Adolfo Valle Campos

\vspace{1 cm}

\Large
Asesores

Prof. Walter Cabrera-Febolá

PhD. G. Christian Baldeviano


\vspace{1 cm}

Lima - Perú

\vspace{.5 cm}

2017

\end{centering}

\vfill
\restoregeometry
\clearpage

\newpage

\tableofcontents

\newpage

\pagenumbering{arabic}

\section{RESUMEN EJECUTIVO}\label{resumen-ejecutivo}

\begin{quote}
RESUMEN A CORREGIR AL FINAL!!!! El presente proyecto tiene como objetivo
identificar antígenos con relevancia clínica contra la malaria vivax.
Para ello, se realizará un perfil a larga escala de anticuerpos en
pacientes sintomáticos a través de microarreglos de 500 proteínas de
\emph{Plasmodium vivax} y 500 de \emph{P. falciparum}. Se contrastarán
parámetros generales de la respuesta humoral y evaluará la reactividad
diferenciada de antígenos en pacientes estratificados por exposición
previa y severidad. Finalmente, se propondrán nuevos antígenos de
\emph{P. vivax} con posible aplicación en la vigilancia serológica de
malaria vivax.
\end{quote}

\section{PLANTEAMIENTO DEL PROBLEMA}\label{planteamiento-del-problema}

\subsection{Introducción}\label{intro}

Malaria es una enfermedad infecciosa de importancia mundial, causada por
protozoarios parásitos del género \emph{Plasmodium} y transmitida por
mosquitos del género \emph{Anopheles}. En el 2015, se estimaron 212
millones de casos y 429,000 muertes atribuidas a esta infección a nivel
mundial.\textsuperscript{\protect\hyperlink{ref-WHO2016world}{1}} Si
bien \emph{P. falciparum} representó el 96 y 99\% de estas cifras, fuera
de África se estimó que \emph{P. vivax} fue responsable del 41 y 86\%,
respectivamente. Más aún, la región de las Américas tuvo la mayor
proporción de estos casos (69\%) donde el Perú fue el tercer país con
más reportes de malaria (19\%), detrás de Brasil y Venezuela, atribuidos
en un 80\% (62,220 en total) a dicha
especie.\textsuperscript{\protect\hyperlink{ref-rosas2016peru}{2}}

La malaria por \emph{P. vivax} han desafiado su tradicional condición de
enfermedad benigna debido a reportes recientes de enfermedad severa y
fatal.\textsuperscript{\protect\hyperlink{ref-quispe2014}{3}} Incluso,
con una patogénesis similar a la desarrollada por \emph{P.
falciparum}.\textsuperscript{\protect\hyperlink{ref-barber2015}{4}} Con
respecto a sus factores condicionantes, Moreno et
al.\textsuperscript{\protect\hyperlink{ref-Moreno2013}{5}} sugirió un
rol de la inmunidad en la reducción de la severidad luego de una
subsecuente exposición a la infección. Por una parte, los enfoques a
larga escala han demostrado que la malaria induce potentes respuestas de
anticuerpos que incrementan con exposiciones
consecutivas\textsuperscript{\protect\hyperlink{ref-crompton2010}{6}}.
Por otra, la cuantificación de títulos de anticuerpos IgG anti-
\emph{Pv} MSP1 entre pacientes según su severidad a la malaria vivax
sugirió un grado similar de exposición
previa.\textsuperscript{\protect\hyperlink{ref-baldevi2013}{7}} Sin
embargo, ningún estudio hasta el momento ha contrastado a una mayor
escala la respuesta inmune humoral entre pacientes con malaria vivax
severa y no complicada.

Por esta razón, el presente estudio tiene el objetivo de comparar la
respuesta de anticuerpos ante la infección con malaria vivax en
pacientes con malaria severa y no-severa provenientes de un estudio
prospectivo conducido en la ciudad de Iquitos entre marzo del 2012 y
junio del 2013, mediante la tecnología del microarreglo de proteínas con
más de 500 antígenos de \emph{P. vivax} expresados en el estadio
eritrocítico de la infección. Primero, se identificarán los antígenos
con reactividad serológica diferenciada entre ambos grupos. Luego, con
el fin de discriminar entre respuestas humorales secundarias y
primarias, se determinarán los antígenos con mayor reactividad en
pacientes con exposición previa a la malaria. Finalmente, se describirán
sus características protéicas junto a los antígenos con mayor
reactividad en toda la muestra con el propósito proponerlos como
candidatos a vigilancia sero-epidemiológica.

\subsection{Preguntas de
investigación}\label{preguntas-de-investigacion}

\begin{enumerate}
\def\labelenumi{\arabic{enumi}.}
\item
  ¿Cuáles son los antígenos de \emph{P. vivax} con reactividad
  serológica diferenciada ante la infección con malaria vivax entre
  pacientes severos y no-severos, de la ciudad de Iquitos (Loreto-Perú)?
\item
  ¿Cuáles son los antígenos de \emph{P. vivax} con reactividad
  serológica diferenciada ante la infección con malaria vivax entre
  pacientes con y sin exposición previa, de la ciudad de Iquitos
  (Loreto-Perú)?
\item
  ¿Cuáles son las características protéicas de los antígenos de \emph{P.
  vivax} con reactividad diferenciada o alta reactividad en pacientes
  con malaria vivax?
\end{enumerate}

\subsection{Hipótesis}\label{hipotesis}

\subsubsection{De diferencia entre
grupos}\label{de-diferencia-entre-grupos}

\begin{enumerate}
\def\labelenumi{\arabic{enumi}.}
\item
  Los pacientes con malaria vivax no-severa poseen mayor reactividad
  serológica contra un subconjunto de antígenos de \emph{P. vivax} ante
  la infección que los pacientes con malatia vivax severa.
\item
  Los pacientes con exposición previa a malaria poseen mayor reactividad
  serológica contra un subconjunto de antígenos de \emph{P. vivax} ante
  la infección que los pacientes sin exposición previa.
\end{enumerate}

\subsubsection{Descriptiva}\label{descriptiva}

\begin{enumerate}
\def\labelenumi{\arabic{enumi}.}
\setcounter{enumi}{2}
\tightlist
\item
  Los antígenos con reactividad diferenciada o alta ractividad poseen
  características propias de proteínas con localización extracelular y
  bajo presión selectiva por el sistema inmune, entre ellas la presencia
  de dominios transmembrana, péptidos señal y razón de mutaciones
  no-sinónimas/sinónimas mayor a 1.
\end{enumerate}

\subsection{Justificación}\label{justif}

El presente estudio se justifica en el descubrimiento de antígenos
discriminantes de condiciones clínicas relevantes como severidad o
exposición. Estos permitirán optimizar y acelerar la ejecución de
vigilancias
sero-epidemiológicas\textsuperscript{\protect\hyperlink{ref-hotspots2015}{8}}
y evaluación de programas de salud pública contra la malaria en el Perú,
en miras a su
eliminación.\textsuperscript{\protect\hyperlink{ref-accelerate2016}{9}}

Históricamente, reemergencias repetitivas de malaria en la amazonía
peruana han demostrado que las estrategias convensionales de monitoreo y
control ya no son
sostenibles.\textsuperscript{\protect\hyperlink{ref-rosas2016peru}{2}}
Una solución a este problema está en el uso de anticuerpos
especie-específicos como biomarcadores para el monitoreo activo de la
exposición e inmunidad. En zonas de baja transmisión, estos ensayos
poseen una mayor sensibilidad y menor gasto económico en comparación a
las estimaciones entomológicas o prevalencias por
parasitemia.\textsuperscript{\protect\hyperlink{ref-elliott2014}{10}}

Por otro lado, un aspecto novedoso del presente trabajo estará en el
desarrollo de un flujo de trabajo reproducible con el software libre de
computación estadística R para el análisis de respuestas serológicas a
larga escala. Su implementación otorgará ventajas al desarrollo,
corrección y reporte final del proyecto. Además, la publicación del
código escrito facilitará su reutilización, estimulando la contribución
de otros grupos y acelerando el progreso en esta área de
investigación.\textsuperscript{\protect\hyperlink{ref-CienciaReproducible2016}{11}}

Como conclusión, tanto la implementación de novedosas herramientas de
análisis con software libre como el descubrimiento de biomarcadores con
relevancia clínica permitirán reducir las brechas técnicas y de
conocimiento que continúan limitando el desarrollo de vigilancias
serológicas programáticas en el Perú.

\section{ANTECEDENTES}\label{antecedentes}

\subsection{Antecedentes de la
investigación}\label{antecedentes-de-la-investigacion}

Crompton PD (2010) en su investigación titulada ``\emph{A prospective
analysis of the Ab response to Plasmodium falciparum before and after a
malaria season by protein microarray}'' menciona el desarrollo de un
sistema de microarreglos de proteínas que permiten la rápida y precisa
de la identificación de targets a vacuna y serodiagnóstico, permitiendo
definir su inmunogenicidad.

Molina DM (2012) en su artículo ``\emph{Plasmodium vivax
Pre-Erythrocytic--Stage Antigen Discovery: Exploiting Naturally Acquired
Humoral Responses}'', afirma que la disponibilidad de la secuencia del
genoma y transcriptoma del \emph{P. vivax} de la cepa \emph{Sal1}
proporciona los medios para analizar las respuestas inmunes específicas
de poblaciones endémicas, brindando la capacidad de seleccionar
antígenos óptimos \emph{P. vivax} para el desarrollo de vacunas.

Luego, Baum E (2015) en el artículo ``\emph{Submicroscopic and
asymptomatic Plasmodium falciparum and Plasmodium vivax infections are
common in western Thailand-molecular and serological evidence}'' nos
informa que los estudios en \emph{P. falciparum} sugieren que la
amplitud y la magnitud de las respuestas de anticuerpos a antígenos del
parásito determinan el nivel de protección, poniendo de relieve la
necesidad de caracterizar plenamente la respuesta de anticuerpos después
de la exposición natural de \emph{P. vivax} para desarrollar una vacuna
anti-infección y anti-enfermedad eficaz.

En armonía con dicha propuesta, el grupo peruano liderado por Torres KJ
(2014) de la Universidad Peruana Cayetano Heredia en el artículo
``\emph{Genome-Level Determination of Plasmodium falciparum Blood Stage
Targets of Malarial Clinical Immunity in the Peruvian Amazon}'' aseveran
que individuos clínicamente inmunes probablemente sean reservorios de
transmisión de la malaria y, por lo tanto, la comprensión de estos
mecanismos inmunológicos adquiridos puede conducir al desarrollo de
nuevas estrategias de vacuna contra la malaria.

\subsection{Bases teóricas}\label{bases-teoricas}

\begin{enumerate}
\def\labelenumi{\alph{enumi}.}
\item
  Epidemiología de la Malaria

  \begin{enumerate}
  \def\labelenumii{\roman{enumii}.}
  \tightlist
  \item
    \textbf{A nivel mundial}
  \end{enumerate}

  La malaria sigue ejerciendo una enfermedad significativa en todo el
  mundo. De cinco especies de Plasmodium que causan la enfermedad en
  humanos, \emph{P. falciparum} y \emph{P. vivax} son responsables de la
  mayoría de los casos de malaria en todo el mundo (Murray CJ, et al.
  2012). Los esfuerzos de investigación se han centrado tradicionalmente
  en \emph{P. falciparum} causa de la mortalidad que causa en África.
  Sin embargo, \emph{P. vivax} es la especie más ampliamente distribuida
  de la malaria, y más de 2,8 millones de personas están en riesgo de
  adquirir \emph{P. vivax} (WHO, World Malaria Report. 2011).

  \begin{enumerate}
  \def\labelenumii{\roman{enumii}.}
  \setcounter{enumii}{1}
  \tightlist
  \item
    \textbf{En el continente Sudamericano}
  \end{enumerate}

  En las Américas, el \emph{P. vivax} es responsable del 70\% de los
  casos de malaria reportados. Además, existe una creciente preocupación
  por el impacto en la salud global de las infecciones por \emph{P.
  vivax} causa de nuestra aparente subestimación de la carga de
  morbilidad potencial y su amplia distribución geográfica
  (Arevalo-Herrera M, et al. 2012). Por último, aun cuando los esfuerzos
  globales de eliminación de la malaria se centran en la reducción de
  los casos de \emph{P. falciparum} en fase arterial, la etapa hepática
  de hipnozoito latente confiere una ventaja de supervivencia única para
  \emph{P. vivax} (Kochar DK, et al. 2005).
\item
  Características de la malaria vivax

  \begin{enumerate}
  \def\labelenumii{\roman{enumii}.}
  \tightlist
  \item
    \textbf{Malaria vivax severa}
  \end{enumerate}

  Informes recientes de malaria grave y fatal han desafiado el dogma de
  que \emph{P. vivax} es un enfermedad benigna. Una amplia variedad de
  complicaciones graves se han descrito y asociado con monoinfecciones
  de \emph{P. vivax}, que van desde el fallo de un solo órgano (cerebro,
  hígado, pulmones, riñones, sangre, bazo, etc.) hasta fallas en
  múltiples órganos que ponen en peligro la vida del paciente (Andrade
  BB, et al. 2010).

  \begin{enumerate}
  \def\labelenumii{\roman{enumii}.}
  \setcounter{enumii}{1}
  \tightlist
  \item
    \textbf{Malaria vivax y coinfecciones}
  \end{enumerate}

  En base a esta nueva evidencia, más investigadores proponen ahora que
  la malaria severa podría ser causada por monoinfecciones de \emph{P.
  vivax}, especialmente cuando las comorbilidades, como el dengue, la
  leptospirosis, hepatitis viral, sepsis bacteriana, y P. falciparum,
  han sido descartadas (Genton B, et al. 2008). Sin embargo, debido a
  que la mayoría de los casos se originan a partir de áreas coendemicas
  de \emph{P. vivax / P. falciparum}, la exposición recurrente previa a
  \emph{P. falciparum} pueden impedirnos ver el papel de \emph{P. vivax}
  en su totalidad como causa única de una enfermedad grave (Alexandre
  MA, et al. 2010).
\end{enumerate}

\subsection{Definición conceptual de acrónimos y
términos}\label{definicion-conceptual-de-acronimos-y-terminos}

\subsubsection{Acrónimos}\label{acronimos}

\textbf{RTS:} Sistema rápido de expresión de proteínas recombinantes
líbre de células.

\textbf{IVTT:} (Proteína) transcrita/traducida \emph{in vitro}.

\textbf{MFI:} Unidad de lectura cruda expresada como \emph{Mean
Fluorescence Intensity} o intensidad fluorescente promedio de todos los
pixeles de cada \emph{spot}, normalizada localmente mediante la
sustracción de la intensidad de fondo presente a su alrededor.

\subsubsection{Términos}\label{terminos}

\textbf{Malaria.-} Enfermedad transmitida por mosquitos,
predominantemente \emph{Anopheles darlingi} en América del Sur, causada
por un parásito del género \emph{Plasmodium}.

\textbf{Malaria severa.-} Clasificación de los pacientes con malaria
según parámetros clínicos y de laboratorio.

\textbf{Antígeno.-} Molécula que se une con productos de la respuesta
inmune, como anticuerpos o receptores de lifocitos T o
B.\textsuperscript{\protect\hyperlink{ref-abbas2012}{12}}

\textbf{Inmunógeno.-} Un antígeno que induce una respuesta
inmunitaria.\textsuperscript{\protect\hyperlink{ref-abbas2012}{12}}

\textbf{Inmunoma.-} Conjunto de todos los inmunógenos que interactúan
con el sistema inmune de determinado
hospedero\textsuperscript{\protect\hyperlink{ref-immunomics2016}{13},\protect\hyperlink{ref-sette2005}{14}}

\textbf{Inmunómica.-} Estudio del
inmunoma.\textsuperscript{\protect\hyperlink{ref-immunomics2016}{13}}

\textbf{Anticuerpo.-} Tipo de molécula glucoprotéica, también llamada
inmunoglobulina (Ig), producida por los linfocitos B, que se une a
antígenos con un grado alto de especificidad y
afinidad.\textsuperscript{\protect\hyperlink{ref-abbas2012}{12}}

\textbf{Anticuerpo secundario.-} Anticuerpo de unión específica a la
fracción constante de los anticuerpos de un hospedero y, en este caso,
conjugado con biotina.

\textbf{Fluoróforo.-} Componente de una molécula que brinda la cualidad
de fluorescencia y, en este caso, conjugado con estreptavidina.

\textbf{Microarreglo de proteínas.-} 1. Técnica de larga escala para
rastrear las interacciones o actividades de proteínas. 2. Matriz con
\emph{spots} de polipéptidos IVTT, impresos en una lámina portaobjetos y
organizados en 8 bloques paralelos, cada uno con 4 arreglos compuestos
por cuadrículas de 17x17 \emph{spots}.

\textbf{Antígeno-IVTT:} Antígeno objetivo o \emph{spot} con proteína
IVTT a partir de un plásmido con DNA insertado del polipéptido, segmento
o exón de interés.

\textbf{Control-IVTT:} Control negativo o \emph{spot} con mix de
expresión RTS y plásmido sin DNA insertado, representante de la
intensidad de fondo específica del paciente.

\textbf{Proteína purificada:} Control de comparación o \emph{spot} con
proteína de antigenicidad conocida expresada dentro de célula.

\textbf{Transformación:} Logaritmo en base dos de la lectura cruda en
MFI de antígenos-IVTT y control-IVTT.

\textbf{Normalización:} Sustracción de la mediana de los control-IVTT a
cada antígeno-IVTT por individuo.

\textbf{Intensidad de antígeno:} Lectura normalizada de cada
antígeno-IVTT entre individuos. También llamada ``Reactividad de
anticuerpos''.

\textbf{Antígeno reactivo:} Antígenos-IVTT con una lectura transformada
mayor o igual a dos veces la mediana de los control-IVTT por individuo.

\textbf{Frecuencia del antígeno:} Porcentaje de individuos con antígeno
reactivo por antígeno-IVTT.

\textbf{Filtrado:} Remoción de todo antígeno que posea una frecuencia
menor al 10\% de todas las muestras.

\textbf{Intensidad de respuesta:} Promedio de intensidades de antígenos
reactivos por individuo

\textbf{Amplitud de respuesta:} Porcentaje de antígenos reactivos por
individuo.

\section{OBJETIVOS}\label{objetivos}

\subsection{General}\label{general}

\begin{itemize}
\tightlist
\item
  Identificar un subconjunto de antígenos de \emph{P. vivax} con alta
  reactividad serológica o discriminante de condiciones clínicas ante
  infecciones con malaria vivax.
\end{itemize}

\subsection{Específicos}\label{especificos}

\begin{itemize}
\item
  Identificar antígenos con reactividad diferenciada según la severidad
  y exposición previa de los pacientes con malaria vivax.
\item
  Identificar los antígenos con mayor reactividad en los pacientes con
  malaria vivax sintomática.
\item
  Describir las características protéicas de los antígenos identificados
  según parámetros disponibles en la base de datos PlasmoDB.
\end{itemize}

\subsection{Exploratorios}\label{exploratorios}

\begin{itemize}
\item
  Comparar la amplitud e intensidad de respuesta de anticuerpos según la
  exposición y severidad de los pacientes con malaria vivax.
\item
  Comparar la amplitud e intensidad de respuesta de anticuerpos por
  rangos de edad de los pacientes con malaria vivax.
\item
  Comparar la amplitud e intensidad de respuesta de anticuerpos según la
  exposición y severidad por rangos de edad de los pacientes con malaria
  vivax.
\end{itemize}

\section{METODOLOGÍA}\label{metodologia}

\subsection{Diseño de la
investigación}\label{diseno-de-la-investigacion}

Analítico y Descriptivo

\subsubsection{Tipo de investigación}\label{tipo-de-investigacion}

prospectivo, caso-control hospital-based

\textbf{Por el tipo de conocimiento previo usado en la investigación:}
Científica

\textbf{Por la naturaleza del objeto de estudio:} Factual Natural

\textbf{Por el tipo de pregunta planteada:} Teórica Descriptiva de
Relaciones no causales

El estudio analizará las diferencias en una misma población con respecto
a dos variables de interés.

\textbf{Por el método de contrastación de la hipótesis:} No experimental

Se describirá al evento bajo diferentes variables para la organización
de la información, mas no para determinar una causalidad.

\textbf{Por el método de estudio de las variables:} Cuantitativa

Evidenciado en la medición de la intensidad de fluorescencia en la
reacción antígeno-anticuerpo

\textbf{Por el número de variables:} Univariada

El estudio, a pesar de poseer varias variables, no establecerá relación
de efecto entre ellas.

\textbf{Por el ambiente en que se realizan:} De laboratorio

La data será generada en laboratorio. El presente trabajo se centrará en
el análisis de la data.

\textbf{Por el enfoque utilitario predominante:} Teórica

Cuyos resultados permitirán conocer nuevas alternativas para el uso de
biomarcadores en vigilancia sero-epidemiológica.

\textbf{Por la profundidad:} Exploratorio

Se validarán las hipótesis formuladas.

\textbf{Por el tiempo de aplicación de la variable:} Transversal

No se realizará una seguimiento de la población.

\subsection{Selección de la población y
muestra}\label{seleccion-de-la-poblacion-y-muestra}

\subsubsection{Población}\label{poblacion}

Pacientes diagnosticados con malaria vivax provenientes de la ciudad de
Iquitos (Loreto - Perú).

\subsubsection{Criterios de inclusión}\label{criterios-de-inclusion}

\subsubsection{Criterios de exclusión}\label{criterios-de-exclusion}

\subsubsection{Muestra}\label{muestra}

\emph{Tipo de muestra:}

No probabilística: La adquisición de la muestra será mediante la
vigilancia pasiva e inclusión voluntaria inmediatamente después de un
diagnóstico positivo para malaria.

\subsection{Recolección de los datos}\label{recoleccion-de-los-datos}

\subsubsection{Técnica para recolectar
datos}\label{tecnica-para-recolectar-datos}

Los individuos diagnosticados con malaria vivax serán enrolados por
detección pasiva de los casos, es decir, los pacientes con síntomas
semejantes a malaria que se atiendan en cualquiera de los dos hospital
mayores de Iquitos y el en Centro de Salud Delta Uno, localizado cerca a
los campos de minería informal en Madre de Dios.

Se extraerán muestras de sangre al momento del diagnóstico, bajo
consentimiento informado y de forma voluntaria. Se realizará una
extracción de DNA para confirmar mono-infección de P. vivax a través de
un ensayo de PCR. El plasma sanguíneo será colectado y conservado a
-80°C hasta su uso.

En el caso de los pacientes de Iquitos, se les clasificó según su
severidad en base al criterio de la OMS Anemia (hemoglobina \textless{}
7 mg/dL), Ictericia (bilirrubina sérica \textgreater{} 2.5 mg/dL),
Hipoglicemia (glucosa \textless{} 40 mg/dL), Daño pulmonar (es decir,
Síndrome de dificultad respiratoria aguda (SDRA), edema pulmonar), Shock
(presión sanguínea sistólica \textless{} 80 mmHg), Daño renal
(creatinina \textgreater{} 3 mg/dl), o malaria cerebral (puntaje Glasgow
\(\le\) 9/14).

\subsubsection{Instrumento de medición y recolección de
datos}\label{instrumento-de-medicion-y-recoleccion-de-datos}

La plataforma de microarreglo de proteínas a utilizar fue desarrollado
por el Dr.~Phil Felgner y el Dr.~Huw Davis de la Universidad de
California en Irvine. Brevemente, se utilizará la tecnología de alto
rendimiento de un sistema de traducción para expresar 500 proteínas de
P. falciparum y 500 proteínas de P. vivax, que representan aprox. 10\%
de proteoma del parásito. Las proteínas fueron impresos en un
portaobjetos y se probaron con plasma diluido 1/200 después de bloquear
con tampón de bloqueo (Whatman).

\subsubsection{Validez y confiabilidad del instrumento de medición y
recolección}\label{validez}

Se agregarán diluciones de proteínas recombinantes expresadas en un
sistema celular de reactividad conocida, controles positivos para
anticuerpos secundarios, control positivo de anticuerpos específicos
anti-hospedero, antígenos de extracción del parásito, control negativo
sin plásmidos de DNA y control negativo con solo buffer. Se colectarán
individuos controles de origen norteamericano que no hayan tenido
contacto alguno con la infección.

\subsubsection{Aplicación de los instrumentos de recolección de
datos}\label{aplicacion-de-los-instrumentos-de-recoleccion-de-datos}

Las muestras de plasma sanguíneo (2uL de plasma por microarreglo) se
depositarán en las lámina con microarreglos, una muestra por bloque.

. Los valores de fluorescencia serán medidos a través de un lector de
microarreglo con láser confocal, la cual otorgará valores para cada uno
de los arreglos de interacción proteína-anticuerpo, obteniéndose una
matriz final con valores numéricos.

\subsubsection{Codificación y creación del archivo de
datos}\label{codificacion-y-creacion-del-archivo-de-datos}

Cada paciente del estudio será identificado con un código de estructura
\texttt{LIM\#\#\#\#}, e.g.: \texttt{LIM1063}. Los antígenos protéicos
serán identificadas el código asignado a sus genes en la base de datos
PlasmoDB, e.g.: \texttt{PF3D7\_0202500} o \texttt{PVX\_091315}, ya sea
un gen de \emph{P. falciparum} o \emph{P. vivax}, respectivamente. En
caso los genes poseean multiples exones, se amplificarán por separado y
se extenderá el código con el número del exón y el total de exones,
e.g.: \texttt{\_1o2} exón 1 de un gen con 2 exones. En caso los genes
poseean una longitud mayor a 3000 nucleótidos, se dividirán en segmentos
sobrelapantes entre 300 y 3000 nt y se extenderá el código con su
respectivo número, e.g.: \texttt{\_S1} para el primer segmento de un
gen.

Los datos serán organizados en dos matrices: (i) archivo
\texttt{samples.csv} con los códigos de los pacientes y sus covariables
epidemiológicas y (ii) archivo \texttt{RawData.csv} con los códigos de
las proteínas, sus lecturas crudas por código de paciente.

\subsection{Variables del estudio}\label{variables-del-estudio}

La reactividad serológica cuantificará la interacción específica de
antígeno y anticuerpo. El producto será una lectura cruda, con MFI como
unidades, la cual será previamente procesada para el análisis. Esta
cuantificación permitirá conocer las dianas de los anticuerpos
secretados por células plasmáticas en respuesta a la infección.

Los pacientes serán clasificados según episodios previos reportados o
criterios OMS de malaria severa, y ambas variables por presencia o
auscencia de dichos criterios. Con esta estratificación se podrá a
prueba la hipótesis de un perfil de anticuerpos dependiente de los
eventos previos y posteriormente de un perfil dependiente de la
severidad.

\subsubsection{Operacionalización}\label{operacionalizacion}

Ver \autoref{tab:opera}.

\begin{table}[ht]
\begin{center}
\hspace*{-1cm}
\begin{tabular}{>{\centering}m{2.4cm} m{2.2cm}m{2.2cm}m{2cm}m{2.2cm}m{1.7cm}m{1.5cm}m{1.6cm} @{}m{0pt}@{} }
  
  \hline
  \multirow{2}{*}{Variable}
  & 
  \multicolumn{2}{c}{Definición} 
  %&
  %\begin{minipage}{2.2cm}
  %Definición\\conceptual
  %\end{minipage}
  %&
  %\begin{minipage}{2.2cm}
  %Definición\\operacional
  %\end{minipage}
  & 
  \multirow{2}{*}{
  \begin{minipage}{2.2cm}
  Instrumento\\de medición
  \end{minipage}
  }
  &
  \multirow{2}{*}{
  \begin{minipage}{2.2cm}
  Criterios\\de medición
  \end{minipage}
  }
  &
  \multirow{2}{*}{
  \begin{minipage}{1.7cm}
  Tipo de\\variable
  \end{minipage}
  }
  &
  \multirow{2}{*}{
  \begin{minipage}{1.5cm}
  Escala de \\medición
  \end{minipage}
  }
  &
  \multirow{2}{*}{
  Fuente
  } &\\[0ex]
  %\hline
  \cline{2-3}
  
  &
  Conceptual
  &
  Operacional
  & 
  &
  &
  & &\\[1ex]
  \hline
  
  \textbf{Dependiente} Reactividad serológica
  & 
  % esCONCEPTUAL: 
  Especificidad de los anticuerpos de respuesta contra un antígeno
  &
  % aOPERACIONAL: 
  Medida indirecta de la reacción antígeno-anticuerpo
  % aDETALLES: medida indirecta de la reacción antígeno-anticuerpo 
  % mediante la lectura de la reacción fluorescente entre 
  % anticuerpo secundario y fluoroforo por spot
  & 
  \begin{minipage}{2.2cm} 
  Lector de\\
  microarreglos
  \end{minipage}
  & 
  \begin{minipage}{2.2cm} 
  \textbf{0-6000} MFI o unidades de\\
  intensidad de \\fluorescencia.
  \end{minipage} 
  &
  Numérica contínua
  & 
  Razón
  &
  Plasma sanguíneo &\\[10ex]
  \hline

  \textbf{Independiente} Severidad
  & 
  % aCONCEPTUAL: 
  Presencia de manifestaciones clínicas severas y complicaciones sistémicas
  &
  % aOPERACIONAL:
  Número de criterios OMS para malaria severa
  & 
  \begin{minipage}{2.2cm} 
  Diagnóstico \\clínico \\y exámenes de \\laboratorio 
  \end{minipage}
  & 
  \begin{minipage}{2.2cm} 
  \textbf{No-severa:} 0 criterios.\\
  \textbf{Severa:} 1 o más criterios.
  \end{minipage}
  &
  Categórica dicotómica
  & 
  Nominal
  &
  Historia clínica y muestra de sangre &\\[10ex]
  \hline
  
  \textbf{Independiente} Exposición previa
  & 
  % aCONCEPTUAL: 
  Presencia de infecciones de malaria en el pasado
  &
  % aOPERACIONAL:
  Número de eventos previos reportados 
  & 
  Encuesta
  & 
  \begin{minipage}{2.2cm} 
  \textbf{Sin:} 0 eventos.\\
  \textbf{Con:} 1 o más eventos.
  \end{minipage}
  &
  Categórica dicotómica
  & 
  Nominal
  &
  Historia clínica &\\[10ex]
  \hline

  \textbf{Interviniente} Edad %Confusora
  & 
  % aCONCEPTUAL: 
  Edad del paciente
  &
  % aOPERACIONAL:
  Años de vida reportados
  & 
  Encuesta
  & 
  \begin{minipage}{2.2cm} 
  \textbf{0-90} años.
  \end{minipage}
  &
  Numérica discreta
  & 
  Razón
  &
  Historia clínica &\\[10ex]
  \hline


  % etc. ...
\end{tabular}
\hspace*{-1cm}
\end{center}
        \captionof{table}{Operacionalización de variables}
        \label{tab:opera}
\end{table}

\subsection{Procesamiento y Análisis de
datos}\label{procesamiento-y-analisis-de-datos}

Todo el análisis estadístico se realizará en el software de computación
estadística R\textsuperscript{\protect\hyperlink{ref-R}{15}}, el cual se
complementará con funciones provenientes de distintos paquetes.

\subsubsection{Procesamiento}\label{procesamiento}

Preliminarmente, describirá la distribución y proporción de las
covariables epidemiológicas, clínicas y de laboratorio de los pacientes
de la muestra. Luego, con las lecturas crudas en MFI se evaluará la
validez y reproducibilidad del ensayo mediante un test de asociación
entre variables contínuas, usando correlación de Pearson (\(r\)) o
Spearman (\(\rho\)), dependiendo de la distribución. Posteriormente, se
procederá con su transformación a escala logarítmica, normalización
entre muestras y filtrado en base al punto de corte establecido. Por
último, se asociarán ambas matrices de datos en un
\texttt{ExpressionSet}, a través de los códigos de pacientes, empleando
el paquete
\texttt{Biobase}.\textsuperscript{\protect\hyperlink{ref-Biobase}{16}}

\subsubsection{Confirmación de
hipótesis}\label{confirmacion-de-hipotesis}

Se pondrán a prueba las hipótesis con el siguiente protocolo. Primero,
se contrastará la amplitud e intensidad de respuesta con un test de
diferencias entre variables contínuas y no pareadas de dos grupos,
usando t-Student o Mann-Whitney, dependiendo de la distribución.
Segundo, se realizará un test de reactividad diferenciada de anticuerpos
entre dos grupos, usando el test t-moderado empírico de Bayes o
eBayes\textsuperscript{\protect\hyperlink{ref-smyth2004ebayes}{17}} con
corrección por comparación múltiple de la razón de falsos
descubrimientos por el método de Benjamini-Hochberg, disponible en el
paquete
\texttt{limma}.\textsuperscript{\protect\hyperlink{ref-limma}{18}}
Tercero, se realizará un \emph{clustering} jerárquico para agrupar a los
antígenos identificados en base a su distancia euclidiana, disponible en
el paquete
\texttt{NMF}.\textsuperscript{\protect\hyperlink{ref-Gaujoux2010NMF}{19}}
Finalmente, se mostrarán las siguiente características protéicas de los
antígenos identificados: presencia de dominios transmembrana, péptido
señal, número de ortólogos en \emph{Plasmodium} y ontología génica,
disponibles en la base de datos
PlasmoDB.\textsuperscript{\protect\hyperlink{ref-plasmodb}{20}}

\subsubsection{Visualización de
resultados}\label{visualizacion-de-resultados}

Los resultados se visualizarán mediante cinco tipos de gráficos.
Diagrama de dispersión para mostrar la correlación entre dos variables
contínuas, Diagramas de cajas para variables contínuas, Diagrama de
barras para frecuencias, \emph{Volcano plots} para mostrar la
reactividad diferenciada de antígenos entre grupos contra el valor P no
ajustado y resaltar los antígenos con valor P ajustado significativo, y
\emph{Heatmaps} para el perfil serológico en escala de dos colores
segmentado por clusters.

\section{ASPECTOS ADMINISTRATIVOS}\label{aspectos-administrativos}

\subsection{Cronograma de actividades}\label{cronograma-de-actividades}

Dado que las muestras ya han sido colectadas, solo se procederá a
recolectar lo datos a través del instrumento de medición descrito.

\begin{longtable}[]{@{}lllllll@{}}
\toprule
\textbf{ACTIVIDAD PROGRAMADA} & & & & & &\tabularnewline
\midrule
\endhead
& \textbf{Jun} & \textbf{Jul} & \textbf{Ago} & \textbf{Sep} &
\textbf{Oct} & \textbf{Nov}\tabularnewline
Aprovación del proyecto de tesis & X & & & & &\tabularnewline
Procesamiento de datos & & X & & & &\tabularnewline
Interpretación de resultados & & & X & & &\tabularnewline
Redacción final & & & & X & &\tabularnewline
Correcciones & & & & & X &\tabularnewline
\textbf{Sustentación} & & & & & & X\tabularnewline
\bottomrule
\end{longtable}

\subsection{Presupuesto y
financiamiento}\label{presupuesto-y-financiamiento}

El proyecto será realizado en El Centro de Enfermedades Tropicales de la
Marina de los Estados Unidos NAMRU-6 y financiado por ****.

\begin{longtable}[]{@{}lcc@{}}
\toprule
\begin{minipage}[b]{0.46\columnwidth}\raggedright\strut
\textbf{DESCRIPCIÓN}\strut
\end{minipage} & \begin{minipage}[b]{0.22\columnwidth}\centering\strut
\textbf{MONTO (S/)}\strut
\end{minipage} & \begin{minipage}[b]{0.22\columnwidth}\centering\strut
\textbf{PORCENTAJE (\%)}\strut
\end{minipage}\tabularnewline
\midrule
\endhead
\begin{minipage}[t]{0.46\columnwidth}\raggedright\strut
\textbf{Bienes}\strut
\end{minipage} & \begin{minipage}[t]{0.22\columnwidth}\centering\strut
\strut
\end{minipage} & \begin{minipage}[t]{0.22\columnwidth}\centering\strut
\strut
\end{minipage}\tabularnewline
\begin{minipage}[t]{0.46\columnwidth}\raggedright\strut
Papelería, útiles y material de oficina\strut
\end{minipage} & \begin{minipage}[t]{0.22\columnwidth}\centering\strut
50\strut
\end{minipage} & \begin{minipage}[t]{0.22\columnwidth}\centering\strut
0.16\strut
\end{minipage}\tabularnewline
\begin{minipage}[t]{0.46\columnwidth}\raggedright\strut
Insumos, instrumental y accesorios de laboratorio\strut
\end{minipage} & \begin{minipage}[t]{0.22\columnwidth}\centering\strut
200\strut
\end{minipage} & \begin{minipage}[t]{0.22\columnwidth}\centering\strut
0.63\strut
\end{minipage}\tabularnewline
\begin{minipage}[t]{0.46\columnwidth}\raggedright\strut
Productos químicos\strut
\end{minipage} & \begin{minipage}[t]{0.22\columnwidth}\centering\strut
25\strut
\end{minipage} & \begin{minipage}[t]{0.22\columnwidth}\centering\strut
0.08\strut
\end{minipage}\tabularnewline
\begin{minipage}[t]{0.46\columnwidth}\raggedright\strut
\textbf{Servicios}\strut
\end{minipage} & \begin{minipage}[t]{0.22\columnwidth}\centering\strut
\strut
\end{minipage} & \begin{minipage}[t]{0.22\columnwidth}\centering\strut
\strut
\end{minipage}\tabularnewline
\begin{minipage}[t]{0.46\columnwidth}\raggedright\strut
Compra, sondeo y lectura de microarreglos\strut
\end{minipage} & \begin{minipage}[t]{0.22\columnwidth}\centering\strut
29610\strut
\end{minipage} & \begin{minipage}[t]{0.22\columnwidth}\centering\strut
92.79\strut
\end{minipage}\tabularnewline
\begin{minipage}[t]{0.46\columnwidth}\raggedright\strut
Gastos en el transporte de muestras\strut
\end{minipage} & \begin{minipage}[t]{0.22\columnwidth}\centering\strut
2000\strut
\end{minipage} & \begin{minipage}[t]{0.22\columnwidth}\centering\strut
6.27\strut
\end{minipage}\tabularnewline
\begin{minipage}[t]{0.46\columnwidth}\raggedright\strut
Impresiones, encuadernación y empastado\strut
\end{minipage} & \begin{minipage}[t]{0.22\columnwidth}\centering\strut
25\strut
\end{minipage} & \begin{minipage}[t]{0.22\columnwidth}\centering\strut
0.08\strut
\end{minipage}\tabularnewline
\begin{minipage}[t]{0.46\columnwidth}\raggedright\strut
\textbf{TOTAL}\strut
\end{minipage} & \begin{minipage}[t]{0.22\columnwidth}\centering\strut
31910\strut
\end{minipage} & \begin{minipage}[t]{0.22\columnwidth}\centering\strut
100\strut
\end{minipage}\tabularnewline
\bottomrule
\end{longtable}

\section{ANEXOS}\label{anexos}

\subsection{Matriz de consistencia}\label{matriz-de-consistencia}

Ver \autoref{tab:consis} al final del documento.

\section{BIBLIOGRAFÍA}\label{bibliografia}

\begin{enumerate}
\def\labelenumi{\arabic{enumi}.}
\item
  \textbf{Referida al tema}

  Alexandre MA, et al. (2010). Severe Plasmodium vivax malaria,
  Brazilian Amazon. Emerg Infect Dis 16: 1611--1614.

  Andrade BB, et al. (2010). Severe Plasmodium vivax malaria exhibits
  marked inflammatory imbalance. Malar J 9: 13.

  Arevalo-Herrera M, et al. (2012). Malaria in selected non-Amazonian
  countries of Latin America. Acta Trop 121: 303--314.

  Baird JK, (2009). Severe and fatal vivax malaria challenges `benign
  tertian malaria' dogma. Ann Trop Paediatr 29: 251--252.

  Bassat Q, et al. (2011). Defying malaria: fathoming severe Plasmodium
  vivax disease. Nat Med 17: 48--49.

  Carlton JM, et al. (2008). Comparative genomics of the neglected human
  malaria parasite Plasmodium vivax. Nature 455: 757--763.

  Genton B, et al. (2008). Plasmodium vivax and mixed infections are
  associated with severe malaria in children: a prospective cohort study
  from Papua New Guinea. PLoS Med 5: e127.

  Kochar DK, et al. (2005). Plasmodium vivax malaria. Emerg Infect Dis
  11: 132--134.

  Murray CJ, et al. (2012). Global malaria mortality between 1980 and
  2010: a systematic analysis. Lancet 379: 413--431.

  Poespoprodjo JR, et al. (2009). Vivax malaria: a major cause of
  morbidity in early infancy. Clin Infect Dis 48: 1704--1712.

  Westenberger SJ, et al. (2010). A systems-based analysis of Plasmodium
  vivax lifecycle transcription from human to mosquito. PLoS Negl Trop
  Dis 4: e653.

  World Health Organization, 2011. World Malaria Report. Available at:
  \url{http://www.who.int/entity/malaria/world/_malaria/_report/_2011/9789241564403/_eng}.
  pdf.
\item
  \textbf{Referida a la metodología de investigación}

  Baum E, et al. (2015). Submicroscopic and asymptomatic Plasmodium
  falciparum and Plasmodium vivax infections are common in western
  Thailand-molecular and serological evidence. Malaria journal, 14(1),
  95.

  Crompton P., et al. (2010). A prospective analysis of the Ab response
  to Plasmodium falciparum before and after a malaria season by protein
  microarray. Proceedings of the National Academy of Sciences, 107(15),
  6958-6963.

  Molina DM, et al. (2012). Plasmodium vivax Pre-Erythrocytic--Stage
  Antigen Discovery: Exploiting Naturally Acquired Humoral Responses.
  The American journal of tropical medicine and hygiene, 87(3), 460-469.

  Torres KJ, et al. (2014). Genome-Level Determination of Plasmodium
  falciparum Blood Stage Targets of Malarial Clinical Immunity in the
  Peruvian Amazon. Journal of Infectious Diseases, jiu614.

  Vigil A, et al. (2010). Defining the humoral immune response to
  infectious agents using high-density protein microarrays. Future
  Microbiol 5: 241--251.
\end{enumerate}

\textbf{De acuerdo al orden en el que han sido citadas:}

\hypertarget{refs}{}
\hypertarget{ref-WHO2016world}{}
1. WHO. World malaria report 2016. \emph{Geneva: WHO}. 2016;13.

\hypertarget{ref-rosas2016peru}{}
2. Rosas-Aguirre A, Gamboa D, Manrique P, et al. Epidemiology of
plasmodium vivax malaria in peru. \emph{The American Journal of Tropical
Medicine and Hygiene}. 2016;95(6 Suppl):133-144.
doi:\href{https://doi.org/https://doi.org/10.4269/ajtmh.16-0268}{https://doi.org/10.4269/ajtmh.16-0268}.

\hypertarget{ref-quispe2014}{}
3. Quispe AM, Pozo E, Guerrero E, et al. Plasmodium vivax
hospitalizations in a monoendemic malaria region: Severe vivax malaria?
\emph{The American journal of tropical medicine and hygiene}.
2014;91(1):11-17.
doi:\href{https://doi.org/https://doi.org/10.4269/ajtmh.12-0610}{https://doi.org/10.4269/ajtmh.12-0610}.

\hypertarget{ref-barber2015}{}
4. Barber BE, William T, Grigg MJ, et al. Parasite biomass-related
inflammation, endothelial activation, microvascular dysfunction and
disease severity in vivax malaria. \emph{PLoS Pathog}. 2015;11(1):1-13.
doi:\href{https://doi.org/10.1371/journal.ppat.1004558}{10.1371/journal.ppat.1004558}.

\hypertarget{ref-Moreno2013}{}
5. Moreno A, Cabrera-Mora M, Garcia A, et al. Plasmodium coatneyi in
rhesus macaques replicates the multisystemic dysfunction of severe
malaria in humans. \emph{Infection and Immunity}. 2013;81(6):1889-1904.
doi:\href{https://doi.org/10.1128/IAI.00027-13}{10.1128/IAI.00027-13}.

\hypertarget{ref-crompton2010}{}
6. Crompton PD, Kayala MA, Traore B, et al. A prospective analysis of
the ab response to plasmodium falciparum before and after a malaria
season by protein microarray. \emph{Proceedings of the National Academy
of Sciences}. 2010;107(15):6958-6963.

\hypertarget{ref-baldevi2013}{}
7. Baldeviano GC, Leiva KP, Quispe AM, et al. Serum markers of severe
clinical complications during plasmodium vivax malaria monoinfections in
the peruvian amazon basin. In: \emph{Abstract Book of the Astmh 62nd
Annual Meeting, Nov. 13--17, Washington d.C., United States}.; 2013:340.
\url{http://www.astmh.org/ASTMH/media/Documents/AbstractBook2013Final.pdf}.

\hypertarget{ref-hotspots2015}{}
8. Rosas-Aguirre NAL-C Angel AND Speybroeck. Hotspots of malaria
transmission in the peruvian amazon: Rapid assessment through a
parasitological and serological survey. \emph{PLOS ONE}.
2015;10(9):1-21.
doi:\href{https://doi.org/10.1371/journal.pone.0137458}{10.1371/journal.pone.0137458}.

\hypertarget{ref-accelerate2016}{}
9. Quispe AM, Llanos-Cuentas A, Rodriguez H, et al. Accelerating to
zero: Strategies to eliminate malaria in the peruvian amazon. \emph{The
American Journal of Tropical Medicine and Hygiene}.
2016;94(6):1200-1207.
doi:\href{https://doi.org/https://doi.org/10.4269/ajtmh.15-0369}{https://doi.org/10.4269/ajtmh.15-0369}.

\hypertarget{ref-elliott2014}{}
10. Elliott SR, Fowkes F, Richards JS, Reiling L, Drew DR, Beeson JG.
Research priorities for the development and implementation of
serological tools for malaria surveillance. \emph{F1000Prime Rep}.
2014;6:100.

\hypertarget{ref-CienciaReproducible2016}{}
11. Rodríguez-Sanchez F, Pérez-Luque AJ, Bartomeus I, Varela S. Ciencia
reproducible: qué, por qué, cómo? \emph{ECOS}. 2016;25(2):83-92.
doi:\href{https://doi.org/10.7818/ecos.2016.25-2.11}{10.7818/ecos.2016.25-2.11}.

\hypertarget{ref-abbas2012}{}
12. Abbas A, Lichtman A, Pillai S. \emph{Inmunología Celular Y Molecular
+ Student Consult}. Elsevier Health Sciences Spain; 2012.
\url{https://books.google.es/books?id=iQCupVehIXQC}.

\hypertarget{ref-immunomics2016}{}
13. De Sousa KP, Doolan DL. Immunomics: A 21st century approach to
vaccine development for complex pathogens. \emph{Parasitology}.
2016;143(02):236-244.

\hypertarget{ref-sette2005}{}
14. Sette A, Fleri W, Peters B, Sathiamurthy M, Bui H-H, Wilson S. A
roadmap for the immunomics of category a--C pathogens. \emph{Immunity}.
2005;22(2):155-161.

\hypertarget{ref-R}{}
15. R Core Team. \emph{R: A Language and Environment for Statistical
Computing}. Vienna, Austria: R Foundation for Statistical Computing;
2016. \url{https://www.R-project.org/}.

\hypertarget{ref-Biobase}{}
16. Huber, W., Carey, et al. Orchestrating high-throughput genomic
analysis with Bioconductor. \emph{Nature Methods}. 2015;12(2):115-121.
\url{http://www.nature.com/nmeth/journal/v12/n2/full/nmeth.3252.html}.

\hypertarget{ref-smyth2004ebayes}{}
17. Smyth GK, others. Linear models and empirical bayes methods for
assessing differential expression in microarray experiments. \emph{Stat
Appl Genet Mol Biol}. 2004;3(1):3.

\hypertarget{ref-limma}{}
18. Ritchie ME, Phipson B, Wu D, et al. limma powers differential
expression analyses for RNA-sequencing and microarray studies.
\emph{Nucleic Acids Research}. 2015;43(7):e47.

\hypertarget{ref-Gaujoux2010NMF}{}
19. Gaujoux R, Seoighe C. A flexible r package for nonnegative matrix
factorization. \emph{BMC Bioinformatics}. 2010;11(1):367.
doi:\href{https://doi.org/10.1186/1471-2105-11-367}{10.1186/1471-2105-11-367}.

\hypertarget{ref-plasmodb}{}
20. Aurrecoechea C, Brestelli J, Brunk BP, et al. PlasmoDB: A functional
genomic database for malaria parasites. \emph{Nucleic acids research}.
2009;37(suppl 1):D539-D543.

\afterpage{
    \clearpage
    \newgeometry{left=6cm,right=3cm,top=1cm,bottom=1cm}
    \thispagestyle{empty}
    \begin{landscape}
        \centering
\begin{center}
\begin{tabular}{ll >{\centering}m{2cm} m{2cm}m{2cm}m{2cm}m{1.5cm}m{1.6cm} @{}m{0pt}@{} }
  \hline
  
  \multirow{2}{*}{Problema} &
  \multirow{2}{*}{Objetivo} & 
  \multirow{2}{*}{Variable} & 
  \multicolumn{5}{c}{Operacionalización de variable} &\\[1ex]
  \cline{4-8}
  
   & 
   & 
   & 
   Dimensión & 
   Indicador & 
   Instrumento & 
   Escala & 
   Fuente &\\[1ex]
  \hline
  
  \multirow{3}{*}{
  \begin{minipage}{5cm}
  \textit{Plasmodium vivax} es responsable del 90\% de los casos 
  de malaria en el Perú, por los que el desarrollo de una vacuna es 
  urgente.\\
  \newline
  Si bien se conocen marcadores de exposición e inmunidad, 
  las respuestas de anticuerpos a larga escala aún no están 
  completamente identificadas.\\
  \newline
  El presente estudio identificará antígenos con reactividad serológica 
  y relevancia clínica en \textit{P. vivax}
  a partir del perfil de anticuerpos en respuesta a la infección.
  \end{minipage}
  } & 
  
  \begin{minipage}{4cm}
  \underline{General}\\
  Identificar un subconjunto de antígenos con reactividad serológica
  y relevancia clínica en \textit{P. vivax}.
  \end{minipage} & 
  
  
  Reactividad serológica & 
  Reactividad antígeno-anticuerpo & 
  \begin{minipage}{2cm} 
  Intensidad de fluorescencia:\\
  \textbf{0-6000} MFI
  \end{minipage} & 
  Lector de microarreglos & 
  Razón &
  Plasma sanguíneo &\\[14ex]
  \cline{2-8}
  
   & 
  \begin{minipage}{4cm}
  \underline{Específico}\\
  Identificar antígenos de \textit{P. vivax} con reactividad diferenciada
  entre pacientes con y sin eventos previos reportados.
  \end{minipage} & 
  
  Exposición & 
  Eventos previos reportados & 
  \begin{minipage}{2cm} 
  \textbf{0:} sin eventos previos.\\
  \textbf{1 o más:} con eventos previos.
  \end{minipage} & 
  Entrevista & 
  Nominal &
  Paciente &\\[16ex]
  \cline{2-8}

   & 
  \begin{minipage}{4cm}
  \underline{Específico}\\
  Identificar antígenos de \textit{P. vivax} con reactividad diferenciada
  entre pacientes con malaria severa y no-severa.
  \end{minipage} & 
  
  Severidad & 
  Criterios OMS de malaria severa & 
  \begin{minipage}{2cm} 
  \textbf{0:} malaria no-severa.\\
  \textbf{1 o más:} malaria severa.
  \end{minipage} & 
  Diagnóstico clínico y exámenes de laboratorio & 
  Nominal &
  Médico y muestra de sangre &\\[16ex]
  \hline

  % etc. ...
\end{tabular}
\end{center}
        \captionof{table}{Matriz de consistencia}
        \label{tab:consis}
    \end{landscape}
    \restoregeometry
    \clearpage
}


\end{document}
